\documentclass[aspectratio=169]{beamer}
\usepackage[utf8]{inputenc}
\usepackage[spanish]{babel}
\usepackage{amsmath}
\usepackage{amssymb}
\usepackage{graphicx}
\usepackage{booktabs}
\usepackage{multirow}
\usepackage{xcolor}

% Tema y colores
\usetheme{Madrid}
\usecolortheme{default}
\setbeamertemplate{navigation symbols}{}
\setbeamertemplate{caption}[numbered]

% Definir colores personalizados
\definecolor{udea}{RGB}{0,102,51}
\setbeamercolor{structure}{fg=udea}
\setbeamercolor{title}{fg=white,bg=udea}
\setbeamercolor{frametitle}{fg=white,bg=udea}
\setbeamercolor{titlelike}{fg=white,bg=udea}
\setbeamercolor{author}{fg=black}
\setbeamercolor{institute}{fg=black}
\setbeamercolor{date}{fg=black}

% Información del documento
\title[Modelo de Ising - Monte Carlo]{Simulación Monte Carlo del Modelo de Ising en Diferentes Tipos de Redes}
\author[Juanjo]{Juanjo}
\institute[UdeA]{Instituto de Física\\Universidad de Antioquia}
\date{9 de octubre de 2025}

\begin{document}

%===============================================================================
% DIAPOSITIVA 1: PORTADA
%===============================================================================
\begin{frame}
\titlepage
\end{frame}

%===============================================================================
% DIAPOSITIVA 2: CONTENIDO
%===============================================================================
\begin{frame}{Contenido}
\tableofcontents
\end{frame}

%===============================================================================
% DIAPOSITIVA 3: INTRODUCCIÓN
%===============================================================================
\section{Introducción}
\begin{frame}{Introducción}
\begin{block}{Modelo de Ising}
Sistema magnético con espines $S_i = \pm 1$ que pueden orientarse en dos direcciones opuestas
\end{block}

\begin{columns}[T]
\column{0.5\textwidth}
\textbf{Importancia:}
\begin{itemize}
\item Transiciones de fase
\item Magnetización espontánea
\item Fenómenos críticos
\item Aplicaciones en materiales reales
\end{itemize}

\column{0.5\textwidth}
\textbf{Soluciones analíticas:}
\begin{itemize}
\item 1D: No hay transición a T finita (Ising, 1925)
\item 2D cuadrada: $T_c = 2.269\frac{J}{k_B}$ (Onsager, 1944)
\item Otros casos: Métodos numéricos
\end{itemize}
\end{columns}

\vspace{0.3cm}
\begin{alertblock}{Objetivo}
Estudiar el modelo de Ising en diferentes topologías de red (z=2,3,4,8) con dilución magnética usando Monte Carlo
\end{alertblock}
\end{frame}

%===============================================================================
% DIAPOSITIVA 4: MARCO TEÓRICO - HAMILTONIANO
%===============================================================================
\section{Marco Teórico}
\begin{frame}{Marco Teórico: Hamiltoniano}
\begin{block}{Hamiltoniano del Modelo de Ising}
$$H = -J \sum_{\langle i,j \rangle} S_i S_j - H \sum_i S_i$$
\end{block}

\begin{columns}[T]
\column{0.5\textwidth}
\textbf{Parámetros:}
\begin{itemize}
\item $S_i = \pm 1$: espín en sitio $i$
\item $J$: constante de acoplamiento
\item $H$: campo magnético externo
\item $\langle i,j \rangle$: primeros vecinos
\end{itemize}

\column{0.5\textwidth}
\textbf{Casos de estudio:}
\begin{itemize}
\item \textcolor{blue}{Paramagneto}: $J=0$
\item \textcolor{red}{Ferromagneto}: $J>0$
\end{itemize}

\vspace{0.3cm}
\textbf{Magnetización:}
$$m = \frac{1}{n} \sum_i S_i$$
\end{columns}

\vspace{0.3cm}
\begin{exampleblock}{Dilución magnética}
Factor $q = n/N$: fracción de sitios ocupados ($q = 0, 0.5, 0.8$)
\end{exampleblock}
\end{frame}

%===============================================================================
% DIAPOSITIVA 5: MÉTODO MONTE CARLO
%===============================================================================
\begin{frame}{Método Monte Carlo - Algoritmo de Metropolis}
\begin{block}{Distribución de Boltzmann}
$$P(\mu) \propto e^{-\beta E_\mu}, \quad \beta = \frac{1}{k_B T}$$
\end{block}

\begin{columns}[T]
\column{0.55\textwidth}
\textbf{Algoritmo de Metropolis:}
\begin{enumerate}
\item Seleccionar espín aleatorio
\item Calcular $\Delta E$ al invertir espín
\item Si $\Delta E \leq 0$: aceptar
\item Si $\Delta E > 0$: aceptar con\\
$P = e^{-\beta \Delta E}$
\item Repetir hasta equilibrio
\end{enumerate}

\column{0.45\textwidth}
\textbf{Parámetros de simulación:}
\begin{itemize}
\item $L$: tamaño de red
\item Pasos MC: 4000-10000
\item Condiciones periódicas
\item Optimización: Numba JIT
\end{itemize}
\end{columns}

\vspace{0.3cm}
\begin{alertblock}{Ley de estados correspondientes (Paramagneto)}
$$m = \tanh\left(\frac{H}{k_B T}\right)$$
\end{alertblock}
\end{frame}

%===============================================================================
% DIAPOSITIVA 6: TOPOLOGÍAS DE RED
%===============================================================================
\section{Implementación y Topologías}
\begin{frame}{Topologías de Red Estudiadas}
\begin{center}
\begin{tabular}{cccc}
\toprule
\textbf{Topología} & \textbf{Dimensión} & \textbf{z} & \textbf{Tamaño (L)} \\
\midrule
Cadena 1D & 1D & 2 & 100 \\
Hexagonal (Honeycomb) & 2D & 3 & 30 \\
Cuadrada & 2D & 4 & 30 \\
BCC (Cúbica Centrada) & 3D & 8 & 15 \\
\bottomrule
\end{tabular}
\end{center}

\vspace{0.5cm}
\begin{columns}[T]
\column{0.5\textwidth}
% Placeholder para imagen de redes
\begin{center}
\framebox[4cm][c]{
\parbox{3.8cm}{\centering
\textcolor{gray}{[Imagen: Topologías\\de red z=2,3,4,8]}
}
}
\end{center}

\column{0.5\textwidth}
\textbf{Características:}
\begin{itemize}
\item Condiciones de frontera periódicas
\item $z$ = número de coordinación
\item Impacto en $T_c$
\item Dilución: $q \in \{0, 0.5, 0.8\}$
\end{itemize}
\end{columns}
\end{frame}

%===============================================================================
% DIAPOSITIVA 7: RESULTADOS PARAMAGNETISMO - CURVAS m vs H
%===============================================================================
\section{Resultados: Paramagnetismo}
\begin{frame}{Paramagnetismo (J=0): Curvas m vs. H}
\begin{columns}[T]
\column{0.5\textwidth}
\begin{center}
\framebox[5cm][c]{
\parbox{4.8cm}{\centering
\textcolor{gray}{[Gráfica: m vs. H\\para 3 temperaturas\\z=2,3,4,8]}
}
}
\end{center}

\column{0.5\textwidth}
\textbf{Observaciones:}
\begin{itemize}
\item Curvas para $T_1 < T_2 < T_3$
\item Mayor T $\Rightarrow$ respuesta más suave
\item Comportamiento similar para todo $z$
\item $q$ solo cambia escala
\end{itemize}

\vspace{0.3cm}
\textbf{Validación cuantitativa:}
\begin{itemize}
\item RMSE $< 0.02$ en todas las redes
\item $R^2 > 0.99$ con teoría
\item Excelente concordancia
\end{itemize}
\end{columns}

\vspace{0.3cm}
\begin{block}{Conclusión}
El número de coordinación \textbf{no afecta} la respuesta paramagnética (sin interacción entre espines)
\end{block}
\end{frame}

%===============================================================================
% DIAPOSITIVA 8: LEY DE ESTADOS CORRESPONDIENTES
%===============================================================================
\begin{frame}{Ley de Estados Correspondientes}
\begin{columns}[T]
\column{0.5\textwidth}
\begin{center}
\framebox[5cm][c]{
\parbox{4.8cm}{\centering
\textcolor{gray}{[Gráfica: m vs. H/T\\colapso universal\\comparación con tanh]}
}
}
\end{center}

\column{0.5\textwidth}
\textbf{Verificación:}
\begin{itemize}
\item Todas las curvas colapsan en función universal
\item $m = \tanh(H/T)$
\item Válido para todas las $T$
\item Independiente de topología
\end{itemize}

\vspace{0.3cm}
\textbf{Métricas:}
\begin{itemize}
\item RMSE: 0.008-0.016
\item $R^2$: 0.9986-0.9995
\item Errores $< 2\%$
\end{itemize}
\end{columns}

\vspace{0.3cm}
\begin{exampleblock}{Predicción teórica confirmada}
La magnetización paramagnética depende únicamente de la razón $H/T$, no de valores absolutos
\end{exampleblock}
\end{frame}

%===============================================================================
% DIAPOSITIVA 9: RELAJACIÓN ENERGÉTICA
%===============================================================================
\begin{frame}{Relajación del Sistema: Energía vs. Pasos MC}
\begin{columns}[T]
\column{0.5\textwidth}
\begin{center}
\framebox[5cm][c]{
\parbox{4.8cm}{\centering
\textcolor{gray}{[Gráfica: E vs. MCS\\para diferentes redes]}
}
}
\end{center}

\column{0.5\textwidth}
\textbf{Análisis de equilibración:}
\begin{itemize}
\item Estado inicial: aleatorio
\item Relajación rápida ($< 1000$ MCS)
\item Fluctuaciones térmicas
\item Convergencia a equilibrio
\end{itemize}

\vspace{0.3cm}
\textbf{Comportamiento:}
\begin{itemize}
\item Mayor $z$ $\Rightarrow$ $|E|$ mayor
\item Mayor $q$ $\Rightarrow$ más espines activos
\item Estabilización clara
\end{itemize}
\end{columns}

\vspace{0.3cm}
\begin{alertblock}{Validación}
Sistema alcanza equilibrio térmico antes de tomar medidas de magnetización
\end{alertblock}
\end{frame}

%===============================================================================
% DIAPOSITIVA 10: RESULTADOS FERROMAGNETISMO - HISTÉRESIS
%===============================================================================
\section{Resultados: Ferromagnetismo}
\begin{frame}{Ferromagnetismo (J=1): Ciclos de Histéresis}
\begin{columns}[T]
\column{0.5\textwidth}
\begin{center}
\framebox[5cm][c]{
\parbox{4.8cm}{\centering
\textcolor{gray}{[Gráfica: Histéresis\\m vs. H para\\diferentes T y q]}
}
}
\end{center}

\column{0.5\textwidth}
\textbf{Características observadas:}
\begin{itemize}
\item \textcolor{red}{$T < T_c$}: Histéresis clara
\item \textcolor{blue}{$T > T_c$}: Respuesta paramagnética
\item Coercitividad $H_c(T)$
\item Remanencia $M_r(T)$
\end{itemize}

\vspace{0.3cm}
\textbf{Dependencias:}
\begin{itemize}
\item $H_c$ y $M_r$ máximos a T baja
\item Disminuyen al acercarse a $T_c$
\item Mayor $z$ $\Rightarrow$ mayor $T_c$
\item Mayor $q$ $\Rightarrow$ mayor $T_c$
\end{itemize}
\end{columns}

\vspace{0.3cm}
\begin{block}{Transición de fase}
Cambio cualitativo de comportamiento ferromagnético a paramagnético en $T_c$
\end{block}
\end{frame}

%===============================================================================
% DIAPOSITIVA 11: MAGNETIZACIÓN VS TEMPERATURA
%===============================================================================
\begin{frame}{Magnetización vs. Temperatura: Transición de Fase}
\begin{columns}[T]
\column{0.5\textwidth}
\begin{center}
\framebox[5cm][c]{
\parbox{4.8cm}{\centering
\textcolor{gray}{[Gráfica: m vs. T\\para z=2,3,4,8\\y q=0,0.5,0.8]}
}
}
\end{center}

\column{0.5\textwidth}
\textbf{Observaciones:}
\begin{itemize}
\item \textcolor{red}{$T < T_c$}: $m \neq 0$ (orden ferromagnético)
\item \textcolor{blue}{$T > T_c$}: $m \approx 0$ (paramagneto)
\item Transición continua (2º orden)
\item $T_c$ depende de $z$ y $q$
\end{itemize}

\vspace{0.3cm}
\textbf{Temperaturas críticas:}
\begin{itemize}
\item $z=2$ (1D): No transición
\item $z=3$: $T_c \approx 1.0$
\item $z=4$: $T_c \approx 2.27$
\item $z=8$: $T_c \approx 6.3$
\end{itemize}
\end{columns}

\vspace{0.3cm}
\begin{exampleblock}{Red cuadrada (z=4)}
$T_c \approx 2.2-2.4$ vs. Onsager: $T_c = 2.269$ (excelente acuerdo)
\end{exampleblock}
\end{frame}

%===============================================================================
% DIAPOSITIVA 12: MICROESTADOS MAGNÉTICOS
%===============================================================================
\begin{frame}{Microestados Magnéticos (Snapshots)}
\begin{columns}[T]
\column{0.5\textwidth}
\begin{center}
\framebox[5cm][c]{
\parbox{4.8cm}{\centering
\textcolor{gray}{[Imágenes: Snapshots\\a $T \ll T_c$, $T \approx T_c$,\\y $T \gg T_c$]}
}
}
\end{center}

\column{0.5\textwidth}
\textbf{Configuraciones:}
\begin{itemize}
\item \textcolor{red}{$\uparrow$}: espín arriba (rojo)
\item \textcolor{blue}{$\downarrow$}: espín abajo (azul)
\item Blanco: sitio desocupado ($q<1$)
\end{itemize}

\vspace{0.3cm}
\textbf{Análisis visual:}
\begin{itemize}
\item $T \ll T_c$: Dominios extensos
\item $T \approx T_c$: Fluctuaciones críticas
\item $T \gg T_c$: Configuración aleatoria
\end{itemize}
\end{columns}

\vspace{0.3cm}
\begin{block}{Interpretación física}
Los microestados revelan la naturaleza de la transición de fase: orden $\rightarrow$ desorden
\end{block}
\end{frame}

%===============================================================================
% DIAPOSITIVA 13: EFECTO DEL NÚMERO DE COORDINACIÓN
%===============================================================================
\section{Análisis y Discusión}
\begin{frame}{Efecto del Número de Coordinación (z)}
\begin{columns}[T]
\column{0.5\textwidth}
\begin{center}
\framebox[5cm][c]{
\parbox{4.8cm}{\centering
\textcolor{gray}{[Gráfica: $T_c$ vs. z\\para diferentes q]}
}
}
\end{center}

\column{0.5\textwidth}
\textbf{Tendencia observada:}
$$T_c \propto z$$

\vspace{0.3cm}
\textbf{Resultados ($q=1.0$):}
\begin{itemize}
\item $z=2$: Sin transición
\item $z=3$: $T_c \sim 1.0$
\item $z=4$: $T_c \sim 2.27$
\item $z=8$: $T_c \sim 6.3$
\end{itemize}

\vspace{0.3cm}
\textbf{Interpretación:}
\begin{itemize}
\item Más vecinos $\Rightarrow$ mayor cooperatividad
\item Estabiliza orden magnético
\item Aumenta $T_c$
\end{itemize}
\end{columns}

\vspace{0.3cm}
\begin{exampleblock}{Teoría de campo medio}
Predice $T_c \propto zJ$ (consistente con observaciones)
\end{exampleblock}
\end{frame}

%===============================================================================
% DIAPOSITIVA 14: EFECTO DE DILUCIÓN MAGNÉTICA
%===============================================================================
\begin{frame}{Efecto de la Dilución Magnética (q)}
\begin{columns}[T]
\column{0.5\textwidth}
\begin{center}
\framebox[5cm][c]{
\parbox{4.8cm}{\centering
\textcolor{gray}{[Gráfica: $T_c$ vs. q\\para diferentes z]}
}
}
\end{center}

\column{0.5\textwidth}
\textbf{Comportamiento:}
\begin{itemize}
\item $T_c$ disminuye con dilución
\item $q=0$: $m \equiv 0$ (sin espines)
\item $q \approx 0.5$: Cerca de percolación
\item $q=1.0$: Sistema completo
\end{itemize}

\vspace{0.3cm}
\textbf{Reducción de $T_c$:}
\begin{itemize}
\item Red cuadrada, $q=0.8$: $-35\%$
\item Red cuadrada, $q=0.5$: $-70\%$
\item Impacto significativo
\end{itemize}
\end{columns}

\vspace{0.3cm}
\begin{alertblock}{Límite de percolación}
Para $q < q_c$ (umbral de percolación), no hay caminos infinitos y $T_c \rightarrow 0$
\end{alertblock}
\end{frame}

%===============================================================================
% DIAPOSITIVA 15: CONCLUSIONES
%===============================================================================
\section{Conclusiones}
\begin{frame}{Conclusiones}
\begin{columns}[T]
\column{0.5\textwidth}
\textbf{Paramagnetismo (J=0):}
\begin{itemize}
\item Verificación de $m = \tanh(H/T)$
\item RMSE $< 0.02$, $R^2 > 0.99$
\item Ley de estados correspondientes confirmada
\item Independiente de $z$
\end{itemize}

\vspace{0.3cm}
\textbf{Ferromagnetismo (J=1):}
\begin{itemize}
\item Transiciones de fase de 2º orden
\item $T_c$ aumenta con $z$
\item $T_c$ disminuye con dilución
\item $T_c$(cuadrada) $\approx$ 2.27 (vs. Onsager: 2.269)
\end{itemize}

\column{0.5\textwidth}
\textbf{Histéresis:}
\begin{itemize}
\item $H_c$ y $M_r$ máximos a T baja
\item Desaparecen para $T > T_c$
\item Comportamiento físico correcto
\end{itemize}

\vspace{0.3cm}
\textbf{Implementación:}
\begin{itemize}
\item 4 topologías: z=2,3,4,8
\item Algoritmo de Metropolis
\item Optimización con Numba
\item Validación cuantitativa completa
\end{itemize}
\end{columns}

\vspace{0.3cm}
\begin{block}{Logro principal}
Simulación exitosa del modelo de Ising con análisis exhaustivo de efectos topológicos y dilución magnética
\end{block}
\end{frame}

%===============================================================================
% DIAPOSITIVA FINAL: PREGUNTAS
%===============================================================================
\begin{frame}[plain]
\begin{center}
\Huge{\textcolor{udea}{\textbf{¿Preguntas?}}}

\vspace{1cm}
\Large{Gracias por su atención}

\vspace{1cm}
\normalsize
\textbf{Juanjo}\\
Instituto de Física\\
Universidad de Antioquia\\
\vspace{0.5cm}
\textit{Miniproyecto \#1: Modelo de Ising - Monte Carlo}
\end{center}
\end{frame}

\end{document}
