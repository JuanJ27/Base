% Project:  TeX PhD Motivation Letter Template
% Author:   Deep Ghuge

% EDIT main.tex directly

\documentclass[11pt]{report}
\usepackage[margin=0.59in]{geometry}
\usepackage{graphicx}
\usepackage{tikz}
\usetikzlibrary{calc}
\usepackage{hyperref}

\newcommand{\mediumsize}{\fontsize{15pt}{16pt}\selectfont}

\begin{document}
\begin{titlepage}

    \begin{minipage}[t]{0.95\textwidth}
        \hfill
        \raggedleft
        \textbf{Juan Montoya} \\
        Medellín, Colombia \\
        \today\\
        \href{mailto:juan.montoya110@udea.edu.co}{juan.montoya110@udea.edu.co} \\
        +57 300-366-8854 \\
        \href{https://github.com/JuanJ27}{github.com/JuanJ27} \\
        \href{https://www.linkedin.com/in/juan-montoya-68262071/}{linkedin.com/in/juan-montoya}
    \end{minipage}

\raggedright \textbf{Summer Student Programme 2025} \\ CERN \\ Switzerland

\vspace{0.7em}

\raggedright Dear Members of the Selection Committee,\\

\vspace{0.4em}

As a child, I liked science shows on channels like National Geographic and Discovery even if I believed that becoming a physicist was only for the “craziest geniuses.”. In my final year of high school, partly influenced by CERN youtubers, I decided to pursue a bachelor’s degree in physics unaware that the university I chose had connections with the CMS experiment.

\vspace{0.4em}

This unexpected connection gave me the opportunity to land in this field. And it happened when a friend invited me to a research group meeting at my university, where I discovered a team involved in the CMS experiment at CERN called GFIF. And so I joined a project analyzing \(b\)-jets and \(\bar{b}\)-jets at low $pT$, being my first step into High-Energy Physics.

\vspace{0.7em}

During this project, I gained experience working with tools such as ROOT in C++, MadGraph, Pythia, and Delphes. I enjoy using Linux for development due to its open-source nature and flexibility. I have spent countless hours configuring the OS, virtual environments, desktops, and even the kernel on my personal laptop. This dedication has built confidence in working within Debian environments. Even friends have asked me for their Ubuntu-related issues, including recovering a laptop without a GUI and managing fragmented storage.

\vspace{0.7em}

Last year, with some friends, I participated in the Datathon organized by the United Nations and the NASA Space Apps Challenge. My team tackled analysis problems with a focus on delivering user-friendly solutions. It was an experience I really enjoyed, not only because I could apply the knowledge from my undergraduate studies but also because of our commitment, often resulting in two hours of sleep per day. To date, we are establishing a company with the objective of leading the sports analytics market in order to enhance the performance of clubs in our country.

\vspace{0.7em}

With GFIF, I also had the opportunity to present our work at the 9th COMHEP conference and was one of the leaders of the CMS Masterclass at 'Pasto\_03Dec2024,' where the talks and feedback were great. Beyond the academic and technical aspects, I love the collaborative environment that HEP fosters. I am currently exploring options to enhance the performance of our project, considering both neural networks in C++/Python and a potential transition to Julia for its advantages in scientific computing.

\vspace{0.7em}

I am hopeful to join the CERN Summer Student Programme to gain firsthand experience with the workflows, methodologies, and technologies used in major particle physics experiments. While I have acquired foundational knowledge with my local CMS-affiliated group, I believe that being at CERN will deepen my understanding and inspiring new ideas for my ongoing and future research projects.

\vspace{0.7em}

Thank you for your time and consideration. I look forward to the possibility of being part of the CERN Summer Student Programme.

\vspace{0.7em}


\raggedright Yours sincerely,\\
\textbf{Juan Montoya}

\end{titlepage}
\end{document}
