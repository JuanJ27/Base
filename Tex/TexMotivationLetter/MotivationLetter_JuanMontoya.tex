% Project:  TeX PhD Motivation Letter Template
% Author:   Deep Ghuge

% EDIT main.tex directly

\documentclass[11pt]{report}
\usepackage[margin=0.59in]{geometry}
\usepackage{graphicx}
\usepackage{tikz}
\usetikzlibrary{calc}
\usepackage{hyperref}

\newcommand{\mediumsize}{\fontsize{15pt}{16pt}\selectfont}

\begin{document}
\begin{titlepage}

    \begin{minipage}[t]{0.95\textwidth}
        \hfill
        \raggedleft
        \textbf{ Juan Montoya} \\
        Medellín, Colombia \\
        \today\\
        \href{mailto:juan.montoya110@udea.edu.co}{juan.montoya110@udea.edu.co} \\
        +57 300-366-8854 \\
        \href{https://github.com/JuanJ27}{github.com/JuanJ27} \\
        \href{https://www.linkedin.com/in/juan-montoya-68262071/}{linkedin.com/in/juan-montoya}
    \end{minipage}

\raggedright \textbf{Summer Student Programme 2025} \\ CERN \\ Meyrin, Switzerland

\vspace{0.7em}

\raggedright Dear Members of the Selection Committee,\\

\vspace{0.4em}

From an early age, I was captivated by science documentaries on channels like National Geographic and Discovery. Although I once believed that becoming a physicist was reserved for only the “craziest geniuses,” my curiosity persisted. Years later, I chose to pursue a Bachelor’s degree in Physics. Initially, I imagined I might become a teacher or a data analyst in my home country, with little time left for true scientific research. However, everything changed when a friend invited me to a research group meeting at my university, where I discovered a team involved in the CMS experiment at CERN. This opened the door for me to join a project analyzing $b$-jets and $ \bar b$-jets at low transverse momentum—my first real step into High-Energy Physics.

\vspace{0.7em}

During this project, I have gained extensive experience working with tools such as ROOT, MadGraph, Pythia, and Delphes. I enjoy using Linux for development due to its open-source nature and flexibility; I have spent countless hours configuring and troubleshooting package managers, virtual environments, desktops, and even the kernel on my personal laptop. My interest in problem-solving also led me to participate in local hackathons, such as the Datathon organized by the United Nations and the NASA Space Apps Challenge. In both competitions, my team tackled data analysis problems with a focus on delivering clear, user-friendly solutions—an experience that sparked my desire to apply scientific methods to real-world issues.

\vspace{0.7em}

Through my involvement in my university’s research group, I have also had the opportunity to present our work at the 9th COMHEP conference, where the discussions and feedback proved invaluable. Beyond the academic and technical aspects, I genuinely appreciate the collaborative environment that High-Energy Physics fosters. I am currently exploring options to enhance the performance of our classification methods, considering both neural networks in C++/Python and a possible transition to Julia for its emerging advantages in scientific computing.

\vspace{0.7em}

I am eager to join the CERN Summer Student Programme to learn first-hand about the workflows, methodologies, and cutting-edge technologies employed in major particle physics experiments. While I have gained foundational knowledge through my local CMS-affiliated group, I believe that immersing myself at CERN will deepen my understanding and ignite new ideas for my ongoing and future research projects. Ultimately, I hope to contribute meaningfully to the worldwide effort of expanding our knowledge of particle physics, while refining the skills I have cultivated thus far.

\vspace{0.7em}

Thank you for your time and consideration. I look forward to the possibility of being part of the CERN Summer Student Programme.

\vspace{0.7em}

\raggedright Yours sincerely,\\
\textbf{Juan Montoya}

\end{titlepage}
\end{document}
