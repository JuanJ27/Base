\documentclass[aspectratio=169,12pt]{beamer}
\usepackage[utf8]{inputenc}
\usepackage[spanish]{babel}
\usepackage{amsmath,amssymb,amsthm}
\usepackage{graphicx}
\usepackage{tikz}
\usepackage{xcolor}

% Configuración para manejar imágenes faltantes
\setkeys{Gin}{draft=false}
\DeclareGraphicsExtensions{.pdf,.png,.jpg,.jpeg}

% Tema y colores
\usetheme{Madrid}
\usecolortheme{seahorse}

% Configuración del título
\title[Física Estadística]{Entropía, Información y Termodinámica}
\author{Juan Montoya}
\date{1 de octubre de 2025}
\institute{Universidad de Antioquia}

% Configuración de itemize con símbolos personalizados
\setbeamertemplate{itemize item}{\color{blue}$\bullet$}
\setbeamertemplate{itemize subitem}{\color{red}$\circ$}

\begin{document}

% Diapositiva 1: Título
\begin{frame}
\titlepage
\end{frame}

% Diapositiva 2: La Revolución de Shannon
\begin{frame}{La Revolución de Shannon: Información como Concepto Físico}
\begin{columns}
\begin{column}{0.6\textwidth}
\textbf{Teoría de la Información (1948)}
\begin{itemize}
\item \textbf{Entropía de Shannon}: $H = -\sum_i p_i \log_2 p_i$
\item Información $=$ Sorpresa promedio
\item Un bit $=$ respuesta a una pregunta sí/no
\end{itemize}

\vspace{0.3cm}
\textbf{Conexión Fundamental}
\begin{itemize}
\item Entropía termodinámica $\leftrightarrow$ Entropía informacional
\item $S = k \ln W$ (Boltzmann) vs. $H = \sum p_i \log p_i$ (Shannon)
\item \textcolor{red}{Información es física} (Landauer)
\end{itemize}
\end{column}
\begin{column}{0.4\textwidth}
% Placeholder para imagen de Shannon
\begin{figure}
\centering
\fbox{\parbox{0.9\textwidth}{\centering \vspace{1cm} \textbf{Imagen:} \\ Claude Shannon \\ trabajando en Bell Labs \vspace{1cm}}}
\end{figure}
\end{column}
\end{columns}
\end{frame}

% Diapositiva 3: Entropía - Más Allá del Desorden
\begin{frame}{Entropía: Muy distinta a "desorden"}
\begin{columns}
\begin{column}{0.5\textwidth}
\textbf{¿Qué NO es la entropía?}
\begin{itemize}
\item No es simplemente "desorden"
\item No depende de percepciones subjetivas
\item No siempre aumenta localmente
\end{itemize}

\vspace{0.3cm}
\textbf{¿Qué SÍ es la entropía?}
\begin{itemize}
\item Número de microestados: $S = k \ln \Omega$
\item Información faltante sobre el sistema
\item Medida de incertidumbre cuantificable
\end{itemize}
\end{column}
\begin{column}{0.5\textwidth}
% Placeholder para imagen de dados/probabilidad
\begin{figure}
\centering
\fbox{\parbox{0.9\textwidth}{\centering \vspace{0.5cm} \textbf{Imagen:} \\ Dados mostrando probabilidades \\ (más formas de obtener 7 que 2) \vspace{0.5cm}}}
\end{figure}

\vspace{0.1cm}
% Placeholder para gráfico de distribución
\begin{figure}
\centering
\fbox{\parbox{0.9\textwidth}{\centering \vspace{0.5cm} \textbf{Gráfico:} \\ Distribución de Boltzmann \\ en sistemas térmicos \vspace{0.5cm}}}
\end{figure}
\end{column}
\end{columns}
\end{frame}

% Diapositiva 4: Segunda Ley como Principio Estadístico
\begin{frame}{La Segunda Ley: Estadística, No Determinismo}
\begin{columns}
\begin{column}{0.6\textwidth}
\textbf{Interpretación Moderna (Boltzmann)}
\begin{itemize}
\item Los sistemas evolucionan hacia estados \textbf{más probables}
\item $\Delta S \geq 0$ por pura \textcolor{blue}{combinatoria}
\item Ejemplo: gas expandiéndose en una caja
\end{itemize}

\vspace{0.3cm}
\textbf{Ejemplo Cuantitativo}
\begin{itemize}
\item Gas de $N = 10^{23}$ partículas
\item Probabilidad de concentrarse en un lado: $P = 2^{-10^{23}}$
\item Tiempo de espera: $> 10^{10^{20}}$ veces la edad del universo
\end{itemize}

\end{column}
\begin{column}{0.4\textwidth}
% Placeholder para imagen de gas expandiéndose
\begin{figure}
\centering
\fbox{\parbox{0.9\textwidth}{\centering \vspace{0.5cm} \textbf{Imagen:} \\ Expansión libre de gas \\ (de ordenado a disperso) \vspace{0.5cm}}}
\end{figure}

% Placeholder para gráfico de probabilidades
\begin{figure}
\centering
\fbox{\parbox{0.9\textwidth}{\centering \vspace{0.5cm} \textbf{Gráfico:} \\ Microestados vs \\ Macroestados \vspace{0.5cm}}}
\end{figure}
\end{column}
\end{columns}
\end{frame}

% Diapositiva 5: Demonio de Maxwell y Límites Fundamentales
\begin{frame}{El Demonio de Maxwell: Información, Trabajo y Borrado}
\begin{columns}
\begin{column}{0.56\textwidth}
\textbf{La Paradoja (1867)}
\begin{itemize}
\item Demonio inteligente separa moléculas
\item Aparentemente viola la segunda ley
\end{itemize}

\vspace{0.3cm}
\textbf{Resolución Moderna}
\begin{itemize}
\item \textbf{Principio de Landauer}: Borrar 1 bit cuesta $k T \ln 2$ de trabajo
\item El demonio debe borrar información
\end{itemize}

\vspace{0.3cm}
\textbf{Aplicaciones Actuales}
\begin{itemize}
\item Límites en computación cuántica
\item Motores de información nanoscópicos
\end{itemize}
\end{column}
\begin{column}{0.45\textwidth}
% Placeholder para imagen del demonio de Maxwell
\begin{figure}
\centering
\fbox{\parbox{0.9\textwidth}{\centering \vspace{0.5cm} \textbf{Imagen:} \\ Demonio de Maxwell \\ clasificando moléculas \vspace{0.5cm}}}
\end{figure}

% Placeholder para diagrama de motor de Szilard
\begin{figure}
\centering
\fbox{\parbox{0.9\textwidth}{\centering \vspace{0.5cm} \textbf{Diagrama:} \\ Motor de Szilard \\ (información → trabajo) \vspace{0.5cm}}}
\end{figure}
\end{column}
\end{columns}
\end{frame}

% Diapositiva 6: Temperatura Negativa
\begin{frame}{Temperatura Negativa: Más Caliente que Infinito}
\begin{columns}
\begin{column}{0.6\textwidth}
\textbf{¿Qué son las temperaturas negativas?}
\begin{itemize}
\item Sistemas con \textbf{energía máxima} finita
\item Inversión de población: más partículas en estados de alta energía
\end{itemize}

\vspace{0.3cm}
\textbf{Ejemplos Reales}
\begin{itemize}
\item Espines nucleares en campos magnéticos
\item Láseres (inversión de población)
\item Gases cuánticos ultrafríos (experimento alemán 2013)
\end{itemize}
\end{column}
\begin{column}{0.4\textwidth}
% Placeholder para gráfico de entropía vs energía
\begin{figure}
\centering
\fbox{\parbox{0.9\textwidth}{\centering \vspace{0.5cm} \textbf{Gráfico:} \\ Entropía vs Energía \\ (máximo donde $T \to \pm\infty$) \vspace{0.5cm}}}
\end{figure}

% Placeholder para imagen de experimento con átomos ultrafríos
\begin{figure}
\centering
\fbox{\parbox{0.9\textwidth}{\centering \vspace{0.5cm} \textbf{Imagen:} \\ Átomos ultrafríos \\ (temperatura negativa) \vspace{0.5cm}}}
\end{figure}
\end{column}
\end{columns}
\end{frame}

% Diapositiva 7: Fronteras Actuales
\begin{frame}{Termodinámica Cuántica y Gravedad Entrópica}
\begin{columns}
\begin{column}{0.5\textwidth}
\textbf{Termodinámica Cuántica}
\begin{itemize}
\item Máquinas térmicas cuánticas
\item Baterías cuánticas
\item Entrelazamiento como recurso termodinámico
\end{itemize}

\vspace{0.3cm}
\textbf{Gravedad Entrópica}
\begin{itemize}
\item ¿Es la gravedad una fuerza emergente?
\item Conexión con entropía de agujeros negros
\item Propuestas de Verlinde y otros
\end{itemize}
\end{column}
\begin{column}{0.5\textwidth}
% Placeholder para imagen de computadora cuántica
\begin{figure}
\centering
\fbox{\parbox{0.9\textwidth}{\centering \vspace{0.5cm} \textbf{Imagen:} \\ Computadora cuántica \\ (nuevos límites termodinámicos) \vspace{0.5cm}}}
\end{figure}

% Placeholder para imagen conceptual de gravedad entrópica
\begin{figure}
\centering
\fbox{\parbox{0.9\textwidth}{\centering \vspace{0.5cm} \textbf{Concepto:} \\ Gravedad emergente \\ (¿entropía → espacio-tiempo?) \vspace{0.5cm}}}
\end{figure}
\end{column}
\end{columns}
\end{frame}

% Diapositiva 8: Conclusiones y Perspectivas
\begin{frame}{Conclusiones}
\begin{columns}
\begin{column}{0.6\textwidth}
\textbf{Unificación Conceptual}
\begin{itemize}
\item Información conecta termodinámica, mecánica cuántica y computación
\item Entropía: de concepto térmico a medida universal de información
\end{itemize}

\vspace{0.2cm}
\textbf{Implicaciones Filosóficas}
\begin{itemize}
\item La realidad física podría ser informacional
\item $"$It from Bit" (Wheeler)
\end{itemize}

\vspace{0.2cm}
\textbf{Aplicaciones Futuras}
\begin{itemize}
\item Computación cuántica eficiente
\item Nuevas tecnologías de refrigeración
\end{itemize}
\end{column}
\begin{column}{0.4\textwidth}
% Placeholder para imagen conceptual de información y realidad
\begin{figure}
\centering
\fbox{\parbox{0.9\textwidth}{\centering \vspace{0.5cm} \textbf{Concepto:} \\ ¿Información como \\ fundamento de la realidad? \vspace{0.5cm}}}
\end{figure}

% Placeholder para imagen de redes de información
\begin{figure}
\centering
\fbox{\parbox{0.9\textwidth}{\centering \vspace{0.5cm} \textbf{Redes:} \\ Información conectando \\ del bit al cosmos \vspace{0.5cm}}}
\end{figure}
\end{column}
\end{columns}

\vspace{0.3cm}
\centering
\end{frame}

\end{document}