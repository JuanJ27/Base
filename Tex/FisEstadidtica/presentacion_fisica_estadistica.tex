\documentclass[aspectratio=169,12pt]{beamer}
\usepackage[utf8]{inputenc}
\usepackage[spanish]{babel}
\usepackage{amsmath,amssymb,amsthm}
\usepackage{graphicx}
\usepackage{tikz}
\usepackage{xcolor}

% Configuración para manejar imágenes faltantes
\setkeys{Gin}{draft=false}
\DeclareGraphicsExtensions{.pdf,.png,.jpg,.jpeg}

% Tema y colores
\usetheme{Madrid}
\usecolortheme{seahorse}

% Configuración del título
\title[Física Estadística]{Entropía, Información y Termodinámica}
\author{Juan Montoya}
\date{1 de octubre de 2025}
\institute{Universidad de Antioquia}

% Configuración de itemize con símbolos personalizados
\setbeamertemplate{itemize item}{\color{blue}$\bullet$}
\setbeamertemplate{itemize subitem}{\color{red}$\circ$}

\begin{document}

% Diapositiva 1: Título
\begin{frame}
\titlepage
\end{frame}

% Diapositiva 2: La Revolución de Shannon
\begin{frame}{La revolución de Shannon: Información como concepto físico}
\begin{columns}
\begin{column}{0.6\textwidth}
\textbf{Teoría de la Información}
\begin{itemize}
\item \textbf{Entropía de Shannon}: $H = -\sum_i p_i \log_2 p_i$
\item Un bit $=$ respuesta a una pregunta sí/no
\item Entropía termodinámica $\leftrightarrow$ Entropía informacional
\item $S = k \ln W$ (Boltzmann) vs. $H = \sum p_i \log p_i$ (Shannon)
\end{itemize}
\end{column}
\begin{column}{0.4\textwidth}
% Placeholder para imagen de Shannon
\begin{figure}
\centering
\includegraphics[width=\textwidth]{Shannon_maze.png}
\end{figure}
\end{column}
\end{columns}
\end{frame}

% Diapositiva 3: Entropía - Más Allá del Desorden
\begin{frame}{Entropía: Muy distinta a "desorden"}
\begin{columns}
\begin{column}{0.5\textwidth}
\textbf{¿Qué NO es la entropía?}
\begin{itemize}
\item No es simplemente "desorden"
\item No depende de percepciones subjetivas
\item No siempre aumenta localmente
\end{itemize}

\vspace{0.3cm}
\textbf{¿Qué SÍ es la entropía?}
\begin{itemize}
\item Número de microestados: $S = k \ln \Omega$
\item Información faltante sobre el sistema
\item Medida de incertidumbre cuantificable
\end{itemize}
\end{column}
\begin{column}{0.5\textwidth}
% Placeholder para imagen de dados/probabilidad
\begin{figure}
\centering
\includegraphics[width=0.9\textwidth]{dados.png}
\end{figure}
\end{column}
\end{columns}
\end{frame}

% Diapositiva 4: Segunda Ley como Principio Estadístico
\begin{frame}{La segunda ley: Estadística, no determinismo}
\begin{columns}
\begin{column}{0.35\textwidth}
\textbf{Interpretación moderna }
\begin{itemize}
\item $\Delta S \geq 0$
\item Ejemplo: gas expandiéndose en una caja
\end{itemize}

\end{column}
\begin{column}{0.65\textwidth}
% Placeholder para imagen de gas expandiéndose
\begin{figure}
\centering
\includegraphics[width=\textwidth]{micro_vs_macro.png}
\end{figure}
\end{column}
\end{columns}
\end{frame}

% Diapositiva 5: Demonio de Maxwell y Límites Fundamentales
\begin{frame}{El demonio de Maxwell: Información, trabajo y borrado}
\begin{columns}
\begin{column}{0.56\textwidth}
\textbf{Paradoja}
\begin{itemize}
\item Demonio inteligente separa moléculas
\item Aparentemente viola la segunda ley
\end{itemize}

\vspace{0.3cm}
\textbf{Resolución}
\begin{itemize}
\item Borrar 1 bit cuesta $k T \ln 2$ de trabajo
\item El demonio debe borrar información
\end{itemize}

\end{column}
\begin{column}{0.45\textwidth}
% Placeholder para imagen del demonio de Maxwell
\begin{figure}
\centering
\includegraphics[width=\textwidth]{demon.png}
\end{figure}
\end{column}
\end{columns}
\end{frame}

% Diapositiva 6: Temperatura Negativa
\begin{frame}{Temperatura negativa: Más caliente que infinito}
\begin{columns}
\begin{column}{0.6\textwidth}
\textbf{¿Qué son las temperaturas negativas?}
\begin{itemize}
\item Sistemas con energía máxima finita
\item Inversión de población: más partículas en estados de alta energía
\end{itemize}

\vspace{0.3cm}
\textbf{Ejemplos reales}
\begin{itemize}
\item Láseres (inversión de población)
\item Gases cuánticos ultrafríos 
\end{itemize}
\end{column}
\begin{column}{0.4\textwidth}
% Placeholder para gráfico de entropía vs energía
\begin{figure}
\centering
\includegraphics[width=\textwidth]{negative_temp.png}
\end{figure}
\end{column}
\end{columns}
\end{frame}

% Diapositiva 7: Fronteras Actuales
\begin{frame}{Termodinámica cuántica y gravedad entrópica}
\begin{columns}
\begin{column}{0.5\textwidth}
\textbf{Termodinámica cuántica}
\begin{itemize}
\item Máquinas térmicas cuánticas
\item Baterías cuánticas
\end{itemize}

\vspace{0.3cm}
\textbf{Gravedad entrópica}
\begin{itemize}
\item Gravedad como fuerza emergente
\item Conexión con entropía de agujeros negros
\end{itemize}
\end{column}
\begin{column}{0.5\textwidth}
% Placeholder para imagen de computadora cuántica
\begin{figure}
\centering
\includegraphics[width=\textwidth]{gravity.png}
\end{figure}
\end{column}
\end{columns}
\end{frame}

% Diapositiva 8: Conclusiones y Perspectivas
\begin{frame}{Conclusiones}

\textbf{Conceptos}
\begin{itemize}
\item Información conecta termodinámica, mecánica cuántica y computación
\item Entropía: de concepto térmico a medida universal de información
\end{itemize}

\vspace{0.2cm}
\textbf{Implicaciones filosóficas}
\begin{itemize}
\item La realidad física podría ser informacional
\item $"$It from Bit"
\end{itemize}

\vspace{0.2cm}
\textbf{Aplicaciones futuras}
\begin{itemize}
\item Computación cuántica eficiente
\item Nuevas tecnologías de refrigeración
\end{itemize}


\vspace{0.3cm}
\centering
\end{frame}

\end{document}