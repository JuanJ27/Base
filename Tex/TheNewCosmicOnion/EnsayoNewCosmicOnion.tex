\documentclass[12pt]{article}
\usepackage[utf8]{inputenc}
\usepackage[T1]{fontenc}
\usepackage[spanish]{babel}
\usepackage{geometry}
\geometry{margin=2cm}
\usepackage{graphicx}
\usepackage{amsmath,amsfonts,amssymb}

\title{Ensayo sobre “The New Cosmic Onion”}
\author{Juan Montoya}
\date{\today}

\begin{document}
\maketitle

\section{Prológo}
El prólogo explica que, tras décadas de descubrimientos, se ha replanteado "The Cosmic Onion" para actualizar nuestro entendimiento de la materia y el universo, especialmente en vísperas del inicio del LHC. Se describe la evolución del conocimiento, desde las primeras ideas de elementos hasta la detección de partículas subatómicas como quarks y electrones, y la confirmación de teorías como la del Big Bang. El autor destaca la importancia de experimentar para distinguir hechos de especulaciones y cómo los avances, tanto teóricos como experimentales, han transformado nuestra visión del cosmos.

\section{Atomos}
El capítulo describe cómo se desarrolló la idea del átomo, desde las primeras concepciones griegas de partículas indivisibles hasta la formulación moderna. Se explica que Dalton propuso la teoría atómica para explicar las combinaciones químicas y que experimentos posteriores, como los de Thomson y especialmente el de Rutherford con la dispersión de partículas alfa, revelaron un núcleo compacto rodeado por electrones. Además, se introduce la revolución cuántica: Bohr asignó a los electrones órbitas cuantizadas, mientras que la hipótesis de de Broglie, al conferirles propiedades ondulatorias, explica la estabilidad de esos niveles de energía.

\section{El nucleo}
El núcleo atómico es un centro compacto y cargado positivamente, formado por protones y neutrones (llamados nucleones), que concentra casi toda la masa del átomo. Los experimentos de Rutherford, mediante la dispersión de partículas alfa, demostraron que la electricidad intensamente concentrada en el núcleo repele a las partículas entrantes, mientras que la aparente “falta” de masa en algunas mediciones sugería la presencia de otra partícula neutra. Rutherford postuló la existencia de los neutrones, confirmada posteriormente por Chadwick en 1932, lo que permitió explicar la masa de núcleos como el oxígeno, donde los protones aportaban la carga pero no toda la masa medida. La combinación y proporción de protones y neutrones determinan la estabilidad nuclear; las fuerzas fuertes mantienen juntos a estos nucleones, mientras que la repulsión eléctrica entre los protones tiende a desestabilizar al núcleo. Estos equilibrios se reflejan además en fenómenos de transmutación nuclear: la desintegración beta (donde un neutrón se convierte en protón) y la emisión de positrones o rayos alfa, procesos que han permitido, mediante reacciones en cadena y técnicas de datación, entender tanto la estructura interna de la materia como el origen de los elementos en el universo.

\section{Fuerzas de la naturaleza}
El texto explica que, tras aislar los componentes básicos de la materia, el siguiente desafío fue entender las cuatro fuerzas fundamentales. La gravedad, dominante en escalas macroscópicas, atrae la materia pero es insignificante a nivel atómico. La fuerza electromagnética, mucho más potente, mantiene a los electrones en sus átomos y forma las moléculas. En el núcleo, la fuerza fuerte—mediada por partículas como el pion—une protones y neutrones a pesar de la repulsión eléctrica, mientras que la fuerza débil, transmitida por los bosones W y Z, permite las transmutaciones nucleares como la desintegración beta y viola la simetría izquierda-derecha. Estas interacciones, cuyas intensidades y rangos están determinados por constantes fundamentales, conforman las bases para la existencia y evolución de la materia en el universo.

\section{Nuclear Particles and the Eightfold Way}
Yukawa propuso que la fuerte interacción, que mantiene unidos a los nucleones en el núcleo, se produce mediante el intercambio de partículas llamadas piones ($\pi$), de tres tipos según su carga ($\pi^{+}$, $\pi^{-}$ y $\pi^{0}$), cuya masa explica el alcance limitado de la fuerza $\approx(10^{-15} m)$. Posteriormente, el creciente “zoo” de partículas extrañas llevó a Gell-Mann y Neuman a idear la “Eightfold Way”, un esquema de clasificación de los hadrones basado en propiedades como carga y extrañeza, que incluso predijo la existencia del $\Omega^{-}$, confirmada experimentalmente.

\section{Cuarks}
El texto muestra que, al observar patrones en los hadrones, Gell-Mann y Zweig propusieron en 1964 que éstos están compuestos de tres tipos de quarks (up, down y strange) con cargas fraccionarias, lo que da lugar a las familias observadas en la “Eightfold Way”. Estas combinaciones explican propiedades como la carga, la extrañeza y el espín de bariones y mesones. Experimentos en SLAC y CERN confirmaron la existencia de quarks y revelaron la presencia de gluones, los portadores de la fuerza que los mantiene confinados, lo que llevó al desarrollo de la teoría de la Cromodinámica Cuántica (QCD).

\section{QCD: Una teoría para los cuarks}
La teoría de la Cromodinámica Cuántica (QCD) explica que los quarks interactúan mediante una fuerza de "color" para superar el problema de exclusión de Pauli en hadrones como el $\Omega^{-}$. Los quarks, que tienen cargas fraccionarias y vienen en tres colores (rojo, azul y amarillo), se unen en grupos (como los bariones de tres quarks) gracias a los gluones, los portadores de la fuerza de color. QCD predice que, a distancias cortas, los quarks se comportan casi libres (libertad asintótica) y, al separarse, la fuerza crece impidiendo que se aíslen (confinamiento), explicando así la estructura interna de los hadrones.

\section{Fuerza electrodebil}
La interacción electrodébil unifica la fuerza débil y la electromagnética. Este capítulo explica que la fuerza débil, responsable de cambiar quarks (por ejemplo, transformando un quark down en up en la desintegración beta del neutrón), fue descrita inicialmente por Fermi y luego extendida por otros (Klein y Schwinger) hasta convertirse en el modelo $SU(2)\times U(1)$. En ese esquema, los bosones $W^{+}$, $W^{-}$ y el neutral $Z^{0}$ (este último formado por la mezcla de $W^{0}$ y $B^{0}$ mediado por el ángulo de Weinberg) son los portadores que explican tanto las corrientes cargadas como las neutrales, lo que se confirmó experimentalmente con masas del orden de 80 y 91 GeV para los bosones $W$ y $Z$, respectivamente.

\section{Desde el charm hasta el top}

El capítulo expone que las asimetrías en las interacciones débiles (como en la desintegración beta) sugerían que el modelo quark, inicialmente compuesto por up, down y strange, carecía de simetría en la parte inferior. Para arreglarlo, se postuló la existencia de un cuarto quark, el charm, que se emparejaría con el strange, formando así la segunda generación de quarks y resolviendo problemas en las corrientes neutrales. La evidencia se presentó en 1974 con el descubrimiento del J/$\psi$, lo que abrió paso a la identificación de mesones y bariones con charm, y, posteriormente, a la predicción y hallazgo de los quarks bottom y top, completando las tres generaciones de quarks y leptones del Modelo Estándar.

\section{La era del LEP}
LEP fue un colisionador de electrones y positrones de 27 km en CERN (en funcionamiento desde 1989) que permitió producir y estudiar millones de partículas $Z$. Estas mediciones precisas determinaron la masa del $Z$ (91.188 ± 0.002 GeV) y su ancho, confirmando que sólo existen tres tipos de neutrinos ligeros y probando las predicciones electrodébiles del Modelo Estándar. Además, al variar la energía se revelaron pequeñas fluctuaciones debidas a contribuciones virtuales del quark top, lo que permitió inferir indirectamente su masa (alrededor de 170-180 GeV) antes de su descubrimiento. Asimismo, LEP exploró la producción de bosones $W$ y buscó indicios del bosón de Higgs, cuya existencia se predice que engendra la masa de las partículas; aunque aún no se observó, los datos sugerían un límite superior alrededor de 200 GeV. En resumen, LEP consolidó el Modelo Estándar al validar la unificación de las interacciones débiles y electromagnéticas, a la vez que planteó nuevas preguntas sobre el origen de las masas y la posible existencia de partículas adicionales.

\section{Violacion de simetría CP y producción B}
El capítulo explica que la violación de CP, descubierta inicialmente en los kaones en 1964, demuestra que la simetría combinada entre carga (C) y paridad (P) no se conserva universalmente. Esta asimetría, aunque muy sutil (alrededor de 1 en 300 en el caso de los kaones), es crucial porque podría ser la clave para entender por qué el universo está dominado por la materia en vez de la antimateria. Además, el estudio detallado en B-factories y otros experimentos busca medir estas diferencias, ampliando nuestro conocimiento sobre las condiciones subyacentes que permitieron el predominio de la materia.

\section{Neutrinos}
El capítulo explica que los neutrinos son partículas sin carga, producidas en procesos beta, que atraviesan la materia casi sin interactuar. Aunque inicialmente se creía que no tenían masa, evidencias de oscilaciones neutrínicas demostraron que tienen masas distintas, ya que pueden transformar de un tipo ($\nu_{e}$, por ejemplo) a otros ($\nu_{\mu}$ o $\nu_{\tau}$) durante su recorrido. Estos descubrimientos no solo confirmaron el mecanismo de fusión en el Sol, sino que también plantean nuevos retos para entender la física más allá del Modelo Estándar, incluyendo posibles vínculos con dimensiones adicionales y materia oscura.

\section{Mas allá del Modelo Estándar: GUTS, SUSY y Higgs}
El Modelo Estándar es una descripción exitosa de la materia hasta energías de $10^{3}$ GeV, pero no explica el origen de las masas ni la aparente diversidad de parámetros. Se prevé que una teoría más profunda — que incluye la unificación de interacciones (GUTs), la supersimetría (SUSY) y el mecanismo de Higgs — complete su panorama. Según esta visión, la ruptura espontánea de simetría otorgaría masa a las partículas a través del bosón de Higgs, que se espera esté por debajo de 1 TeV, mientras que SUSY ayudaría a enfocar las constantes de acoplamiento y lograr la unificación de la fuerza fuerte, débil y electromagnética a energías muy altas.

\section{Cosmologia, fisica de particulas y el Big Bang}
El universo comenzó como un estado infinitamente caliente y denso y, al expandirse, se enfrió hasta las 3K actuales. Las observaciones de Edwin Hubble (redshift) y los estudios de la radiación cósmica de fondo (COBE, WMAP) permiten estimar la edad del universo en 12 a 14 mil millones de años, confirmando el modelo del Big Bang. Además, la abundancia de elementos ligeros y la evidencia de que la mayor parte de la materia y energía del cosmos es oscura indican que el universo de hoy es el remanente enfriado y estructurado de un pasado caótico y altamente simétrico en el que confluyeron la física de partículas y la cosmología.

\section{Conclusión}
El texto expone las dos posibilidades para el destino final del universo. Una opción es que se expanda indefinidamente hasta que, a través de la desintegración de protones con un tiempo de vida del orden de $10^{3}$ años, toda la materia se erosione y quede solo radiación fría. La otra opción es la colapso gravitacional si neutrinos, que se producen en el Big Bang en gran cantidad, tienen masa suficiente para aportar la mayor parte de la masa total del universo, superando la masa de la materia visible. Se discute además que, aunque las oscilaciones neutrínicas demuestran que al menos una de las tres variedades tiene masa (con diferencias muy pequeñas), las simulaciones y observaciones de estructuras galácticas sugieren que los neutrinos no constituyen toda la materia oscura necesaria para provocar un colapso.

En el epílogo, el autor reflexiona sobre el progreso obtenido en los casi 100 años transcurridos desde la revelación nuclear de Becquerel, recordando avances como la formulación de la Teoría Electrodébil, la confirmación del Modelo Estándar con la aparición de tres generaciones de partículas (incluyendo el sorprendente quark top), y los recientes descubrimientos en cosmología, como la “planitud” del universo y la existencia tanto de materia oscura como de energía oscura. Estas evidencias, junto con conceptos emergentes como las teorías de supercuerdas, indican que el Modelo Estándar es sólo una aproximación y que aún quedan grandes misterios por resolver, tales como el origen de la masa y la asimetría entre materia y antimateria, que futuros experimentos en el LHC y otros instrumentos podrán arrojar luz.

\end{document}