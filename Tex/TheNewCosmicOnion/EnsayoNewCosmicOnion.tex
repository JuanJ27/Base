\documentclass[12pt]{article}
\usepackage[utf8]{inputenc}
\usepackage[T1]{fontenc}
\usepackage[spanish]{babel}
\usepackage{geometry}
\geometry{margin=2cm}
\usepackage{graphicx}
\usepackage{amsmath,amsfonts,amssymb}

\title{Ensayo sobre “The New Cosmic Onion”}
\author{Juan Montoya}
\date{\today}

\begin{document}
\maketitle

\section{Introducción y motivaciones}
Este libro, “The New Cosmic Onion”, es una actualización de “The Cosmic Onion” de Frank Close, que explica de manera accesible los avances en física de partículas y cosmología desde la publicación original en 1983. Mis motivaciones para leer este libro fueron dos, la primera\newcommand{\comment}[1]{}\comment{se desespera, se encojona si se lo hecho afuera} es que hace parte de las actividades de la electiva "\textit{Introducción a la fisica de particulas fundamentales}"\text{ y }la segunda es que se me hizo llamativo el título del libro, ya que me recordó a la famosa frase de Shrek: “Los ogros son como las cebollas, tienen capas”. En este ensayo, se resumirán los capítulos del libro y se comentarán algunos aspectos relevantes.

\section{Prológo}
En el prólogo se cuenta, de manera simplificada, que después de muchas décadas de descubrimientos (sí, muchas décadas) se replanteó "The Cosmic Onion" para actualizar nuestro entendimiento sobre la materia y el universo. Se dice que antes se creía en unas ideas y ahora, pues, se ha avanzado hasta entender partículas como quarks y electrones, y se confirma que el Big Bang es una buena explicación. El autor se asegura en dejar claro que experimentar es clave para separar hechos de especulaciones.

\section{Atomos}
Este capítulo habla de la evolución de la idea del átomo. Se cuenta que las concepciones griegas eran muy básicas, hasta que Dalton llegó y dijo “mira, esto sirve para explicar las combinaciones químicas”. Luego se mencionan experimentos de Thomson, y luego el de Rutherford que mostró que había un núcleo con electrones girando alrededor. También se comenta de forma que Bohr puso a los electrones en órbitas cuantizadas, y de Broglie explicó que tienen propiedades ondulatorias.

\section{El nucleo}
Aquí se explica de forma sencilla que el núcleo es la parte central del átomo, formado por protones y neutrones, y que en su mayoría concentra la masa. Se relata que los experimentos de Rutherford demostraron que el núcleo era compacto y que era importante para entender la repulsión eléctrica. Además, se introduce a los neutrones (luego confirmados por Chadwick en 1932) pero se vuelve a decir lo mismo: el núcleo tiene protones y neutrones, y su equilibrio es esencial. Se hace énfasis en que la estabilidad depende de las fuerzas fuertes, que mantienen a los nucleones juntos, y se menciona la desintegración beta y la emisión de positrones o rayos alfa.

\section{Fuerzas de la naturaleza}
En este capítulo se explica que, después de conocer la materia, el siguiente reto fue entender las fuerzas fundamentales: la gravedad actúa en lo macroscópico pero es casi insignificante en el átomo, mientras que la fuerza electromagnética mantiene a los electrones en sus órbitas. Se comenta la fuerza fuerte, que une el núcleo a pesar de la repulsión eléctrica, y la fuerza débil, que permite transmutaciones como la desintegración beta. Se reitera en varias formas cómo estas interacciones se relacionan con constantes fundamentales.

\section{Particulas nucleares y el Camino Óctuple}
Este capítulo menciona que Yukawa propuso que la interacción fuerte se mediaba por los piones ($\pi$), explicando que hay tres tipos: $\pi^{+}$, $\pi^{-}$ y $\pi^{0}$. Se comenta que la masa de estos piones explica el alcance limitado de la fuerza (alrededor de $10^{-15} m$). Después se menciona que, cuando aparecieron muchas partículas extrañas, Gell-Mann y Neuman idearon el “Camino Óctuple” para clasificarlas, volviendo a repetir que esto predijo la existencia del $\Omega^{-}$, luego confirmado experimentalmente.

\section{Cuarks}
Aquí el texto muestra que Gell-Mann y Zweig propusieron en 1964 que los hadrones estaban compuestos de tres tipos de quarks (up, down y strange). Se dice una y otra vez que estos quarks tienen cargas fraccionarias y que de ahí se explican las familias en la “Camino Óctuple”. Se menciona de nuevo que experimentos en SLAC y CERN confirmaron la existencia de quarks y de gluones, los cuales mantienen a los quarks juntos. En definitiva, muestra que QCD es la teoría detrás de todo esto.

\section{QCD: Una teoría para los cuarks}
En este capítulo se explica de manera muy simple que la Cromodinámica Cuántica (QCD) es la teoría que explica las interacciones entre quarks mediante una fuerza de "color". Se enfatiza que los quarks tienen cargas fraccionarias y que vienen en tres colores (rojo, azul y amarillo) y que se unen en grupos gracias a los gluones. Se comenta de forma poco sofisticada que QCD predice que, a distancias cortas, los quarks se comportan casi como si fueran libres (libertad asintótica) y que al separarse, la fuerza crece impidiendo su aislamiento (confinamiento).

\section{Fuerza electrodebil}
El texto explica que la interacción electrodébil unifica la fuerza débil y la electromagnética. Se expone que la fuerza débil es la que cambia quarks (como pasar de down a up en la desintegración beta del neutrón) y se menciona el modelo $SU(2)\times U(1)$, en el cual los bosones $W^{+}$, $W^{-}$ y el neutral $Z^{0}$ (resultante de la mezcla de $W^{0}$ y $B^{0}$, debido al ángulo de Weinberg) se encargan de estas interacciones. Se reitera que estos bosones tienen masas del orden de 80 y 91 GeV aproximadamente.

\section{Desde el charm hasta el top}
Se menciona que, según el texto, las asimetrías en las interacciones débiles mostraron que el modelo quark de tres tipos (up, down y strange) era insuficiente. Así que se postuló un cuarto quark, el charm, que se empareja con el strange para formar la segunda generación. Se menciona que en 1974, con el descubrimiento del J/$\psi$, se evidenció la existencia del charm, abriendo camino a descubrir mesones y bariones con charm y posteriormente a identificar quarks bottom y top.

\section{La era del LEP}
Aquí se explica que LEP fue un colisionador de electrones y positrones de 27 km en CERN, en funcionamiento desde 1989, que produjo y estudió millones de partículas $Z$. Se repite que estos datos precisos determinaron la masa del $Z$ (91.188 ± 0.002 GeV) y su ancho, confirmando que solo existen tres tipos de neutrinos ligeros. Además, se comenta que al variar la energía se veían fluctuaciones pequeñas por contribuciones del quark top, lo cual permitió inferir su masa (alrededor de 170-180 GeV) antes de su descubrimiento. Todo esto para recalcar que LEP validó el Modelo Estándar.

\section{Violacion de simetría CP y producción B}
Se explica, de manera muy simple y repetitiva, que la violación de CP, descubierta inicialmente en los kaones en 1964, muestra que la simetría combinada de carga (C) y paridad (P) no se conserva en general. Se insiste en que esta asimetría, aunque sutil (alrededor de una de cada 300 veces en los kaones), es muy importante para entender por qué predomina la materia sobre la antimateria. Además, se menciona que estudios en B-factories y otros experimentos buscan medir estas diferencias, repitiendo la idea para enfatizar su relevancia.

\section{Neutrinos}
En este capítulo se menciona que los neutrinos son partículas sin carga, producidas en procesos beta, y que atraviesan la materia casi sin interactuar. Al final menciona que aunque se pensaba que no tenían masa, las oscilaciones neutrínicas demostraron que sí tienen masas varias, ya que se transforman de un tipo a otro. Se menciona que esto es importante tanto para el mecanismo de fusión en el Sol como para entender la física más allá del Modelo Estándar, de forma simplificada.

\section{Mas allá del Modelo Estándar: GUTS, SUSY y Higgs}
El Modelo Estándar se presenta como un modelo exitoso hasta energías de $10^{3}$ GeV, pero que no explica de forma completa el origen de las masas ni la diversidad de parámetros. Se espera una teoría más profunda que incluya GUTs, SUSY y el mecanismo de Higgs. Según esta visión, la ruptura espontánea de simetría otorga masa a las partículas a través del bosón de Higgs (que se espera esté por debajo de 1 TeV), y SUSY ayudaría a unificar la fuerza fuerte, débil y electromagnética a energías altísimas. Se insiste en la importancia de estas ideas sin entrar en muchos detalles complejos.

\section{Cosmologia, fisica de particulas y el Big Bang}
El capítulo explica que el universo comenzó en un estado infinitamente caliente y denso y que, al expandirse, se enfrió hasta los 3K actuales. Las observaciones de Edwin Hubble (redshift) y estudios de la radiación cósmica de fondo (COBE, WMAP) permiten estimar la edad del universo en unos 12 a 14 mil millones de años, confirmando el modelo del Big Bang. Se menciona que la abundancia de elementos ligeros y la evidencia de la materia oscura indican que el universo actual es el resultado enfriado de un pasado caótico.

\section{Conclusión}
Se concluye que hay dos posibilidades para el destino final del universo. Una opción es que se expanda indefinidamente hasta que, a través de procesos como la desintegración de protones (con un tiempo de vida del orden de $10^{3}$ años), toda la materia se erosione y quede solo radiación fría. La otra es el colapso gravitacional, especialmente si los neutrinos, que se producen en el Big Bang, tienen masa suficiente para aportar la mayor parte de la masa del universo, superando a la materia visible. Se repite la idea de que, aunque los neutrinos tienen masas pequeñas (confirmado por oscilaciones neutrínicas), no parecen ser la única causa de la materia oscura necesaria para provocar un colapso. Se cierra reiterando que en casi 100 años desde la revelación nuclear de Becquerel se han logrado muchos avances, pero aún quedan grandes misterios por resolver, como el origen de la masa y la asimetría entre materia y antimateria, esperando que futuros experimentos en el LHC aclaren todo esto.

\end{document}
