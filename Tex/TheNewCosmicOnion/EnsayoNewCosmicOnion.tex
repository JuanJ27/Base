\documentclass[12pt]{article}
\usepackage[utf8]{inputenc}
\usepackage[T1]{fontenc}
\usepackage[spanish]{babel}
\usepackage{geometry}
\geometry{margin=2cm}
\usepackage{graphicx}
\usepackage{amsmath,amsfonts,amssymb}

\title{Ensayo sobre “The New Cosmic Onion”}
\author{Juan Montoya}
\date{\today}

\begin{document}
\maketitle

\section{Introducción y motivaciones}
Este libro, “The New Cosmic Onion”, es una actualización de “The Cosmic Onion” de Frank Close, que explica de manera accesible los avances en física de partículas y cosmología desde la publicación original en 1983. Mis motivaciones para leer este libro fueron dos, la primera\newcommand{\comment}[1]{}\comment{se desespera, se encojona si se lo hecho afuera} es que hace parte de las actividades de la electiva "\textit{Introducción a la fisica de particulas fundamentales}"\text{ y }la segunda es que se me hizo llamativo el título del libro, ya que me recordó a la famosa frase de Shrek: “Los ogros son como las cebollas, tienen capas”. En este ensayo, se resumirán los capítulos del libro y se comentarán algunos aspectos relevantes.

\section{Prológo}
En el prólogo se cuenta, de manera simplificada, que después de muchas décadas de descubrimientos se replanteó "The Cosmic Onion"\text{ }para actualizar el contenido sobre la materia y el universo. La humanidad ha ido desvelando las capas de una “cebolla cósmica”, descubriendo progresivamente que la realidad se compone de niveles cada vez más profundos de estructura y simetría,  avanzado hasta entender partículas como cuarks y electrones. El autor se asegura en dejar claro que experimentar es clave para separar hechos de especulaciones.

\section{Atomos}
El capítulo examina la evolución de la idea del átomo, mostrando cómo nuestro entendimiento se ha profundizado a lo largo del tiempo. Desde los griegos, que planteaban una visión básica de la materia, hasta cuando Dalton propuso que los átomos eran la clave para explicar las combinaciones químicas. Experimentos posteriores, como el de Thomson, revelaron la existencia de partículas subatómicas (electrones) y el de Rutherford demostró que el átomo posee un núcleo cargado positivamente.

Richard Feynman resumió esto con la frase: «We’re made of atoms», enfatizando que aunque parezca sorprendente, somos producto de entidades subatómicas. Sin embargo, a principios del siglo XX se volvia cada vez más evidente que los átomos no eran las entidades más fundamentales de la naturaleza.

Posteriormente, Niels Bohr, se da cuenta de que la experiencia con sistemas macroscópicos no era adecuada para el mundo microscópico y formuló un modelo donde los electrones solo podían ocupar órbitas cuantizadas; esta idea se complementó con la hipótesisd de Broglie, que muestra que los electrones exhiben propiedades corpusculares y ondulatorias. Estableciendo una union entre ideas clásicas y conceptos de la cuántica, cuyo desarrollo posterior por parte de Schrodinger, Heisenberg y Dirac dio lugar a una descripción mucho más profunda y completa de la estructura atómica.

Además, se ilustra que, aunque cada átomo parece vacío en cuanto a su contenido de partículas albergando en su interior electrones y un núcleo, los campos eléctricos y magnéticos que estos generan son tan intensos que impiden, que una persona se hunda en la Tierra. Mientras que si comparamos un átomo con el sistema solar la diferencia en escala parecería menor, en realidad un átomo debería colapsar según las leyes clásicas, pues los electrones, al orbitar, deberían emitir energía y caer al núcleo. Este problema fue resuelto gracias a la introducción del modelo cuántico, donde la cuantización de las órbitas garantiza la estabilidad de los átomos.

El capítulo nos guía a través de la evolución de la teoría atómica, desde los primeros modelos de los griegos, pasando por Dalton, Thomson, Rutherford, Bohr y de Broglie, hasta llegar a la moderna mecánica cuántica. La integración de estos conceptos revela un nuevo entendimiento de la materia con la que interactuamos.

\section{El nucleo}
El libro explica que el núcleo es la parte central del átomo, formado fundamentalmente por protones y neutrones. El núcleo, cargado positivamente, genera campos eléctricos que mantienen a los electrones en órbita y también repelen partículas cargadas, como las partículas alfa, lo que sirvieron para adivinar la existencia de un centro compacto. Rutherford bombardeaba átomos de nitrógeno con partículas alfa, para mostrar la repulsión de estas partículas se debía a la presencia de un núcleo denso; sin embargo, al comparar las masas atómicas se dió cuenta que los protones solo aportaban la mitad de la masa, lo que llevó a postular la existencia del neutrón, una partícula similar en masa a la del protón pero sin carga, confirmada posteriormente por Chadwick en 1932.

Destaca que cuando los nucleones se unen, la interacción entre ellos es tan poderosa debido a la fuerza fuerte, la cual compensa la repulsión eléctrica entre los protones y permite la estabilidad del núcleo. Este equilibrio esta involucrado con la desintegración beta, en la que un neutrón se transforma en un protón, electron y antineutrino, o la emisión de positrones en la llamada “radiación beta positiva,” utilizados incluso en aplicaciones médicas como la tomografía por emisión de positrones (PET).

La comprensión del núcleo se amplió al conocer que la estabilidad de los elementos depende de la relación entre la cantidad de protones y neutrones, siendo más estables aquellos núcleos en los que estos números se equilibran de forma óptima. Esta estructura nuclear ayuda a comprender procesos astrofísicos como la fusión nuclear en el interior del Sol. En resumen se muestra cómo la estabilidad de la materia surge de la interacción entre componentes subatómicos, regidos por la fuerza fuerte.

\section{Fuerzas de la naturaleza}
En este capítulo se explica que, después de conocer la materia, el siguiente reto fue entender las fuerzas fundamentales: la gravedad actúa en lo macroscópico pero es casi insignificante en el átomo, mientras que la fuerza electromagnética mantiene a los electrones en sus órbitas. Se comenta la fuerza fuerte, que une el núcleo a pesar de la repulsión eléctrica, y la fuerza débil, que permite transmutaciones como la desintegración beta. Se reitera en varias formas cómo estas interacciones se relacionan con constantes fundamentales.

\section{Particulas nucleares y el Camino Óctuple}
Este capítulo menciona que Yukawa propuso que la interacción fuerte se mediaba por los piones ($\pi$), explicando que hay tres tipos: $\pi^{+}$, $\pi^{-}$ y $\pi^{0}$. Se comenta que la masa de estos piones explica el alcance limitado de la fuerza (alrededor de $10^{-15} m$). Después se menciona que, cuando aparecieron muchas partículas extrañas, Gell-Mann y Neuman idearon el “Camino Óctuple” para clasificarlas, volviendo a repetir que esto predijo la existencia del $\Omega^{-}$, luego confirmado experimentalmente.

\section{Cuarks}
A partir de la revolución experimental y teórica de la década de 1960, se descubrió que los hadrones están formados por cuarks. Gell-Mann y Zweig propusieron en 1964 el modelo del “Camino óctuple” que podemos ver representado en la figura 6.3(a) del libro, en el que tan solo los cuarks up, down y strange, con cargas fraccionarias (+2/3 para el up y -1/3 para el down), pueden explicar las familias observadas en el agrupamiento de hadrones. Estos al combinarse en tríos, dan lugar a bariones, mientras que la formación de mesones se da por la combinación de un quark y un antiquark. 

\begin{center}
\includegraphics[width=0.85\textwidth]{/home/juan/Imágenes/Screenshot_20250224_231548.png}
\end{center}

Experimentos realizados en SLAC y CERN confirmaron la existencia de cuarks, al mostrar indirectamente, mediante la dispersión de electrone, que los protones presentan una estructura interna de tres cuarks. También, se detectó la presencia de gluones, las partículas portadoras de interacción fuerte, que “pegan” los cuarks dentro de los hadrones, dándoles confinamiento. Aunque los cuarks parecen comportarse casi libres en experimentos de alta energía, en condiciones normales permanecen unidos de manera permanente dentro de los hadrones. Este problema dio pie al desarrollo y consolidación de la teoría de la cromodinámica cuántica (QCD), que explica las interacciones entre cuarks a través del intercambio de gluones.

Adicionalmente quiero destacar la buena labor didactica del autor con sus graficos, en este caso con los cuarks;

\begin{center}
\includegraphics[width=0.45\textwidth]{/home/juan/Imágenes/Screenshot_20250224_232030.png}
\end{center}

\section{QCD: Una teoría para los cuarks}
En este capítulo se explica de manera muy simple que la Cromodinámica Cuántica (QCD) es la teoría que explica las interacciones entre cuarks mediante una fuerza de "\text{c}olor". Se enfatiza que los cuarks tienen cargas fraccionarias y que vienen en tres colores (rojo, azul y amarillo) y que se unen en grupos gracias a los gluones. Se comenta de que QCD predice que, a distancias cortas, los cuarks se comportan casi como si fueran libres (libertad asintótica) y que al separarse, la fuerza crece impidiendo su aislamiento (confinamiento).

\section{Fuerza electrodebil}
El texto explica que la interacción electrodébil unifica la fuerza débil y la electromagnética. Se expone que la fuerza débil es la que cambia cuarks (como pasar de down a up en la desintegración beta del neutrón) y se menciona el modelo $SU(2)\times U(1)$, en el cual los bosones $W^{+}$, $W^{-}$ y el neutral $Z^{0}$ (resultante de la mezcla de $W^{0}$ y $B^{0}$, debido al ángulo de Weinberg) se encargan de estas interacciones. Se reitera que estos bosones tienen masas del orden de 80 y 91 GeV aproximadamente.

\section{Desde el charm hasta el top}
Se menciona que, según el texto, las asimetrías en las interacciones débiles mostraron que el modelo quark de tres tipos (up, down y strange) era insuficiente. Así que se postuló un cuarto quark, el charm, que se empareja con el strange para formar la segunda generación. Se menciona que en 1974, con el descubrimiento del J/$\psi$, se evidenció la existencia del charm, abriendo camino a descubrir mesones y bariones con charm y posteriormente a identificar cuarks bottom y top.

\section{La era del LEP}
Aquí se explica que LEP fue un colisionador de electrones y positrones de 27 km en CERN, en funcionamiento desde 1989, que produjo y estudió millones de partículas $Z$. Se repite que estos datos precisos determinaron la masa del $Z$ (91.188 ± 0.002 GeV) y su ancho, confirmando que solo existen tres tipos de neutrinos ligeros. Además, se comenta que al variar la energía se veían fluctuaciones pequeñas por contribuciones del quark top, lo cual permitió inferir su masa (alrededor de 170-180 GeV) antes de su descubrimiento. Todo esto para recalcar que LEP validó el Modelo Estándar.

\section{Violacion de simetría CP y producción B}
Se explica, de manera muy simple y repetitiva, que la violación de CP, descubierta inicialmente en los kaones en 1964, muestra que la simetría combinada de carga (C) y paridad (P) no se conserva en general. Se insiste en que esta asimetría, aunque sutil (alrededor de una de cada 300 veces en los kaones), es muy importante para entender por qué predomina la materia sobre la antimateria. Además, se menciona que estudios en B-factories y otros experimentos buscan medir estas diferencias, repitiendo la idea para enfatizar su relevancia.

\section{Neutrinos}

En este capítulo se explica que los neutrinos son partículas fundamentales sin carga, producidas en procesos beta junto con los electrones, lo que sugiere una relación similar a la de los cuarks up y down en la desintegración nuclear. Se pensaba que los neutrinos no tenian masa, pero gracias a las observaciones de oscilaciones neutrínicas—detectadas en experimentos tanto con rayos cósmicos como en líneas base largas (como en los experimentos de KEK, MINOS y SNO) se comprobó que existen tres variedades ($\nu_{e}$, $\nu_{\mu}$ y $\nu_{\tau}$) con masas diferentes, permitiendo que los electron-neutrinos del el Sol se transformen en los otros tipos antes de llegar a la Tierra. 

Adicionalmente, se menciona su posible papel en la física más allá del Modelo Estándar. Una hipótesis, basada en el mecanismo see-saw, sugiere que si existen neutrinos derechos supermasivos, estos podrían explicar la pequeña masa de los neutrinos conocidos mediante fluctuaciones muy raras. Este resultado implica una disparidad inmensa y ofrece la primera pista de un mundo de ultra-altas energías, posiblemente vinculado a la materia oscura y a la formación de estructuras cosmológicas. Siendo la comprensión actual de los neutrinos y sus oscilaciones importante para explicar el mecanismo de fusión solar y también abre la puerta a descubrir nuevos horizontes en la física fundamental, permitiendo avanzar hacia una teoría más completa que supere al Modelo Estándar.

\section{Mas allá del Modelo Estándar: GUTS, SUSY y Higgs}
El Modelo Estándar se presenta como un modelo exitoso hasta energías de $10^{3}$ GeV, pero que no explica de forma completa el origen de las masas ni la diversidad de parámetros. Se espera una teoría más profunda que incluya GUTs, SUSY y el mecanismo de Higgs. Según esta visión, la ruptura espontánea de simetría otorga masa a las partículas a través del bosón de Higgs (que se espera esté por debajo de 1 TeV), y SUSY ayudaría a unificar la fuerza fuerte, débil y electromagnética a energías altísimas. Se insiste en la importancia de estas ideas sin entrar en muchos detalles complejos.

\section{Cosmologia, fisica de particulas y el Big Bang}
El capítulo explica que el universo comenzó en un estado infinitamente caliente y denso y que, al expandirse, se enfrió hasta los 3K actuales. Las observaciones de Edwin Hubble (redshift) y estudios de la radiación cósmica de fondo (COBE, WMAP) permiten estimar la edad del universo en unos 12 a 14 mil millones de años, confirmando el modelo del Big Bang. Se menciona que la abundancia de elementos ligeros y la evidencia de la materia oscura indican que el universo actual es el resultado enfriado de un pasado caótico.

\section{Epílogo}
Se concluye que hay dos posibilidades para el destino final del universo. Una opción es que se expanda indefinidamente hasta que, a través de procesos como la desintegración de protones (con un tiempo de vida del orden de $10^{3}$ años), toda la materia se erosione y quede solo radiación fría. La otra es el colapso gravitacional, especialmente si los neutrinos, que se producen en el Big Bang, tienen masa suficiente para aportar la mayor parte de la masa del universo, superando a la materia visible. Aunque los neutrinos tienen masas pequeñas (confirmado por oscilaciones neutrínicas), no parecen ser la única causa de la materia oscura necesaria para provocar un colapso. Se cierra reiterando que en casi 100 años desde la revelación nuclear de Becquerel se han logrado muchos avances, pero aún quedan grandes misterios por resolver, como el origen de la masa y la asimetría entre materia y antimateria, esperando que futuros experimentos en el LHC aclaren todo esto.

\section{Conclusión y comentarios}
En “The New Cosmic Onion” se invita conocer las capas profundas de la materia y del universo, desde las primeras ideas atomistas de los antiguos griegos hasta los últimos avances en la física de partículas y la cosmología. El autor articula cómo con descubrimientos revolucionarios la humanidad ha ido desvelando un universo cada vez más complejo. Se destacan experimentos en colisionadores como LEP y el LHC, que permiten recrear condiciones del Big Bang y ponen de relieve la posibilidad de una unificación de las fuerzas fundamentales.


El libro plantea preguntas abiertas que animan a futuros descubrimientos. Siendo un testimonio fuertemente divulgativo y accesible de la evolución de la física de partículas y la cosmología, tanto asi que hasta el propio autor del libro participo de un show de televisión con este mismo esporitu llamado \textit{The Cosmic Onion}.

\begin{center}
    \includegraphics[width = 0.5\textwidth]{/home/juan/Imágenes/Screenshot_20250224_233254.png}
\end{center}


\end{document}
