\documentclass[
    aps,                % American Physical Society
    pra,                % choose journal (pra, prb, prc, prd, pre, prl, rmp)
    reprint,            % two-column layout
    longbibliography,
    superscriptaddress % author affiliations as superscripts
]{revtex4-2}

\usepackage[utf8]{inputenc}
\usepackage[T1]{fontenc}
\usepackage[spanish, english]{babel}
\usepackage{amsmath,amssymb}
\usepackage{graphicx}
\usepackage{siunitx}
\usepackage{braket}
\usepackage[colorlinks=true, allcolors=blue]{hyperref}
\usepackage{setspace}
%\usepackage[top=2.5cm, left=2cm, right=2cm]{geometry}
\setstretch{1.3}  % Ajusta este valor según necesites

\begin{document}
\selectlanguage{spanish}
\title{\LARGE Aplicación del Quantum Machine Learning para la identificación de jets en el LHC}

\author{Juan Montoya}
\email{juan.montoya110@udea.edu.co}
\affiliation{Instituto de Física, Universidad de Antioquia, Colombia}

\date{\today}

\maketitle

\section{Introducción}
\label{sec:intro}

La computación cuántica se presenta como un campo computacional nuevo que, diferente de la computación clásica, utiliza los principios de la mecánica cuántica para procesar información \cite{ref:qml}. Su creciente relevancia en el entorno científico e industrial actual se da debido a su potencial para resolver problemas computacionalmente intratables para sistemas clásicos. Este potencial tiene origen en dos propiedades fundamentales: la superposición cuántica y el entrelazamiento.

A diferencia de los bits clásicos que son binarios, osea, únicamente pueden representar valores de 0 o 1, los bits cuánticos o qubits pueden existir en una superposición de estados $\ket{0}$ y $\ket{1}$ \cite{superposition_entanglement}. Esta propiedad permite que un sistema de $n$ qubits pueda representar de forma simultanea $2^n$ estados, mientras que los bits clásicos solo podrian representar un único estado entre $2^n$ posibilidades. Por otro lado el entrelazamiento, establece correlaciones entre los qubits, permitiendo que el estado de un sistema cuántico no pueda descomponerse como producto de estados individuales, permitiendonos la creación de algoritmos con la capacidad de resolver ciertos problemas con ventajas exponenciales respecto a sus contrapartes clásicas.

Los circuitos cuánticos son el entorno practico de los algoritmos cuánticos. Están formados por puertas logicas cuánticas que con diferencia de las puertas lógicas clásicas, son reversibles y capaces de entrelazar los qubits \cite{quantum_gates}. La propiedad de reversibilidad es fundamental, ya que garantiza la unitariedad de las operaciones y que propiamente la mecanica cuantica sea bien representada ya que esta misma es reversible.

Ahora estamos en la denominada era NISQ (Noisy Intermediate-Scale Quantum), caracterizada por dispositivos cuánticos de escala intermedia (orden de $10$ a $10^{2}$ qubits) y pese que son limitados por el ruido y la falta de corrección de errores, muestran potencial para aplicaciones prácticas en diferentes dominios \cite{nisq_era}. Asi, el Machine Learning Cuántico (QML) surge como un campos prometedor para mostrar posibles ventajas prácticas de la computación cuántica.

El QML une la computacion cuántica con técnicas de aprendizaje automático, explorando cómo los circuitos pueden mejorar o complementar los algoritmos de ML clásicos. Estos circuitos se plantean como análogos cuánticos de los modelos de ML clásicos, donde los parámetros se optimizan con calculos iterados para minimizar funciones de coste. La hipótesis central del QML plantea que las propiedades cuánticas, como pueden ser la capacidad de procesar estados muy grandes a través de la superposición y el entrelazamiento, proporcionarian ventajas computacionales grandes para ciertos casos del aprendizaje automático relacionados con datos de alta dimensionalidad \cite{qml_hypothesis}.

Una aplicación del QML esta en la física de partículas, específicamente en el análisis de datos generados por el Gran Colisionador de Hadrones (LHC) en el CERN. Este acelerador produce anualmente alrededor de 50 petabytes de datos que requieren de poder computacional para ser procesados \cite{lhc_data_volume}. Entre los desafíos de este analisis se encuentra la clasificación de jets hadrónicos, estructuras que emergen de las colisiones de partículas.

Los jets son chorros de partículas producidos mediante la fragmentación y hadronización de quarks y gluones que emergen de las colisiones de los protones. Son importantes porque son el observable para los quarks y gluones, porque debido al confinamiento de color no pueden ser detectados directamente \cite{color_confinement}. La clasificación de jets y la identificación de la carga de los quarks que forman los hadrones del jet son un problema computacionalmente complejo debido a las correlaciones no lineales entre las numerosas partículas que conforman estas estructuras ademas del gran volumen de datos.

El QML ofrece una direccion distinta para abordar este problema de clasificación usando los circuitos cuánticos. Empezamos con la incrustacion de los datos clásicos del jet en un estado cuántico, conocido como $"$embedding$"$. Entre los procesos de incrustacion se encuentra el $"$angle embedding$"$ \cite{angle_embedding}, que incrusta la información clásica como ángulos de rotación en las puertas cuánticas. Así, cada característica del jet (como momentum, carga, etc.) se representa mediante rotaciones aplicadas a qubits, pasando del espacio de características clásico al estado cuántico dentro del espacio de Hilbert.

La esfera de Bloch \cite{bloch_sphere} da una representación geométrica intuitiva para entender esta incrustacion. En esta representación esferica, los estados $|0\rangle$ y $|1\rangle$ son los polos norte y sur respectivamente y cualquier punto sobre la superficie de la esfera representa un estado en superposición. El angle embedding asigna al dato clásico un punto específicos en la esfera. Despues sigue aplicar un conjunto de puertas parametrizadas para entrelazar los qubits, creando correlaciones que potencialmente capturaren relaciones complejas entre las características del jet más eficientemente que el ML clasico.

La superposición permite explorar paralelamente múltiples configuraciones, mientras el entrelazamiento explora correlaciones entre las partículas del jet. Esto es interesante para el etiquetado de carga en jets-$b$, donde estas correlaciones de las partículas contienen información sobre la carga del quark original \cite{ref:qmlhiggs}.

Para evaluar el potencial del QML para el etiquetado de carga de jets-$b$, usaremos datos simulados del experimento LHCb y CMS \cite{ref:lhcb}. El LHCb, especializado en el estudio de hadrones que contienen quarks $b$ y $c$, presenta investigaciones previas al respecto, útiles para nuestro análisis debido a que plantea un estandar y una referencia para comparar el desarrollo aqui presentado. El problema específico que abordamos se formula como una clasificación binaria entre jets-$b$ (originados por quarks $b$) y jets-$\bar{b}$ (originados por antiquarks $\bar{b}$), utilizando como variables de entrada las propiedades cinemáticas y geometricas de las partículas constituyentes del jet para el caso del experimento del CMS.

A nivel de metodo planteamos el analisis de la implementación de los circuitos cuánticos variacionales optimizados específicamente para este problema. Empleamos simuladores cuánticos de uso de GPU para evaluar diferentes arquitecturas de circuito, estrategias de embedding y protocolos de entrenamiento. Este enfoque pretende investigar sistemáticamente cómo las propiedades de los cicuitos cuánticos influyen en el rendimiento del clasificador.

Nuestro estudio se sitúa en un contexto más amplio de cuánticas en física de altas energías. En este entorno la comunidad cientifca, a través de iniciativas como el grupo Quantum Computing for HEP (QC4HEP) \cite{PRXQuantum.4.027001}, ha planteado tres áreas principales donde la computación cuántica podría ofrecer ventajas significativas:

1. Algoritmos de operación de detectores, encargados de discriminar señal frente a ruido en flujos masivos de datos crudos.

2. Algoritmos de identificación y reconstrucción, cuyo objetivo es transformar señales de píxeles, tiempos y energía en objetos fundamentales (electrones, muones, jets) y extraer sus propiedades.

3. Herramientas de simulación e inferencia, que confrontan datos experimentales con predicciones teóricas mediante modelos estadísticos paramétricos.

Nuestra aplicación de QML al etiquetado de carga en jets-$b$ se enfoca principalmente en la segunda categoría. Por lo tanto, evaluamos dos arquitecturas cuánticas principales:

1. Quantum Neural Networks (QNN) \cite{qnn_architecture}: Implementan transformaciones no lineales en el espacio de Hilbert mediante secuencias alternadas de capas de rotación y entrelazamiento, permitiendo generar separaciones de hiperplanos complejos para la clasificación.

2. Quantum Convolutional Neural Networks (QCNN): Adaptan los principios de convolución al dominio cuántico, aplicando operaciones locales seguidas de pooling para extraer características jerárquicas de los datos de jets.

Este estudio no solo contribuye al repertorio de aplicaciones que tiene el QML en física de partículas (como separación señal-fondo \cite{ref:qmlappllhiggs}, detección de anomalías \cite{ref:qoptjets} y reconstrucción de trazas \cite{ref:qmlappltracking}), sino que también explora aplicaciones prácticas de la computación cuántica en la era NISQ. En las siguientes secciones presentamos metodológicamente nuestra implementación de circuitos cuánticos variacionales para el etiquetado de carga en jets-$b$ y discutimos los resultados obtenidos en comparación con técnicas clásicas.





\onecolumngrid
\section{Referencias}   
\bibliography{references} % archivo references.bib

\end{document}