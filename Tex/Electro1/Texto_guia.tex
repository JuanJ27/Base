\documentclass[12pt,a4paper]{book}
\usepackage[utf8]{inputenc}
\usepackage[spanish]{babel}
\usepackage{amsmath,amssymb,amsfonts}
\usepackage{graphicx}
\usepackage{geometry}
\geometry{margin=2.5cm}
\usepackage{hyperref}
\hypersetup{colorlinks=true,linkcolor=blue}

\begin{document}

% Portada
\begin{titlepage}
\begin{center}
\vspace*{3cm}
{\huge \textbf{Electrodinámica I}}\\[2cm]

{\Large E. Reyes Gómez}\\[6cm]

{\large Notas para un curso de Pregrado}\\[5cm]

{\small 1803}\\[1cm]

{\small Universidad de Antioquia\\
Facultad de Ciencias Exactas y Naturales\\
Instituto de Física\\
Medellín}\\[1cm]

{\small Última actualización: 06 de Febrero de 2012}
\end{center}
\end{titlepage}

% Prefacio
\chapter*{Prefacio}
\addcontentsline{toc}{chapter}{Prefacio}

Las notas de Electrodinámica I han sido escritas tomando como base las conferencias impartidas por el autor en el Instituto de Física de la Universidad de Antioquia desde el segundo semestre del año 2006 hasta la actualidad. Dichas notas están dirigidas fundamentalmente a los estudiantes de la especialidad de Física que asisten al curso Electromagnetismo I dictado en el sexto semestre de la carrera, pero pueden ser de interés también para aquellos que se sientan atraídos por la Física Teórica. Para una mejor comprensión de las cuestiones analizadas en las notas es recomendable que el lector posea conocimientos de Matemáticas Superiores, que abarquen desde el cálculo diferencial e integral en una y varias variables hasta la solución de ecuaciones diferenciales en derivadas parciales de segundo orden. No pretendemos que el presente material sustituya textos clásicos como por ejemplo el de Landau y Lifshitz, o como el célebre libro de Jackson. Sólo aspiramos a que las notas que se presentan a continuación constituyan una modesta ayuda a la comprensión por parte del lector de los temas fundamentales que se tratan aquí.

Un agradecimiento especial a los Profesores del Instituto de Física Nicolás Raigoza y Lorenzo de la Torre por la lectura crítica de la primera versión del material, por sus comentarios, útiles sugerencias y necesarios señalamientos que ya han sido corregidos. Deseo extender también mi gratitud a un grupo de estudiantes que han cooperado con el mejoramiento sistemático de estas notas. Ellos son:

\begin{itemize}
\item David Muñetón y Sebastián Sánchez Goez, estudiantes del curso Electromagnetismo I dictado en el primer semestre del año 2009, cuyas dudas contribuyeron al esclarecimiento de algunos temas en la versión inicial de las notas, así como a la corrección de algunas cuestiones de la sección 4.3 que no habían sido correctamente presentadas.

\item Andrés Ordoñez, estudiante del curso Electromagnetismo I dictado en el primer semestre del año 2010, quien corrigió un error en la expresión (1.70) de la versión inicial del manuscrito, además de contribuir a la claridad en la presentación de la sección 4.2.1.
\end{itemize}

\begin{flushright}
E. Reyes Gómez\\
Medellín, Septiembre de 2010
\end{flushright}

% Índice general
\tableofcontents

\chapter{Electrostática del vacío}
\section{Principios básicos de la Electrostática}
\subsection{Ley de Coulomb}

La ley de Coulomb posee un basamento experimental, y fue formulada por Charles-Augustin de Coulomb en 1785 [1]. Esta ley empírica plantea que la fuerza de interacción entre dos cargas eléctricas es directamente proporcional al producto de las cargas interactuantes, inversamente proporcional al cuadrado de la distancia que las separa, y actúa a lo largo de la dirección del segmento de recta que las une. La constante de proporcionalidad depende, obviamente, del sistema de unidades utilizado. Nosotros trabajaremos aquí en el Sistema Internacional de Unidades (SI).
Matemáticamente,
\begin{equation}
\vec{F}_{1,2} = \frac{1}{4\pi\epsilon_0}\frac{q_1 q_2}{r^2}\vec{e}_r,
\end{equation}
siendo $\epsilon_0 = 8, 8541878176\times10^{-12}$ F/m la permitividad dieléctrica del vacío, $q_1$ y $q_2$ son los valores de las cargas 1 y 2, respectivamente, $r$ es la distancia entre las cargas 1 y 2, y $\vec{e}_r = \vec{r}/r$ es el vector unitario a lo largo de la dirección del segmento de recta que une las cargas $q_1$ y $q_2$. El sentido de la fuerza de Coulomb depende de los signos de las cargas $q_1$ y $q_2$. En la naturaleza se verifican únicamente las situaciones $q > 0$ (carga eléctrica positiva), $q < 0$ (carga eléctrica negativa), o $q = 0$ (carga eléctrica nula o ausencia de carga). La ley de Coulomb es válida sólo cuando no existe movimiento relativo entre las cargas eléctricas, razón por la cual la fuerza de Coulomb es llamada en ocasiones fuerza electrostática.

\subsection{Principio de Superposición. Intensidad del campo electrostático}

Una idea de crucial importancia en la formulación de la Electrodinámica, que no es deducible a partir de la ley de Coulomb, es el llamado Principio de Superposición. Resumido en palabras, el Principio de Superposición plantea que la fuerza de interacción entre dos cargas\footnote{Para abreviar, cada vez que nos refiramos a una carga eléctrica o conjunto de cargas eléctricas, utilizaremos simplemente las palabras carga o cargas, respectivamente.} no es afectada por la presencia de una tercera. Examinemos la primera consecuencia del Principio de Superposición. Supongamos que tenemos $N$ cargas puntuales\footnote{El modelo de carga puntual en Electrodinámica es el equivalente al modelo del punto material en Mecánica. Toda la carga se asume concentrada en un punto del espacio caracterizado por el radio vector de posición $\vec{r}$.} $q_1$, $q_2$, ..., $q_N$, y otra carga puntual que llamaremos de prueba y que denotaremos con la letra $Q_p$. La fuerza que ejerce el sistema de las $N$ cargas sobre la carga de prueba $Q_p$ es igual a la suma vectorial de las fuerzas que ejercen, por separado, cada una de las cargas $q_i$ localizadas en las posiciones $\vec{r}_i$ sobre la carga $Q_p$ ubicada en la posición $\vec{r}_{Q_p}$, es decir,
\begin{equation}
\vec{F}_{Q_p} = \sum_{i=1}^{N} \vec{F}_{i,Q_p} = \sum_{i=1}^{N} \frac{1}{4\pi\epsilon_0}\frac{q_i Q_p}{|\vec{r}_{Q_p} - \vec{r}_i|^2}\vec{e}_{iQ_p},
\end{equation}
o, teniendo en cuenta la definición de $\vec{e}_r$,
\begin{equation}
\vec{F}_{Q_p} = \sum_{i=1}^{N} q_i Q_p \frac{\vec{r}_{Q_p} - \vec{r}_i}{4\pi\epsilon_0 |\vec{r}_{Q_p} - \vec{r}_i|^3} = Q_p \vec{E}(\vec{r}_{Q_p}),
\end{equation}
siendo 
\begin{equation}
\vec{E}(\vec{r}) = \sum_{i=1}^{N} \frac{1}{4\pi\epsilon_0} q_i \frac{\vec{r} - \vec{r}_i}{|\vec{r} - \vec{r}_i|^3}.
\end{equation}

El vector $\vec{E}$ es un campo vectorial dependiente de la posición, asociado a la presencia del sistema de cargas, e independiente de la carga de prueba $Q_p$. Este campo vectorial, que depende únicamente de la distribución espacial y valores de las cargas puntuales $q_i$, recibe el nombre de intensidad del campo electrostático en el punto $\vec{r}$.

Otra consecuencia importante del Principio de Superposición es que este conduce a ecuaciones lineales para la determinación del estado físico de las cargas y/o del campo electrostático.

A pesar de que en la naturaleza las cargas aparecen siempre en forma discreta (la carga eléctrica es una magnitud cuantizada), suele utilizarse en ocasiones el modelo continuo, que asume la carga eléctrica distribuida continuamente sobre determinadas regiones del espacio, o bien sobre todo el espacio. Ambas concepciones pueden ser tratadas matemáticamente desde el punto de vista de la formulación del modelo continuo. Definamos la función escalar
\begin{equation}
\rho(\vec{r}) = \sum_{i=1}^{N} q_i \delta(\vec{r} - \vec{r}_i).
\end{equation}

Nótese que, integrando sobre todo el espacio $\mathbb{R}^3$, se tiene que la carga neta $Q$ del sistema es
\begin{equation}
Q = \int_{\mathbb{R}^3} \rho(\vec{r})dV = \sum_{i=1}^{N} q_i.
\end{equation}

La función $\rho$ recibe el nombre de densidad de carga. Haciendo uso de ella no es difícil ver que,
\begin{equation}
\vec{E}(\vec{r}) = \frac{1}{4\pi\epsilon_0} \int_{\mathbb{R}^3} \rho(\vec{r}') \frac{\vec{r} - \vec{r}'}{|\vec{r} - \vec{r}'|^3}dV'.
\end{equation}

La expresión anterior es general y válida tanto para distribuciones de carga discretas como continuas, siempre que dichas distribuciones estén definidas sobre todo el espacio tridimensional $\mathbb{R}^3$. En el caso de una distribución continua de cargas definida únicamente en el subconjunto $V \subset \mathbb{R}^3$ será igualmente válida la expresión
\begin{equation}
Q = \int_{V} \rho(\vec{r})dV.
\end{equation}

La expresión equivalente para $\vec{E}$ reemplazando $\mathbb{R}^3$ por $V$ en (1.6) será correcta únicamente si no se le imponen condiciones adicionales, denominadas condiciones de frontera, al campo electrostático en la frontera de $V$. En la mayoría de los casos el valor de $\vec{E}$ depende, en general, de las propiedades del campo electrostático en la superficie que limita al volumen $V$.

\begin{figure}
\centering
% Aquí podrías incluir la imagen de la Figura 1.1 si la tienes disponible
\caption{Tómese el origen de coordenadas en el punto $O$ o en el punto $A$. El valor de la densidad de carga en el punto $M$ no depende de dicha elección.}
\label{fig:1.1}
\end{figure}

El valor de la densidad de carga en un punto $M \in \mathbb{R}^3$ no depende de la elección del origen de coordenadas. Sean $O$ y $A$ otros dos puntos arbitrarios de $\mathbb{R}^3$ diferentes de $M$ (ver Fig. 1.1). Supongamos que tenemos el origen de coordenadas en el punto $O$ y que $\vec{r}'$ es el vector de posición de $M$ con respecto a $O$. Supongamos ahora que trasladamos el origen de coordenadas al punto $A$ y sea $\vec{r}''$ el vector de posición de $M$ medido desde $A$. Obviamente $\vec{r}'' = \vec{r}' - \vec{r}_A$, siendo $\vec{r}_A$ el vector de posición de $A$ con respecto a $O$. La independencia de la densidad de carga en $M$ de la elección del origen de coordenadas puede expresarse matemáticamente de la siguiente forma:
\begin{equation}
\rho(M) \equiv \rho(\vec{r}') = \tilde{\rho}(\vec{r}'') = \tilde{\rho}(\vec{r}' - \vec{r}_A),
\end{equation}
siendo $\rho$ y $\tilde{\rho}$ la densidad de carga medida desde $O$ y desde $A$, respectivamente.

\subsection{Ley de Gauss}

Tomemos la divergencia del vector $\vec{E}$ en (1.6).
\begin{equation}
\nabla \cdot \vec{E}(\vec{r}) = \nabla \cdot \frac{1}{4\pi\epsilon_0} \int_{\mathbb{R}^3} \rho(\vec{r}') \frac{\vec{r}-\vec{r}'}{|\vec{r}-\vec{r}'|^3}dV' = \frac{1}{4\pi\epsilon_0} \int_{\mathbb{R}^3} \rho(\vec{r}') \nabla \cdot \left( \frac{\vec{r}-\vec{r}'}{|\vec{r}-\vec{r}'|^3} \right) dV'.
\end{equation}

Pero sabemos que
\begin{equation}
\nabla \cdot \left( \frac{\vec{r}-\vec{r}'}{|\vec{r}-\vec{r}'|^3} \right) = 4\pi\delta(\vec{r} - \vec{r}').
\end{equation}

En consecuencia,
\begin{equation}
\nabla \cdot \vec{E}(\vec{r}) = \frac{\rho(\vec{r})}{\epsilon_0}.
\end{equation}

La relación (1.10) es la llamada forma diferencial de la ley de Gauss. Interpretemos el significado físico de esta ley. Sea $V \subset \mathbb{R}^3$ un volumen, limitado por una superficie $S$ suave a pedazos y orientable, sobre el cual está definida la densidad de carga $\rho$ que es una función integrable en $V$, y supongamos además que $\vec{E}$ posee derivadas parciales de primer orden continuas sobre $V$ (ver Fig. 1.2). Entonces
\begin{equation}
\int_V \nabla \cdot \vec{E}dV = \frac{1}{\epsilon_0}\int_V \rho dV = \frac{Q}{\epsilon_0},
\end{equation}
y en virtud del Teorema de Gauss-Ostrogradsky,
\begin{equation}
\oint_S \vec{E} \cdot \vec{n} dS = \frac{Q}{\epsilon_0}.
\end{equation}

\begin{figure}
\centering
% Aquí podrías incluir la imagen de la Figura 1.2 si la tienes disponible
\caption{Volumen $V$ limitado por una superficie $S$ suave a pedazos y orientable. El vector normal exterior es $\vec{n}$.}
\label{fig:1.2}
\end{figure}

Es decir, el flujo del vector $\vec{E}$ a través de la superficie $S$ es proporcional a la carga neta encerrada dentro de ella. La ecuación (1.11) es conocida como forma integral de la ley de Gauss.

Suele adoptarse el convenio siguiente: Si el vector $\vec{E}$ fluye desde la región interior hacia la región exterior a la superficie $S$, entonces se asume que $Q > 0$. Por el contrario, si el vector $\vec{E}$ fluye desde la región exterior hacia la región interior a la superficie $S$, entonces se dice que $Q < 0$.

Una implicación fundamental de la ley de Gauss es que las cargas eléctricas estacionarias son las fuentes del campo electrostático.

\subsection{Potencial electrostático. Ecuación de Poisson}

El vector $\vec{E}$ solución de la ecuación (1.10) está indeterminado en el rotacional de un campo vectorial arbitrario. Sea $\vec{E}' = \vec{E} + \nabla \times \vec{A}$, siendo $\vec{A}$ un campo vectorial con derivadas parciales de primer orden continuas. Entonces es fácil verificar que $\nabla \cdot \vec{E}' = \nabla \cdot \vec{E} = \rho/\epsilon_0$, o sea, los campos $\vec{E}'$ y $\vec{E}$ son ambos soluciones de la ley de Gauss. Resulta necesario entonces la introducción de una magnitud física, característica del campo, que nos permita describirlo unívocamente.

En virtud de la expresión (1.6) tenemos que
\begin{equation}
\vec{E}(\vec{r}) = \frac{1}{4\pi\epsilon_0} \int_{\mathbb{R}^3} \rho(\vec{r}') \frac{\vec{r}-\vec{r}'}{|\vec{r}-\vec{r}'|^3}dV' = -\frac{1}{4\pi\epsilon_0} \int_{\mathbb{R}^3} \rho(\vec{r}') \nabla_r \left( \frac{1}{|\vec{r}-\vec{r}'|} \right) dV',
\end{equation}
de donde
\begin{equation}
\vec{E}(\vec{r}) = -\nabla_r \left[ \frac{1}{4\pi\epsilon_0} \int_{\mathbb{R}^3} \frac{\rho(\vec{r}')}{|\vec{r}-\vec{r}'|}dV' \right].
\end{equation}

Definiendo la función
\begin{equation}
\phi(\vec{r}) = \frac{1}{4\pi\epsilon_0} \int_{\mathbb{R}^3} \frac{\rho(\vec{r}')}{|\vec{r}-\vec{r}'|}dV',
\end{equation}
encontramos que
\begin{equation}
\vec{E}(\vec{r}) = -\nabla\phi(\vec{r}).
\end{equation}

La función $\phi$ dada por (1.12) recibe el nombre de potencial electrostático. Analicemos el significado físico de dicha función. Supongamos que queremos llevar una carga $q$ desde el punto $A \in \mathbb{R}^3$ hasta el punto $B \in \mathbb{R}^3$ en presencia de un campo electrostático externo $\vec{E}$. La velocidad de la partícula con respecto a la fuente del campo externo debe ser infinitamente pequeña para que la ley de Coulomb sea aplicable. El trabajo que hay que realizar sobre la carga en contra del campo electrostático es, en este caso,
\begin{equation}
W_{AB} = -\int_{A}^{B} \vec{F} \cdot d\vec{l} = -q \int_{A}^{B} \vec{E} \cdot d\vec{l} = q \int_{A}^{B} \nabla\phi \cdot d\vec{l} = q \int_{A}^{B} d\phi = q[\phi(B) - \phi(A)].
\end{equation}

El trabajo realizado por el campo sobre la carga será obviamente $-W_{AB}$. La integral de línea de segundo tipo anterior no depende del camino que une los puntos $A$ y $B$. Nótese además que $W_{BA} = -W_{AB}$, de donde se desprende que $W_{AA} = 0$ (o bien $W_{BB} = 0$). En otras palabras,
\begin{equation}
\oint_{C} \vec{E} \cdot d\vec{l} = 0,
\end{equation}
siendo $C$ un contorno cerrado que contiene a los puntos $A$ y $B$. De acuerdo con el Teorema de Green, de la expresión anterior se deriva que
\begin{equation}
\nabla \times \vec{E} = 0,
\end{equation}
lo cual puede comprobarse también mediante el cálculo directo.

El campo electrostático es, en consecuencia, un campo conservativo, y tiene sentido definir la energía potencial de la carga $q$ en presencia del campo electrostático $\vec{E}$. Esta es
\begin{equation}
U(\vec{r}) = q\phi(\vec{r}).
\end{equation}

En presencia de un campo electrostático, la energía potencial de la carga eléctrica en la posición $\vec{r}$ es directamente proporcional al valor del potencial electrostático en dicha posición. El trabajo realizado por el campo electrostático sobre la carga para moverla desde el punto $A$ hasta el punto $B$ es
\begin{equation}
W = -\Delta U = -[U(B) - U(A)].
\end{equation}

Nótese que el potencial electrostático está indeterminado en una constante. Sean $\phi'$ y $\phi$ tales que $\phi'(\vec{r}) = \phi(\vec{r}) + C_0$, siendo $C_0$ una constante numérica. Es evidente que $-\nabla\phi'(\vec{r}) = -\nabla\phi(\vec{r}) = \vec{E}(\vec{r})$. Esta indeterminación suele utilizarse para calibrar convenientemente el valor del potencial electrostático en determinada región del espacio. Una de las calibraciones más utilizadas en la electrostática es elegir la constante $C_0$ de forma tal que
\begin{equation}
\lim_{r\to+\infty} \phi(\vec{r}) = 0.
\end{equation}

La indeterminación en el potencial electrostático conduce a su vez a una indeterminación en la energía potencial, pero esto carece de importancia puesto que lo que realmente interesa, desde el punto de vista físico, no es el valor de la energía potencial en sí sino su diferencia, de la cual puede extraerse el trabajo realizado por el campo (o en contra de él) para mover una carga eléctrica de un punto del espacio a otro.

Combinando las expresiones (1.10) y (1.13), no es difícil ver que
\begin{equation}
\nabla \cdot \vec{E} = -\nabla \cdot \nabla\phi = -\nabla^2\phi = \frac{\rho}{\epsilon_0},
\end{equation}

de donde se halla finalmente que
\begin{equation}
\nabla^2 \phi(\vec{r}) = -\frac{\rho(\vec{r})}{\epsilon_0}.
\end{equation}

Esta es la llamada ecuación de Poisson para el potencial electrostático. En regiones del espacio donde la densidad de carga es nula, la ecuación de Poisson se transforma en la ecuación de Laplace
\begin{equation}
\nabla^2 \phi(\vec{r}) = 0.
\end{equation}

El potencial (1.12) es la solución exacta de la ecuación de Poisson en el espacio abierto $\mathbb{R}^3$. Tal afirmación puede comprobarse de forma directa calculando el Laplaciano de $\phi$. Es decir,
\begin{equation}
\nabla^2 \phi(\vec{r}) = \nabla^2 \frac{1}{4\pi\epsilon_0} \int_{\mathbb{R}^3} \frac{\rho(\vec{r}')}{|\vec{r} - \vec{r}'|}dV' = \frac{1}{4\pi\epsilon_0} \int_{\mathbb{R}^3} \rho(\vec{r}') \nabla^2 \left( \frac{1}{|\vec{r} - \vec{r}'|} \right) dV'.
\end{equation}

Teniendo en cuenta que
\begin{equation}
\nabla^2 \left( \frac{1}{|\vec{r} - \vec{r}'|} \right) = -4\pi\delta(\vec{r} - \vec{r}'),
\end{equation}
la ecuación (1.19) se recupera sin dificultad.

\section{Método de la función de Green para la solución del problema electrostático}

Dada una distribución estática de cargas con densidad $\rho = \rho(\vec{r})$, el problema fundamental de la electrostática consiste en encontrar las propiedades y estructura matemática del campo electrostático producido por dicha distribución de cargas. Supongamos que tenemos un conjunto volumétrico abierto $V \subset \mathbb{R}^3$ sobre el cual está definido la densidad de carga $\rho$, y que está limitado por una superficie $S$ suave a pedazos y orientable. En el interior de $V$ la solución del problema fundamental de la electrostática equivale a resolver la ecuación de Poisson (1.19), mientras que en el exterior de $V$ (donde no está definida la densidad de carga), la solución de dicho problema equivale a resolver la ecuación de Laplace (1.20). Obviamente, en ambos casos necesitamos conocer alguna información de las propiedades del campo electrostático sobre la superficie $S$.

De forma general, el problema fundamental de la electrostática para el interior de $V$ puede ser enunciado como a continuación se indica.

Para casi todo punto $M \in V$\footnote{La frase casi todo punto $M \in V$... indica la posibilidad de la existencia de puntos de $V$ sobre los cuales el potencial electrostático posee singularidades esenciales. De cualquier manera, el volumen $V$ puede ser redefinido, en la mayoría de los casos, excluyendo los puntos singulares de $\phi$. Este hecho es particularmente notable en el caso de un sistema de cargas puntuales, donde el potencial es una función divergente en la posición de cada partícula. En general, para distribuciones continuas de carga se exige que el potencial $\phi$ sea una función acotada en su dominio.} de vector de posición $\vec{r}$, encontrar el potencial electrostático $\phi = \phi(\vec{r})$ que es solución de la ecuación de Poisson
\begin{equation}
\nabla^2 \phi(\vec{r}) = -\frac{\rho(\vec{r})}{\epsilon_0},
\end{equation}
y que satisface una de las tres condiciones siguientes en la superficie $S$ frontera de $V$:

\begin{enumerate}
\item $\phi = f_1$ sobre $S$ (condición de frontera de primer tipo o de Dirichlet),
\item $\frac{\partial\phi}{\partial n} = f_2$ sobre $S$ (condición de frontera de segundo tipo o de Neumann),
\item $\frac{\partial\phi}{\partial n} + f_3(\phi - f_4) = 0$ sobre $S$ (condición de frontera de tercer tipo\footnote{El problema matemático de la solución de una ecuación diferencial en derivadas parciales de segundo orden elíptica con una condición de frontera de tercer tipo es un problema llamado usualmente mal planteado o no bien definido. Este tipo de problema mal definido no es, en general, apropiado para las aplicaciones en Física y no lo examinaremos aquí.}),
\end{enumerate}

siendo $f_1$, $f_2$, $f_3$ y $f_4$ funciones conocidas definidas sobre la superficie $S$, y 
\begin{equation}
\frac{\partial\phi}{\partial n} = \vec{n} \cdot \nabla\phi
\end{equation}
es la derivada direccional del potencial $\phi$ en la dirección de la normal $\vec{n}$ exterior a la superficie $S$.

\subsection{Identidad de Green. Funciones de Green}

Sean las funciones $\Phi$ y $\Psi$ con derivadas parciales continuas hasta el segundo orden, definidas en cierto conjunto volumétrico abierto $V \subset \mathbb{R}^3$ limitado por una superficie $S$ suave a pedazos y orientable. Definamos el campo vectorial
\begin{equation}
\vec{F}(\vec{r}) = \Phi(\vec{r})\nabla\Psi(\vec{r}) - \Psi(\vec{r})\nabla\Phi(\vec{r}).
\end{equation}

El Teorema de Gauss-Ostrogradsky aplicado al campo vectorial $\vec{F}$ establece que
\begin{equation}
\int_V \nabla \cdot \vec{F}dV = \oint_S \vec{F} \cdot \vec{n}dS.
\end{equation}

Pero
\begin{equation}
\nabla \cdot \vec{F} = \Phi\nabla^2\Psi - \Psi\nabla^2\Phi
\end{equation}
y
\begin{equation}
\vec{F} \cdot \vec{n} = \Phi \vec{n} \cdot \nabla\Psi - \Psi \vec{n} \cdot \nabla\Phi = \Phi \frac{\partial\Psi}{\partial n} - \Psi \frac{\partial\Phi}{\partial n}.
\end{equation}

En consecuencia,
\begin{equation}
\int_V \left[ \Phi(\vec{r})\nabla^2\Psi(\vec{r}) - \Psi(\vec{r})\nabla^2\Phi(\vec{r}) \right] dV = \oint_S \left[ \Phi(\vec{r})\frac{\partial\Psi(\vec{r})}{\partial n} - \Psi(\vec{r})\frac{\partial\Phi(\vec{r})}{\partial n} \right] dS.
\end{equation}

Esta es la llamada identidad de Green.

Supongamos ahora que existe cierta función $G = G(\vec{r},\vec{r}')$ tal que
\begin{equation}
\nabla^2 G(\vec{r},\vec{r}') = -\delta(\vec{r} - \vec{r}').
\end{equation}

Dicha función se conoce con el nombre función de Green\footnote{En algunos textos se le llama función fuente o, en Inglés, source function.}. El significado de la función de Green es inmediato. En un sistema de unidades en el que la carga eléctrica tenga las mismas unidades que $\epsilon_0$, la función de Green no es más que el potencial generado en el punto $M \in \mathbb{R}^3$ con vector de posición $\vec{r}$ por una carga puntual de valor $\epsilon_0$ situada en el punto $M' \in \mathbb{R}^3$ ($M' \neq M$) con vector de posición $\vec{r}'$. Una propiedad importante de la función de Green es su simetría, es decir,
\begin{equation}
G(\vec{r},\vec{r}') = G(\vec{r}',\vec{r}).
\end{equation}

La expresión anterior, conocida como Teorema de Lyapunov [2], es una consecuencia del llamado Principio de Reciprocidad: una fuente situada en el punto $M'$ produce sobre el punto $M$ el mismo efecto que produciría sobre $M'$ la misma fuente ubicada sobre el punto $M$. Como corolario del Principio de Reciprocidad, es evidente que
\begin{equation}
\nabla'^2G(\vec{r},\vec{r}') = -\delta(\vec{r} - \vec{r}').
\end{equation}

Supongamos ahora que $\phi$ es la solución de la ecuación de Poisson (1.19) y hagamos en la identidad de Green (1.24) $\Phi \equiv \phi$ y $\Psi \equiv G$. Entonces
\begin{equation}
\int_V \left[ \phi(\vec{r}')\nabla'^2 G(\vec{r},\vec{r}') - G(\vec{r},\vec{r}')\nabla'^2\phi(\vec{r}') \right] dV' = \oint_S \left[ \phi(\vec{r}')\frac{\partial G(\vec{r},\vec{r}')}{\partial n'} - G(\vec{r},\vec{r}')\frac{\partial\phi(\vec{r}')}{\partial n'} \right] dS'.
\end{equation}

Teniendo en cuenta las ecuaciones (1.19) y (1.27), hallamos finalmente que
\begin{equation}
\phi(\vec{r}) = \frac{1}{\epsilon_0} \int_V \rho(\vec{r}')G(\vec{r},\vec{r}')dV' + \oint_S \left[ G(\vec{r},\vec{r}')\frac{\partial\phi(\vec{r}')}{\partial n'} - \phi(\vec{r}')\frac{\partial G(\vec{r},\vec{r}')}{\partial n'} \right] dS'.
\end{equation}

La ecuación anterior es la solución general de la ecuación de Poisson. El cálculo de la función de Green puede ser realizado haciendo uso de los teoremas de integración propios de la Teoría de Variable Compleja. Nosotros haremos aquí uso de la propiedad de la función delta de Dirac, según la cual
\begin{equation}
\nabla^2\left( \frac{1}{|\vec{r} - \vec{r}'|} \right) = -4\pi \delta(\vec{r} - \vec{r}').
\end{equation}

De acuerdo con dicha propiedad, es evidente que
\begin{equation}
G(\vec{r},\vec{r}') = \frac{1}{4\pi|\vec{r} - \vec{r}'|} + F(\vec{r},\vec{r}'),
\end{equation}
siendo $F(\vec{r},\vec{r}')$ una función armónica en los conjuntos a los que pertenecen las variables $\vec{r}$ y $\vec{r}'$, es decir, $\nabla^2F(\vec{r},\vec{r}') = \nabla'^2 F(\vec{r},\vec{r}') = 0$.

Supongamos que $V(R)$ es una bola, centrada en el origen de coordenadas, limitada por una superficie esférica $S(R)$ de radio $R$. Si calibramos el potencial electrostático de acuerdo con la relación (1.18), entonces es fácil ver que
\begin{equation}
\lim_{R\to\infty} \oint_{S(R)} \left[ G(\vec{r},\vec{r}')\frac{\partial\phi(\vec{r}')}{\partial n'} - \phi(\vec{r}')\frac{\partial G(\vec{r},\vec{r}')}{\partial n'} \right] dS' = 0,
\end{equation}
y en consecuencia,
\begin{equation}
\phi(\vec{r}) = \lim_{R\to\infty} \frac{1}{\epsilon_0} \int_{V(R)} \rho(\vec{r}')G(\vec{r},\vec{r}')dV' = \frac{1}{4\pi\epsilon_0} \int_{\mathbb{R}^3} \frac{\rho(\vec{r}')}{|\vec{r} - \vec{r}'|}dV',
\end{equation}
que es la solución (1.12) ya conocida de la ecuación de Poisson (1.19) en todo el espacio $\mathbb{R}^3$.

La solución (1.28) depende obviamente del tipo de condición de frontera impuesta al potencial electrostático sobre la superficie $S$. Veamos esta situación a continuación.

\subsection{Problemas de Dirichlet y Neumann}

Consideremos la solución del problema de Dirichlet
\begin{equation}
\begin{cases}
\nabla^2\phi(\vec{r}) = -\frac{\rho(\vec{r})}{\epsilon_0} & \text{en } V, \\
\phi(\vec{r})|_S = D(\vec{r})
\end{cases}
\end{equation}
siendo $D$ una función conocida. Puesto que la función de Green está indeterminada en una función armónica, podemos, una vez fijada la geometría de la superficie $S$, elegir una función $F$ armónica en $V$ tal que
\begin{equation}
G|_S = 0.
\end{equation}

A partir de (1.28) es posible ver que la solución del problema de Dirichlet será entonces
\begin{equation}
\phi(\vec{r}) = \frac{1}{\epsilon_0} \int_V \rho(\vec{r}')G(\vec{r},\vec{r}') dV' - \oint_S D(\vec{r}') \frac{\partial G(\vec{r},\vec{r}')}{\partial n'} dS'.
\end{equation}

Probemos que esta solución es única y para ello recurramos al conocido método lógico de demostración por reducción al absurdo. Supongamos que $\phi_1$ y $\phi_2$ son soluciones diferentes del problema de Dirichlet y construyamos la función $U = \phi_1 - \phi_2$, la cual, obviamente, es la solución del problema
\begin{equation}
\begin{cases}
\nabla^2U(\vec{r}) = 0 & \text{en } V, \\
U(\vec{r})|_S = 0
\end{cases}
\end{equation}

Es evidente que
\begin{equation}
0 = \oint_S U(\vec{r})\frac{\partial U(\vec{r})}{\partial n} dS = \oint_S U(\vec{r})\nabla U(\vec{r}) \cdot \vec{n}dS.
\end{equation}

Aplicando ahora el teorema de Gauss-Ostrogradsky en la expresión anterior obtenemos que
\begin{equation}
0 = \int_V \nabla \cdot [U(\vec{r})\nabla U(\vec{r})] dV = \int_V [\nabla U(\vec{r})]^2 dV + \int_V U(\vec{r})\nabla^2 U(\vec{r}) dV = \int_V [\nabla U(\vec{r})]^2 dV,
\end{equation}
donde hemos tenido en cuenta que $\nabla^2 U(\vec{r}) = 0$ en el volumen $V$. Debido a la arbitrariedad del volumen $V$ sigue inmediatamente que $\nabla U(\vec{r}) = 0$ en $V$, es decir, $U$ es constante en $V$. Pero como $U$ debe ser una función continua en su dominio y $U(\vec{r})|_S = 0$, entonces $U(\vec{r}) = 0$ en $V$. Así, $\phi_1(\vec{r}) = \phi_2(\vec{r})$ en $V$, lo cual es un absurdo pues habíamos supuesto como hipótesis que $\phi_1(\vec{r}) \neq \phi_2(\vec{r})$ en $V$. Esta contradicción demuestra la unicidad de la solución del problema de Dirichlet.

Consideremos ahora el problema de Neumann
\begin{equation}
\begin{cases}
\nabla^2 \phi(\vec{r}) = -\frac{\rho(\vec{r})}{\epsilon_0} & \text{en } V, \\
\frac{\partial\phi(\vec{r})}{\partial n}\bigg|_S = N(\vec{r})
\end{cases}
\end{equation}
donde $N(\vec{r})$ es una función conocida. En este caso la parte armónica de la función de Green puede ser elegida de tal forma que
\begin{equation}
\frac{\partial G}{\partial n}\bigg|_S = -\frac{1}{A(S)},
\end{equation}
siendo $A(S)$ el área de la superficie $S$\footnote{Conviene aclarar que la elección - aparentemente evidente - $\frac{\partial G}{\partial n}\big|_S = 0$ conduce a una contradicción, pues $\oint_S \frac{\partial G}{\partial n'} dS' = \int_V \nabla' \cdot \nabla' G dV' = \int_V \nabla'^2 G dV' = -\int_V \delta(\vec{r} - \vec{r}')dV' = -1$. Esto indica que $\frac{\partial G}{\partial n}\big|_S$ no puede tomarse igual a cero sobre la superficie $S$.}. teniendo en cuenta (1.28) podemos ver que
\begin{equation}
\phi(\vec{r}) = \frac{1}{\epsilon_0} \int_V \rho(\vec{r}')G(\vec{r},\vec{r}') dV' + \oint_S G(\vec{r},\vec{r}')N(\vec{r}') dS' + \langle\phi\rangle_S,
\end{equation}
siendo
\begin{equation}
\langle\phi\rangle_S = \oint_S \frac{\phi(\vec{r})}{A(S)} dS
\end{equation}
el valor medio del potencial sobre la superficie $S$. Análogamente a como se hizo en el caso del problema de Dirichlet, también puede demostrarse que la solución anterior para el problema de Neumann es única.

\subsection{Ejemplos de obtención de la función de Green}

La obtención de la función de Green para un problema de frontera específico es una tarea en general compleja. Sin embargo, la fortaleza de este método justifica completamente los esfuerzos que hay que realizar para encontrar dicha función: Una vez conocida la función de Green para un problema de frontera de Dirichlet o Neumann asociado a una superficie y volumen específicos, podemos resolver cualquier problema electrostático del tipo Dirichlet o Neumann, respectivamente, asociado a la misma superficie y volumen. Examinemos a continuación algunos ejemplos sencillos de obtención de la función de Green.

\subsubsection{Función de Green en el segmento unidimensional (0, l) con condiciones de Dirichlet en los extremos}

Sea $l \in \mathbb{R}$ tal que $l > 0$, y consideremos el problema unidimensional de Dirichlet para la función de Green
\begin{equation}
\begin{cases}
\frac{\partial^2}{\partial x^2}G(x, x') = -\delta(x - x') & (x, x') \in (0, l) \times (0, l) \\
G(0, x') = G(l, x') = 0 & x' \in (0, l)
\end{cases}
\end{equation}

\subsubsection{Función de Green en el segmento unidimensional (0, l) con condiciones de Dirichlet en los extremos}

Sea $l \in \mathbb{R}$ tal que $l > 0$, y consideremos el problema unidimensional de Dirichlet para la función de Green
\begin{equation}
\begin{cases}
\frac{\partial^2}{\partial x^2}G(x, x') = -\delta(x - x') & (x, x') \in (0, l) \times (0, l) \\
G(0, x') = G(l, x') = 0 & x' \in (0, l)
\end{cases}
\end{equation}

La solución para $G$ puede ser representada como una serie de las funciones $\sin\left(\frac{n\pi}{l}x\right)$, siendo $n$ un número natural, las cuales satisfacen la condición de Dirichlet en los extremos del intervalo $(0, l)$ para cualquier valor de $n$ y son un conjunto completo en este intervalo. Por lo tanto,
\begin{equation}
G(x, x') = \sum_{n=1}^{+\infty} C_n(x') \sin\left(\frac{n\pi}{l}x\right).
\end{equation}

Los coeficientes $C_n(x')$ pueden calcularse a partir de la condición
\begin{equation}
C_n(x') = \frac{2}{l}\int_{0}^{l} G(x, x') \sin\left(\frac{n\pi}{l}x\right) dx.
\end{equation}

Sustituyendo (1.42) en (1.41) obtenemos que
\begin{equation}
G(x, x') = \frac{2}{l}\sum_{n=1}^{+\infty} \int_{0}^{l} G(x'', x') \sin\left(\frac{n\pi}{l}x''\right) \sin\left(\frac{n\pi}{l}x\right) dx'',
\end{equation}
de donde
\begin{equation}
\delta(x - x'') = \frac{2}{l}\sum_{n=1}^{+\infty} \sin\left(\frac{n\pi}{l}x\right) \sin\left(\frac{n\pi}{l}x''\right).
\end{equation}

Esta es la llamada relación de completitud.

Sustituyendo el desarrollo para $G$ y la relación de completitud en la ecuación diferencial obtenemos
\begin{equation}
\sum_{n=1}^{+\infty} C_n(x')\left(\frac{n\pi}{l}\right)^2 \sin\left(\frac{n\pi}{l}x\right) = \frac{2}{l}\sum_{n=1}^{+\infty} \sin\left(\frac{n\pi}{l}x\right) \sin\left(\frac{n\pi}{l}x'\right),
\end{equation}
de donde
\begin{equation}
C_n(x') = \frac{2}{l}\frac{\sin\left(\frac{n\pi}{l}x'\right)}{\left(\frac{n\pi}{l}\right)^2}.
\end{equation}

Luego,
\begin{equation}
G(x, x') = \frac{2}{l}\sum_{n=1}^{+\infty} \frac{\sin\left(\frac{n\pi}{l}x\right)\sin\left(\frac{n\pi}{l}x'\right)}{\left(\frac{n\pi}{l}\right)^2}.
\end{equation}

La expresión (1.46) nos da la función de Green unidimensional buscada para el problema de Dirichlet en el segmento $(0, l)$.

\subsubsection{Función de Green en una sección circular con condiciones de Dirichlet en la frontera}

Consideremos la sección plana del círculo de radio $R$ mostrada en la figura 1.3. La función de Green dentro de la región analizada satisface, en coordenadas polares $(r = \sqrt{x^2 + y^2}, \phi)$, la ecuación diferencial
\begin{equation}
\left[\frac{1}{r}\frac{\partial}{\partial r}\left(r\frac{\partial}{\partial r}\right) + \frac{1}{r^2}\frac{\partial^2}{\partial\phi^2}\right] G(r, r', \phi, \phi') = -\frac{1}{r}\delta(r - r')\delta(\phi - \phi').
\end{equation}

\begin{figure}
\centering
% Aquí podrías incluir la imagen de la Figura 1.3 si la tienes disponible
\caption{Sección circular de radio $R$ y apertura angular $\alpha$. La función de Green satisface la condición de Dirichlet en la frontera.}
\label{fig:1.3}
\end{figure}

La función de Green satisface, además, las condiciones de frontera
\begin{equation}
\begin{cases}
G(r, r', 0, \phi') = G(r, r', \phi, 0) = 0 \\
G(R, r', \phi, \phi') = G(r, R, \phi, \phi') = 0 \\
G(r, r', \alpha, \phi') = G(r, r', \phi, \alpha) = 0
\end{cases}
\end{equation}

La función de Green que satisface las condiciones de frontera angulares puede proponerse como
\begin{equation}
G(r, r', \phi, \phi') = \sum_{n=1}^{+\infty} \sin(\alpha_n \phi) \sin(\alpha_n\phi') f_n(r, r'),
\end{equation}
siendo $\alpha_n = \frac{n\pi}{\alpha}$ y $f_n$ son funciones a determinar para cada valor de $n$. Por otra parte, de acuerdo con (1.44),
\begin{equation}
\delta(\phi - \phi') = \frac{2}{\alpha}\sum_{n=1}^{+\infty} \sin(\alpha_n \phi) \sin(\alpha_n\phi').
\end{equation}

Sustituyendo (1.49) y (1.50) en (1.47) obtenemos la siguiente ecuación diferencial para las funciones $f_n$:
\begin{equation}
\frac{\partial}{\partial r}\left(r \frac{\partial}{\partial r}f_n(r, r')\right) - \frac{\alpha_n^2}{r}f_n(r, r') = -\frac{2}{\alpha}\delta(r - r').
\end{equation}

Si $r \neq r'$, la ecuación para $f_n$ es una ecuación diferencial homogénea de Euler con coeficientes variables, y su solución general tiene la forma
\begin{equation}
f_n(r, r') = A(r')r^{\alpha_n} + B(r')r^{-\alpha_n}.
\end{equation}

El arco $r = r'$ define dos regiones disjuntas. Si $r < r'$, entonces $f_n \sim A_n(r')r^{\alpha_n}$, pues $f_n(0, r') = 0$. Si por el contrario $r > r'$, entonces $f_n \sim A(r')r^{\alpha_n} + B(r')r^{-\alpha_n}$, pero como $f_n(R, r') = 0$, entonces $B(r') = -A(r')R^{2\alpha_n}$. En consecuencia
\begin{equation}
f_n \sim A(r') \left[r^{\alpha_n} - \left(\frac{R^2}{r}\right)^{\alpha_n}\right]
\end{equation}
siempre que $r$ sea mayor que $r'$. Definiendo $C_n(r') = A(r')R^{\alpha_n}$ hallamos que $f_n \sim C_n(r') \left[\left(\frac{r}{R}\right)^{\alpha_n} - \left(\frac{R}{r}\right)^{\alpha_n}\right]$ siempre que $r$ sea mayor que $r'$. Podemos escribir entonces
\begin{equation}
f_n(r, r') = 
\begin{cases}
A_n(r')r^{\alpha_n} & \text{si } r < r' \\
C_n(r')\left[\left(\frac{r}{R}\right)^{\alpha_n} - \left(\frac{R}{r}\right)^{\alpha_n}\right] & \text{si } r > r'
\end{cases}
\end{equation}

En virtud del principio de reciprocidad, $f_n(r, r') = f_n(r', r)$. En consecuencia,
\begin{equation}
f_n(r, r') = f_n(r', r) = 
\begin{cases}
A_n(r)r'^{\alpha_n} & \text{si } r' < r \\
C_n(r)\left[\left(\frac{r'}{R}\right)^{\alpha_n} - \left(\frac{R}{r'}\right)^{\alpha_n}\right] & \text{si } r' > r
\end{cases}
\end{equation}

Comparando las dos expresiones anteriores podemos ver que
\begin{equation}
A_n(r') = \beta_n \left[\left(\frac{r'}{R}\right)^{\alpha_n} - \left(\frac{R}{r'}\right)^{\alpha_n}\right]
\end{equation}
y
\begin{equation}
C_n(r') = \beta_n r'^{\alpha_n},
\end{equation}
siendo $\beta_n$ una constante numérica que, atendiendo al propio principio de reciprocidad, debe tener el mismo valor cuando $r > r'$ que cuando $r < r'$ para cada valor dado de $n$.

Así, la solución general para $f_n$ tiene la forma
\begin{equation}
f_n(r, r') = 
\begin{cases}
\beta_n r^{\alpha_n}\left[\left(\frac{r'}{R}\right)^{\alpha_n} - \left(\frac{R}{r'}\right)^{\alpha_n}\right] & \text{si } r < r' \\
\beta_n r'^{\alpha_n}\left[\left(\frac{r}{R}\right)^{\alpha_n} - \left(\frac{R}{r}\right)^{\alpha_n}\right] & \text{si } r > r'
\end{cases}
\end{equation}
o bien
\begin{equation}
f_n(r, r') = \beta_n r_-^{\alpha_n} \left[\left(\frac{r_+}{R}\right)^{\alpha_n} - \left(\frac{R}{r_+}\right)^{\alpha_n}\right],
\end{equation}
siendo $r_- = \min(r, r')$ y $r_+ = \max(r, r')$.

Calculemos ahora la constante $\beta_n$. Para ello integremos la ecuación diferencial (1.51)
\begin{equation}
\frac{\partial}{\partial r}\left(r \frac{\partial}{\partial r}f_n(r, r')\right) - \frac{\alpha_n^2}{r}f_n(r, r') = -\frac{2}{\alpha}\delta(r - r')
\end{equation}
entre $r = r' - \varepsilon$ y $r = r' + \varepsilon$ (siendo $\varepsilon > 0$ tal que $r' - \varepsilon > 0$ $\forall r'$), y luego tomemos el límite cuando $\varepsilon$ tiende a cero. Así,
\begin{equation}
\int_{r'-\varepsilon}^{r'+\varepsilon} \frac{\partial}{\partial r}\left(r \frac{\partial}{\partial r}f_n(r, r')\right) dr - \alpha_n^2 \int_{r'-\varepsilon}^{r'+\varepsilon} \frac{f_n(r, r')}{r} dr = -\frac{2}{\alpha},
\end{equation}
de donde
\begin{equation}
(r' + \varepsilon)\left.\frac{\partial f_n}{\partial r}\right|_{r'+\varepsilon} - (r' - \varepsilon)\left.\frac{\partial f_n}{\partial r}\right|_{r'-\varepsilon} - \alpha_n^2 \int_{r'-\varepsilon}^{r'+\varepsilon} \frac{f_n(r, r')}{r} dr = -\frac{2}{\alpha}.
\end{equation}

Pero
\begin{equation}
\left.\frac{\partial f_n}{\partial r}\right|_{r'+\varepsilon} = \beta_n \alpha_n (r')^{\alpha_n} \left[\left(\frac{r'+\varepsilon}{R}\right)^{\alpha_n-1} + \left(\frac{R}{r'+\varepsilon}\right)^{\alpha_n+1}\right] \frac{1}{R^{\alpha_n}}
\end{equation}
y
\begin{equation}
\left.\frac{\partial f_n}{\partial r}\right|_{r'-\varepsilon} = \beta_n \alpha_n \left[\left(\frac{r'}{R}\right)^{\alpha_n} - \left(\frac{R}{r'}\right)^{\alpha_n}\right](r'-\varepsilon)^{\alpha_n-1}.
\end{equation}

Luego,
\begin{equation}
\begin{split}
\beta_n \alpha_n (r'+\varepsilon) & (r')^{\alpha_n} \left[\left(\frac{r'+\varepsilon}{R}\right)^{\alpha_n-1} + \left(\frac{R}{r'+\varepsilon}\right)^{\alpha_n+1}\right] \frac{1}{R^{\alpha_n}}\\
- \beta_n \alpha_n (r'-\varepsilon) & \left[\left(\frac{r'}{R}\right)^{\alpha_n} - \left(\frac{R}{r'}\right)^{\alpha_n}\right](r'-\varepsilon)^{\alpha_n-1}\\
- \alpha_n^2 \int_{r'-\varepsilon}^{r'+\varepsilon} & \frac{f_n(r, r')}{r} dr = -\frac{2}{\alpha}
\end{split}
\end{equation}

Tomando el límite cuando $\varepsilon$ tiende a cero hallamos finalmente que
\begin{equation}
\beta_n = -\frac{1}{\alpha\alpha_n R^{\alpha_n}}.
\end{equation}

La función de Green buscada será entonces
\begin{equation}
G(r, r', \phi, \phi') = -\frac{1}{\alpha}\sum_{n=1}^{+\infty} \frac{\sin(\alpha_n \phi) \sin(\alpha_n \phi')}{\alpha_n} \left(\frac{r_-}{R}\right)^{\alpha_n} \left[\left(\frac{r_+}{R}\right)^{\alpha_n} - \left(\frac{R}{r_+}\right)^{\alpha_n}\right].
\end{equation}

\section{Método de las imágenes para la solución del problema electrostático}

La solución de los problemas de Dirichlet o Neumann en el interior de cierto volumen $V$ limitado por una superficie $S$ exige del conocimiento de $\phi|_S$ o $\frac{\partial\phi}{\partial n}\big|_S$, respectivamente. Las funciones de Green correspondientes a estos problemas, como ya hemos visto, están indeterminadas en una función $F$ que es armónica en el interior de $V$. Podemos entonces utilizar esta indeterminación para proponer funciones $F$ tales que las funciones de Green satisfagan las condiciones de frontera sobre $S$. Para la elección apropiada de $F$ suele recurrirse a disímiles procedimientos. Uno de ellos utiliza el concepto de imagen proveniente de la Óptica para la colocación adecuada de distribuciones de carga, en el exterior de $V$, tales que la suma de todas las contribuciones de las cargas involucradas en el problema y de las cargas colocadas en el exterior (usualmente denominadas imágenes) reproduzcan sobre $S$ las condiciones de frontera deseadas.

Ya hemos visto que la función de Green $G(\vec{r},\vec{r}')$ puede ser interpretada, excepto por una constante de proporcionalidad asociada al sistema de unidades utilizado, como el potencial generado en el punto de vector de posición $\vec{r}$ por una carga puntual de valor $\epsilon_0$ ubicada en el punto de vector de posición $\vec{r}' \in V$, cuyo valor sobre la frontera $S$ de $V$ está determinado por las condiciones de Dirichlet o Neumann, es decir, es la solución de la correspondiente ecuación de Poisson en el interior de $V$. Puesto que la distribución de cargas imágenes es colocada en el exterior de $V$, entonces la contribución de estas cargas al potencial electrostático será una función armónica en $V$, la cual puede ser identificada sin ambigüedades como la función $F$.

Veamos a continuación la aplicación del método de las imágenes a la solución de dos problemas concretos.

\subsection{Carga puntual frente a un plano conductor conectado a tierra}

Supongamos que tenemos un bloque metálico conectado a tierra (potencial cero) que se extiende sobre la región de $\mathbb{R}^3$ tal que $(x, y, x) \in (-\infty, 0) \times (-\infty, +\infty) \times (-\infty, +\infty)$. Supongamos además que tenemos una carga $Q$ en el punto $\vec{r}_Q = (x_Q, y_Q, z_Q)$ ($x_Q > 0$). Queremos encontrar el potencial en todo punto del espacio $(0, +\infty) \times (-\infty, +\infty) \times (-\infty, +\infty)$.

\begin{figure}
\centering
% Aquí podrías incluir la imagen de la Figura 1.4 si la tienes disponible
\caption{Carga puntual frente a un bloque metálico rectangular y seminfinito a potencial cero.}
\label{fig:1.4}
\end{figure}

El problema puede resolverse fácilmente colocando una carga imagen $Q_i$ en la posición $\vec{r}_i = (-x_i, y_i, z_i)$ ($x_i > 0$) dentro del bloque metálico. Nótese que esta carga es exterior al volumen de interés situado a la derecha del eje $y$ (ver Fig. 1.4). El potencial asociado al sistema de cargas es
\begin{equation}
\phi(\vec{r}) = \frac{1}{4\pi\epsilon_0}\left[\frac{Q}{|\vec{r} - \vec{r}_Q|} + \frac{Q_i}{|\vec{r} - \vec{r}_i|}\right] = \frac{1}{4\pi\epsilon_0}\left[\frac{Q}{\sqrt{(x - x_Q)^2 + (y - y_Q)^2 + (z - z_Q)^2}} + \frac{Q_i}{\sqrt{(x + x_i)^2 + (y - y_i)^2 + (z - z_i)^2}}\right].
\end{equation}

Hay que imponer ahora la condición $\phi|_S = 0 = \phi(0, y, z)$. En consecuencia,
\begin{equation}
\frac{Q}{\sqrt{x_Q^2 + (y - y_Q)^2 + (z - z_Q)^2}} + \frac{Q_i}{\sqrt{x_i^2 + (y - y_i)^2 + (z - z_i)^2}} = 0,
\end{equation}
de donde es fácil ver que $Q_i = -Q$, $x_i = x_Q$, $y_i = y_Q$ y $z_i = z_Q$. Por lo tanto,
\begin{equation}
\phi(\vec{r}) = \frac{Q}{4\pi\epsilon_0}\left[\frac{1}{\sqrt{(x - x_Q)^2 + (y - y_Q)^2 + (z - z_Q)^2}} - \frac{1}{\sqrt{(x + x_Q)^2 + (y - y_Q)^2 + (z - z_Q)^2}}\right].
\end{equation}

En este caso la función de Green tiene la forma
\begin{equation}
G(\vec{r},\vec{r}') = \frac{1}{4\pi}\left[\frac{1}{\sqrt{(x - x')^2 + (y - y')^2 + (z - z')^2}} - \frac{1}{\sqrt{(x + x')^2 + (y - y')^2 + (z - z')^2}}\right].
\end{equation}

\subsection{Carga puntual frente a un cascarón esférico metálico a potencial cero}

\begin{figure}
\centering
% Aquí podrías incluir la imagen de la Figura 1.5 si la tienes disponible
\caption{Carga puntual frente a un cascarón esférico metálico a potencial cero.}
\label{fig:1.5}
\end{figure}

Supongamos ahora que tenemos una carga puntual $Q$ situada fuera del volumen limitado por un cascarón esférico de radio $a$ (ver Fig. 1.5). Consideremos además que dicho cascarón es metálico y que se encuentra a potencial cero. Nótese que el potencial producido por la carga $Q$ sobre el punto del espacio cuyo vector de posición respecto del centro de la esfera es $\vec{r}$ es invariante ante rotaciones del sistema alrededor del eje que une el centro de la esfera con la carga $Q$. En consecuencia, la carga imagen debe estar localizada en algún punto del volumen interior limitado por el cascarón y sobre el segmento que une el origen con la carga $Q$. Hemos supuesto anticipadamente que el signo de la carga imagen es opuesto al de la carga $Q$. En consecuencia,
\begin{equation}
\phi(\vec{r}) = \frac{1}{4\pi\epsilon_0}\left[\frac{Q}{|\vec{r} - \vec{r}'|} - \frac{Q_i}{|\vec{r} - \vec{r}''|}\right].
\end{equation}

En $\vec{r} = \vec{a}$ se tiene que $\phi(\vec{a}) = 0$. Por lo tanto
\begin{equation}
\frac{Q}{|\vec{a} - \vec{r}'|} - \frac{Q_i}{|\vec{a} - \vec{r}''|} = 0,
\end{equation}
o bien,
\begin{equation}
Q^2|\vec{a} - \vec{r}''|^2 = Q_i^2|\vec{a} - \vec{r}'|^2.
\end{equation}

De aquí se obtiene inmediatamente que
\begin{equation}
Q^2(a^2 + 2\vec{a} \cdot Q_i^2 \vec{r}' - Q^2 \vec{r}'' + Q^2 r''^2 - Q_i^2 r'^2) = 0.
\end{equation}

Nótese que el producto escalar de $\vec{a}$ con $Q_i^2 \vec{r}' - Q^2\vec{r}''$ depende del ángulo formado entre esos dos vectores. Puesto que la dirección del vector $\vec{a}$ es arbitraria la ecuación anterior no debe ser una función de dicho ángulo, lo cual tiene lugar solamente si $|Q_i^2 \vec{r}' -Q^2\vec{r}''| = 0$, o bien, $Q_i^2 \vec{r}' = Q^2 \vec{r}''$. De aquí se concluye que
\begin{equation}
Q_i^2 = Q^2 \frac{r''}{r'}.
\end{equation}

Sustituyendo (1.58) en (1.57) hallamos que
\begin{equation}
Q^2\left[1 - \frac{r''}{r'}a^2 + Q^2 r''^2 - Q^2 \frac{r''}{r'}r'^2\right] = 0,
\end{equation}
de donde
\begin{equation}
(r' - r'')(a^2 - r' r'') = 0.
\end{equation}

De la ecuación anterior se obtienen un par de soluciones. La primera de ellas es $r' = r''$, la cual no es general y sólo es válida sobre la superficie del cascarón. La segunda solución $r'' = \frac{a^2}{r'}$ (o en forma vectorial, $\vec{r}'' = \frac{a^2}{r'^2}\vec{r}'$) es más general y contiene a la primera. Sustituyéndola en (1.58) hallamos
\begin{equation}
Q_i^2 = Q^2 \frac{a^2}{r'^2},
\end{equation}
donde sólo la solución positiva $Q_i = Q \frac{a}{r'}$ es de interés.

De esta manera obtenemos la siguiente expresión para el potencial electrostático:
\begin{equation}
\phi(\vec{r}) = \frac{Q}{4\pi\epsilon_0}\left[\frac{1}{|\vec{r} - \vec{r}'|} - \frac{\frac{a}{r'}}{|\vec{r} - \frac{a^2}{r'^2}\vec{r}'|}\right].
\end{equation}

La función de Green para el problema planteado ($r \geq a$) será entonces
\begin{equation}
G(\vec{r},\vec{r}') = \frac{1}{4\pi}\left[\frac{1}{|\vec{r} - \vec{r}'|} - \frac{\frac{a}{r'}}{|\vec{r} - \frac{a^2}{r'^2}\vec{r}'|}\right].
\end{equation}

\section{Método de separación de variables para la solución de la ecuación de Laplace}

El método de separación de variables es uno de los más utilizados para la solución de ecuaciones diferenciales en derivadas parciales lineales y homogéneas como es el caso de la ecuación de Laplace. La idea del método es sencilla: proponer una solución de la ecuación diferencial como el producto de funciones que dependan cada una de una sola variable independiente, y sustituir dicha solución en la ecuación original, para encontrar ecuaciones diferenciales ordinarias, generalmente más simples de resolver. En esta sección presentaremos brevemente la técnica de separación de variables para la solución de la ecuación de Laplace en algunos sistemas de coordenadas.

\subsection{Solución de la ecuación de Laplace en coordenadas cartesianas}

La ecuación de Laplace, en coordenadas cartesianas, puede ser escrita en la forma
\begin{equation}
\left(\frac{\partial^2}{\partial x^2} + \frac{\partial^2}{\partial y^2} + \frac{\partial^2}{\partial z^2}\right) \phi(x, y, z) = 0.
\end{equation}

Se propone, como solución de (1.61), una función del tipo
\begin{equation}
\phi(x, y, z) = X(x)Y(y)Z(z).
\end{equation}
Sustituyendo en la ecuación diferencial hallamos
\begin{equation}
\frac{d^2X}{dx^2}Y(y)Z(z) + X(x)\frac{d^2Y}{dy^2}Z(z) + X(x)Y(y)\frac{d^2Z}{dz^2} = 0,
\end{equation}
o bien
\begin{equation}
\frac{1}{X}\frac{d^2X}{dx^2} + \frac{1}{Y}\frac{d^2Y}{dy^2} + \frac{1}{Z}\frac{d^2Z}{dz^2} = 0.
\end{equation}

Puesto que las variables $x$, $y$ y $z$ son independientes, entonces cada término de la ecuación anterior debe ser una constante numérica. Definamos
\begin{equation}
-\alpha^2 = \frac{1}{X}\frac{d^2X}{dx^2}, \quad -\beta^2 = \frac{1}{Y}\frac{d^2Y}{dy^2}, \quad \gamma^2 = \frac{1}{Z}\frac{d^2Z}{dz^2},
\end{equation}
donde $\gamma^2 = \alpha^2 + \beta^2$, es decir, los números $\alpha$, $\beta$ y $\gamma$ son números pitagóricos.

Reescribamos ahora de forma conveniente las tres ecuaciones anteriores, es decir,
\begin{equation}
\frac{d^2X}{dx^2} + \alpha^2X(x) = 0,
\end{equation}
\begin{equation}
\frac{d^2Y}{dy^2} + \beta^2Y(y) = 0,
\end{equation}
\begin{equation}
\frac{d^2Z}{dz^2} - \gamma^2Z(z) = 0.
\end{equation}

Dependiendo de los valores que tomen las constantes de integración $\alpha$, $\beta$ y $\gamma$, determinados por las condiciones de frontera para el potencial o sus primeras derivadas, pueden tener lugar varios tipos de soluciones para el potencial (1.62). Veamos.

\begin{itemize}
\item Caso 1: $\alpha \neq 0$, $\beta \neq 0$.

Las soluciones generales para $X$, $Y$ y $Z$ vienen dadas por las expresiones
\begin{align}
X(x) &= a_1 e^{i\alpha x} + a_2 e^{-i\alpha x}, \\
Y(y) &= b_1 e^{i\beta y} + b_2 e^{-i\beta y}
\end{align}
y
\begin{equation}
Z(z) = c_1 e^{\gamma z} + c_2 e^{-\gamma z},
\end{equation}
respectivamente.

\item Caso 2: $\alpha = 0$, $\beta = \gamma \neq 0$.

En este caso la solución para la función $X$ es polinómica, es decir,
\begin{equation}
X(x) = a_1 x + a_2.
\end{equation}
Además,
\begin{equation}
Y(y) = b_1 e^{i\beta y} + b_2 e^{-i\beta y}
\end{equation}
y
\begin{equation}
Z(z) = c_1 e^{\beta z} + c_2 e^{-\beta z}.
\end{equation}

\item Caso 3: $\alpha = \beta = \gamma = 0$.

En este caso las tres funciones $X$, $Y$ y $Z$ son polinomios de primer grado en sus respectivas variables, o sea,
\begin{align}
X(x) &= a_1 x + a_2, \\
Y(y) &= b_1 y + b_2
\end{align}
y
\begin{equation}
Z(z) = c_1 z + c_2.
\end{equation}
\end{itemize}

En todos los casos anteriores las constantes $a_1$, $a_2$, $b_1$, $b_2$, $c_1$ y $c_2$ se determinan también a partir de las condiciones de frontera específicas de cada problema concreto.

Ilustremos el procedimiento de solución de la ecuación de Laplace en coordenadas cartesianas a partir de un ejemplo sencillo. Consideremos un cubo de lado $a$ con dos caras opuestas conectadas a un potencial $V$ y el resto de sus caras conectadas a tierra (potencial cero), tal y como se muestra en la figura 1.6. Queremos encontrar el valor del potencial en el interior del cubo. Se propone, como solución de la ecuación diferencial de Laplace, una función del tipo (1.62). De acuerdo con el método de separación de variables, elijamos las constantes $\alpha$, $\beta$ y $\gamma$ de manera que se satisfagan las relaciones (1.63). Si tomamos el origen de coordenadas en uno de los vértices del cubo, las funciones $X$, $Y$ y $Z$ satisfacen las condiciones de frontera
\begin{align}
X(0) = X(a) &= 0, \\
Y(0) = Y(a) &= 0
\end{align}
y
\begin{equation}
Z(0) = Z(a) = V,
\end{equation}
respectivamente.

La función $X$ será solución de la ecuación diferencial (1.64), y se propone como
\begin{equation}
X(x) = A_1 \sin(\alpha x) + A_2 \cos(\alpha x).
\end{equation}
Pero $X(0) = 0 \Rightarrow A_2 = 0$. Luego,
\begin{equation}
X(x) = A_1 \sin(\alpha x).
\end{equation}

Por otra parte $X(a) = A_1 \sin(\alpha a) = 0$, de donde es evidente que $\alpha \equiv \alpha_n = \frac{n\pi}{a}$, siendo $n$ un número natural ($n = 0$ conduce a la solución trivial $X = 0$, la cual no posee sentido físico). Existen entonces infinitas soluciones para $X$, que denotaremos como $X_n$. Así,
\begin{equation}
X_n(x) = A_n \sin(\alpha_n x).
\end{equation}

Análogamente
\begin{equation}
Y_m(y) = B_m \sin(\beta_m y),
\end{equation}
donde $\beta_m = \frac{m\pi}{a}$.

Analicemos ahora la solución $Z$ de la ecuación diferencial (1.66), la cual se propone como
\begin{equation}
Z_{nm}(z) = C_{nm} \sinh(\gamma_{nm} z) + D_{nm} \cosh(\gamma_{nm} z)
\end{equation}
siendo $\gamma_{nm} = \sqrt{\alpha_n^2 + \beta_m^2} = \frac{\pi}{a}\sqrt{n^2 + m^2}$. De la condición de frontera $Z_{nm}(0) = V$ se desprende automáticamente que $D_{nm} = V$ para todos los valores posibles de los números $n$ y $m$. Luego
\begin{equation}
Z_{nm}(z) = C_{nm} \sinh(\gamma_{nm} z) + V \cosh(\gamma_{nm} z).
\end{equation}

Por otra parte, de la condición $Z_{nm}(a) = V$ se obtiene inmediatamente que
\begin{equation}
C_{nm} = V\frac{1 - \cosh(\gamma_{nm}a)}{\sinh(\gamma_{nm}a)}.
\end{equation}

En consecuencia
\begin{equation}
Z_{nm}(z) = V\left[\frac{1 - \cosh(\gamma_{nm}a)}{\sinh(\gamma_{nm}a)}\sinh(\gamma_{nm}z) + \cosh(\gamma_{nm}z)\right].
\end{equation}

Una solución particular para la ecuación de Laplace en el interior del cubo es
\begin{equation}
\phi_{nm}(x, y, z) = V A_n\sin(\alpha_n x)B_m\sin(\beta_m y)\left[\frac{1 - \cosh(\gamma_{nm}a)}{\sinh(\gamma_{nm}a)}\sinh(\gamma_{nm}z) + \cosh(\gamma_{nm}z)\right],
\end{equation}
mientras que la solución general puede ser escrita como
\begin{align}
\phi(x, y, z) &= \sum_{n=1}^{+\infty}\sum_{m=1}^{+\infty}\phi_{nm}(x, y, z) \\
&= V\sum_{n=1}^{+\infty}\sum_{m=1}^{+\infty}A_n\sin(\alpha_n x)B_m\sin(\beta_m y)\left[\frac{1 - \cosh(\gamma_{nm}a)}{\sinh(\gamma_{nm}a)}\sinh(\gamma_{nm}z) + \cosh(\gamma_{nm}z)\right].
\end{align}

Debemos encontrar ahora los valores de las constantes $A_n$ y $B_m$. Es evidente que la solución anterior satisface las condiciones de frontera en las variables $x$ y $y$. En la variable $z$ tenemos que
\begin{align}
\phi(x, y, 0) = \phi(x, y, a) &= V\sum_{n=1}^{+\infty}\sum_{m=1}^{+\infty}A_n\sin(\alpha_n x)B_m\sin(\beta_m y) \\
&= V\left[\sum_{n=1}^{+\infty}A_n\sin(\alpha_n x)\right]\left[\sum_{m=1}^{+\infty}B_m\sin(\beta_m y)\right].
\end{align}

Las condiciones de frontera en la variable $z$ son solamente satisfechas si $A_n$ y $B_m$ son las representaciones de Fourier de la función unidad en el intervalo $(0, a)$. Si
\begin{equation}
1 = \sum_{n=1}^{+\infty}A_n\sin(\alpha_n x),
\end{equation}
entonces
\begin{equation}
A_n = \frac{2}{a}\int_0^a\sin(\alpha_n x)dx = \frac{2}{n\pi}[1-(-1)^n].
\end{equation}

Análogamente
\begin{equation}
B_m = \frac{2}{m\pi}[1-(-1)^m].
\end{equation}

La solución buscada para el potencial dentro de la caja cúbica será entonces
\begin{equation}
\phi(x, y, z) = V\left(\frac{2}{\pi}\right)^2\sum_{n=1}^{+\infty}\sum_{m=1}^{+\infty}\frac{[1-(-1)^n][1-(-1)^m]}{nm}\sin(\alpha_n x)\sin(\beta_m y)\left[\frac{1 - \cosh(\gamma_{nm}a)}{\sinh(\gamma_{nm}a)}\sinh(\gamma_{nm}z) + \cosh(\gamma_{nm}z)\right].
\end{equation}

\subsection{Solución de la ecuación de Laplace en coordenadas cilíndricas}

La ecuación de Laplace en coordenadas cilíndricas tiene la forma
\begin{equation}
\frac{1}{r}\frac{\partial}{\partial r}\left(r\frac{\partial\phi}{\partial r}\right) + \frac{1}{r^2}\frac{\partial^2\phi}{\partial\varphi^2} + \frac{\partial^2\phi}{\partial z^2} = 0,
\end{equation}
siendo $r = \sqrt{x^2 + y^2}$, $\varphi = \arctan\left(\frac{y}{x}\right)$, y $\phi$ una función periódica de período $2\pi$ en la variable $\varphi$, es decir, $\phi(r, \varphi, z) = \phi(r, \varphi + 2\pi, z)$ para todos los valores permitidos de $r$ y $z$.

Se propone, como solución de (1.71), una función del tipo
\begin{equation}
\phi(r, \varphi, z) = R(r)\Phi(\varphi)Z(z),
\end{equation}
donde $\Phi$ es, obviamente, una función $2\pi$-periódica. Sustituyendo (1.72) en la ecuación diferencial obtenemos
\begin{equation}
\frac{1}{rR(r)}\frac{d}{dr}\left(r\frac{dR}{dr}\right) + \frac{1}{r^2\Phi(\varphi)}\frac{d^2\Phi}{d\varphi^2} + \frac{1}{Z(z)}\frac{d^2Z}{dz^2} = 0.
\end{equation}

Aplicando ahora el método de separación de variables podemos escribir
\begin{equation}
\frac{1}{Z(z)}\frac{d^2Z}{dz^2} = k^2
\end{equation}
y
\begin{equation}
\frac{1}{\Phi(\varphi)}\frac{d^2\Phi}{d\varphi^2} = -\nu^2.
\end{equation}

El factor $-\nu^2$ se escoge negativo con el fin de obtener funciones que tengan la propiedad de periodicidad en $\varphi$ (funciones $\sin$ y $\cos$). La ecuación para $Z(z)$ tiene como solución funciones exponenciales,
\begin{equation}
Z(z) = C_1 e^{kz} + C_2 e^{-kz},
\end{equation}
en donde $C_1$ y $C_2$ son constantes, mientras que la ecuación para $\Phi(\varphi)$ produce soluciones del tipo
\begin{equation}
\Phi(\varphi) = A\cos\nu\varphi + B\sin\nu\varphi,
\end{equation}
con $A$ y $B$ constantes. Como $\Phi(\varphi + 2\pi) = \Phi(\varphi)$ para todo valor de $\varphi$, debe tenerse que $\nu = 0, 1, 2, \ldots$.

Reemplazando ahora (1.74) y (1.75) en (1.73) se obtiene la ecuación para $R(r)$:
\begin{equation}
\frac{1}{rR(r)}\frac{d}{dr}\left(r\frac{dR}{dr}\right) = -k^2 + \frac{\nu^2}{r^2},
\end{equation}
que puede reescribirse como
\begin{equation}
r\frac{d}{dr}\left(r\frac{dR}{dr}\right) - \nu^2 R + k^2r^2R = 0.
\end{equation}

Esta es una ecuación diferencial de Bessel, cuya solución general viene dada en términos de las funciones de Bessel,
\begin{equation}
R(r) = C_3 J_{\nu}(kr) + C_4 Y_{\nu}(kr),
\end{equation}
donde $J_{\nu}$ y $Y_{\nu}$ son, respectivamente, las funciones de Bessel de primera y segunda especie, de orden $\nu$, y $C_3$ y $C_4$ son constantes. En el caso $k = 0$ la solución toma la forma
\begin{equation}
R(r) = C_3 r^{\nu} + C_4 r^{-\nu},
\end{equation}
con $C_3$ y $C_4$ constantes. Si, además, $\nu = 0$, se obtiene
\begin{equation}
R(r) = C_3 + C_4 \ln r.
\end{equation}

\subsubsection{Potencial independiente de $\varphi$ y $z$}

Si el potencial es independiente de $\varphi$ y de $z$, la ecuación (1.71) se reduce a
\begin{equation}
\frac{1}{r}\frac{d}{dr}\left(r\frac{d\phi}{dr}\right) = 0.
\end{equation}

La ecuación anterior puede resolverse mediante integración directa. Tenemos que
\begin{equation}
r\frac{d\phi}{dr} = c_1,
\end{equation}
siendo $c_1$ una constante. Integrando nuevamente obtenemos
\begin{equation}
\phi(r) = c_1\ln r + c_2,
\end{equation}
donde $c_2$ es otra constante.

\subsubsection{Potencial independiente de $z$}

Dividiendo por $RZ$ obtenemos
\begin{equation}
\frac{1}{rR}\frac{d}{dr}\left(r\frac{dR}{dr}\right) + \frac{1}{Z}\frac{d^2Z}{dz^2} = 0.
\end{equation}

El primer término depende solamente de $r$, mientras que el segundo solo depende de $z$. Para que la ecuación se satisfaga, cada uno de estos términos debe ser igual a una constante. Llamemos $k^2$ a esta constante. Entonces
\begin{equation}
\frac{1}{rR}\frac{d}{dr}\left(r\frac{dR}{dr}\right) = -k^2,
\end{equation}
\begin{equation}
\frac{1}{Z}\frac{d^2Z}{dz^2} = k^2.
\end{equation}

La ecuación (1.104) tiene como solución general
\begin{equation}
Z(z) = Ae^{kz} + Be^{-kz},
\end{equation}
donde $A$ y $B$ son constantes arbitrarias.

Por otra parte, la ecuación (1.103) puede reescribirse como
\begin{equation}
r\frac{d}{dr}\left(r\frac{dR}{dr}\right) + k^2r^2R = 0,
\end{equation}
o bien,
\begin{equation}
r^2\frac{d^2R}{dr^2} + r\frac{dR}{dr} + k^2r^2R = 0.
\end{equation}

Esta es una ecuación diferencial de Bessel de orden cero, y sus soluciones linealmente independientes son la función de Bessel de primera especie de orden cero, $J_0(kr)$, y la función de Bessel de segunda especie de orden cero, $Y_0(kr)$. Por lo tanto, la solución general de (1.107) es
\begin{equation}
R(r) = CJ_0(kr) + DY_0(kr),
\end{equation}
siendo $C$ y $D$ constantes.

La solución general de la ecuación (1.99) puede expresarse entonces como
\begin{equation}
\phi(r, z) = \int_{0}^{\infty} \left[A(k)e^{kz} + B(k)e^{-kz}\right]\left[C(k)J_0(kr) + D(k)Y_0(kr)\right]dk,
\end{equation}
donde $A(k)$, $B(k)$, $C(k)$ y $D(k)$ son constantes (posiblemente dependientes de $k$) a determinar a partir de las condiciones de frontera.

\subsection{Solución de la ecuación de Laplace en coordenadas esféricas}

La ecuación de Laplace en coordenadas esféricas $(r, \theta, \varphi)$ es
\begin{equation}
\nabla^2\phi = \frac{1}{r^2}\frac{\partial}{\partial r}\left(r^2\frac{\partial \phi}{\partial r}\right) + \frac{1}{r^2\sin\theta}\frac{\partial}{\partial\theta}\left(\sin\theta\frac{\partial\phi}{\partial\theta}\right) + \frac{1}{r^2\sin^2\theta}\frac{\partial^2\phi}{\partial\varphi^2} = 0.
\end{equation}

En muchos problemas físicos importantes, la simetría del sistema conduce a que el potencial sea independiente del ángulo azimutal $\varphi$, es decir, $\phi = \phi(r, \theta)$. En este caso, la ecuación de Laplace se reduce a
\begin{equation}
\frac{1}{r^2}\frac{\partial}{\partial r}\left(r^2\frac{\partial \phi}{\partial r}\right) + \frac{1}{r^2\sin\theta}\frac{\partial}{\partial\theta}\left(\sin\theta\frac{\partial\phi}{\partial\theta}\right) = 0.
\end{equation}

Usando nuevamente el método de separación de variables, proponemos
\begin{equation}
\phi(r, \theta) = R(r)\Theta(\theta).
\end{equation}

Sustituyendo esta expresión en la ecuación (1.111) obtenemos
\begin{equation}
\frac{\Theta}{r^2}\frac{d}{dr}\left(r^2\frac{dR}{dr}\right) + \frac{R}{r^2\sin\theta}\frac{d}{d\theta}\left(\sin\theta\frac{d\Theta}{d\theta}\right) = 0.
\end{equation}

Multiplicando por $\frac{r^2\sin\theta}{R\Theta}$ hallamos
\begin{equation}
\frac{\sin\theta}{R}\frac{d}{dr}\left(r^2\frac{dR}{dr}\right) + \frac{1}{\Theta}\frac{d}{d\theta}\left(\sin\theta\frac{d\Theta}{d\theta}\right) = 0.
\end{equation}

El primer término de esta ecuación depende solamente de $r$, mientras que el segundo solo depende de $\theta$. Para que la ecuación se satisfaga, cada uno de estos términos debe ser igual a una constante. Llamemos $l(l+1)$ a esta constante. Entonces
\begin{equation}
\frac{\sin\theta}{R}\frac{d}{dr}\left(r^2\frac{dR}{dr}\right) = -l(l+1)\sin\theta,
\end{equation}
\begin{equation}
\frac{1}{\Theta}\frac{d}{d\theta}\left(\sin\theta\frac{d\Theta}{d\theta}\right) = l(l+1).
\end{equation}

La ecuación (1.115) puede simplificarse a
\begin{equation}
\frac{1}{R}\frac{d}{dr}\left(r^2\frac{dR}{dr}\right) = -l(l+1),
\end{equation}
o bien,
\begin{equation}
r^2\frac{d^2R}{dr^2} + 2r\frac{dR}{dr} - l(l+1)R = 0.
\end{equation}

Esta es una ecuación diferencial ordinaria de Euler, cuya solución general es
\begin{equation}
R(r) = Ar^l + Br^{-(l+1)},
\end{equation}
siendo $A$ y $B$ constantes arbitrarias.

Por otra parte, la ecuación (1.116) puede reescribirse como
\begin{equation}
\frac{d}{d\theta}\left(\sin\theta\frac{d\Theta}{d\theta}\right) - l(l+1)\sin\theta\Theta = 0,
\end{equation}
o bien,
\begin{equation}
\sin\theta\frac{d^2\Theta}{d\theta^2} + \cos\theta\frac{d\Theta}{d\theta} - l(l+1)\sin\theta\Theta = 0.
\end{equation}

Haciendo el cambio de variable $\mu = \cos\theta$, la ecuación anterior se transforma en
\begin{equation}
\frac{d}{d\mu}\left[(1-\mu^2)\frac{d\Theta}{d\mu}\right] + l(l+1)\Theta = 0,
\end{equation}
que es la ecuación de Legendre. Las soluciones linealmente independientes de esta ecuación son los polinomios de Legendre de primera especie, $P_l(\mu)$, y los polinomios de Legendre de segunda especie, $Q_l(\mu)$. Por lo tanto, la solución general de (1.121) es
\begin{equation}
\Theta(\theta) = CP_l(\cos\theta) + DQ_l(\cos\theta),
\end{equation}
donde $C$ y $D$ son constantes.

La solución general de la ecuación de Laplace para $\phi(r, \theta)$ puede expresarse entonces como
\begin{equation}
\phi(r, \theta) = \sum_{l=0}^{\infty}\left[A_l r^l + B_l r^{-(l+1)}\right]\left[C_l P_l(\cos\theta) + D_l Q_l(\cos\theta)\right].
\end{equation}

En el caso particular de problemas donde el potencial debe ser finito en el origen y en el infinito, se tienen restricciones adicionales. Para $r \to 0$, el término $r^{-(l+1)}$ diverge, por lo que $B_l = 0$ si el potencial debe ser finito en el origen. Análogamente, para $r \to \infty$, el término $r^l$ diverge para $l > 0$, por lo que $A_l = 0$ (excepto posiblemente para $l = 0$) si el potencial debe ser finito en el infinito.

Además, los polinomios de Legendre de segunda especie, $Q_l(\cos\theta)$, tienen singularidades en $\theta = 0$ y $\theta = \pi$. Por lo tanto, si el potencial debe ser regular en estos puntos, se requiere que $D_l = 0$.

En el caso más general, donde el potencial depende también del ángulo azimutal $\varphi$, la solución de la ecuación de Laplace puede expresarse en términos de los armónicos esféricos $Y_{lm}(\theta, \varphi)$:
\begin{equation}
\phi(r, \theta, \varphi) = \sum_{l=0}^{\infty}\sum_{m=-l}^{l}\left[A_{lm}r^l + B_{lm}r^{-(l+1)}\right]Y_{lm}(\theta, \varphi).
\end{equation}

Los armónicos esféricos están definidos como
\begin{equation}
Y_{lm}(\theta, \varphi) = \sqrt{\frac{(2l+1)(l-m)!}{4\pi(l+m)!}}P_l^m(\cos\theta)e^{im\varphi},
\end{equation}
donde $P_l^m(\cos\theta)$ son las funciones asociadas de Legendre.

\section{Desarrollo en multipolos del campo electrostático}

El campo electrostático producido por un sistema de cargas que ocupa un volumen finito puede ser descrito, para distancias suficientemente grandes comparadas con las dimensiones del sistema, mediante el llamado desarrollo en multipolos. Este método consiste en expresar el potencial electrostático como una serie de términos que representan contribuciones de diferentes órdenes (monopolo, dipolo, cuadrupolo, etc.) y que decrecen en importancia a medida que la distancia aumenta.

Supongamos que tenemos una distribución de carga con densidad $\rho(\vec{r'})$ confinada a un volumen $V$ de dimensiones características $a$. El potencial electrostático en un punto $\vec{r}$ exterior a la distribución de carga ($r > a$) está dado por
\begin{equation}
\phi(\vec{r}) = \frac{1}{4\pi\epsilon_0}\int_V \frac{\rho(\vec{r'})}{|\vec{r} - \vec{r'}|} dV'.
\end{equation}

Para distancias $r \gg a$, podemos desarrollar el factor $\frac{1}{|\vec{r} - \vec{r'}|}$ en serie de potencias de $\frac{r'}{r}$. Para ello, recordemos que
\begin{equation}
\frac{1}{|\vec{r} - \vec{r'}|} = \frac{1}{r}\frac{1}{\sqrt{1 - 2\frac{\vec{r}\cdot\vec{r'}}{r^2} + \frac{r'^2}{r^2}}}.
\end{equation}

Para $r > r'$, esta expresión puede expandirse como
\begin{equation}
\frac{1}{|\vec{r} - \vec{r'}|} = \frac{1}{r}\sum_{l=0}^{\infty}\left(\frac{r'}{r}\right)^l P_l(\cos\gamma),
\end{equation}
donde $P_l(\cos\gamma)$ son los polinomios de Legendre y $\gamma$ es el ángulo entre los vectores $\vec{r}$ y $\vec{r'}$.

Utilizando la fórmula del coseno del ángulo entre dos vectores, $\cos\gamma = \frac{\vec{r}\cdot\vec{r'}}{rr'}$, podemos reescribir el desarrollo como
\begin{equation}
\frac{1}{|\vec{r} - \vec{r'}|} = \frac{1}{r}\sum_{l=0}^{\infty}\left(\frac{r'}{r}\right)^l P_l\left(\frac{\vec{r}\cdot\vec{r'}}{rr'}\right).
\end{equation}

Sustituyendo este desarrollo en la expresión del potencial, obtenemos
\begin{equation}
\phi(\vec{r}) = \frac{1}{4\pi\epsilon_0}\frac{1}{r}\sum_{l=0}^{\infty}\frac{1}{r^l}\int_V \rho(\vec{r'})r'^l P_l\left(\frac{\vec{r}\cdot\vec{r'}}{rr'}\right) dV'.
\end{equation}

Para simplificar la notación, definimos los momentos multipolares del sistema como
\begin{equation}
Q_l = \int_V \rho(\vec{r'})r'^l P_l\left(\frac{\vec{r}\cdot\vec{r'}}{rr'}\right) dV'.
\end{equation}

Así, el potencial electrostático puede expresarse como
\begin{equation}
\phi(\vec{r}) = \frac{1}{4\pi\epsilon_0}\frac{1}{r}\sum_{l=0}^{\infty}\frac{Q_l}{r^l}.
\end{equation}

Analizaremos ahora los primeros términos de este desarrollo.

\subsection{Momentos multipolos y algunos de sus propiedades}

\subsubsection{Momento monopolar (término $l=0$)}

Para $l=0$ tenemos $P_0(x) = 1$, y por lo tanto
\begin{equation}
Q_0 = \int_V \rho(\vec{r'}) dV' = Q,
\end{equation}
donde $Q$ es la carga total del sistema. El término monopolar del potencial es
\begin{equation}
\phi_{\text{monopolo}}(\vec{r}) = \frac{1}{4\pi\epsilon_0}\frac{Q}{r},
\end{equation}
que corresponde al potencial de una carga puntual.

\subsubsection{Momento dipolar (término $l=1$)}

Para $l=1$ tenemos $P_1(x) = x$, y por lo tanto
\begin{equation}
Q_1 = \int_V \rho(\vec{r'})r' \frac{\vec{r}\cdot\vec{r'}}{rr'} dV' = \frac{\vec{r}}{r}\cdot\int_V \rho(\vec{r'})\vec{r'} dV' = \frac{\vec{r}}{r}\cdot\vec{p},
\end{equation}
donde $\vec{p} = \int_V \rho(\vec{r'})\vec{r'} dV'$ es el momento dipolar eléctrico. El término dipolar del potencial es
\begin{equation}
\phi_{\text{dipolo}}(\vec{r}) = \frac{1}{4\pi\epsilon_0}\frac{1}{r^2}\frac{\vec{r}}{r}\cdot\vec{p} = \frac{1}{4\pi\epsilon_0}\frac{\vec{r}\cdot\vec{p}}{r^3}.
\end{equation}

Alternativamente, podemos escribir
\begin{equation}
\phi_{\text{dipolo}}(\vec{r}) = \frac{1}{4\pi\epsilon_0}\frac{\vec{p}\cdot\vec{r}}{r^3} = \frac{1}{4\pi\epsilon_0}\frac{p\cos\theta}{r^2},
\end{equation}
donde $\theta$ es el ángulo entre $\vec{p}$ y $\vec{r}$.

El campo eléctrico correspondiente al momento dipolar es
\begin{equation}
\vec{E}_{\text{dipolo}}(\vec{r}) = -\nabla\phi_{\text{dipolo}}(\vec{r}) = \frac{1}{4\pi\epsilon_0}\left[\frac{3(\vec{p}\cdot\vec{r})\vec{r}}{r^5} - \frac{\vec{p}}{r^3}\right].
\end{equation}

\subsubsection{Momento cuadrupolar (término $l=2$)}

Para $l=2$ tenemos $P_2(x) = \frac{1}{2}(3x^2 - 1)$, y por lo tanto
\begin{equation}
Q_2 = \int_V \rho(\vec{r'})r'^2 \frac{1}{2}\left[3\left(\frac{\vec{r}\cdot\vec{r'}}{rr'}\right)^2 - 1\right] dV'.
\end{equation}

Este término puede reescribirse de la forma
\begin{equation}
Q_2 = \frac{1}{2}\sum_{i,j=1}^{3}\frac{r_i r_j}{r^2}Q_{ij},
\end{equation}
donde $Q_{ij}$ es el tensor momento cuadrupolar, definido como
\begin{equation}
Q_{ij} = \int_V \rho(\vec{r'})(3r'_i r'_j - r'^2\delta_{ij}) dV'.
\end{equation}

El término cuadrupolar del potencial es
\begin{equation}
\phi_{\text{cuadrupolo}}(\vec{r}) = \frac{1}{8\pi\epsilon_0}\frac{1}{r^3}\sum_{i,j=1}^{3}\frac{r_i r_j}{r^2}Q_{ij}.
\end{equation}

El tensor $Q_{ij}$ tiene las siguientes propiedades:
\begin{enumerate}
\item Es simétrico: $Q_{ij} = Q_{ji}$
\item Su traza es nula: $\sum_{i=1}^{3}Q_{ii} = 0$
\end{enumerate}

En consecuencia, $Q_{ij}$ tiene solo 5 componentes independientes.

\subsection{Expansión de multipolo de una distribución de cargas puntuales}

Para una distribución discreta de cargas puntuales, $\rho(\vec{r'}) = \sum_{i=1}^{N} q_i \delta(\vec{r'} - \vec{r}_i)$, los momentos multipolares toman la forma:

\subsubsection{Momento monopolar}
\begin{equation}
Q = \sum_{i=1}^{N} q_i
\end{equation}

\subsubsection{Momento dipolar}
\begin{equation}
\vec{p} = \sum_{i=1}^{N} q_i \vec{r}_i
\end{equation}

\subsubsection{Momento cuadrupolar}
\begin{equation}
Q_{ij} = \sum_{i=1}^{N} q_i (3r_{ki} r_{kj} - r_k^2\delta_{ij})
\end{equation}

En el caso particular de un sistema neutro ($Q = 0$) y con momento dipolar nulo ($\vec{p} = \vec{0}$), el primer término no nulo en el desarrollo en multipolos corresponde al momento cuadrupolar, y el potencial electrostático a grandes distancias se comporta como $r^{-3}$.

El desarrollo en multipolos es particularmente útil cuando estamos interesados en el comportamiento del campo electrostático a distancias grandes comparadas con las dimensiones del sistema de cargas. La convergencia de la serie es más rápida cuanto mayor sea la relación $r/a$, donde $a$ es la dimensión característica del sistema.

\section{Energía del campo electrostático}

La energía de un sistema de cargas interactuantes puede calcularse de diversas formas. Una de ellas consiste en determinar el trabajo necesario para construir el sistema, trayendo las cargas desde el infinito hasta sus posiciones finales. Otra forma es calcular la energía almacenada en el propio campo electrostático generado por las cargas.

Consideremos primeramente un sistema de $N$ cargas puntuales $q_1, q_2, \ldots, q_N$ situadas en las posiciones $\vec{r}_1, \vec{r}_2, \ldots, \vec{r}_N$. El potencial electrostático en un punto $\vec{r}$ debido a este sistema es
\begin{equation}
\phi(\vec{r}) = \frac{1}{4\pi\epsilon_0} \sum_{i=1}^{N} \frac{q_i}{|\vec{r} - \vec{r}_i|}.
\end{equation}

La energía potencial de una carga de prueba $q$ colocada en la posición $\vec{r}$ es $U = q\phi(\vec{r})$. Para calcular la energía del sistema completo, imaginemos el proceso de construcción del mismo. Suponemos que inicialmente todas las cargas están separadas a distancias infinitas entre sí (donde la energía potencial es cero). Traemos la primera carga $q_1$ desde el infinito hasta la posición $\vec{r}_1$. Como no hay otras cargas presentes, no se realiza trabajo en este paso. Luego traemos la segunda carga $q_2$ desde el infinito hasta la posición $\vec{r}_2$ en presencia del campo creado por $q_1$. El trabajo necesario es $W_2 = q_2\phi_1(\vec{r}_2)$, donde $\phi_1$ es el potencial debido a $q_1$. Continuamos el proceso trayendo la tercera carga, y así sucesivamente.

El trabajo total para construir el sistema será
\begin{equation}
W = \sum_{i=2}^{N} q_i \sum_{j=1}^{i-1} \frac{1}{4\pi\epsilon_0} \frac{q_j}{|\vec{r}_i - \vec{r}_j|} = \frac{1}{8\pi\epsilon_0} \sum_{i=1}^{N} \sum_{j=1, j\neq i}^{N} \frac{q_i q_j}{|\vec{r}_i - \vec{r}_j|},
\end{equation}
donde hemos utilizado que $\sum_{i=2}^{N} q_i \sum_{j=1}^{i-1} \frac{q_j}{|\vec{r}_i - \vec{r}_j|} = \frac{1}{2} \sum_{i=1}^{N} \sum_{j=1, j\neq i}^{N} \frac{q_i q_j}{|\vec{r}_i - \vec{r}_j|}$. Este trabajo representa la energía potencial electrostática del sistema.

Para una distribución continua de carga con densidad $\rho(\vec{r})$, la energía electrostática puede expresarse como
\begin{equation}
W = \frac{1}{2} \int \rho(\vec{r}) \phi(\vec{r}) dV,
\end{equation}
donde la integral se extiende sobre todo el volumen donde $\rho \neq 0$. Sustituyendo la expresión para el potencial, obtenemos
\begin{equation}
W = \frac{1}{8\pi\epsilon_0} \int\int \frac{\rho(\vec{r})\rho(\vec{r}')}{|\vec{r} - \vec{r}'|} dV dV'.
\end{equation}

Otra forma de expresar la energía electrostática es en términos del campo eléctrico. Utilizando la ley de Gauss $\nabla \cdot \vec{E} = \rho/\epsilon_0$ y el hecho de que $\vec{E} = -\nabla\phi$, podemos escribir
\begin{equation}
\rho = \epsilon_0 \nabla \cdot \vec{E} = -\epsilon_0 \nabla^2\phi.
\end{equation}

Sustituyendo en la expresión de la energía, obtenemos
\begin{equation}
W = -\frac{\epsilon_0}{2} \int \phi \nabla^2\phi\, dV.
\end{equation}

Utilizando la identidad $\phi \nabla^2\phi = \nabla \cdot (\phi \nabla\phi) - |\nabla\phi|^2$ y el teorema de la divergencia, podemos transformar esta expresión en
\begin{equation}
W = \frac{\epsilon_0}{2} \int |\nabla\phi|^2 dV - \frac{\epsilon_0}{2} \oint \phi \nabla\phi \cdot d\vec{S}.
\end{equation}

Si las cargas están confinadas en una región finita del espacio, el término de superficie se anula cuando la integración se extiende al infinito, ya que tanto $\phi$ como $\nabla\phi$ decaen suficientemente rápido. Por lo tanto,
\begin{equation}
W = \frac{\epsilon_0}{2} \int |\nabla\phi|^2 dV = \frac{\epsilon_0}{2} \int |\vec{E}|^2 dV = \frac{1}{2} \int \epsilon_0 |\vec{E}|^2 dV.
\end{equation}

Esta expresión muestra que la energía del sistema puede interpretarse como la energía almacenada en el propio campo eléctrico. La cantidad $w_e = \frac{1}{2}\epsilon_0 |\vec{E}|^2$ representa la densidad de energía del campo eléctrico, es decir, la energía por unidad de volumen.

\section{Fuerzas de origen electrostático}

Las fuerzas de origen electrostático son aquellas que se ejercen entre cargas eléctricas debido a su interacción. Estas fuerzas pueden calcularse utilizando diferentes enfoques. Analizaremos algunos casos importantes.

\subsection{Fuerza sobre una carga puntual}

La fuerza que experimenta una carga puntual $q$ situada en un punto $\vec{r}$ bajo la acción de un campo eléctrico $\vec{E}(\vec{r})$ viene dada por
\begin{equation}
\vec{F} = q\vec{E}(\vec{r}).
\end{equation}

Si el campo es creado por otra carga puntual $q'$ situada en $\vec{r}'$, entonces
\begin{equation}
\vec{E}(\vec{r}) = \frac{1}{4\pi\epsilon_0} \frac{q'}{|\vec{r} - \vec{r}'|^2} \frac{\vec{r} - \vec{r}'}{|\vec{r} - \vec{r}'|},
\end{equation}
y la fuerza sobre la carga $q$ es
\begin{equation}
\vec{F} = \frac{1}{4\pi\epsilon_0} \frac{qq'}{|\vec{r} - \vec{r}'|^2} \frac{\vec{r} - \vec{r}'}{|\vec{r} - \vec{r}'|}.
\end{equation}
Esta es la conocida ley de Coulomb.

\subsection{Fuerza sobre una distribución de carga}

Para una distribución continua de carga con densidad $\rho(\vec{r})$, la fuerza total ejercida por un campo eléctrico externo $\vec{E}_{ext}(\vec{r})$ es
\begin{equation}
\vec{F} = \int \rho(\vec{r}) \vec{E}_{ext}(\vec{r}) dV.
\end{equation}

Es importante notar que $\vec{E}_{ext}$ es el campo eléctrico producido por fuentes externas y no incluye el campo generado por la propia distribución $\rho(\vec{r})$, ya que las fuerzas internas se cancelan entre sí según el principio de acción y reacción.

\subsection{Tensor de tensiones de Maxwell}

Una forma alternativa de calcular fuerzas electrostáticas es mediante el tensor de tensiones de Maxwell. Este enfoque considera las fuerzas como resultado de tensiones en el campo electrostático.

Para el campo electrostático, el tensor de tensiones de Maxwell tiene componentes
\begin{equation}
T_{ij} = \epsilon_0 \left( E_i E_j - \frac{1}{2} \delta_{ij} |\vec{E}|^2 \right),
\end{equation}
donde $E_i$ y $E_j$ son las componentes del campo eléctrico, y $\delta_{ij}$ es la delta de Kronecker.

La fuerza sobre una distribución de carga contenida en un volumen $V$ puede calcularse como
\begin{equation}
F_i = \oint_S T_{ij} n_j dS,
\end{equation}
donde la integral se realiza sobre una superficie cerrada $S$ que encierra el volumen $V$, y $n_j$ son las componentes del vector unitario normal a la superficie.

\subsection{Presión electrostática}

Un caso particular de interés es la presión electrostática en la superficie de un conductor. En la superficie de un conductor en equilibrio electrostático, el campo eléctrico es perpendicular a la superficie con magnitud $E = \sigma/\epsilon_0$, donde $\sigma$ es la densidad superficial de carga.

La presión electrostática sobre la superficie es
\begin{equation}
p = \frac{1}{2} \epsilon_0 E^2 = \frac{\sigma^2}{2\epsilon_0}.
\end{equation}
Esta presión siempre actúa hacia afuera del conductor.

\subsection{Fuerzas entre conductores}

La fuerza entre conductores cargados puede calcularse a partir de la energía del sistema. Si $W$ es la energía electrostática y $x$ es la coordenada que describe la posición relativa entre los conductores, la fuerza en la dirección $x$ es
\begin{equation}
F_x = -\frac{\partial W}{\partial x}.
\end{equation}

Para un sistema de conductores mantenidos a potenciales constantes $\phi_1, \phi_2, \ldots, \phi_N$, la energía es $W = \frac{1}{2} \sum_{i,j} C_{ij}\phi_i\phi_j$, donde $C_{ij}$ son los coeficientes de capacidad e inducción. La fuerza viene dada por
\begin{equation}
F_x = -\frac{1}{2} \sum_{i,j} \frac{\partial C_{ij}}{\partial x}\phi_i\phi_j.
\end{equation}

Si, en cambio, los conductores mantienen cargas constantes $q_1, q_2, \ldots, q_N$, la energía es $W = \frac{1}{2} \sum_{i,j} p_{ij}q_i q_j$, donde $p_{ij}$ son elementos de la matriz inversa a $C_{ij}$. La fuerza en este caso es
\begin{equation}
F_x = -\frac{1}{2} \sum_{i,j} \frac{\partial p_{ij}}{\partial x}q_i q_j.
\end{equation}
\subsection{Dipolo eléctrico en presencia de un campo electrostático}

Un dipolo eléctrico en presencia de un campo electrostático experimenta tanto una fuerza como un torque. Consideremos un dipolo con momento dipolar $\vec{p}$ colocado en un campo eléctrico externo $\vec{E}(\vec{r})$.

La energía potencial del dipolo en el campo es
\begin{equation}
U = -\vec{p} \cdot \vec{E}.
\end{equation}

Esta energía da lugar a un torque que tiende a alinear el dipolo con el campo eléctrico:
\begin{equation}
\vec{\tau} = \vec{p} \times \vec{E}.
\end{equation}

Si el campo eléctrico no es uniforme, el dipolo también experimenta una fuerza neta:
\begin{equation}
\vec{F} = (\vec{p} \cdot \nabla)\vec{E}.
\end{equation}

En componentes cartesianas, esta fuerza puede escribirse como
\begin{equation}
F_i = p_j \frac{\partial E_i}{\partial x_j},
\end{equation}
donde se utiliza la convención de suma sobre índices repetidos.

Si expresamos el campo en términos del potencial, $\vec{E} = -\nabla \phi$, la fuerza puede escribirse como
\begin{equation}
\vec{F} = -(\vec{p} \cdot \nabla)\nabla \phi = -\nabla(\vec{p} \cdot \nabla \phi) + \nabla \times (\vec{p} \times \nabla \phi).
\end{equation}

Para un dipolo con momento dipolar constante, el segundo término se anula y la fuerza es
\begin{equation}
\vec{F} = -\nabla(\vec{p} \cdot \nabla \phi) = -\nabla(\vec{p} \cdot \vec{E}).
\end{equation}

Esta expresión muestra que la fuerza sobre un dipolo eléctrico puede interpretarse como el gradiente negativo de la energía potencial.

\subsection{Radio clásico del electrón}

El radio clásico del electrón, también conocido como radio de Lorentz o longitud de Thomson, es un parámetro fundamental en la electrodinámica clásica. Se define como la distancia a la cual la energía electrostática de una esfera uniformemente cargada iguala la energía de masa del electrón según la relación de Einstein $E = mc^2$.

Para una esfera de radio $r_e$ con carga $e$ uniformemente distribuida, la energía electrostática es
\begin{equation}
U_e = \frac{1}{2} \frac{e^2}{4\pi\epsilon_0 r_e}.
\end{equation}

Igualando esta energía a la energía de masa del electrón, obtenemos
\begin{equation}
\frac{1}{2} \frac{e^2}{4\pi\epsilon_0 r_e} = m_e c^2,
\end{equation}
donde $m_e$ es la masa del electrón y $c$ es la velocidad de la luz.

Despejando $r_e$, encontramos el radio clásico del electrón:
\begin{equation}
r_e = \frac{1}{4\pi\epsilon_0} \frac{e^2}{2m_e c^2} \approx 2.82 \times 10^{-15} \, \textrm{m}.
\end{equation}

Este valor representa la escala de longitud en la que los efectos de la mecánica cuántica y la electrodinámica cuántica se vuelven importantes para describir correctamente el comportamiento del electrón. La teoría clásica del electrón como una esfera cargada falla a distancias del orden del radio clásico.

El radio clásico del electrón aparece naturalmente en la fórmula de la sección eficaz de dispersión de Thomson para la dispersión de radiación electromagnética por electrones libres:
\begin{equation}
\sigma_T = \frac{8\pi}{3} r_e^2.
\end{equation}

Es importante señalar que, según la mecánica cuántica, el electrón es una partícula puntual sin estructura interna conocida. El radio clásico es simplemente un parámetro que surge del tratamiento clásico y no debe interpretarse como el tamaño físico real del electrón.

\section{Problemas propuestos}

\begin{enumerate}
\item Del octante $x \geq 0$; $y \geq 0$; $z \geq 0$ de la bola $x^2 + y^2 + z^2 \leq c^2$ se ha sustraído el cuerpo limitado por los planos coordenados y por el plano $\frac{x}{a} + \frac{y}{b} + \frac{z}{c} = 1$ ($a > 0$, $b > 0$, $c > 0$, $a \leq c$, $b \leq c$). Halle la carga del cuerpo si la densidad de carga es $\rho(\vec{r}) = \lambda_0 z$.

\item Determine el potencial del campo electrostático $\vec{E}(\vec{r}) = ay\vec{i} + (ax + bz)\vec{j} + by\vec{k}$, siendo $a$ y $b$ constantes. Calcule el trabajo realizado por el campo al llevar la carga $q$ desde el origen de coordenadas hasta el punto $\left(\frac{E_0}{a}, \frac{E_0}{\sqrt{2}}, \frac{E_0}{b}\sqrt{\frac{a^2}{a^2+b^2}}\right)$, siendo $E_0$ otra constante.

\item Calcule la función de Green para el interior de un sector circular plano de radio $R$ con condiciones de frontera de Dirichlet en la frontera.

\item Calcule la función de Green para el interior de una bola de radio $R$ con la condición de frontera de Dirichlet. Repita el cálculo utilizando la condición de frontera de Neumann.

\item Sea el rectángulo seminfinito $R = (0, L) \times (0, +\infty)$ y $(x, y) \in R$. Supongamos que el potencial electrostático es conocido en la frontera de este rectángulo: en particular $\phi(0, y) = \phi(L, y) = 0$ $\forall y \in (0, +\infty)$ y $\phi(x, 0) = V\sin\left(\frac{4\pi}{L}x\right)$ $\forall x \in (0, L)$. Calcule el potencial electrostático en el interior de $R$.

\item Sea una superficie cilíndrica de radio $R$ y altura $2a + b$ cerrada en ambos extremos por dos tapas circulares de radio $R$ conectadas a potencial cero. Supongamos además que el potencial electrostático sobre la superficie lateral viene dado por la expresión
\begin{equation*}
V(\varphi, z) = 
\begin{cases}
0 & \text{si } 0 < z < a \quad \forall \varphi \in (0, 2\pi) \\
V_1 & \text{si } a < z < a + b \quad \forall \varphi \in (0, 2\pi) \\
0 & \text{si } a + b < z < 2a + b \quad \forall \varphi \in (0, 2\pi)
\end{cases}
\end{equation*}
Calcule el potencial electrostático en el conjunto que tiene como frontera exterior a la superficie dada.

\item El potencial electrostático sobre un cascarón esférico de radio $R$ viene dado por la expresión
\begin{equation*}
V(\theta, \varphi) = 
\begin{cases}
V_1 & \text{si } 0 < \theta < \pi/2 \quad \forall \varphi \in (0, 2\pi) \\
0 & \text{si } \pi/2 < \theta < \pi \quad \forall \varphi \in (0, 2\pi)
\end{cases}
\end{equation*}
Encuentre el potencial electrostático en el interior del cascarón.

\item Demuestre que el vector intensidad del campo eléctrico correspondiente a un dipolo eléctrico de momento dipolar $\vec{p}$ viene dado por la expresión
\begin{equation*}
\vec{E} = \frac{1}{4\pi\epsilon_0}\left[3\frac{(\vec{p} \cdot \vec{r})\vec{r}}{r^5} - \frac{\vec{p}}{r^3}\right]
\end{equation*}

\item Demuestre que el vector intensidad del campo eléctrico correspondiente a un cuadrupolo eléctrico de momento cuadrupolar con componentes $D_{ij}$ posee componentes dadas por la expresión
\begin{equation*}
E_k = \frac{1}{24\pi\epsilon_0}\left[\frac{15D_{ij}x_ix_jx_k}{r^7} - \frac{3}{r^5}(x_i\delta_{jk} + x_k\delta_{ij} + x_j\delta_{ki})\right]
\end{equation*}

\item Demuestre que un dipolo eléctrico puntual de momento dipolar $\vec{p}$ responde a una densidad de carga $\rho = -(\vec{p} \cdot \nabla)\delta(\vec{r} - \vec{r}_0)$, siendo $\vec{r}_0$ el vector de posición del dipolo.

\item Sean las cargas puntuales $q_1 = -e$ situada en el punto de coordenadas $(0, a, 0)$, $q_2 = e$ localizada en $(a, 0, 0)$, $q_3 = -e$ ubicada en el punto $(0, -a, 0)$ y $q_4 = e$ localizada en $(-a, 0, 0)$. Demuestre que el momento cuadrupolar eléctrico asociado a esta distribución de cargas viene dado por la expresión
\begin{equation*}
(D_{ij}) = 6ea^2 \begin{pmatrix} 
1 & 0 & 0 \\
0 & -1 & 0 \\
0 & 0 & 0
\end{pmatrix}
\end{equation*}

\item Un átomo de Hidrógeno en su estado básico tiene la densidad de carga electrónica
\begin{equation*}
\rho(\vec{r}) = -\frac{e}{\pi a_0^3}\exp\left[-\frac{2r}{a_0}\right],
\end{equation*}
siendo $e$ el valor absoluto de la carga del electrón y $a_0$ el radio de Bohr. El núcleo se supone puntual y está localizado en el origen. Verifique que
\begin{equation*}
\int_{\mathbb{R}^3}\rho(\vec{r})dV = -e.
\end{equation*}
Halle el potencial escalar y el vector intensidad del campo eléctrico en todo el espacio. Calcule la energía de interacción electrón-núcleo.

\item En cierto estado excitado del átomo de Hidrógeno, la nube electrónica tiene la densidad de carga, escrita en coordenadas esféricas,
\begin{equation*}
\rho(\vec{r}) = -\frac{e}{4\pi^3 8a_0^7}r^4\exp\left[-\frac{2r}{3a_0}\right]\sin^4(\theta).
\end{equation*}
Demuestre que el momento dipolar eléctrico del átomo de Hidrógeno asociado a este estado excitado es cero. Demuestre además que el momento cuadrupolar eléctrico en dicho estado tiene la forma
\begin{equation*}
(D_{ij}) = 36ea_0^2 \begin{pmatrix}
-1 & 0 & 0 \\
0 & -1 & 0 \\
0 & 0 & 2
\end{pmatrix}
\end{equation*}

\item En el problema anterior, calcule la energía de interacción entre el núcleo positivo y la nube electrónica del átomo en el estado dado, utilizando tanto la definición (1.132) como el desarrollo en multipolos (1.135).
\end{enumerate}

\chapter{Magnetostática del vacío}
\section{Principios básicos de la Magnetostática}

La magnetostática es la rama de la electrodinámica que estudia los campos magnéticos producidos por corrientes eléctricas estacionarias. Al igual que la electrostática, la magnetostática se basa en un conjunto de principios fundamentales derivados de las ecuaciones de Maxwell en el caso estático.

Para una distribución de corriente estacionaria, las ecuaciones de Maxwell que describen el campo magnético son:
\begin{align}
\nabla \times \vec{B} &= \mu_0 \vec{J},\\
\nabla \cdot \vec{B} &= 0,
\end{align}
donde $\vec{B}$ es el campo magnético (o inducción magnética), $\vec{J}$ es la densidad de corriente, y $\mu_0 = 4\pi \times 10^{-7} \, \textrm{H/m}$ es la permeabilidad del vacío.

La primera ecuación, conocida como ley de Ampère, establece que las corrientes eléctricas son las fuentes del campo magnético. La segunda ecuación expresa la ausencia de monopolos magnéticos, es decir, el campo magnético es solenoidal o tiene divergencia nula.

\subsection{Ley de fuerzas de Ampère y ley de Biot-Savart}

La ley de fuerzas de Ampère describe la interacción magnética entre dos elementos de corriente. Si consideramos dos elementos de corriente $I_1 d\vec{l}_1$ e $I_2 d\vec{l}_2$ separados por un vector $\vec{r}$ (dirigido del primer elemento al segundo), la fuerza magnética que ejerce el primer elemento sobre el segundo viene dada por
\begin{equation}
d\vec{F}_{21} = \frac{\mu_0}{4\pi} \frac{I_1 I_2}{r^2} [d\vec{l}_1 \times (d\vec{l}_2 \times \hat{r})],
\end{equation}
donde $\hat{r} = \vec{r}/r$ es el vector unitario en la dirección de $\vec{r}$.

Esta fuerza se puede reescribir utilizando identidades vectoriales como
\begin{equation}
d\vec{F}_{21} = \frac{\mu_0}{4\pi} I_1 I_2 \left[ \frac{(d\vec{l}_1 \cdot d\vec{l}_2)}{r^2} \hat{r} - \frac{(d\vec{l}_1 \cdot \hat{r})(d\vec{l}_2 \cdot \hat{r})}{r^2} \hat{r} \right].
\end{equation}

La ley de Biot-Savart proporciona el campo magnético producido por un elemento de corriente. Si un elemento de corriente $I d\vec{l}$ está situado en el origen, el campo magnético que produce en un punto $\vec{r}$ es
\begin{equation}
d\vec{B} = \frac{\mu_0}{4\pi} \frac{I d\vec{l} \times \hat{r}}{r^2}.
\end{equation}

Para una distribución de corriente general, el campo magnético se obtiene integrando las contribuciones de todos los elementos de corriente:
\begin{equation}
\vec{B}(\vec{r}) = \frac{\mu_0}{4\pi} \int \frac{\vec{J}(\vec{r}') \times (\vec{r} - \vec{r}')}{|\vec{r} - \vec{r}'|^3} dV',
\end{equation}
donde la integral se extiende sobre todo el volumen donde existe la densidad de corriente $\vec{J}$.

Para un circuito filiforme con corriente $I$, esta expresión se simplifica a
\begin{equation}
\vec{B}(\vec{r}) = \frac{\mu_0 I}{4\pi} \oint \frac{d\vec{l}' \times (\vec{r} - \vec{r}')}{|\vec{r} - \vec{r}'|^3},
\end{equation}
donde la integral se realiza a lo largo del circuito.

\subsection{Campo magnético y densidad de flujo magnético}

En el vacío, el campo magnético $\vec{B}$ y la intensidad de campo magnético $\vec{H}$ están relacionados por
\begin{equation}
\vec{B} = \mu_0 \vec{H}.
\end{equation}

En el contexto de la magnetostática, $\vec{B}$ se conoce a menudo como densidad de flujo magnético o inducción magnética, mientras que $\vec{H}$ se denomina intensidad de campo magnético. En el vacío, estas cantidades difieren solo por un factor constante $\mu_0$, pero en medios materiales su relación es más compleja.

El flujo magnético $\Phi_B$ a través de una superficie $S$ se define como
\begin{equation}
\Phi_B = \int_S \vec{B} \cdot d\vec{S}.
\end{equation}

Debido a que $\nabla \cdot \vec{B} = 0$, el flujo magnético neto a través de cualquier superficie cerrada es cero (ley de Gauss para el magnetismo):
\begin{equation}
\oint_S \vec{B} \cdot d\vec{S} = 0.
\end{equation}

Esta ley refleja la ausencia de monopolos magnéticos en la naturaleza.

\section{Fuerza de Lorentz}

La fuerza de Lorentz es la fuerza experimentada por una partícula cargada que se mueve en un campo electromagnético. En presencia de un campo eléctrico $\vec{E}$ y un campo magnético $\vec{B}$, la fuerza sobre una partícula con carga $q$ y velocidad $\vec{v}$ viene dada por
\begin{equation}
\vec{F} = q(\vec{E} + \vec{v} \times \vec{B}).
\end{equation}

En el contexto de la magnetostática, donde los campos eléctricos y magnéticos son estáticos ($\partial \vec{E}/\partial t = 0$ y $\partial \vec{B}/\partial t = 0$), esta fuerza se divide naturalmente en dos componentes: la fuerza eléctrica $q\vec{E}$ y la fuerza magnética $q\vec{v} \times \vec{B}$.

\subsection{Partícula cargada en presencia de un campo magnético}

Consideremos una partícula de carga $q$ y masa $m$ moviéndose con velocidad $\vec{v}$ en un campo magnético uniforme $\vec{B}$. La ecuación de movimiento de la partícula es
\begin{equation}
m\frac{d\vec{v}}{dt} = q\vec{v} \times \vec{B}.
\end{equation}

Varias propiedades importantes se derivan de esta ecuación:

1. La fuerza magnética es siempre perpendicular a la velocidad de la partícula, por lo que no realiza trabajo sobre ella. En consecuencia, la energía cinética de la partícula permanece constante: $\frac{d}{dt}(\frac{1}{2}mv^2) = 0$.

2. La componente de la velocidad paralela al campo magnético, $v_{\parallel}$, no se ve afectada por la fuerza magnética, ya que $\vec{v}_{\parallel} \times \vec{B} = 0$. Por lo tanto, $v_{\parallel}$ es constante.

3. La componente de la velocidad perpendicular al campo magnético, $\vec{v}_{\perp}$, experimenta una fuerza que es perpendicular tanto a $\vec{v}_{\perp}$ como a $\vec{B}$. Esto da lugar a un movimiento circular en el plano perpendicular a $\vec{B}$.

La frecuencia angular de este movimiento circular, conocida como frecuencia ciclotrónica o frecuencia de Larmor, es
\begin{equation}
\omega_c = \frac{|q|B}{m}.
\end{equation}

El radio del círculo, llamado radio de Larmor o radio ciclotrónico, es
\begin{equation}
r_L = \frac{mv_{\perp}}{|q|B}.
\end{equation}

Combinando el movimiento circular en el plano perpendicular a $\vec{B}$ con el movimiento uniforme en la dirección de $\vec{B}$, la trayectoria resultante es una hélice con eje paralelo al campo magnético.

\subsection{Fuerza magnética en presencia de un campo magnético externo}

La fuerza magnética sobre una corriente eléctrica en presencia de un campo magnético externo se puede derivar considerando el efecto del campo sobre los portadores de carga en movimiento que constituyen la corriente.

Para un elemento de corriente $I d\vec{l}$ en un campo magnético externo $\vec{B}$, la fuerza magnética es
\begin{equation}
d\vec{F} = I d\vec{l} \times \vec{B}.
\end{equation}

Esta expresión, conocida como ley de Laplace, es consistente con la ley de Lorentz aplicada a los portadores de carga que forman la corriente.

Para un circuito cerrado en un campo magnético, la fuerza total es
\begin{equation}
\vec{F} = I \oint d\vec{l} \times \vec{B}.
\end{equation}

Si el campo magnético es uniforme, esta fuerza puede expresarse en términos del momento magnético del circuito. Para un circuito plano, el momento magnético es $\vec{\mu} = I\vec{S}$, donde $\vec{S}$ es el vector área del circuito. La fuerza sobre el circuito en un campo magnético uniforme es entonces
\begin{equation}
\vec{F} = \nabla(\vec{\mu} \cdot \vec{B}),
\end{equation}
donde el gradiente actúa sobre las coordenadas del circuito.

En el caso general de una distribución de corriente con densidad $\vec{J}$ en un campo magnético $\vec{B}$, la fuerza por unidad de volumen es
\begin{equation}
\vec{f} = \vec{J} \times \vec{B}.
\end{equation}

\section{Desarrollo en multipolos para el campo magnético}

De manera similar al desarrollo multipolar en electrostática, el campo magnético producido por una distribución de corriente localizada puede expresarse como una serie de términos que representan contribuciones de complejidad creciente: monopolo, dipolo, cuadrupolo, etc.

El potencial vector magnético $\vec{A}$ para una distribución de corriente localizada, a grandes distancias, puede desarrollarse como
\begin{equation}
\vec{A}(\vec{r}) = \frac{\mu_0}{4\pi} \left[ \frac{\vec{\mu} \times \hat{r}}{r^2} + \frac{1}{r^3} \sum_{i,j} Q_{ij} \hat{r}_j \hat{\epsilon}_i + \ldots \right],
\end{equation}
donde $\vec{\mu}$ es el momento dipolar magnético y $Q_{ij}$ son las componentes del tensor cuadrupolar magnético.

Es importante notar que, debido a que $\nabla \cdot \vec{B} = 0$ (ausencia de monopolos magnéticos), el término monopolar está ausente en este desarrollo. El primer término no nulo es el dipolar.

El campo magnético correspondiente se obtiene mediante $\vec{B} = \nabla \times \vec{A}$. A grandes distancias, el término dominante es el dipolar:
\begin{equation}
\vec{B}(\vec{r}) \approx \frac{\mu_0}{4\pi} \left[ \frac{3(\vec{\mu} \cdot \hat{r})\hat{r} - \vec{\mu}}{r^3} \right].
\end{equation}

Esta expresión es formalmente idéntica al campo eléctrico de un dipolo eléctrico, con el momento dipolar eléctrico $\vec{p}$ reemplazado por el momento dipolar magnético $\vec{\mu}$.

\subsection{Momentum magnético de un sistema de partículas puntuales}

El momento magnético (o momento dipolar magnético) es una medida de la intensidad y orientación del campo magnético producido por una distribución de corriente.

Para una partícula cargada en movimiento, el momento magnético asociado a su movimiento orbital alrededor de un punto de referencia es
\begin{equation}
\vec{\mu} = \frac{q}{2} \vec{r} \times \vec{v},
\end{equation}
donde $q$ es la carga de la partícula, $\vec{r}$ es el vector de posición respecto al punto de referencia, y $\vec{v}$ es la velocidad.

Esta expresión puede reescribirse en términos del momento angular orbital $\vec{L} = m\vec{r} \times \vec{v}$ como
\begin{equation}
\vec{\mu} = \frac{q}{2m} \vec{L}.
\end{equation}

La relación $\gamma = q/(2m)$ se conoce como razón giromagnética o factor giromagnético.

Para un sistema de partículas cargadas, el momento magnético total es la suma de las contribuciones individuales:
\begin{equation}
\vec{\mu} = \sum_i \vec{\mu}_i = \sum_i \frac{q_i}{2} \vec{r}_i \times \vec{v}_i.
\end{equation}

En el caso de una distribución continua de corriente, el momento magnético puede calcularse como
\begin{equation}
\vec{\mu} = \frac{1}{2} \int \vec{r} \times \vec{J}(\vec{r}) \, dV,
\end{equation}
donde $\vec{J}$ es la densidad de corriente.

Para un circuito filiforme plano con corriente $I$, el momento magnético es simplemente
\begin{equation}
\vec{\mu} = I \vec{S},
\end{equation}
donde $\vec{S}$ es el vector de área del circuito.

\section{Energía de interacción entre una corriente y un campo magnético exterior a ella}

La energía de interacción entre una distribución de corriente y un campo magnético externo puede calcularse a partir del trabajo necesario para establecer la corriente en presencia del campo, o equivalentemente, para mover la corriente desde una región sin campo hasta la región donde existe el campo.

Para una distribución de corriente con densidad $\vec{J}(\vec{r})$ en presencia de un campo magnético externo $\vec{B}_{ext}(\vec{r})$, la energía de interacción puede expresarse como
\begin{equation}
U = -\int \vec{J}(\vec{r}) \cdot \vec{A}_{ext}(\vec{r}) \, dV,
\end{equation}
donde $\vec{A}_{ext}$ es el potencial vector asociado al campo magnético externo.

Para un circuito filiforme con corriente $I$, esta expresión se reduce a
\begin{equation}
U = -I \oint \vec{A}_{ext} \cdot d\vec{l},
\end{equation}
donde la integral se realiza a lo largo del circuito.

\subsection{Energía de interacción en la aproximación de dipolo}

Si el tamaño característico de la distribución de corriente es pequeño en comparación con la escala de variación espacial del campo magnético externo, podemos aproximar la energía de interacción utilizando el momento magnético de la distribución.

Expandiendo el potencial vector $\vec{A}_{ext}$ alrededor de un punto de referencia $\vec{r}_0$ dentro de la distribución de corriente:
\begin{equation}
\vec{A}_{ext}(\vec{r}) \approx \vec{A}_{ext}(\vec{r}_0) + (\vec{r} - \vec{r}_0) \cdot \nabla \vec{A}_{ext}|_{\vec{r}_0} + \ldots
\end{equation}

Sustituyendo esta expansión en la expresión para la energía y considerando solo el primer término no nulo, obtenemos:
\begin{equation}
U \approx -\vec{\mu} \cdot \vec{B}_{ext}(\vec{r}_0),
\end{equation}
donde $\vec{\mu}$ es el momento magnético de la distribución de corriente.

Esta expresión muestra que, en la aproximación dipolar, la energía de interacción es directamente proporcional al producto escalar del momento magnético y el campo magnético externo. La configuración de mínima energía corresponde a la alineación del momento magnético con el campo externo.

\section{Dipolo magnético en presencia de un campo magnético externo}

Un dipolo magnético en presencia de un campo magnético externo experimenta tanto un torque como una fuerza, dependiendo de la uniformidad del campo.

El torque sobre un dipolo magnético $\vec{\mu}$ en un campo magnético $\vec{B}$ viene dado por
\begin{equation}
\vec{\tau} = \vec{\mu} \times \vec{B}.
\end{equation}

Este torque tiende a alinear el dipolo magnético con la dirección del campo externo. La energía potencial asociada a esta interacción es, como vimos en la sección anterior,
\begin{equation}
U = -\vec{\mu} \cdot \vec{B}.
\end{equation}

Si el campo magnético no es uniforme, el dipolo también experimenta una fuerza neta dada por
\begin{equation}
\vec{F} = \nabla(\vec{\mu} \cdot \vec{B}),
\end{equation}
donde el gradiente actúa sobre las coordenadas del dipolo.

Para un dipolo con momento magnético constante, esta fuerza puede expresarse como
\begin{equation}
\vec{F} = (\vec{\mu} \cdot \nabla) \vec{B}.
\end{equation}

En componentes cartesianas,
\begin{equation}
F_i = \mu_j \frac{\partial B_i}{\partial x_j},
\end{equation}
donde se utiliza la convención de suma sobre índices repetidos.

Esta fuerza tiende a mover el dipolo hacia regiones donde el campo magnético es más intenso, siempre que el dipolo tenga una componente paralela al gradiente del campo.

\section{Magnetostática de Lorentz}

La magnetostática de Lorentz se refiere al estudio del comportamiento de un dipolo magnético en presencia de un campo magnético externo, considerando los efectos de la fuerza y el torque que actúan sobre él. Este análisis es particularmente relevante para entender fenómenos como la precesión de un dipolo magnético y el magnetismo de los materiales.

Un aspecto importante de la magnetostática de Lorentz es la ecuación de movimiento para un dipolo magnético en un campo externo. Si consideramos un dipolo con momento magnético $\vec{\mu}$ y momento angular intrínseco $\vec{L}$ relacionados por $\vec{\mu} = \gamma \vec{L}$, donde $\gamma$ es la razón giromagnética, entonces la ecuación de movimiento es:
\begin{equation}
\frac{d\vec{\mu}}{dt} = \gamma \vec{\mu} \times \vec{B}.
\end{equation}

Esta ecuación describe la precesión del momento magnético alrededor de la dirección del campo magnético, un fenómeno conocido como precesión de Larmor.

\subsection{Precesión de un dipolo en un campo magnético uniforme y constante}

Consideremos un dipolo magnético con momento $\vec{\mu}$ en un campo magnético uniforme y constante $\vec{B}$. De acuerdo con la ecuación de movimiento, el dipolo experimenta un torque $\vec{\tau} = \vec{\mu} \times \vec{B}$ que causa una variación en su momento angular.

Si el momento magnético y el momento angular están relacionados por $\vec{\mu} = \gamma \vec{L}$, donde $\gamma$ es la razón giromagnética, entonces
\begin{equation}
\frac{d\vec{\mu}}{dt} = \gamma \frac{d\vec{L}}{dt} = \gamma \vec{\tau} = \gamma \vec{\mu} \times \vec{B}.
\end{equation}

Esta ecuación describe un movimiento de precesión del vector $\vec{\mu}$ alrededor de la dirección del campo $\vec{B}$. La frecuencia angular de esta precesión, conocida como frecuencia de Larmor, es
\begin{equation}
\omega_L = \gamma B.
\end{equation}

Durante la precesión, el ángulo entre $\vec{\mu}$ y $\vec{B}$ permanece constante, y la componente de $\vec{\mu}$ en la dirección de $\vec{B}$ no cambia. La energía del dipolo $U = -\vec{\mu} \cdot \vec{B}$ también se mantiene constante durante la precesión.

La precesión de Larmor es un fenómeno fundamental en muchos contextos físicos, incluyendo la resonancia magnética nuclear (RMN) y la resonancia paramagnética electrónica (RPE), donde se utiliza para manipular y medir propiedades de espines nucleares y electrónicos.

En el caso de una partícula cargada orbitando en un campo magnético, la precesión de Larmor se manifiesta como una precesión de la órbita alrededor de la dirección del campo, con la misma frecuencia angular $\omega_L = qB/(2m)$.

\section{Problemas propuestos}

\begin{enumerate}
\item Por un cilindro muy largo de radio $a$, tan largo que puede ser considerado infinito, circula una corriente constante de densidad $\vec{J}$. Calcule el valor del vector inducción magnética $\vec{B}$ en el exterior del cilindro.

\item Describa el movimiento de una partícula no relativista de carga $q$ y masa $m_0$ en presencia de un campo magnético de inducción $\vec{B}$ y de un campo eléctrico de intensidad $\vec{E}$, ambos uniformes y constantes.

\item Demuestre que el potencial magnético vectorial de un campo magnético uniforme y constante de inducción $\vec{B}$ viene dado por la expresión
\begin{equation*}
\vec{A} = \frac{1}{2}\vec{B} \times \vec{r}.
\end{equation*}

\item Sea una bola de radio $a$ sobre cuyo volumen se encuentra distribuida uniformemente la carga $Q$. Si la esfera rota con velocidad constante $\omega$ alrededor de uno de sus diámetros, calcule el momento dipolar de la bola. Calcule además el vector inducción magnética en el centro de la bola.

\item Suponiendo que la bola que rota en el problema anterior es el electrón no puntual con radio clásico $r_0$ determinado en la sección 1.8 del Capítulo 1, encuentre cuál debe ser el valor de la velocidad angular $\omega$ para que el momento magnético $\vec{m}$ coincida con el momento magnético de spin del electrón. Halle la velocidad tangencial de un punto cualquiera sobre el ecuador de la bola. ¿El valor de velocidad obtenido puede ser admitido como válido?

\item Una partícula tiene el momento magnético $\vec{m}$. Demuestre que esta partícula tiene asociada la densidad de corriente $\vec{J} = (\vec{m} \times \nabla)\delta(\vec{r} - \vec{r}_0)$, siendo $\vec{r}_0$ el vector de posición de la partícula.

\item En cierto estado excitado del átomo de Hidrógeno la nube electrónica tiene asociada una densidad de corriente cuyas componentes en coordenadas esféricas son $J_r = J_\theta = 0$ y
\begin{equation*}
J_\varphi = \frac{e\hbar}{2\pi^3 8m_0 a_0^7}r^3\exp\left[-\frac{2r}{3a_0}\right]\sin^3(\theta),
\end{equation*}
siendo $e$ y $m_0$ la carga y la masa de reposo del electrón, respectivamente, y $a_0$ es el radio de Bohr. Calcule el valor del vector inducción magnética $\vec{B}$ en el origen de coordenadas.

\item En otro estado excitado del átomo de Hidrógeno la densidad de corriente asociada a la nube electrónica es, en coordenadas esféricas, $J_r = J_\theta = 0$ y
\begin{equation*}
J_\varphi = \frac{e\hbar}{64\pi m_0 a_0^5}r\exp\left[-\frac{r}{a_0}\right]\sin(\theta).
\end{equation*}
Suponga que el núcleo del átomo tiene un momento magnético $\vec{m}_n$. Calcule la energía de interacción electrón-núcleo asociada con el movimiento orbital del electrón. Ignore el valor del momento magnético de spin del electrón.
\end{enumerate}

\chapter{Ecuaciones de Maxwell}
\section{Ley de conservación de carga}

La ley de conservación de carga es uno de los principios fundamentales de la física. Establece que la carga eléctrica total de un sistema aislado se mantiene constante a lo largo del tiempo. En forma diferencial, esta ley puede expresarse como:
\begin{equation}
\frac{\partial \rho}{\partial t} + \nabla \cdot \vec{J} = 0,
\end{equation}
donde $\rho$ es la densidad de carga y $\vec{J}$ es la densidad de corriente.

Esta ecuación, conocida como ecuación de continuidad, relaciona la variación temporal de la densidad de carga en un punto con la divergencia de la densidad de corriente en ese mismo punto. Físicamente, indica que si la carga disminuye en una región (término $\partial \rho / \partial t$ negativo), debe haber un flujo neto de carga saliendo de esa región (término $\nabla \cdot \vec{J}$ positivo), y viceversa.

En forma integral, para un volumen $V$ limitado por una superficie cerrada $S$, la ley de conservación de carga se expresa como:
\begin{equation}
\frac{d}{dt} \int_V \rho \, dV = -\oint_S \vec{J} \cdot d\vec{S},
\end{equation}
que establece que la variación temporal de la carga total contenida en el volumen $V$ es igual al flujo neto de corriente a través de la superficie $S$ con signo negativo.

La ley de conservación de carga es consistente con las ecuaciones de Maxwell y, de hecho, puede derivarse a partir de ellas tomando la divergencia de la ley de Ampère-Maxwell y combinándola con la ley de Gauss para el campo eléctrico.

\section{Ley de Lenz y ley de Faraday}

La ley de Faraday, formulada por Michael Faraday en 1831, es una de las leyes fundamentales del electromagnetismo. Establece que un campo magnético variable en el tiempo induce una fuerza electromotriz (fem) en un circuito conductor. La magnitud de esta fem es proporcional a la tasa de variación del flujo magnético a través del circuito.

Matemáticamente, la ley de Faraday se expresa como:
\begin{equation}
\mathcal{E} = -\frac{d\Phi_B}{dt},
\end{equation}
donde $\mathcal{E}$ es la fuerza electromotriz inducida y $\Phi_B$ es el flujo magnético a través del circuito.

El signo negativo en esta ecuación es consecuencia de la ley de Lenz, que establece que la corriente inducida en un circuito siempre fluye en una dirección tal que crea un campo magnético que se opone al cambio de flujo magnético que la produjo. Esta ley es una manifestación del principio de conservación de la energía.

En forma diferencial, la ley de Faraday se expresa como una de las ecuaciones de Maxwell:
\begin{equation}
\nabla \times \vec{E} = -\frac{\partial \vec{B}}{\partial t},
\end{equation}
que relaciona el rotacional del campo eléctrico con la variación temporal del campo magnético.

En forma integral, para una superficie $S$ limitada por un contorno cerrado $C$, la ley de Faraday-Lenz se expresa como:
\begin{equation}
\oint_C \vec{E} \cdot d\vec{l} = -\frac{d}{dt} \int_S \vec{B} \cdot d\vec{S}.
\end{equation}

Esta ley tiene numerosas aplicaciones prácticas, como en la generación de electricidad en generadores y alternadores, y en la inducción electromagnética utilizada en transformadores y motores eléctricos.

\section{Ley de Ampère-Maxwell}

La ley de Ampère, formulada inicialmente para la magnetostática, establece que la circulación del campo magnético a lo largo de un contorno cerrado es proporcional a la corriente eléctrica que atraviesa la superficie limitada por dicho contorno. Sin embargo, esta ley es incompleta cuando se consideran campos variables en el tiempo.

James Clerk Maxwell modificó la ley de Ampère añadiendo un término adicional, conocido como corriente de desplazamiento, que involucra la variación temporal del campo eléctrico. La ley de Ampère-Maxwell completa se expresa en forma diferencial como:
\begin{equation}
\nabla \times \vec{B} = \mu_0 \vec{J} + \mu_0 \epsilon_0 \frac{\partial \vec{E}}{\partial t},
\end{equation}
donde el término $\mu_0 \epsilon_0 \frac{\partial \vec{E}}{\partial t}$ representa la densidad de corriente de desplazamiento.

En forma integral, para un contorno cerrado $C$ que limita una superficie $S$, la ley de Ampère-Maxwell se expresa como:
\begin{equation}
\oint_C \vec{B} \cdot d\vec{l} = \mu_0 \int_S \vec{J} \cdot d\vec{S} + \mu_0 \epsilon_0 \int_S \frac{\partial \vec{E}}{\partial t} \cdot d\vec{S}.
\end{equation}

La inclusión de la corriente de desplazamiento fue una de las contribuciones más importantes de Maxwell a la electrodinámica. Este término garantiza la conservación de la carga y permite la existencia de ondas electromagnéticas, como veremos más adelante.

La importancia de la corriente de desplazamiento se hace evidente en situaciones donde las corrientes de conducción no son continuas, como en un condensador que se está cargando o descargando. En tales casos, la corriente de desplazamiento proporciona la continuidad necesaria para la corriente total.

\section{Interpretación física de las ecuaciones de Maxwell}

Las ecuaciones de Maxwell constituyen un conjunto completo de ecuaciones que describen todos los fenómenos electromagnéticos clásicos. Resumidas en forma diferencial, estas ecuaciones son:

\begin{align}
\nabla \cdot \vec{E} &= \frac{\rho}{\epsilon_0} \quad \text{(Ley de Gauss para el campo eléctrico)} \\
\nabla \cdot \vec{B} &= 0 \quad \text{(Ley de Gauss para el campo magnético)} \\
\nabla \times \vec{E} &= -\frac{\partial \vec{B}}{\partial t} \quad \text{(Ley de Faraday)} \\
\nabla \times \vec{B} &= \mu_0 \vec{J} + \mu_0 \epsilon_0 \frac{\partial \vec{E}}{\partial t} \quad \text{(Ley de Ampère-Maxwell)}
\end{align}

Interpretación física de cada ecuación:

1. La ley de Gauss para el campo eléctrico establece que las cargas eléctricas son las fuentes del campo eléctrico. El flujo eléctrico que emerge de una superficie cerrada es proporcional a la carga encerrada dentro de ella.

2. La ley de Gauss para el campo magnético expresa la inexistencia de monopolos magnéticos. El flujo magnético a través de cualquier superficie cerrada es siempre cero, lo que implica que las líneas de campo magnético siempre forman lazos cerrados.

3. La ley de Faraday describe cómo un campo magnético variable en el tiempo induce un campo eléctrico. Esta ley es fundamental para entender la inducción electromagnética y su aplicación en generadores eléctricos y transformadores.

4. La ley de Ampère-Maxwell relaciona el campo magnético con sus fuentes: las corrientes eléctricas y los campos eléctricos variables en el tiempo. La inclusión del término de corriente de desplazamiento por Maxwell completó esta ley.

Estas ecuaciones no solo describen cómo los campos eléctricos y magnéticos son generados por cargas y corrientes, sino también cómo interactúan entre sí. Un campo eléctrico variable genera un campo magnético, y viceversa, lo que conduce a la existencia de ondas electromagnéticas que pueden propagarse incluso en el vacío.

Una consecuencia fundamental de las ecuaciones de Maxwell es la predicción de la velocidad de la luz. Combinando estas ecuaciones se puede demostrar que las ondas electromagnéticas se propagan a una velocidad $c = 1/\sqrt{\mu_0\epsilon_0}$, que es precisamente la velocidad de la luz en el vacío.

Las ecuaciones de Maxwell representan una unificación de los fenómenos eléctricos y magnéticos, demostrando que ambos son manifestaciones de un único fenómeno: el electromagnetismo.

\section{Potenciales del campo electromagnético}

Los campos electromagnéticos $\vec{E}$ y $\vec{B}$ pueden derivarse de potenciales, lo que simplifica considerablemente la resolución de problemas en electrodinámica. Dado que $\nabla \cdot \vec{B} = 0$, el campo magnético puede expresarse como el rotacional de un campo vectorial $\vec{A}$, denominado potencial vector magnético:
\begin{equation}
\vec{B} = \nabla \times \vec{A}.
\end{equation}

Sustituyendo esta expresión en la ley de Faraday, obtenemos:
\begin{equation}
\nabla \times \vec{E} = -\frac{\partial}{\partial t}(\nabla \times \vec{A}) = -\nabla \times \frac{\partial \vec{A}}{\partial t}.
\end{equation}

Esto implica que:
\begin{equation}
\nabla \times \left(\vec{E} + \frac{\partial \vec{A}}{\partial t}\right) = 0.
\end{equation}

Como el rotacional de un gradiente es siempre cero, podemos expresar:
\begin{equation}
\vec{E} + \frac{\partial \vec{A}}{\partial t} = -\nabla \phi,
\end{equation}
donde $\phi$ es un potencial escalar, análogo al potencial electrostático en el caso estático.

Por lo tanto, los campos electromagnéticos pueden expresarse en términos de los potenciales $\phi$ y $\vec{A}$ como:
\begin{align}
\vec{E} &= -\nabla \phi - \frac{\partial \vec{A}}{\partial t}, \\
\vec{B} &= \nabla \times \vec{A}.
\end{align}

Estos potenciales no son únicos, ya que podemos realizar ciertas transformaciones, conocidas como transformaciones de gauge, que modifican los potenciales pero dejan invariantes los campos físicos $\vec{E}$ y $\vec{B}$.

\subsection{Transformación gauge y condición de Lorentz}

Los potenciales electromagnéticos $\phi$ y $\vec{A}$ no están completamente determinados por los campos $\vec{E}$ y $\vec{B}$. Es posible realizar transformaciones de los potenciales que no afecten a los campos físicos. Estas transformaciones se conocen como transformaciones de gauge.

Consideremos la siguiente transformación:
\begin{align}
\vec{A}' &= \vec{A} + \nabla \Lambda, \\
\phi' &= \phi - \frac{\partial \Lambda}{\partial t},
\end{align}
donde $\Lambda(\vec{r}, t)$ es una función escalar arbitraria.

Es fácil verificar que los campos electromagnéticos permanecen invariantes bajo esta transformación:
\begin{align}
\vec{B}' &= \nabla \times \vec{A}' = \nabla \times (\vec{A} + \nabla \Lambda) = \nabla \times \vec{A} = \vec{B}, \\
\vec{E}' &= -\nabla \phi' - \frac{\partial \vec{A}'}{\partial t} = -\nabla \left(\phi - \frac{\partial \Lambda}{\partial t}\right) - \frac{\partial}{\partial t}\left(\vec{A} + \nabla \Lambda\right) \\
&= -\nabla \phi + \nabla \frac{\partial \Lambda}{\partial t} - \frac{\partial \vec{A}}{\partial t} - \frac{\partial}{\partial t}\nabla \Lambda \\
&= -\nabla \phi - \frac{\partial \vec{A}}{\partial t} = \vec{E}.
\end{align}

Esta libertad de gauge permite imponer condiciones adicionales sobre los potenciales para simplificar las ecuaciones que los gobiernan. Una elección particularmente útil es la condición de Lorentz:
\begin{equation}
\nabla \cdot \vec{A} + \mu_0 \epsilon_0 \frac{\partial \phi}{\partial t} = 0.
\end{equation}

La condición de Lorentz no restringe los fenómenos físicos, sino que simplemente corresponde a una elección conveniente del gauge. Esta condición es covariante bajo transformaciones de Lorentz, lo que la hace particularmente útil en la teoría de la relatividad especial.

\subsection{Ecuaciones de D'Alembert para los potenciales del campo}

Utilizando la condición de Lorentz, podemos obtener ecuaciones de onda para los potenciales electromagnéticos. Partimos de las ecuaciones de Maxwell en términos de los potenciales:
\begin{align}
\nabla \cdot \vec{E} &= \frac{\rho}{\epsilon_0}, \\
\nabla \times \vec{B} &= \mu_0 \vec{J} + \mu_0 \epsilon_0 \frac{\partial \vec{E}}{\partial t}.
\end{align}

Sustituyendo $\vec{E} = -\nabla \phi - \frac{\partial \vec{A}}{\partial t}$ y $\vec{B} = \nabla \times \vec{A}$, y aplicando la condición de Lorentz, obtenemos después de algunas manipulaciones algebraicas:
\begin{align}
\nabla^2 \phi - \mu_0 \epsilon_0 \frac{\partial^2 \phi}{\partial t^2} &= -\frac{\rho}{\epsilon_0}, \\
\nabla^2 \vec{A} - \mu_0 \epsilon_0 \frac{\partial^2 \vec{A}}{\partial t^2} &= -\mu_0 \vec{J}.
\end{align}

Estas son las ecuaciones de D'Alembert o ecuaciones de onda inhomogéneas para los potenciales electromagnéticos. En forma compacta, podemos escribirlas como:
\begin{align}
\Box \phi &= -\frac{\rho}{\epsilon_0}, \\
\Box \vec{A} &= -\mu_0 \vec{J},
\end{align}
donde $\Box = \nabla^2 - \frac{1}{c^2}\frac{\partial^2}{\partial t^2}$ es el operador de D'Alembert, y $c = \frac{1}{\sqrt{\mu_0 \epsilon_0}}$ es la velocidad de la luz en el vacío.

Las soluciones de estas ecuaciones son los potenciales retardados, que tienen en cuenta el tiempo finito que tarda la influencia electromagnética en propagarse desde la fuente hasta el punto de observación.

En el caso estático, donde $\frac{\partial}{\partial t} = 0$, estas ecuaciones se reducen a las ecuaciones de Poisson para los potenciales:
\begin{align}
\nabla^2 \phi &= -\frac{\rho}{\epsilon_0}, \\
\nabla^2 \vec{A} &= -\mu_0 \vec{J}.
\end{align}

La simplificación de las ecuaciones de Maxwell a ecuaciones de onda para los potenciales es una de las ventajas más importantes del uso de potenciales en electrodinámica.

\section{Ley de conservación de la energía del campo electromagnético}

La energía es una cantidad conservada en electrodinámica, y las ecuaciones de Maxwell permiten derivar una ley de conservación de la energía electromagnética.

Partimos de la ley de Ampère-Maxwell y la ley de Faraday:
\begin{align}
\nabla \times \vec{B} &= \mu_0 \vec{J} + \mu_0 \epsilon_0 \frac{\partial \vec{E}}{\partial t}, \\
\nabla \times \vec{E} &= -\frac{\partial \vec{B}}{\partial t}.
\end{align}

Multiplicando escalarmente la primera ecuación por $\vec{E}$ y la segunda por $\vec{B}/\mu_0$, tenemos:
\begin{align}
\vec{E} \cdot (\nabla \times \vec{B}) &= \mu_0 \vec{E} \cdot \vec{J} + \mu_0 \epsilon_0 \vec{E} \cdot \frac{\partial \vec{E}}{\partial t}, \\
\frac{\vec{B}}{\mu_0} \cdot (\nabla \times \vec{E}) &= -\frac{\vec{B}}{\mu_0} \cdot \frac{\partial \vec{B}}{\partial t}.
\end{align}

Restando la segunda ecuación de la primera y utilizando la identidad vectorial $\nabla \cdot (\vec{E} \times \vec{B}) = \vec{B} \cdot (\nabla \times \vec{E}) - \vec{E} \cdot (\nabla \times \vec{B})$, obtenemos:
\begin{equation}
-\nabla \cdot (\vec{E} \times \vec{B}) = \mu_0 \vec{E} \cdot \vec{J} + \mu_0 \epsilon_0 \vec{E} \cdot \frac{\partial \vec{E}}{\partial t} + \frac{\vec{B}}{\mu_0} \cdot \frac{\partial \vec{B}}{\partial t}.
\end{equation}

Los dos últimos términos pueden reescribirse como derivadas temporales:
\begin{equation}
\mu_0 \epsilon_0 \vec{E} \cdot \frac{\partial \vec{E}}{\partial t} = \frac{\partial}{\partial t}\left(\frac{1}{2}\epsilon_0 E^2\right), \quad \frac{\vec{B}}{\mu_0} \cdot \frac{\partial \vec{B}}{\partial t} = \frac{\partial}{\partial t}\left(\frac{1}{2\mu_0}B^2\right).
\end{equation}

Así, obtenemos la ecuación de conservación de la energía electromagnética:
\begin{equation}
\nabla \cdot \vec{S} + \frac{\partial u}{\partial t} = -\vec{J} \cdot \vec{E},
\end{equation}
donde:
\begin{align}
\vec{S} &= \frac{1}{\mu_0}\vec{E} \times \vec{B}, \\
u &= \frac{1}{2}\epsilon_0 E^2 + \frac{1}{2\mu_0}B^2.
\end{align}

Aquí, $\vec{S}$ es el vector de Poynting, que representa el flujo de energía electromagnética por unidad de área y tiempo, y $u$ es la densidad de energía electromagnética. El término $-\vec{J} \cdot \vec{E}$ representa la potencia por unidad de volumen transferida del campo a la materia (si es positivo) o de la materia al campo (si es negativo).
\section{Ley de conservación del momentum lineal del campo electromagnético}

Así como existe una ley de conservación para la energía electromagnética, también existe una ley de conservación para el momentum lineal. Esta ley se deriva de las ecuaciones de Maxwell y la expresión de la fuerza de Lorentz.

Consideremos la fuerza electromagnética sobre una distribución de carga con densidad $\rho$ y densidad de corriente $\vec{J}$:
\begin{equation}
\vec{f} = \rho \vec{E} + \vec{J} \times \vec{B}.
\end{equation}

Utilizando las ecuaciones de Maxwell, podemos reescribir esta expresión en términos de los campos. Después de algunas manipulaciones algebraicas, se obtiene:
\begin{equation}
\vec{f} = \epsilon_0 (\nabla \cdot \vec{E})\vec{E} + \frac{1}{\mu_0}(\nabla \times \vec{B}) \times \vec{B} - \epsilon_0 \frac{\partial}{\partial t}(\vec{E} \times \vec{B}).
\end{equation}

Esta expresión puede reformularse utilizando el tensor de tensiones de Maxwell $\mathbf{T}$, cuyas componentes en coordenadas cartesianas son:
\begin{equation}
T_{ij} = \epsilon_0 \left(E_i E_j - \frac{1}{2}\delta_{ij}E^2\right) + \frac{1}{\mu_0}\left(B_i B_j - \frac{1}{2}\delta_{ij}B^2\right),
\end{equation}
donde $\delta_{ij}$ es la delta de Kronecker.

La fuerza por unidad de volumen puede entonces expresarse como:
\begin{equation}
f_i = \sum_j \frac{\partial T_{ij}}{\partial x_j} - \epsilon_0 \mu_0 \frac{\partial}{\partial t}([\vec{E} \times \vec{B}]_i).
\end{equation}

Esta ecuación se puede interpretar como una ley de conservación del momentum, donde $\mathbf{T}$ representa el flujo de momentum electromagnético, y $\vec{g} = \epsilon_0 \mu_0 \vec{E} \times \vec{B} = \vec{S}/c^2$ es la densidad de momentum electromagnético.

En forma integral, para un volumen $V$ limitado por una superficie $S$, la ley de conservación del momentum se expresa como:
\begin{equation}
\frac{d}{dt}\int_V \vec{g} \, dV + \oint_S \mathbf{T} \cdot d\vec{S} = \int_V \vec{f} \, dV,
\end{equation}
donde $\mathbf{T} \cdot d\vec{S}$ representa el tensor de tensiones de Maxwell contraído con el vector de superficie.

Esta ecuación establece que la variación temporal del momentum electromagnético contenido en un volumen, más el flujo neto de momentum a través de la superficie que limita ese volumen, es igual a la fuerza total ejercida sobre la materia dentro del volumen.

El tensor de tensiones de Maxwell tiene una interpretación física clara: sus componentes diagonales representan presiones (fuerza por unidad de área) y sus componentes no diagonales representan tensiones de cizalladura en el "medio" electromagnético.

La ley de conservación del momentum electromagnético tiene importantes aplicaciones, como la explicación de la presión de radiación y el efecto de la transferencia de momentum en la interacción de la radiación con la materia.

\section{Ondas electromagnéticas}

Una de las consecuencias más importantes de las ecuaciones de Maxwell es la predicción de la existencia de ondas electromagnéticas que pueden propagarse incluso en el vacío. Estas ondas son perturbaciones de los campos eléctrico y magnético que se propagan a la velocidad de la luz.

Para derivar la ecuación de onda electromagnética, consideramos las ecuaciones de Maxwell en una región del espacio libre de cargas y corrientes ($\rho = 0$, $\vec{J} = 0$):
\begin{align}
\nabla \cdot \vec{E} &= 0, \\
\nabla \cdot \vec{B} &= 0, \\
\nabla \times \vec{E} &= -\frac{\partial \vec{B}}{\partial t}, \\
\nabla \times \vec{B} &= \mu_0 \epsilon_0 \frac{\partial \vec{E}}{\partial t}.
\end{align}

Tomando el rotacional de la tercera ecuación y utilizando la cuarta, obtenemos:
\begin{align}
\nabla \times (\nabla \times \vec{E}) &= -\frac{\partial}{\partial t}(\nabla \times \vec{B}) = -\mu_0 \epsilon_0 \frac{\partial^2 \vec{E}}{\partial t^2}.
\end{align}

Utilizando la identidad vectorial $\nabla \times (\nabla \times \vec{E}) = \nabla(\nabla \cdot \vec{E}) - \nabla^2\vec{E}$ y considerando que $\nabla \cdot \vec{E} = 0$, obtenemos la ecuación de onda para el campo eléctrico:
\begin{equation}
\nabla^2\vec{E} - \mu_0 \epsilon_0 \frac{\partial^2 \vec{E}}{\partial t^2} = 0.
\end{equation}

De manera similar, podemos obtener la ecuación de onda para el campo magnético:
\begin{equation}
\nabla^2\vec{B} - \mu_0 \epsilon_0 \frac{\partial^2 \vec{B}}{\partial t^2} = 0.
\end{equation}

Estas son ecuaciones de onda tridimensionales, donde la velocidad de propagación es $c = 1/\sqrt{\mu_0 \epsilon_0}$, que es exactamente la velocidad de la luz en el vacío.

\subsection{Ondas planas}

Una solución particular de las ecuaciones de onda electromagnética son las ondas planas, donde los campos eléctrico y magnético son uniformes en planos perpendiculares a la dirección de propagación. En una onda plana monocromática que se propaga en la dirección del vector unitario $\hat{n}$, los campos tienen la forma:
\begin{align}
\vec{E}(\vec{r}, t) &= \vec{E}_0 e^{i(\vec{k} \cdot \vec{r} - \omega t)}, \\
\vec{B}(\vec{r}, t) &= \vec{B}_0 e^{i(\vec{k} \cdot \vec{r} - \omega t)},
\end{align}
donde $\vec{k} = k\hat{n}$ es el vector de onda, con $k = \omega/c$ siendo el número de onda, y $\omega$ es la frecuencia angular de la onda.

Estas expresiones deben satisfacer las ecuaciones de Maxwell, lo que impone las siguientes condiciones:
\begin{align}
\vec{k} \cdot \vec{E}_0 &= 0, \\
\vec{k} \cdot \vec{B}_0 &= 0, \\
\vec{k} \times \vec{E}_0 &= \omega \vec{B}_0, \\
\vec{k} \times \vec{B}_0 &= -\frac{\omega}{c^2} \vec{E}_0.
\end{align}

Las dos primeras ecuaciones indican que los campos $\vec{E}$ y $\vec{B}$ son transversales a la dirección de propagación. Las dos últimas ecuaciones relacionan las amplitudes y direcciones de los campos eléctrico y magnético.

De estas relaciones se deduce que:
\begin{align}
\vec{B}_0 &= \frac{1}{c}\hat{n} \times \vec{E}_0, \\
|\vec{B}_0| &= \frac{1}{c}|\vec{E}_0|.
\end{align}

Es decir, en una onda plana electromagnética en el vacío:
1. Los campos $\vec{E}$, $\vec{B}$ y la dirección de propagación $\hat{n}$ son mutuamente perpendiculares, formando un triedro rectángulo.
2. La magnitud del campo magnético es $1/c$ veces la magnitud del campo eléctrico.
3. Los campos $\vec{E}$ y $\vec{B}$ oscilan en fase.

La solución general de las ecuaciones de onda se obtiene como superposición de ondas planas monocromáticas con diferentes frecuencias y direcciones de propagación.

\subsection{Densidad de energía y vector de Poynting de la onda plana}

La densidad de energía electromagnética en el vacío está dada por:
\begin{equation}
u = \frac{1}{2}\epsilon_0 E^2 + \frac{1}{2\mu_0}B^2.
\end{equation}

Para una onda plana, donde $B = E/c$, la densidad de energía se puede expresar como:
\begin{equation}
u = \frac{1}{2}\epsilon_0 E^2 + \frac{1}{2\mu_0}\frac{E^2}{c^2} = \frac{1}{2}\epsilon_0 E^2 + \frac{1}{2}\epsilon_0 E^2 = \epsilon_0 E^2,
\end{equation}
donde hemos utilizado que $\mu_0 \epsilon_0 = 1/c^2$.

Esto muestra que, en una onda plana, la energía se reparte equitativamente entre los campos eléctrico y magnético.

El vector de Poynting, que representa el flujo de energía electromagnética por unidad de área y tiempo, está dado por:
\begin{equation}
\vec{S} = \frac{1}{\mu_0}\vec{E} \times \vec{B}.
\end{equation}

Para una onda plana con $\vec{B} = \frac{1}{c}\hat{n} \times \vec{E}$, el vector de Poynting es:
\begin{equation}
\vec{S} = \frac{1}{\mu_0}\vec{E} \times \left(\frac{1}{c}\hat{n} \times \vec{E}\right).
\end{equation}

Utilizando la identidad vectorial $\vec{A} \times (\vec{B} \times \vec{C}) = \vec{B}(\vec{A} \cdot \vec{C}) - \vec{C}(\vec{A} \cdot \vec{B})$ y considerando que $\vec{E} \perp \hat{n}$, obtenemos:
\begin{equation}
\vec{S} = \frac{1}{\mu_0 c}E^2 \hat{n} = \epsilon_0 c E^2 \hat{n}.
\end{equation}

Comparando con la expresión de la densidad de energía, vemos que:
\begin{equation}
\vec{S} = u c \hat{n}.
\end{equation}

Esto significa que la energía de la onda electromagnética se propaga en la dirección $\hat{n}$ con velocidad $c$, como era de esperarse.

La intensidad de la onda, definida como la potencia media por unidad de área, se obtiene promediando el vector de Poynting sobre un período:
\begin{equation}
I = \langle |\vec{S}| \rangle = \frac{1}{2}\epsilon_0 c E_0^2,
\end{equation}
donde $E_0$ es la amplitud del campo eléctrico.

\subsection{Polarización de las ondas planas}

La polarización de una onda electromagnética describe la orientación del campo eléctrico en el plano perpendicular a la dirección de propagación. Como el campo eléctrico y el campo magnético son perpendiculares entre sí, conociendo la polarización del campo eléctrico se determina automáticamente la orientación del campo magnético.

Para una onda plana monocromática que se propaga en la dirección $z$, el campo eléctrico puede escribirse como:
\begin{equation}
\vec{E}(\vec{r}, t) = \vec{E}_0 e^{i(kz - \omega t)},
\end{equation}
donde $\vec{E}_0$ es un vector complejo que yace en el plano $xy$.

Existen varios tipos de polarización:

1. Polarización lineal: Cuando el campo eléctrico oscila a lo largo de una línea recta. En este caso, $\vec{E}_0$ es un vector real (o complejo con fase constante). Por ejemplo, $\vec{E}_0 = E_0 \hat{x}$ corresponde a una onda linealmente polarizada en la dirección $x$.

2. Polarización circular: Cuando la punta del vector campo eléctrico describe una circunferencia en el plano perpendicular a la dirección de propagación. Esto ocurre cuando las componentes $x$ e $y$ del campo eléctrico tienen igual amplitud pero difieren en fase en $\pm \pi/2$. Por ejemplo:
   \begin{equation}
   \vec{E}_0 = \frac{E_0}{\sqrt{2}}(\hat{x} \pm i\hat{y})
   \end{equation}
   corresponde a polarización circular derecha (+) o izquierda (-).

3. Polarización elíptica: Es el caso general, donde la punta del vector campo eléctrico describe una elipse en el plano perpendicular a la dirección de propagación.

La polarización de una onda electromagnética es importante en muchas aplicaciones, como en comunicaciones, óptica, y en el estudio de la interacción de la radiación con la materia.

Una onda no polarizada, como la luz natural, es una superposición de ondas con diferentes polarizaciones que varían rápidamente de manera aleatoria. Esta luz puede polarizarse mediante filtros polarizadores, que permiten el paso solo de la componente del campo eléctrico a lo largo de una dirección específica.

\section{Paquete de ondas}

Las ondas planas monocromáticas se extienden infinitamente en el espacio y el tiempo, lo que no corresponde a situaciones físicas reales. En la práctica, las perturbaciones electromagnéticas están localizadas en el espacio y tienen duración finita. Estas perturbaciones pueden representarse mediante paquetes de ondas, que son superposiciones de ondas planas monocromáticas con diferentes frecuencias y direcciones de propagación.

Un paquete de ondas unidimensional puede expresarse como:
\begin{equation}
\vec{E}(z, t) = \int_{-\infty}^{\infty} \vec{A}(k) e^{i(kz - \omega(k)t)} dk,
\end{equation}
donde $\vec{A}(k)$ es la amplitud espectral de la onda con número de onda $k$, y $\omega(k)$ es la relación de dispersión que relaciona la frecuencia angular con el número de onda.

En el vacío, la relación de dispersión para ondas electromagnéticas es lineal: $\omega(k) = ck$, donde $c$ es la velocidad de la luz. Esto significa que todas las componentes del paquete de ondas se propagan con la misma velocidad, lo que permite que el paquete mantenga su forma durante la propagación.

Para un paquete de ondas gaussiano, la amplitud espectral tiene la forma:
\begin{equation}
\vec{A}(k) = \vec{A}_0 e^{-(k-k_0)^2/(2\Delta k^2)},
\end{equation}
donde $k_0$ es el número de onda central y $\Delta k$ es la anchura espectral del paquete.

La transformada de Fourier inversa de esta expresión da un paquete de ondas con una envolvente gaussiana en el espacio:
\begin{equation}
\vec{E}(z, t) = \vec{E}_0 e^{-(z-ct)^2/(2\Delta z^2)} e^{i(k_0 z - \omega_0 t)},
\end{equation}
donde $\Delta z = 1/\Delta k$ es la anchura espacial del paquete y $\omega_0 = ck_0$ es la frecuencia angular central.

La velocidad con la que se propaga la envolvente del paquete se denomina velocidad de grupo:
\begin{equation}
v_g = \frac{d\omega}{dk}.
\end{equation}

En el vacío, $v_g = c$, lo que significa que la envolvente del paquete se propaga a la velocidad de la luz. Sin embargo, en medios dispersivos, donde la relación de dispersión no es lineal, la velocidad de grupo puede diferir de la velocidad de fase $v_p = \omega/k$, y el paquete puede cambiar de forma durante la propagación.

La anchura espectral $\Delta \omega$ y la duración temporal $\Delta t$ del paquete están relacionadas por el principio de incertidumbre:
\begin{equation}
\Delta \omega \Delta t \geq \frac{1}{2}.
\end{equation}

Esta relación impone un límite fundamental a la localización simultánea en frecuencia y tiempo de una señal electromagnética.

\section{Problemas propuestos}

\begin{enumerate}
\item La partícula cargada de carga $q$ se mueve en el plano $XY$ a lo largo de la recta $y = x - a$ con velocidad constante $\vec{v}$ alejándose del origen de coordenadas. En el instante de tiempo $t = 0$ la partícula se encontraba sobre el eje $x$. Calcule las densidades de carga y corriente asociadas a la partícula y verifique el cumplimiento de la ecuación de continuidad.

\item La partícula cargada con carga $q$ se mueve a lo largo del eje $x$ según la ley $x(t) = A \sin(\omega t)$. Calcule las densidades de carga y corriente asociadas a la partícula y verifique el cumplimiento de la ecuación de continuidad. Halle los valores medios $\langle\rho\rangle$ y $\langle\vec{J}\rangle$ de las densidades de carga y corriente, respectivamente, en el período $T = \frac{2\pi}{\omega}$. Demuestre que
\begin{equation*}
\int \langle\rho\rangle dV = q.
\end{equation*}

\item La partícula de carga $q$ se mueve en el plano $XY$ siguiendo una trayectoria circular de radio $a$ con velocidad angular constante $\omega$. La trayectoria circular tiene su centro en el origen de coordenadas, y en el instante de tiempo $t = 0$ la partícula se encontraba sobre la rama positiva del eje $x$. Halle las densidades de carga y corriente asociadas a la partícula y escríbalas en coordenadas cilíndricas. Demuestre que
\begin{equation*}
\int \langle\rho\rangle dV = q
\end{equation*}
y que
\begin{equation*}
\int dr \int dz \int \langle J_\varphi\rangle dV = \frac{q}{T},
\end{equation*}
siendo $\langle J_\varphi \rangle$ el valor medio, en un período de revolución $T$ de la partícula, de la componente $\varphi$ de la densidad de corriente.

\item En el plano $XY$ se encuentra un hilo conductor que describe la parábola $y = \alpha x^2$ ($\alpha > 0$). Un campo magnético homogéneo y constante de inducción $\vec{B}$ está aplicado en la dirección perpendicular al plano de la parábola. En el instante de tiempo $t = 0$ comienza a desplazarse, con aceleración constante $a$ dirigida a lo largo del eje $y$, un puente metálico que une las dos ramas de la parábola y que es paralelo al eje $x$. Hallar la fuerza electromotriz inducida en el circuito formado por la parábola y el puente en función de $y$.

\item Calcule los valores medios de la densidad de energía y del vector de Poynting asociados a una onda electromagnética plana de frecuencia $\omega$ y amplitud $\vec{E}_0$.

\item En el plano $z = 0$ el vector $\vec{E}$ del campo electromagnético de una onda monocromática plana que se propaga en la dirección del eje $z$ es
\begin{equation*}
\vec{E} = E_0 \left[ e^{-i(\omega t - \pi/4)}\vec{e}_x + e^{-i\omega t}\vec{e}_y \right],
\end{equation*}
siendo $\vec{e}_x$ y $\vec{e}_y$ los vectores unitarios a lo largo de las direcciones de los ejes $x$ y $y$, respectivamente, $\omega$ es la frecuencia de la onda y $E_0$ es un número real. Encuentre una expresión que vincula las partes reales de las dos componentes dadas de $\vec{E}$. ¿Qué tipo de polarización posee la onda dada?
\end{enumerate}

\chapter{Teoría Clásica de la Radiación}
\section{Solución de las ecuaciones para los potenciales del campo electromagnético}

Ya hemos visto que los potenciales $\phi$ y $\vec{A}$ del campo electromagnético satisfacen las ecuaciones no-homogéneas de D'Alembert [ver ecuaciones (3.24) y (3.25) en la sección 3.5.2 del Capítulo 3]
\begin{equation}
\Box\phi(\vec{r}, t) = -\frac{\rho(\vec{r}, t)}{\epsilon_0}
\end{equation}
y
\begin{equation}
\Box\vec{A}(\vec{r}, t) = -\mu_0 \vec{J}(\vec{r}, t),
\end{equation}
respectivamente. La forma explícita para los potenciales del campo electromagnético puede ser encontrada con el auxilio de la conocida identidad de Kirchhoff y de la función de Green para la ecuación de D'Alembert en el espacio abierto. Veamos.

\subsection{Identidad de Kirchhoff}

Para la deducción de la identidad de Kirchhoff nos hemos basado en el formalismo desarrollado en la referencia [2].

Estudiemos la solución del problema diferencial
\begin{equation}
\nabla^2 - \frac{1}{c^2}\frac{\partial^2}{\partial t^2}u(\vec{r}, t) = \Box u(\vec{r}, t) = -f(\vec{r}, t)
\end{equation}
y para ello consideremos la superficie definida por la ecuación
\begin{equation}
|\vec{r} - \vec{r}_0| = \frac{1}{c}|t - t_0|,
\end{equation}
llamada usualmente cono luminoso con vértice en el punto $M_0$ de radiovector de posición $\vec{r}_0$ en el instante $t_0$. Nótese que si en el instante de tiempo $t = 0$ se emite una onda electromagnética desde el punto $M_0$, entonces en el instante de tiempo $t > t_0$ la señal llegará al punto $M$ de radio vector de posición $\vec{r}$ y, obviamente,
\begin{equation}
|\vec{r} - \vec{r}_0| = \frac{1}{c}(t - t_0), \quad t > t_0.
\end{equation}
Análogamente, si en el instante de tiempo $t < t_0$ se emite desde el punto $M$ de radiovector de posición $\vec{r}$ una onda electromagnética, entonces la señal arribará al punto $M_0$ en el instante de tiempo $t_0$ tal que
\begin{equation}
|\vec{r} - \vec{r}_0| = \frac{1}{c}(t_0 - t), \quad t < t_0.
\end{equation}

Las ecuaciones (4.3) y (4.4) definen la parte superior y la parte inferior, respectivamente, del cono luminoso con respecto a su vértice $M_0$.

Efectuemos en (4.1) el cambio de variables
\begin{equation}
t^* = t - t_0 - \frac{|\vec{r} - \vec{r}_0|}{c}
\end{equation}
y pasemos a las coordenadas esféricas $(r, \theta, \phi)$ tomando el origen de coordenadas en el punto $M_0$. Denotemos
\begin{equation}
U(r, \theta, \phi, t^*) = u(r, \theta, \phi, t)
\end{equation}
y
\begin{equation}
F(r, \theta, \phi, t^*) = f(r, \theta, \phi, t).
\end{equation}

Nótese que
\begin{align}
\frac{\partial u}{\partial r} &= \frac{\partial U}{\partial r} + \frac{1}{c}\frac{\partial U}{\partial t^*}, \\
\frac{\partial^2 u}{\partial r^2} &= \frac{\partial^2 U}{\partial r^2} + \frac{2}{c}\frac{\partial^2 U}{\partial r \partial t^*} + \frac{1}{c^2}\frac{\partial^2 U}{\partial t^{*2}}, \\
\frac{\partial u}{\partial \theta} &= \frac{\partial U}{\partial \theta}, \\
\frac{\partial^2 u}{\partial \theta^2} &= \frac{\partial^2 U}{\partial \theta^2}, \\
\frac{\partial u}{\partial \phi} &= \frac{\partial U}{\partial \phi}, \\
\frac{\partial^2 u}{\partial \phi^2} &= \frac{\partial^2 U}{\partial \phi^2}, \\
\frac{\partial u}{\partial t} &= \frac{\partial U}{\partial t^*}, \\
\frac{\partial^2 u}{\partial t^2} &= \frac{\partial^2 U}{\partial t^{*2}}.
\end{align}

En consecuencia,
\begin{align}
\Box u &= \frac{\partial^2 U}{\partial r^2} + \frac{2}{r}\frac{\partial U}{\partial r} + \frac{1}{r^2 \sin(\theta)}\frac{\partial}{\partial \theta}\left(\sin(\theta)\frac{\partial U}{\partial \theta}\right) + \frac{1}{r^2 \sin^2(\theta)}\frac{\partial^2 U}{\partial \phi^2} + \frac{2}{cr}\frac{\partial U}{\partial t^*} + \frac{2}{c}\frac{\partial^2 U}{\partial r \partial t^*} \nonumber \\
&+ \frac{1}{c^2}\frac{\partial^2 U}{\partial t^{*2}} - \frac{1}{c^2}\frac{\partial^2 U}{\partial t^{*2}} = -F,
\end{align}
de donde obtenemos que
\begin{equation}
\nabla^2 U(\vec{r}, t^*) = -F(\vec{r}, t^*) - \frac{2}{cr}\frac{\partial}{\partial r}\left(r\frac{\partial U(\vec{r}, t^*)}{\partial t^*}\right).
\end{equation}

Esta es una ecuación no-homogénea de Poisson, en la cual $t^*$ actúa como un parámetro.

Sea $V \subset \mathbb{R}^3$ un conjunto volumétrico limitado por una superficie orientable $S$ con vector normal $\vec{n}$ y supongamos que $V$ contiene el origen de coordenadas, es decir, $M_0 \in V$. Ya hemos visto en el Capítulo 1 que la solución del problema diferencial
\begin{equation}
\nabla^2 u(\vec{r}) = -f(\vec{r})
\end{equation}
admite la solución general
\begin{equation}
u(\vec{r}) = \int_V G(\vec{r},\vec{r}\,')f(\vec{r}\,')\,dV' + \oint_S \left[G(\vec{r},\vec{r}\,')\frac{\partial u(\vec{r}\,')}{\partial n'} - \frac{\partial}{\partial n'}G(\vec{r},\vec{r}\,')u(\vec{r}\,')\right]\,dS',
\end{equation}
siendo
\begin{equation}
G(\vec{r},\vec{r}\,') = \frac{1}{4\pi|\vec{r} - \vec{r}\,'|}
\end{equation}
la función de Green para la ecuación de Poisson. Por lo tanto la solución de la ecuación (4.8) en el punto $M_0$ ($\vec{r}_0 = 0$) en el instante $t^*$ es
\begin{equation}
U(0, t^*) = \frac{1}{4\pi}\oint_S \left[\frac{1}{r}\frac{\partial U(\vec{r}, t^*)}{\partial n} - U(\vec{r}, t^*)\frac{\partial}{\partial n}\frac{1}{r}\right]\,dS + \frac{1}{4\pi}\int_V \frac{1}{cr^2}\frac{\partial}{\partial r}\left(r\frac{\partial U(\vec{r}, t^*)}{\partial t^*}\right)\,dV + \frac{1}{4\pi}\int_V \frac{F(\vec{r}, t^*)}{r}\,dV.
\end{equation}

Consideremos la integral
\begin{equation}
I = \int_V \frac{1}{r^2}\frac{\partial}{\partial r}\left(r\frac{\partial U(\vec{r}, t^*)}{\partial t^*}\right)\,dV.
\end{equation}

Entonces la expresión (4.9) puede ser reescrita como
\begin{equation}
U(0, t^*) = \frac{1}{4\pi}\oint_S \left[\frac{1}{r}\frac{\partial U(\vec{r}, t^*)}{\partial n} - U(\vec{r}, t^*)\frac{\partial}{\partial n}\frac{1}{r}\right]\,dS + \frac{1}{4\pi}\frac{1}{c}I + \frac{1}{4\pi}\int_V \frac{F(\vec{r}, t^*)}{r}\,dV.
\end{equation}

Dado un campo vectorial $\vec{H}$ tal que
\begin{equation}
\vec{H} = H_r \vec{e}_r + H_\theta \vec{e}_\theta + H_\phi \vec{e}_\phi
\end{equation}
escrito en coordenadas esféricas, su divergencia puede calcularse mediante la expresión
\begin{equation}
\nabla \cdot \vec{H} = \frac{1}{r^2 \sin(\theta)}\left[\frac{\partial}{\partial r}(r^2 \sin(\theta) H_r) + \frac{\partial}{\partial \theta}(r \sin(\theta) H_\theta) + \frac{\partial}{\partial \phi}(r H_\phi)\right].
\end{equation}
Si en particular $H_\theta = H_\phi = 0$, entonces
\begin{equation}
\nabla \cdot \vec{H} = \frac{1}{r^2}\frac{\partial}{\partial r}(r^2 H_r).
\end{equation}
Definamos
\begin{equation}
\vec{H}(\vec{r}, t^*) = \frac{1}{r}\frac{\partial U(\vec{r}, t^*)}{\partial t^*}\vec{e}_r.
\end{equation}
Entonces
\begin{equation}
I = \int_V \nabla \cdot \vec{H}(\vec{r}, t^*)\,dV = \oint_S \vec{H}(\vec{r}, t^*) \cdot \vec{n}\,dS = \oint_S \frac{1}{r}\frac{\partial U(\vec{r}, t^*)}{\partial t^*}\vec{e}_r \cdot \vec{n}\,dS.
\end{equation}
Pero
\begin{equation}
\vec{e}_r \cdot \vec{n} = \nabla r \cdot \vec{n} = \frac{dr}{dn}.
\end{equation}
Luego
\begin{equation}
I = \oint_S \frac{1}{r}\frac{\partial U(\vec{r}, t^*)}{\partial t^*}\frac{dr}{dn}\,dS.
\end{equation}

Sustituyendo (4.12) en (4.11) encontramos que
\begin{equation}
U(0, t^*) = \frac{1}{4\pi}\oint_S \left[\frac{1}{r}\frac{\partial U(\vec{r}, t^*)}{\partial n} - U(\vec{r}, t^*)\frac{\partial}{\partial n}\frac{1}{r} + \frac{1}{cr}\frac{dr}{dn}\frac{\partial U(\vec{r}, t^*)}{\partial t^*}\right]\,dS + \frac{1}{4\pi}\int_V \frac{F(\vec{r}, t^*)}{r}\,dV.
\end{equation}

Retornemos ahora a las variables originales $(\vec{r}, t)$. Puesto que en virtud de (4.6) se tiene que $u(\vec{r}, t) = U(\vec{r}, t^*)$, entonces
\begin{equation}
\frac{\partial u}{\partial n} = \nabla u \cdot \vec{n} = \nabla U \cdot \vec{n} + \nabla t^* \cdot \vec{n}\frac{\partial U}{\partial t^*}.
\end{equation}
Pero de acuerdo con la expresión (4.5)
\begin{equation}
\nabla t^* \cdot \vec{n} = -\frac{1}{c}\nabla r \cdot \vec{n} = -\frac{1}{c}\frac{dr}{dn}.
\end{equation}
Luego
\begin{equation}
\frac{\partial u}{\partial n} = \frac{\partial U}{\partial n} - \frac{1}{c}\frac{dr}{dn}\frac{\partial U}{\partial t^*}.
\end{equation}

Supongamos ahora que $t^* = 0$ en (4.5). En este caso
\begin{equation}
t = t_0 + \frac{|\vec{r} - \vec{r}_0|}{c}
\end{equation}
y
\begin{equation}
rU(\vec{r}, 0) = u\left(\vec{r}, t_0 + \frac{r}{c}\right),
\end{equation}
donde $\vec{r}_0 = 0$ por haber tomado el origen de coordenadas sobre el punto $M_0$. Nótese que si $\vec{r} = \vec{r}_0 = 0$ entonces $t = t_0$. Por lo tanto
\begin{equation}
U(0, 0) = u(\vec{r}_0, t_0).
\end{equation}

Evaluando la expresión (4.13) en $t^* = 0$ y teniendo en cuenta las expresiones (4.6), (4.7), (4.14), (4.15), (4.16) y (4.17) encontramos
\begin{equation}
u(\vec{r}_0, t_0) = \frac{1}{4\pi}\oint_S \left[\frac{1}{r}\frac{\partial u\left(\vec{r}, t_0 + \frac{r}{c}\right)}{\partial n} - u\left(\vec{r}, t_0 + \frac{r}{c}\right)\frac{\partial}{\partial n}\frac{1}{r} + \frac{1}{cr}\frac{dr}{dn}\frac{\partial u\left(\vec{r}, t_0 + \frac{r}{c}\right)}{\partial t}\right]\,dS + \frac{1}{4\pi}\int_V \frac{f\left(\vec{r}, t_0 + \frac{r}{c}\right)}{r}\,dV.
\end{equation}

Esta es la solución general de la ecuación diferencial (4.1) en el vértice del cono luminoso. La expresión anterior puede ser fácilmente extendida para cualesquiera valores de $\vec{r}$ y $t$. Así,
\begin{equation}
u(\vec{r}, t) = \frac{1}{4\pi}\oint_S \left[\frac{1}{|\vec{r} - \vec{r}'|}\frac{\partial [u]}{\partial n'} - [u]\frac{\partial}{\partial n'}\frac{1}{|\vec{r} - \vec{r}'|} + \frac{1}{c|\vec{r} - \vec{r}'|}\frac{d|\vec{r} - \vec{r}'|}{dn'}\frac{\partial [u]}{\partial t'}\right]\,dS' + \frac{1}{4\pi}\int_V \frac{[f]}{|\vec{r} - \vec{r}'|}\,dV',
\end{equation}
donde hemos definido
\begin{equation}
[u] = u\left(\vec{r}', t - \frac{|\vec{r} - \vec{r}'|}{c}\right),
\end{equation}
\begin{equation}
\left[\frac{\partial u}{\partial n'}\right] = \frac{\partial u}{\partial n'}\left(\vec{r}', t - \frac{|\vec{r} - \vec{r}'|}{c}\right),
\end{equation}
\begin{equation}
\left[\frac{\partial u}{\partial t'}\right] = \frac{\partial u}{\partial t'}\left(\vec{r}', t - \frac{|\vec{r} - \vec{r}'|}{c}\right)
\end{equation}
y
\begin{equation}
[f] = f\left(\vec{r}', t - \frac{|\vec{r} - \vec{r}'|}{c}\right).
\end{equation}

La expresión (4.19) se conoce con el nombre de fórmula de Kirchhoff o identidad de Kirchhoff.

\subsection{Función de Green para la ecuación de D'Alembert en el espacio abierto}

Consideremos el problema diferencial
\begin{align}
\Box G(\vec{r},\vec{r}', t) &= -\delta(\vec{r} - \vec{r}')\delta(t), \quad \vec{r} \in \mathbb{R}^3, \vec{r}' \in \mathbb{R}^3, t \in (0, +\infty), \\
G(\vec{r},\vec{r}', t) &\xrightarrow{r\to\infty} 0, \\
\frac{\partial G(\vec{r},\vec{r}', t)}{\partial n} &\xrightarrow{r\to\infty} 0,
\end{align}
siendo $\vec{n}$ un vector normal a una esfera de radio $R$ centrada en el origen de coordenadas. Obviamente el límite $r \to \infty$ en la expresión anterior equivale a tomar el límite $R \to \infty$\footnote{Esta aclaración es válida también para cualquier problema diferencial en el espacio abierto, como los dados por las expresiones (4.27), (4.29) y (4.30).}.

Haciendo $f(\vec{r}, t) = \delta(\vec{r} - \vec{r}')\delta(t)$ en la identidad de Kirchhoff (4.19) y teniendo en cuenta las condiciones de frontera para $G$ hallamos
\begin{equation}
G(\vec{r},\vec{r}', t) = \frac{1}{4\pi}\int_V \frac{[f]}{|\vec{r} - \vec{r}''|}\,dV'' = \frac{1}{4\pi}\int_V \frac{1}{|\vec{r} - \vec{r}''|}\delta(\vec{r}'' - \vec{r}')\delta\left(t - \frac{|\vec{r} - \vec{r}''|}{c}\right)\,dV''.
\end{equation}

En consecuencia,
\begin{equation}
G(\vec{r},\vec{r}', t) = \frac{1}{4\pi|\vec{r} - \vec{r}'|}\delta\left(t - \frac{|\vec{r} - \vec{r}'|}{c}\right).
\end{equation}

Esta es la función de Green para la ecuación de D'Alembert en el espacio abierto. Dado el problema diferencial
\begin{align}
\Box u(\vec{r}, t) &= -f(\vec{r}, t), \quad \vec{r} \in \mathbb{R}^3, t \in (0, +\infty), \\
u(\vec{r}, t) &\xrightarrow{r\to\infty} 0, \\
\frac{\partial u(\vec{r}, t)}{\partial n} &\xrightarrow{r\to\infty} 0,
\end{align}
la solución $u$ viene dada por la convolución de la inhomogeneidad $f$ con la función de Green (4.26), es decir [2],
\begin{equation}
u(\vec{r}, t) = \int_0^{+\infty} d\tau \int_{\mathbb{R}^3} f(\vec{r}', \tau)G(\vec{r},\vec{r}', t - \tau)\,dV'.
\end{equation}

\subsection{Potenciales retardados}

Ahora estamos en condiciones de resolver las ecuaciones para los potenciales del campo electromagnético en el espacio abierto. Estas ecuaciones son
\begin{align}
\Box\phi(\vec{r}, t) &= -\frac{\rho(\vec{r}, t)}{\epsilon_0}, \quad \vec{r} \in \mathbb{R}^3, t \in (0, +\infty), \\
\phi(\vec{r}, t) &\xrightarrow{r\to\infty} 0, \\
\frac{\partial \phi(\vec{r}, t)}{\partial n} &\xrightarrow{r\to\infty} 0
\end{align}
y
\begin{align}
\Box\vec{A}(\vec{r}, t) &= -\mu_0 \vec{J}(\vec{r}, t), \quad \vec{r} \in \mathbb{R}^3, t \in (0, +\infty), \\
\vec{A}(\vec{r}, t) &\xrightarrow{r\to\infty} 0, \\
\frac{\partial \vec{A}(\vec{r}, t)}{\partial n} &\xrightarrow{r\to\infty} 0
\end{align}
para los potenciales escalar y vectorial, respectivamente.

De acuerdo con (4.28) es evidente que
\begin{equation}
\phi(\vec{r}, t) = \frac{1}{4\pi\epsilon_0}\int_0^{+\infty} d\tau \int_{\mathbb{R}^3} \frac{\rho(\vec{r}', \tau)}{|\vec{r} - \vec{r}'|}\delta\left(t - \tau - \frac{|\vec{r} - \vec{r}'|}{c}\right)\,dV'
\end{equation}
y
\begin{equation}
\vec{A}(\vec{r}, t) = \frac{\mu_0}{4\pi}\int_0^{+\infty} d\tau \int_{\mathbb{R}^3} \frac{\vec{J}(\vec{r}', \tau)}{|\vec{r} - \vec{r}'|}\delta\left(t - \tau - \frac{|\vec{r} - \vec{r}'|}{c}\right)\,dV',
\end{equation}
de donde
\begin{equation}
\phi(\vec{r}, t) = \frac{1}{4\pi\epsilon_0}\int_{\mathbb{R}^3} \frac{\rho\left(\vec{r}', t - \frac{|\vec{r} - \vec{r}'|}{c}\right)}{|\vec{r} - \vec{r}'|}\,dV' = \frac{1}{4\pi\epsilon_0}\int_{\mathbb{R}^3} \frac{[\rho]}{|\vec{r} - \vec{r}'|}\,dV'
\end{equation}
y
\begin{equation}
\vec{A}(\vec{r}, t) = \frac{\mu_0}{4\pi}\int_{\mathbb{R}^3} \frac{\vec{J}\left(\vec{r}', t - \frac{|\vec{r} - \vec{r}'|}{c}\right)}{|\vec{r} - \vec{r}'|}\,dV' = \frac{\mu_0}{4\pi}\int_{\mathbb{R}^3} \frac{[\vec{J}]}{|\vec{r} - \vec{r}'|}\,dV'.
\end{equation}

Por razones obvias los potenciales dados por las expresiones (4.31) y (4.32) reciben el nombre de potenciales retardados. Por ejemplo, si tenemos un diferencial de carga dependiente del tiempo situado en el punto de radiovector $\vec{r}'$, el diferencial de potencial que creará la carga infinitesimal sobre un punto del espacio de radiovector $\vec{r}$ en el instante $t$ dependerá del estado de dicha carga infinitesimal en el instante $t - t'$, siendo $t' = \frac{1}{c}|\vec{r} - \vec{r}'|$ el tiempo que demora la interacción en propagarse desde el punto de radiovector $\vec{r}'$ hasta el punto de radiovector $\vec{r}$. Este hecho refleja la finitud de la velocidad de propagación de la interacción en el campo electromagnético.

\section{Campo electromagnético de una partícula cargada en movimiento arbitrario}
\subsection{Potenciales de Liénard-Wiechert}

De especial importancia para la Teoría de la Radiación resultan ser los potenciales retardados relacionados con el movimiento arbitrario de una partícula cargada. Estos potenciales, estudiados por los físicos Alfred-Marie Liénard [19] y Emil Johann Wiechert [20], se conocen con el nombre de potenciales de Liénard-Wiechert.

Supongamos que tenemos una partícula de carga $q$ y radiovector de posición $\vec{r}_0 = \vec{r}_0(t)$ moviéndose con la velocidad $\vec{v}(t) = \dot{\vec{r}}_0(t)$. Como ya es sabido, la densidad de carga asociada a la partícula es
\begin{equation}
\rho(\vec{r}, t) = q\delta[\vec{r} - \vec{r}_0(t)].
\end{equation}

El potencial escalar retardado será entonces
\begin{equation}
\phi(\vec{r}, t) = \frac{1}{4\pi\epsilon_0}\int_{\mathbb{R}^3} \frac{[\rho]}{|\vec{r} - \vec{r}'|}\,dV' = \frac{q}{4\pi\epsilon_0}\int_{\mathbb{R}^3} \frac{\delta\left[\vec{r}' - \vec{r}_0\left(t - \frac{|\vec{r} - \vec{r}'|}{c}\right)\right]}{|\vec{r} - \vec{r}'|}\,dV'.
\end{equation}

La integral anterior puede ser reescrita como
\begin{equation}
\phi(\vec{r}, t) = \frac{q}{4\pi\epsilon_0}\int_{\mathbb{R}^3} \frac{dV'}{|\vec{r} - \vec{r}'|}\int d\tau\,\delta[\vec{r}' - \vec{r}_0(\tau)]\delta\left(\tau - t + \frac{|\vec{r} - \vec{r}'|}{c}\right),
\end{equation}
de donde
\begin{equation}
\phi(\vec{r}, t) = \frac{q}{4\pi\epsilon_0}\frac{1}{|\vec{r} - \vec{r}_0(t')|},
\end{equation}
siendo $t'$ la solución de la ecuación
\begin{equation}
t' = t - \frac{|\vec{r} - \vec{r}_0(t')|}{c}.
\end{equation}

La ecuación anterior admite una interpretación física muy sencilla. Si en el instante $t'$ la partícula cargada emite una señal electromagnética en forma de luz, entonces esta señal arribará al punto de radiovector $\vec{r}$ en el instante $t$. El tiempo que demora la señal en propagarse es $\Delta t = t - t' = \frac{|\vec{r} - \vec{r}_0(t')|}{c}$.

Nos interesa saber cuál es la corriente $\vec{J}$ asociada a la carga. Para ello recordemos que $\vec{J} = \rho\vec{v}$. Por lo tanto
\begin{equation}
\vec{J}(\vec{r}, t) = q\vec{v}(t)\delta[\vec{r} - \vec{r}_0(t)],
\end{equation}
de donde
\begin{equation}
\vec{A}(\vec{r}, t) = \frac{\mu_0 q}{4\pi}\frac{\vec{v}(t')}{|\vec{r} - \vec{r}_0(t')|} = \frac{\mu_0 q}{4\pi}\frac{\dot{\vec{r}}_0(t')}{|\vec{r} - \vec{r}_0(t')|}.
\end{equation}

Definamos el vector $\vec{R}(t') = \vec{r} - \vec{r}_0(t')$. Dicho vector apunta desde la posición de la partícula en el instante $t'$ hasta el punto del espacio de radiovector $\vec{r}$ en el cual deseamos determinar los potenciales y los campos. El módulo de $\vec{R}(t')$ suele denotarse como $R(t') = |\vec{R}(t')|$. Así,
\begin{equation}
\phi(\vec{r}, t) = \frac{q}{4\pi\epsilon_0}\frac{1}{R(t')}
\end{equation}
y
\begin{equation}
\vec{A}(\vec{r}, t) = \frac{\mu_0 q}{4\pi}\frac{\vec{v}(t')}{R(t')}.
\end{equation}

Utilizando la expresión para la velocidad de la luz en función de las constantes eléctrica y magnética del vacío, es decir, $\mu_0\epsilon_0 = \frac{1}{c^2}$, la expresión (4.39) puede reescribirse como
\begin{equation}
\vec{A}(\vec{r}, t) = \frac{1}{c^2}\frac{\vec{v}(t')}{R(t')}\frac{q}{4\pi\epsilon_0} = \frac{\vec{v}(t')}{c^2}\phi(\vec{r}, t).
\end{equation}

Es interesante ver cómo se transforman los potenciales (4.38) y (4.39) en el caso en que la carga $q$ describa un movimiento relativista. Para ello introducimos el tiempo retardado $t' = t - \frac{R(t')}{c}$ y veamos cómo puede relacionarse una variación infinitesimal $dt'$ con la correspondiente variación infinitesimal $dt$.

Para ello diferenciamos la expresión para el tiempo retardado obteniendo
\begin{equation}
dt' = dt - \frac{1}{c}dR(t').
\end{equation}

Calculemos $dR(t')$. Tenemos que
\begin{equation}
R(t') = |\vec{R}(t')| = |\vec{r} - \vec{r}_0(t')| = \sqrt{[\vec{r} - \vec{r}_0(t')]^2}.
\end{equation}
Luego
\begin{equation}
dR(t') = \frac{d|\vec{R}(t')|}{dt'}dt' = \frac{d}{dt'}\sqrt{[\vec{r} - \vec{r}_0(t')]^2}\,dt' = \frac{1}{2}\frac{1}{\sqrt{[\vec{r} - \vec{r}_0(t')]^2}}\frac{d}{dt'}[\vec{r} - \vec{r}_0(t')]^2\,dt'.
\end{equation}
Pero
\begin{equation}
\frac{d}{dt'}[\vec{r} - \vec{r}_0(t')]^2 = \frac{d}{dt'}[(\vec{r} - \vec{r}_0(t')) \cdot (\vec{r} - \vec{r}_0(t'))] = -2(\vec{r} - \vec{r}_0(t')) \cdot \frac{d\vec{r}_0(t')}{dt'} = -2\vec{R}(t') \cdot \vec{v}(t').
\end{equation}
Por lo tanto
\begin{equation}
dR(t') = -\frac{\vec{R}(t') \cdot \vec{v}(t')}{|\vec{R}(t')|}\,dt' = -\frac{\vec{R}(t') \cdot \vec{v}(t')}{R(t')}\,dt'.
\end{equation}

Sustituyendo (4.42) en (4.41) encontramos
\begin{equation}
dt' = dt + \frac{1}{c}\frac{\vec{R}(t') \cdot \vec{v}(t')}{R(t')}\,dt' = dt + \frac{\vec{n}(t') \cdot \vec{v}(t')}{c}\,dt',
\end{equation}
de donde
\begin{equation}
dt' = \frac{dt}{1 - \frac{\vec{n}(t') \cdot \vec{v}(t')}{c}},
\end{equation}
siendo $\vec{n}(t') = \frac{\vec{R}(t')}{R(t')}$ un vector unitario en la dirección del vector $\vec{R}(t')$. La expresión (4.43) muestra que $dt' \neq dt$, salvo para el caso de una partícula en reposo ($\vec{v} = \vec{0}$).

Definamos la cantidad
\begin{equation}
\kappa = 1 - \frac{\vec{n}(t') \cdot \vec{v}(t')}{c} = 1 - \frac{\vec{R}(t') \cdot \vec{v}(t')}{R(t')c} = 1 - \beta\cos(\alpha),
\end{equation}
donde $\beta = \frac{v}{c}$, $v = |\vec{v}(t')|$ y $\alpha$ es el ángulo entre $\vec{R}(t')$ y $\vec{v}(t')$. Con ello (4.43) puede ser reescrita como
\begin{equation}
dt' = \frac{dt}{\kappa}.
\end{equation}

Veamos ahora cómo queda $\phi(\vec{r}, t)$ cuando lo escribimos en términos del tiempo no retardado. Para ello nos aprovecharemos del hecho de que $q\,dt' = q\,dt$. Entonces
\begin{equation}
\phi(\vec{r}, t) = \frac{q}{4\pi\epsilon_0}\frac{1}{R(t')} = \frac{q\,dt'}{4\pi\epsilon_0}\frac{1}{R(t')}\frac{1}{dt'} = \frac{q\,dt}{4\pi\epsilon_0}\frac{1}{R(t')}\frac{1}{dt'} = \frac{q\,dt}{4\pi\epsilon_0}\frac{1}{R(t')}\frac{\kappa}{dt} = \frac{q}{4\pi\epsilon_0}\frac{\kappa}{R(t')}.
\end{equation}

De manera análoga encontramos para el potencial vectorial
\begin{equation}
\vec{A}(\vec{r}, t) = \frac{\mu_0 q}{4\pi}\frac{\vec{v}(t')}{R(t')} = \frac{\mu_0 q\,dt'}{4\pi}\frac{\vec{v}(t')}{R(t')}\frac{1}{dt'} = \frac{\mu_0 q\,dt}{4\pi}\frac{\vec{v}(t')}{R(t')}\frac{1}{dt'} = \frac{\mu_0 q\,dt}{4\pi}\frac{\vec{v}(t')}{R(t')}\frac{\kappa}{dt} = \frac{\mu_0 q}{4\pi}\frac{\kappa\vec{v}(t')}{R(t')}.
\end{equation}

Las expresiones (4.46) y (4.47) son las expresiones relativísticamente correctas para los potenciales de Liénard-Wiechert. Haciendo uso de la relación $\mu_0\epsilon_0 = \frac{1}{c^2}$ podemos reescribir (4.47) como
\begin{equation}
\vec{A}(\vec{r}, t) = \frac{\mu_0 q}{4\pi}\frac{\kappa\vec{v}(t')}{R(t')} = \frac{1}{c^2}\frac{\kappa\vec{v}(t')}{R(t')}\frac{q}{4\pi\epsilon_0} = \frac{\vec{v}(t')}{c^2}\phi(\vec{r}, t).
\end{equation}

En lo sucesivo omitiremos la dependencia funcional explícita en $t'$. Así, escribiremos $\vec{R} = \vec{r} - \vec{r}_0$, $\vec{v} = \dot{\vec{r}}_0$, $\vec{n} = \frac{\vec{R}}{R}$, $\kappa = 1 - \frac{\vec{n} \cdot \vec{v}}{c}$, etcétera.

\subsection{Campos de Liénard-Wiechert}

Una vez determinados los potenciales del campo electromagnético (potenciales de Liénard-Wiechert) podemos proceder a calcular los campos $\vec{E}$ y $\vec{B}$ generados por una partícula cargada en movimiento arbitrario.

Del electromagnetismo sabemos que
\begin{equation}
\vec{E} = -\nabla\phi - \frac{\partial\vec{A}}{\partial t}
\end{equation}
y
\begin{equation}
\vec{B} = \nabla \times \vec{A}.
\end{equation}

Para calcular $\vec{E}$ debemos, en primer lugar, hallar $\nabla\phi$. De acuerdo con (4.46) tenemos
\begin{equation}
\phi(\vec{r}, t) = \frac{q}{4\pi\epsilon_0}\frac{\kappa}{R}.
\end{equation}
Entonces
\begin{equation}
\nabla\phi = \frac{q}{4\pi\epsilon_0}\nabla\left(\frac{\kappa}{R}\right) = \frac{q}{4\pi\epsilon_0}\left[\frac{\nabla\kappa}{R} - \frac{\kappa\nabla R}{R^2}\right].
\end{equation}

Para el cálculo de $\nabla\kappa$ tengamos en cuenta que $\kappa = 1 - \frac{\vec{n} \cdot \vec{v}}{c}$. Como
\begin{equation}
\vec{n} = \frac{\vec{R}}{R} = \frac{\vec{r} - \vec{r}_0}{|\vec{r} - \vec{r}_0|},
\end{equation}
podemos escribir
\begin{equation}
\nabla\kappa = -\frac{1}{c}\nabla(\vec{n} \cdot \vec{v}) = -\frac{1}{c}\nabla\left(\frac{\vec{R}}{R} \cdot \vec{v}\right) = -\frac{1}{c}\nabla\left(\frac{\vec{R} \cdot \vec{v}}{R}\right).
\end{equation}

Debemos tener presente que $\vec{r}_0$ y $\vec{v}$ son funciones del tiempo retardado $t'$, el cual a su vez depende del vector posición $\vec{r}$. Así, cuando calculamos $\nabla\kappa$ estamos derivando respecto a $\vec{r}$ y debemos hacer uso de la regla de la cadena.

Veamos primeramente cuál es la relación entre $\nabla$ y $\nabla'$, siendo $\nabla'$ el operador gradiente con respecto a $\vec{r}_0$. Sabemos que
\begin{equation}
\vec{R} = \vec{r} - \vec{r}_0.
\end{equation}
Luego
\begin{equation}
\nabla\vec{R} = \nabla\vec{r} - \nabla\vec{r}_0 = \mathbb{I} - \nabla\vec{r}_0,
\end{equation}
donde $\mathbb{I}$ es el tensor unitario ($\mathbb{I} = \sum_{i=1}^3 \vec{e}_i \otimes \vec{e}_i$, siendo $\vec{e}_i$, $i = 1, 2, 3$, los vectores de la base canónica en $\mathbb{R}^3$). 

De la ecuación que define al tiempo retardado
\begin{equation}
t' = t - \frac{R}{c}
\end{equation}
obtenemos
\begin{equation}
\nabla t' = \nabla t - \frac{\nabla R}{c} = -\frac{\nabla R}{c} = -\frac{\vec{R}}{Rc},
\end{equation}
ya que $\nabla R = \frac{\vec{R}}{R}$.

Por lo tanto,
\begin{equation}
\nabla\vec{r}_0 = \frac{d\vec{r}_0}{dt'}\nabla t' = \vec{v}\nabla t' = -\frac{\vec{v}\vec{R}}{Rc}
\end{equation}
y
\begin{equation}
\nabla\vec{R} = \mathbb{I} - \nabla\vec{r}_0 = \mathbb{I} + \frac{\vec{v}\vec{R}}{Rc}.
\end{equation}

El gradiente de un escalar $\varphi$ que depende de $\vec{r}_0$, es decir, $\varphi = \varphi(\vec{r}_0)$, puede calcularse como
\begin{equation}
\nabla\varphi = \nabla'\varphi \cdot \nabla\vec{r}_0 = -\frac{\vec{R}}{Rc}\nabla'\varphi \cdot \vec{v},
\end{equation}
donde usamos la expresión (4.56). Para una función vectorial $\vec{F} = \vec{F}(\vec{r}_0)$ se tiene que
\begin{equation}
\nabla\vec{F} = \nabla'\vec{F} \cdot \nabla\vec{r}_0 = -\frac{\vec{R}}{Rc}\nabla'\vec{F} \cdot \vec{v}.
\end{equation}

Continuemos ahora con el cálculo de $\nabla\kappa$. Se tiene que
\begin{align}
\nabla\kappa &= -\frac{1}{c}\nabla\left(\frac{\vec{R} \cdot \vec{v}}{R}\right) = -\frac{1}{c}\left[\frac{\nabla(\vec{R} \cdot \vec{v})}{R} - \frac{(\vec{R} \cdot \vec{v})\nabla R}{R^2}\right] \nonumber \\
&= -\frac{1}{c}\left[\frac{\nabla\vec{R} \cdot \vec{v} + \vec{R} \cdot \nabla\vec{v}}{R} - \frac{(\vec{R} \cdot \vec{v})\vec{R}}{R^3}\right].
\end{align}

De la expresión (4.57) obtenemos
\begin{equation}
\nabla\vec{R} \cdot \vec{v} = \left(\mathbb{I} + \frac{\vec{v}\vec{R}}{Rc}\right) \cdot \vec{v} = \vec{v} + \frac{v^2\vec{R}}{Rc} = \vec{v} + \frac{v^2\vec{n}}{c}.
\end{equation}

En virtud de (4.59) se tiene
\begin{equation}
\vec{R} \cdot \nabla\vec{v} = -\frac{\vec{R} \cdot \vec{R}}{Rc}\nabla'\vec{v} \cdot \vec{v} = -\frac{R^2}{Rc}\nabla'\vec{v} \cdot \vec{v} = -\frac{R}{c}\nabla'\vec{v} \cdot \vec{v} = -\frac{R}{c}\frac{d}{dt'}\left(\frac{v^2}{2}\right) = -\frac{R}{c}\vec{v} \cdot \dot{\vec{v}} = -\frac{R}{c}\vec{v} \cdot \vec{a},
\end{equation}
donde $\vec{a} = \dot{\vec{v}} = \ddot{\vec{r}}_0$ es la aceleración de la partícula.

Sustituyendo (4.61) y (4.62) en (4.60) encontramos que
\begin{align}
\nabla\kappa &= -\frac{1}{c}\left[\frac{\vec{v} + \frac{v^2\vec{n}}{c} - \frac{R}{c}\vec{v} \cdot \vec{a}}{R} - \frac{(\vec{R} \cdot \vec{v})\vec{R}}{R^3}\right] \nonumber \\
&= -\frac{1}{c}\left[\frac{\vec{v}}{R} + \frac{v^2\vec{n}}{Rc} - \frac{\vec{v} \cdot \vec{a}}{c} - \frac{(\vec{R} \cdot \vec{v})\vec{n}}{R^2}\right] \nonumber \\
&= -\frac{1}{c}\left[\frac{\vec{v}}{R} + \frac{v^2\vec{n}}{Rc} - \frac{\vec{v} \cdot \vec{a}}{c} - \frac{(\vec{n} \cdot \vec{v})\vec{n}}{R}\right] \nonumber \\
&= -\frac{1}{c}\left[\frac{\vec{v} - (\vec{n} \cdot \vec{v})\vec{n}}{R} + \frac{v^2\vec{n}}{Rc} - \frac{\vec{v} \cdot \vec{a}}{c}\right] \nonumber \\
&= -\frac{1}{c}\left[\frac{(\vec{v} \cdot \vec{n})\vec{n} - \vec{v}}{R} + \frac{v^2\vec{n}}{Rc} - \frac{\vec{v} \cdot \vec{a}}{c}\right].
\end{align}

Tengamos en cuenta que
\begin{equation}
\vec{v} - (\vec{n} \cdot \vec{v})\vec{n} = \vec{v} - \vec{n}(\vec{n} \cdot \vec{v})
\end{equation}
es la componente de $\vec{v}$ perpendicular a $\vec{n}$. Escribamos $\vec{v}_{\perp} = \vec{v} - (\vec{n} \cdot \vec{v})\vec{n}$. Entonces (4.63) adquiere la forma
\begin{equation}
\nabla\kappa = \frac{1}{c}\left[\frac{\vec{v}_{\perp}}{R} - \frac{v^2\vec{n}}{Rc} + \frac{\vec{v} \cdot \vec{a}}{c}\right].
\end{equation}

Ahora calculemos $\nabla R$:
\begin{equation}
\nabla R = \nabla|\vec{R}| = \frac{\vec{R}}{R} = \vec{n}.
\end{equation}

Sustituyendo (4.65) y (4.66) en (4.52) se obtiene
\begin{align}
\nabla\phi &= \frac{q}{4\pi\epsilon_0}\left[\frac{\nabla\kappa}{R} - \frac{\kappa\nabla R}{R^2}\right] = \frac{q}{4\pi\epsilon_0}\left[\frac{1}{R}\frac{1}{c}\left(\frac{\vec{v}_{\perp}}{R} - \frac{v^2\vec{n}}{Rc} + \frac{\vec{v} \cdot \vec{a}}{c}\right) - \frac{\kappa\vec{n}}{R^2}\right] \nonumber \\
&= \frac{q}{4\pi\epsilon_0}\left[\frac{1}{Rc}\left(\frac{\vec{v}_{\perp}}{R} - \frac{v^2\vec{n}}{Rc} + \frac{\vec{v} \cdot \vec{a}}{c}\right) - \frac{\kappa\vec{n}}{R^2}\right] \nonumber \\
&= \frac{q}{4\pi\epsilon_0}\left[\frac{\vec{v}_{\perp}}{R^2c} - \frac{v^2\vec{n}}{R^2c^2} + \frac{\vec{v} \cdot \vec{a}}{Rc^2} - \frac{\kappa\vec{n}}{R^2}\right].
\end{align}

Ahora necesitamos calcular el término $\frac{\partial\vec{A}}{\partial t}$. De la expresión (4.47) se tiene que
\begin{equation}
\vec{A}(\vec{r}, t) = \frac{\mu_0 q}{4\pi}\frac{\kappa\vec{v}}{R}.
\end{equation}
Por tanto,
\begin{equation}
\frac{\partial\vec{A}}{\partial t} = \frac{\mu_0 q}{4\pi}\frac{\partial}{\partial t}\left(\frac{\kappa\vec{v}}{R}\right) = \frac{\mu_0 q}{4\pi}\left[\frac{\frac{\partial\kappa}{\partial t}\vec{v} + \kappa\frac{\partial\vec{v}}{\partial t}}{R} - \frac{\kappa\vec{v}}{R^2}\frac{\partial R}{\partial t}\right].
\end{equation}

Para el cálculo de $\frac{\partial\kappa}{\partial t}$, $\frac{\partial\vec{v}}{\partial t}$ y $\frac{\partial R}{\partial t}$ haremos uso de la regla de la cadena y de la relación entre la derivada con respecto a $t$ y la derivada con respecto a $t'$:
\begin{equation}
\frac{\partial}{\partial t} = \frac{\partial t'}{\partial t}\frac{\partial}{\partial t'} = \frac{1}{\frac{\partial t}{\partial t'}}\frac{\partial}{\partial t'} = \frac{1}{\kappa}\frac{\partial}{\partial t'},
\end{equation}
donde en el último paso hemos usado (4.45).

La derivada de $\kappa$ con respecto a $t$ es
\begin{equation}
\frac{\partial\kappa}{\partial t} = \frac{1}{\kappa}\frac{\partial\kappa}{\partial t'} = \frac{1}{\kappa}\frac{\partial}{\partial t'}\left(1 - \frac{\vec{n} \cdot \vec{v}}{c}\right) = -\frac{1}{\kappa}\frac{1}{c}\frac{\partial}{\partial t'}(\vec{n} \cdot \vec{v}) = -\frac{1}{\kappa}\frac{1}{c}\left(\frac{\partial\vec{n}}{\partial t'} \cdot \vec{v} + \vec{n} \cdot \frac{\partial\vec{v}}{\partial t'}\right).
\end{equation}

Para calcular $\frac{\partial\vec{n}}{\partial t'}$ tengamos en cuenta que $\vec{n} = \frac{\vec{R}}{R}$. Luego
\begin{equation}
\frac{\partial\vec{n}}{\partial t'} = \frac{\partial}{\partial t'}\left(\frac{\vec{R}}{R}\right) = \frac{\frac{\partial\vec{R}}{\partial t'}}{R} - \frac{\vec{R}}{R^2}\frac{\partial R}{\partial t'}.
\end{equation}

Pero $\vec{R} = \vec{r} - \vec{r}_0$. Por tanto, $\frac{\partial\vec{R}}{\partial t'} = -\frac{\partial\vec{r}_0}{\partial t'} = -\vec{v}$. Además, $\frac{\partial R}{\partial t'} = \frac{\partial|\vec{R}|}{\partial t'} = \frac{\partial}{\partial t'}\sqrt{\vec{R} \cdot \vec{R}} = \frac{1}{2}\frac{1}{\sqrt{\vec{R} \cdot \vec{R}}}\frac{\partial}{\partial t'}(\vec{R} \cdot \vec{R}) = \frac{1}{2}\frac{1}{R}2\vec{R} \cdot \frac{\partial\vec{R}}{\partial t'} = \frac{\vec{R} \cdot \frac{\partial\vec{R}}{\partial t'}}{R} = \frac{\vec{R} \cdot (-\vec{v})}{R} = -\frac{\vec{R} \cdot \vec{v}}{R} = -\vec{n} \cdot \vec{v}$.

Por ende
\begin{equation}
\frac{\partial\vec{n}}{\partial t'} = \frac{-\vec{v}}{R} - \frac{\vec{R}}{R^2}(-\vec{n} \cdot \vec{v}) = -\frac{\vec{v}}{R} + \frac{\vec{n}(\vec{n} \cdot \vec{v})}{R} = \frac{\vec{n}(\vec{n} \cdot \vec{v}) - \vec{v}}{R} = -\frac{\vec{v}_{\perp}}{R}.
\end{equation}

Usando (4.73) en (4.71) obtenemos
\begin{equation}
\frac{\partial\kappa}{\partial t} = -\frac{1}{\kappa}\frac{1}{c}\left(-\frac{\vec{v}_{\perp}}{R} \cdot \vec{v} + \vec{n} \cdot \vec{a}\right) = \frac{1}{\kappa}\frac{1}{c}\left(\frac{\vec{v}_{\perp} \cdot \vec{v}}{R} - \vec{n} \cdot \vec{a}\right).
\end{equation}

Pero $\vec{v}_{\perp} \cdot \vec{v} = [\vec{v} - \vec{n}(\vec{n} \cdot \vec{v})] \cdot \vec{v} = \vec{v} \cdot \vec{v} - (\vec{n} \cdot \vec{v})(\vec{n} \cdot \vec{v}) = v^2 - (\vec{n} \cdot \vec{v})^2$. Así,
\begin{equation}
\frac{\partial\kappa}{\partial t} = \frac{1}{\kappa}\frac{1}{c}\left[\frac{v^2 - (\vec{n} \cdot \vec{v})^2}{R} - \vec{n} \cdot \vec{a}\right].
\end{equation}

Para $\frac{\partial\vec{v}}{\partial t}$ se tiene que
\begin{equation}
\frac{\partial\vec{v}}{\partial t} = \frac{1}{\kappa}\frac{\partial\vec{v}}{\partial t'} = \frac{\vec{a}}{\kappa}.
\end{equation}

En cuanto a $\frac{\partial R}{\partial t}$, podemos escribir
\begin{equation}
\frac{\partial R}{\partial t} = \frac{1}{\kappa}\frac{\partial R}{\partial t'} = \frac{1}{\kappa}(-\vec{n} \cdot \vec{v}) = -\frac{\vec{n} \cdot \vec{v}}{\kappa}.
\end{equation}

Sustituyendo (4.75), (4.76) y (4.77) en (4.69) hallamos
\begin{align}
\frac{\partial\vec{A}}{\partial t} &= \frac{\mu_0 q}{4\pi}\left[\frac{\frac{1}{\kappa}\frac{1}{c}\left[\frac{v^2 - (\vec{n} \cdot \vec{v})^2}{R} - \vec{n} \cdot \vec{a}\right]\vec{v} + \kappa\frac{\vec{a}}{\kappa}}{R} - \frac{\kappa\vec{v}}{R^2}\left(-\frac{\vec{n} \cdot \vec{v}}{\kappa}\right)\right] \nonumber \\
&= \frac{\mu_0 q}{4\pi}\left[\frac{\frac{1}{\kappa}\frac{1}{c}\left[\frac{v^2 - (\vec{n} \cdot \vec{v})^2}{R} - \vec{n} \cdot \vec{a}\right]\vec{v} + \vec{a}}{R} + \frac{\vec{v}(\vec{n} \cdot \vec{v})}{R^2}\right] \nonumber \\
&= \frac{\mu_0 q}{4\pi}\left[\frac{1}{Rc\kappa}[v^2 - (\vec{n} \cdot \vec{v})^2]\vec{v} - \frac{1}{Rc\kappa}(\vec{n} \cdot \vec{a})\vec{v} + \frac{\vec{a}}{R} + \frac{\vec{v}(\vec{n} \cdot \vec{v})}{R^2}\right].
\end{align}

Tengamos en cuenta que $\mu_0 = \frac{1}{\epsilon_0 c^2}$. En consecuencia,
\begin{align}
\frac{\partial\vec{A}}{\partial t} &= \frac{q}{4\pi\epsilon_0 c^2}\left[\frac{1}{Rc\kappa}[v^2 - (\vec{n} \cdot \vec{v})^2]\vec{v} - \frac{1}{Rc\kappa}(\vec{n} \cdot \vec{a})\vec{v} + \frac{\vec{a}}{R} + \frac{\vec{v}(\vec{n} \cdot \vec{v})}{R^2}\right] \nonumber \\
&= \frac{q}{4\pi\epsilon_0}\left[\frac{1}{R^2c^3\kappa}[v^2 - (\vec{n} \cdot \vec{v})^2]\vec{v} - \frac{1}{Rc^3\kappa}(\vec{n} \cdot \vec{a})\vec{v} + \frac{\vec{a}}{Rc^2} + \frac{\vec{v}(\vec{n} \cdot \vec{v})}{R^2c^2}\right].
\end{align}

Finalmente, podemos calcular el campo eléctrico:
\begin{align}
\vec{E} &= -\nabla\phi - \frac{\partial\vec{A}}{\partial t} \nonumber \\
&= -\frac{q}{4\pi\epsilon_0}\left[\frac{\vec{v}_{\perp}}{R^2c} - \frac{v^2\vec{n}}{R^2c^2} + \frac{\vec{v} \cdot \vec{a}}{Rc^2} - \frac{\kappa\vec{n}}{R^2}\right] \nonumber \\
&- \frac{q}{4\pi\epsilon_0}\left[\frac{1}{R^2c^3\kappa}[v^2 - (\vec{n} \cdot \vec{v})^2]\vec{v} - \frac{1}{Rc^3\kappa}(\vec{n} \cdot \vec{a})\vec{v} + \frac{\vec{a}}{Rc^2} + \frac{\vec{v}(\vec{n} \cdot \vec{v})}{R^2c^2}\right] \nonumber \\
&= \frac{q}{4\pi\epsilon_0}\left[-\frac{\vec{v}_{\perp}}{R^2c} + \frac{v^2\vec{n}}{R^2c^2} - \frac{\vec{v} \cdot \vec{a}}{Rc^2} + \frac{\kappa\vec{n}}{R^2} \right. \nonumber \\
&\left. -\frac{1}{R^2c^3\kappa}[v^2 - (\vec{n} \cdot \vec{v})^2]\vec{v} + \frac{1}{Rc^3\kappa}(\vec{n} \cdot \vec{a})\vec{v} - \frac{\vec{a}}{Rc^2} - \frac{\vec{v}(\vec{n} \cdot \vec{v})}{R^2c^2}\right].
\end{align}

Agrupando términos convenientemente obtenemos
\begin{align}
\vec{E} &= \frac{q}{4\pi\epsilon_0}\left[\frac{\kappa\vec{n}}{R^2} + \frac{v^2\vec{n}}{R^2c^2} - \frac{\vec{v}_{\perp}}{R^2c} - \frac{\vec{v}(\vec{n} \cdot \vec{v})}{R^2c^2} \right. \nonumber \\
&\left. -\frac{1}{R^2c^3\kappa}[v^2 - (\vec{n} \cdot \vec{v})^2]\vec{v} - \frac{\vec{v} \cdot \vec{a}}{Rc^2} + \frac{1}{Rc^3\kappa}(\vec{n} \cdot \vec{a})\vec{v} - \frac{\vec{a}}{Rc^2}\right].
\end{align}

\subsection{Campo estático y campo de radiación}

Los vectores $\vec{E}$ y $\vec{B}$ asociados al campo electromagnético producido por el movimiento de la partícula pueden ser calculados a partir de las expresiones
\begin{equation}
\vec{E} = -\nabla\phi - \frac{\partial\vec{A}}{\partial t}
\end{equation}
y
\begin{equation}
\vec{B} = \nabla \times \vec{A}.
\end{equation}

Sustituyendo (4.46) y (4.47) en las expresiones anteriores es posible demostrar fácilmente que
\begin{equation}
\vec{E}(\vec{r}, t) = \frac{q}{4\pi\epsilon_0}\left[\frac{\left[1 - \beta^2(\tau)\right][\vec{n}(\tau) - \vec{\beta}(\tau)] + \frac{1}{c}R(\tau)\vec{n}(\tau) \times \left\{\left[\vec{n}(\tau) - \vec{\beta}(\tau)\right] \times \dot{\vec{\beta}}(\tau)\right\}}{\left[1 - \vec{n}(\tau) \cdot \vec{\beta}(\tau)\right]^3 R^2(\tau)}\right]_{\tau=\tau_R(t)}
\end{equation}
y
\begin{equation}
\vec{B}(\vec{r}, t) = \frac{1}{c}\vec{n}[\tau_R(t)] \times \vec{E}(\vec{r}, t).
\end{equation}

La demostración de las expresiones anteriores se deja en manos del lector (¡inténtelo!).

La componente $\vec{E}$ del campo electromagnético puede ser reescrita como
\begin{equation}
\vec{E}(\vec{r}, t) = \vec{E}_1(\vec{r}, t) + \vec{E}_2(\vec{r}, t),
\end{equation}
siendo
\begin{equation}
\vec{E}_1(\vec{r}, t) = \frac{q}{4\pi\epsilon_0}\left[\frac{\left[1 - \beta^2(\tau)\right][\vec{n}(\tau) - \vec{\beta}(\tau)]}{\left[1 - \vec{n}(\tau) \cdot \vec{\beta}(\tau)\right]^3 R^2(\tau)}\right]_{\tau=\tau_R(t)}
\end{equation}
y
\begin{equation}
\vec{E}_2(\vec{r}, t) = \frac{q}{4\pi\epsilon_0c}\left[\frac{\vec{n}(\tau) \times \left\{\left[\vec{n}(\tau) - \vec{\beta}(\tau)\right] \times \dot{\vec{\beta}}(\tau)\right\}}{\left[1 - \vec{n}(\tau) \cdot \vec{\beta}(\tau)\right]^3 R(\tau)}\right]_{\tau=\tau_R(t)}.
\end{equation}

El término $\vec{E}_1$ posee una estructura que recuerda al campo electrostático de una carga puntual. Analicemos el límite no relativista de la componente $\vec{E}_1$. El límite no relativista puede ser alcanzado en las ecuaciones mediante la condición $\vec{\beta} = |\vec{\beta}| \to 0$ o, equivalentemente, $c \to +\infty$.

Puesto que $\tau_R(t)$ es la raíz simple de la ecuación $\tau - t + \frac{1}{c}R(\tau) = 0$, es evidente que
\begin{equation}
\lim_{\beta \to 0}\tau_R(t) = \lim_{c \to +\infty}\tau_R(t) = t.
\end{equation}

En consecuencia,
\begin{equation}
\vec{E}_1(\vec{r}, t) \xrightarrow[\beta \to 0]{} \frac{q}{4\pi\epsilon_0}\frac{\vec{n}(t)}{R^2(t)},
\end{equation}
expresión que tiene la misma estructura que la del campo electrostático de la partícula puntual, excepto por la dependencia temporal en $\vec{R}$. La componente $\vec{E}_1$ decae a cero como $\frac{1}{R^2}$ cuando $R$ tiende a infinito, y recibe el nombre de campo estático.

La componente $\vec{E}_2$ dada por (4.52) no tiene un análogo electrostático, y tiende a cero como $\frac{1}{R}$ cuando $R$ tiende a infinito. Este es el llamado campo de radiación, y es el término dominante para grandes valores de $R$. Nótese que el campo de radiación es nulo en ausencia de aceleración, es decir, cuando $\dot{\vec{\beta}} = 0$.

\subsection{Vector de Poynting en la región alejada de la fuente}

En una región espacial alejada de la partícula, es decir, de la fuente del campo electromagnético, el campo eléctrico (4.50) está esencialmente dominado por el término de radiación, es decir,
\begin{equation}
\vec{E}(\vec{r}, t) \simeq \vec{E}_2(\vec{r}, t).
\end{equation}

El vector de Poynting asociado al campo electromagnético de la partícula en movimiento arbitrario es
\begin{equation}
\vec{S} = \epsilon_0 c^2\vec{E} \times \vec{B} = \epsilon_0 c\left[\vec{E} \times (\vec{n} \times \vec{E})\right] = \epsilon_0 c\left[\vec{n}E^2 - \vec{E}(\vec{n} \cdot \vec{E})\right].
\end{equation}

Nótese que en la región alejada de la fuente se tiene que
\begin{equation}
\vec{n} \cdot \vec{E} \simeq \vec{n} \cdot \vec{E}_2 = 0.
\end{equation}

Luego
\begin{equation}
\vec{S} = \epsilon_0 cE_2^2\vec{n},
\end{equation}
de donde
\begin{equation}
\vec{S}(\vec{r}, t) = \frac{q^2}{16\pi^2\epsilon_0c}\left[\frac{\left|\vec{n}(\tau) \times \left\{\left[\vec{n}(\tau) - \vec{\beta}(\tau)\right] \times \dot{\vec{\beta}}(\tau)\right\}\right|^2}{\left[1 - \vec{n}(\tau) \cdot \vec{\beta}(\tau)\right]^3 R^2(\tau)}\vec{n}(\tau)\right]_{\tau=\tau_R(t)}.
\end{equation}

En lo sucesivo nos dedicaremos, por simplicidad, a tratar la formulación de la Teoría Clásica de la Radiación en el límite no relativista. Para una formulación relativista de esta teoría el lector puede consultar la referencia [21].

El vector de Poynting en el límite no relativista puede ser escrito como
\begin{equation}
\vec{S}(\vec{r}, t) = \frac{q^2}{16\pi^2\epsilon_0c}\left|\vec{n}(t) \times \left[\vec{n}(t) \times \dot{\vec{\beta}}(t)\right]\right|^2\frac{\vec{n}(t)}{R^2(t)},
\end{equation}
o bien
\begin{equation}
\vec{S}(\vec{r}, t) = \frac{q^2}{16\pi^2\epsilon_0c^3}\left|\vec{n}(t) \times \left[\vec{n}(t) \times \dot{\vec{v}}(t)\right]\right|^2\frac{\vec{n}(t)}{R^2(t)}.
\end{equation}

Pero
\begin{equation}
\vec{n} \times [\vec{n} \times \dot{\vec{v}}] = \vec{n}(\vec{n} \cdot \dot{\vec{v}}) - \dot{\vec{v}}.
\end{equation}

Por lo tanto
\begin{align}
\left|\vec{n} \times [\vec{n} \times \dot{\vec{v}}]\right|^2 &= \left|\vec{n}(\vec{n} \cdot \dot{\vec{v}}) - \dot{\vec{v}}\right|^2 = (\vec{n} \cdot \dot{\vec{v}})^2 + \dot{v}^2 - 2(\vec{n} \cdot \dot{\vec{v}})^2 \nonumber \\
&= \dot{v}^2 - (\vec{n} \cdot \dot{\vec{v}})^2 = \dot{v}^2\left(1 - \cos^2[\theta(t)]\right) = \dot{v}^2\sin^2[\theta(t)],
\end{align}
siendo $\theta = \theta(t)$ el ángulo formado entre los vectores $\dot{\vec{v}}$ y $\vec{n}$. Sustituyendo el resultado anterior en la expresión (4.56) tenemos finalmente que
\begin{equation}
\vec{S}(\vec{r}, t) = \frac{q^2\dot{v}^2(t)\sin^2[\theta(t)]}{16\pi^2\epsilon_0c^3R^2(t)}\vec{n}(t).
\end{equation}

Este es el valor del vector de Poynting en la región alejada de la fuente del campo electromagnético.

\subsection{Potencia radiada. Fórmula de Larmor}

Como vimos antes, la potencia total radiada en todas las direcciones es
\begin{equation}
P = \oint_S \vec{S} \cdot d\vec{A},
\end{equation}
donde la integración se extiende sobre una superficie cerrada que contiene en su interior a la partícula. Para efectuar esta integración, resulta conveniente tomar una superficie esférica de radio $R$ centrada en el origen.

La forma más natural de realizar la integración anterior es considerar que $d\vec{A} = R^2\vec{n}\,d\Omega$, siendo $d\Omega$ el elemento de ángulo sólido. Con ello se tiene que
\begin{equation}
P = \oint_S \vec{S} \cdot d\vec{A} = \oint_S \vec{S} \cdot \vec{n}R^2\,d\Omega = \oint_S S_nR^2\,d\Omega,
\end{equation}
donde $S_n$ es la componente normal del vector de Poynting, la cual, como vimos antes, vale
\begin{equation}
S_n = \frac{q^2\dot{v}^2\sin^2\theta}{16\pi^2\epsilon_0c^3R^2}.
\end{equation}

Sustituyendo este valor en la expresión de la potencia radiada obtenemos
\begin{equation}
P = \oint_S \frac{q^2\dot{v}^2\sin^2\theta}{16\pi^2\epsilon_0c^3R^2}R^2\,d\Omega = \frac{q^2\dot{v}^2}{16\pi^2\epsilon_0c^3}\oint_S \sin^2\theta\,d\Omega.
\end{equation}

Para realizar la integral $\oint_S \sin^2\theta\,d\Omega$ haremos uso de un sistema de referencia cuyo eje $z$ sea paralelo al vector aceleración $\dot{\vec{v}}$. En tal sistema tendremos que
\begin{equation}
\oint_S \sin^2\theta\,d\Omega = \int_0^{2\pi}d\phi\int_0^{\pi}\sin^2\theta\sin\theta\,d\theta = 2\pi\int_0^{\pi}\sin^3\theta\,d\theta = 2\pi\cdot\frac{4}{3} = \frac{8\pi}{3}.
\end{equation}

Por ende, la potencia total radiada por la partícula cargada en movimiento a velocidades no relativistas es
\begin{equation}
P = \frac{q^2\dot{v}^2}{16\pi^2\epsilon_0c^3}\cdot\frac{8\pi}{3} = \frac{q^2\dot{v}^2}{6\pi\epsilon_0c^3}.
\end{equation}

Esta es la conocida fórmula de Larmor para la radiación de una carga puntual. Como puede apreciarse, la potencia radiada es proporcional al cuadrado de la aceleración de la partícula, en el límite no relativista.

\section{Dispersión de la radiación}

\subsection{Dispersión de la radiación por partículas aisladas}

La interacción entre la radiación (ondas electromagnéticas) y la materia conduce a efectos como: (a) absorción de la radiación (disminución del flujo de energía a través de una superficie dada), (b) dispersión o esparcimiento (scattering) (redirección de la radiación en múltiples direcciones), (c) refracción (cambio de la dirección de propagación), (d) reflexión (redirección de la radiación en una única dirección), etcétera. Aquí nos ocuparemos de estudiar el fenómeno de dispersión de la radiación.

La dispersión de la radiación por un medio material se debe a la interacción de la onda electromagnética incidente con las partículas microscópicas que conforman el medio. Consideremos en principio el caso más sencillo: dispersión por una partícula aislada.

Supongamos una onda electromagnética plana y monocromática que incide sobre una partícula cargada aislada de masa $m$ y carga $q$. El campo eléctrico de la onda incidente puede escribirse como
\begin{equation}
\vec{E}_i = \vec{E}_0e^{i(k_iz-\omega t)} = \vec{E}_0\cos(k_iz-\omega t) + i\vec{E}_0\sin(k_iz-\omega t),
\end{equation}
donde hemos considerado, sin pérdida de generalidad, que la onda se propaga a lo largo del eje $z$ positivo.

La partícula expuesta a este campo oscilante adquiere una fuerza dada por $\vec{F} = q\vec{E}_i$. Si no hay otras fuerzas actuando sobre la partícula (o si estas son despreciables frente a la fuerza del campo), entonces la ecuación de movimiento es
\begin{equation}
m\ddot{\vec{r}} = q\vec{E}_i,
\end{equation}
de donde
\begin{equation}
\ddot{\vec{r}} = \frac{q}{m}\vec{E}_0e^{i(k_iz-\omega t)}.
\end{equation}

Asumamos que la partícula estaba originalmente en reposo en la posición $\vec{r} = \vec{0}$. Entonces, dado que estamos tratando con un campo eléctrico armónico, esperamos que el movimiento de la partícula sea también armónico con la misma frecuencia que el campo. Integrando dos veces con respecto al tiempo, obtenemos
\begin{equation}
\vec{r}(t) = -\frac{q\vec{E}_0}{m\omega^2}e^{i(k_iz-\omega t)}.
\end{equation}

La partícula que se mueve de esta forma acelera, y esta aceleración conduce a la radiación de ondas electromagnéticas. En otras palabras, la partícula expuesta a la radiación incidente se convierte en un emisor de radiación.

La aceleración de la partícula es
\begin{equation}
\ddot{\vec{r}}(t) = \frac{q\vec{E}_0}{m}e^{i(k_iz-\omega t)}.
\end{equation}

Para un observador distante, esta aceleración da lugar a un campo eléctrico dispersado (scattered) en la dirección $\vec{n}$ dado por
\begin{equation}
\vec{E}_s = \frac{q}{4\pi\epsilon_0c^2}\frac{\vec{n} \times (\vec{n} \times \ddot{\vec{r}})}{R},
\end{equation}
donde $R$ es la distancia al observador y $\vec{n}$ es un vector unitario en la dirección del observador.

Sustituyendo la aceleración en la expresión anterior, hallamos
\begin{equation}
\vec{E}_s = \frac{q^2}{4\pi\epsilon_0c^2m}\frac{\vec{n} \times (\vec{n} \times \vec{E}_0)}{R}e^{i(k_iz-\omega t)}.
\end{equation}

Como la onda dispersada se propaga en la dirección $\vec{n}$, el campo en el punto de observación exhibirá un retardo asociado al tiempo que tarda la radiación en viajar desde la partícula hasta el observador. Suponiendo que la partícula se encuentra en el origen de coordenadas, este retardo se expresa simplemente como $\frac{R}{c}$. Por lo tanto, la expresión correcta para el campo eléctrico dispersado es
\begin{equation}
\vec{E}_s = \frac{q^2}{4\pi\epsilon_0c^2m}\frac{\vec{n} \times (\vec{n} \times \vec{E}_0)}{R}e^{i(k_iz-\omega t-\frac{\omega R}{c})}.
\end{equation}

La intensidad de la radiación dispersada es proporcional al módulo al cuadrado del campo eléctrico. Así,
\begin{equation}
I_s \propto |\vec{E}_s|^2 \propto \left|\frac{q^2}{4\pi\epsilon_0c^2m}\frac{\vec{n} \times (\vec{n} \times \vec{E}_0)}{R}\right|^2.
\end{equation}

Usemos la identidad vectorial $\vec{n} \times (\vec{n} \times \vec{E}_0) = \vec{n}(\vec{n} \cdot \vec{E}_0) - \vec{E}_0$. Puesto que la onda incidente es una onda plana, tenemos que $\vec{E}_0 \perp \vec{k}_i$, es decir, $\vec{E}_0 \perp \hat{z}$. Además, si denotamos por $\theta$ el ángulo entre la dirección de dispersión $\vec{n}$ y la dirección de incidencia $\hat{z}$, y por $\phi$ el ángulo azimutal medido desde la dirección de polarización $\vec{E}_0$ en el plano perpendicular a $\hat{z}$, entonces
\begin{equation}
\vec{n} \cdot \vec{E}_0 = |\vec{E}_0|\sin\theta\cos\phi.
\end{equation}

Usando la identidad vectorial mencionada y después de algunos cálculos, podemos mostrar que
\begin{equation}
|\vec{n} \times (\vec{n} \times \vec{E}_0)|^2 = |\vec{E}_0|^2(1 - \sin^2\theta\cos^2\phi).
\end{equation}

Para calcular la sección eficaz diferencial de dispersión, debemos relacionar la intensidad de la radiación dispersada con la intensidad de la radiación incidente. La intensidad de la radiación incidente es
\begin{equation}
I_i = \frac{1}{2}\epsilon_0c|\vec{E}_0|^2.
\end{equation}

La sección eficaz diferencial de dispersión, $\frac{d\sigma}{d\Omega}$, se define como
\begin{equation}
\frac{d\sigma}{d\Omega} = \frac{R^2I_s}{I_i}.
\end{equation}

Sustituyendo las expresiones para $I_s$ e $I_i$, obtenemos
\begin{equation}
\frac{d\sigma}{d\Omega} = \frac{r_0^2}{2}(1 - \sin^2\theta\cos^2\phi),
\end{equation}
donde hemos definido
\begin{equation}
r_0 = \frac{q^2}{4\pi\epsilon_0mc^2},
\end{equation}
que tiene dimensiones de longitud. Para el caso del electrón, $r_0 = 2.82 \times 10^{-15}$ m, valor conocido como el radio clásico del electrón.

Si la radiación incidente no está polarizada, debemos promediar sobre todas las posibles orientaciones de $\vec{E}_0$ en el plano perpendicular a la dirección de incidencia. Esto equivale a promediar sobre el ángulo $\phi$. Como
\begin{equation}
\langle\cos^2\phi\rangle = \frac{1}{2\pi}\int_0^{2\pi}\cos^2\phi\,d\phi = \frac{1}{2},
\end{equation}
tenemos
\begin{equation}
\frac{d\sigma}{d\Omega} = r_0^2\left(1 - \frac{1}{2}\sin^2\theta\right) = r_0^2\left(\frac{1 + \cos^2\theta}{2}\right).
\end{equation}

Esta es la fórmula de Thomson para la dispersión de radiación por una partícula libre cargada. La sección eficaz total de dispersión se obtiene integrando sobre todos los ángulos sólidos:
\begin{equation}
\sigma_T = \int\frac{d\sigma}{d\Omega}\,d\Omega = r_0^2\int_0^{2\pi}d\phi\int_0^{\pi}\frac{1 + \cos^2\theta}{2}\sin\theta\,d\theta = \frac{8\pi}{3}r_0^2.
\end{equation}

Para el electrón, $\sigma_T = \frac{8\pi}{3}r_0^2 = 6.65 \times 10^{-29}$ m$^2$, valor conocido como la sección eficaz de Thomson.

\subsection{Dispersión de la radiación por partículas elásticamente ligadas}

En el estudio anterior consideramos el caso de una partícula libre (sin ligaduras) expuesta a radiación electromagnética. En la realidad, los electrones en la materia ordinaria no están libres sino ligados a átomos o moléculas mediante fuerzas de interacción. A continuación, estudiaremos el caso de un electrón ligado a una posición de equilibrio mediante una fuerza de restitución elástica.

Para un electrón ligado elásticamente, la ecuación de movimiento puede escribirse como
\begin{equation}
m\ddot{\vec{r}} = q\vec{E}_i - K\vec{r},
\end{equation}
donde $K$ es la constante de la fuerza restauradora y $\vec{E}_i$ es el campo eléctrico de la onda incidente.

Si el campo eléctrico incidente es armónico, $\vec{E}_i = \vec{E}_0e^{i(k_iz-\omega t)}$, y definimos la frecuencia natural o característica del oscilador como $\omega_0 = \sqrt{\frac{K}{m}}$, entonces la ecuación de movimiento se reescribe como
\begin{equation}
\ddot{\vec{r}} + \omega_0^2\vec{r} = \frac{q}{m}\vec{E}_0e^{i(k_iz-\omega t)}.
\end{equation}

Esta ecuación diferencial corresponde a un oscilador armónico forzado. La solución general consta de la solución homogénea (sin término de fuerza) más una solución particular. La solución homogénea describe oscilaciones a la frecuencia natural $\omega_0$ cuya amplitud disminuye con el tiempo debido a la emisión de radiación. Para tiempos suficientemente largos, sólo queda la solución particular, que describe oscilaciones a la frecuencia de la fuerza aplicada, es decir, a la frecuencia $\omega$ de la radiación incidente.

La solución particular para la ecuación del oscilador forzado es
\begin{equation}
\vec{r}(t) = -\frac{q\vec{E}_0}{m(\omega^2-\omega_0^2)}e^{i(k_iz-\omega t)}.
\end{equation}

Nótese que cuando $\omega \to \omega_0$, la amplitud de oscilación tiende a infinito. Esto es una consecuencia de haber despreciado los mecanismos de amortiguamiento en nuestro modelo. En situaciones reales, siempre existen mecanismos que limitan la amplitud máxima de oscilación.

La aceleración de la partícula es
\begin{equation}
\ddot{\vec{r}}(t) = \frac{q\omega^2\vec{E}_0}{m(\omega^2-\omega_0^2)}e^{i(k_iz-\omega t)}.
\end{equation}

Siguiendo un procedimiento similar al caso de la partícula libre, el campo eléctrico dispersado en la dirección $\vec{n}$ es
\begin{equation}
\vec{E}_s = \frac{q^2\omega^2}{4\pi\epsilon_0c^2m(\omega^2-\omega_0^2)}\frac{\vec{n} \times (\vec{n} \times \vec{E}_0)}{R}e^{i(k_iz-\omega t-\frac{\omega R}{c})}.
\end{equation}

La sección eficaz diferencial de dispersión resulta
\begin{equation}
\frac{d\sigma}{d\Omega} = r_0^2\left(\frac{\omega^2}{\omega^2-\omega_0^2}\right)^2\frac{1 + \cos^2\theta}{2},
\end{equation}
donde $r_0 = \frac{q^2}{4\pi\epsilon_0mc^2}$ es el radio clásico del electrón.

Para radiación no polarizada, la sección eficaz total de dispersión es
\begin{equation}
\sigma = \frac{8\pi}{3}r_0^2\left(\frac{\omega^2}{\omega^2-\omega_0^2}\right)^2.
\end{equation}

Esta fórmula muestra características importantes del fenómeno de dispersión por partículas ligadas:

1. Para $\omega \gg \omega_0$ (frecuencias de la radiación incidente mucho mayores que la frecuencia natural), la sección eficaz tiende al valor de Thomson: $\sigma \to \sigma_T = \frac{8\pi}{3}r_0^2$.

2. Para $\omega \ll \omega_0$ (frecuencias incidentes mucho menores que la frecuencia natural), la sección eficaz es proporcional a $\omega^4$: $\sigma \approx \frac{8\pi}{3}r_0^2\left(\frac{\omega}{\omega_0}\right)^4$. Este comportamiento explica por qué el cielo es azul: la dispersión de la luz solar por las moléculas de la atmósfera es más efectiva para las frecuencias altas (azul) que para las bajas (rojo).

3. Cerca de la resonancia ($\omega \approx \omega_0$), la sección eficaz aumenta dramáticamente. En la realidad, la amplitud está limitada por mecanismos de amortiguamiento.

Un modelo más realista debe incluir un término de amortiguamiento en la ecuación de movimiento:
\begin{equation}
\ddot{\vec{r}} + \gamma\dot{\vec{r}} + \omega_0^2\vec{r} = \frac{q}{m}\vec{E}_0e^{i(k_iz-\omega t)},
\end{equation}
donde $\gamma$ es el coeficiente de amortiguamiento. La solución particular para esta ecuación es
\begin{equation}
\vec{r}(t) = -\frac{q\vec{E}_0}{m[(\omega^2-\omega_0^2) + i\gamma\omega]}e^{i(k_iz-\omega t)}.
\end{equation}

En este caso, la sección eficaz diferencial de dispersión es
\begin{equation}
\frac{d\sigma}{d\Omega} = r_0^2\left|\frac{\omega^2}{(\omega^2-\omega_0^2) + i\gamma\omega}\right|^2\frac{1 + \cos^2\theta}{2},
\end{equation}
o, equivalentemente,
\begin{equation}
\frac{d\sigma}{d\Omega} = r_0^2\frac{\omega^4}{(\omega^2-\omega_0^2)^2 + \gamma^2\omega^2}\frac{1 + \cos^2\theta}{2}.
\end{equation}

La sección eficaz total para radiación no polarizada es
\begin{equation}
\sigma = \frac{8\pi}{3}r_0^2\frac{\omega^4}{(\omega^2-\omega_0^2)^2 + \gamma^2\omega^2}.
\end{equation}

Esta expresión es la fórmula clásica para la dispersión resonante. Cuando $\omega = \omega_0$ (resonancia exacta), la sección eficaz alcanza su valor máximo:
\begin{equation}
\sigma_{\text{max}} = \frac{8\pi}{3}r_0^2\frac{\omega_0^2}{\gamma^2}.
\end{equation}

\subsection{Energía absorbida en el proceso de dispersión}

Cuando una onda electromagnética incide sobre una partícula cargada, parte de la energía de la onda es absorbida por la partícula y posteriormente reemitida en todas direcciones. Este proceso de absorción y reemisión constituye el fenómeno de dispersión.

La potencia media absorbida por una partícula cargada puede calcularse como el producto de la fuerza que ejerce el campo sobre la partícula por la velocidad de la misma, promediado en el tiempo:
\begin{equation}
\langle P_{\text{abs}} \rangle = \langle \vec{F} \cdot \dot{\vec{r}} \rangle = q\langle \vec{E} \cdot \dot{\vec{r}} \rangle.
\end{equation}

Para una partícula elásticamente ligada y con amortiguamiento, tenemos
\begin{equation}
\dot{\vec{r}} = -\frac{iq\omega\vec{E}_0}{m[(\omega^2-\omega_0^2) + i\gamma\omega]}e^{i(k_iz-\omega t)},
\end{equation}
y
\begin{equation}
\vec{E} = \vec{E}_0e^{i(k_iz-\omega t)}.
\end{equation}

El promedio temporal del producto escalar $\vec{E} \cdot \dot{\vec{r}}$ es
\begin{equation}
\langle \vec{E} \cdot \dot{\vec{r}} \rangle = -\frac{iq\omega|\vec{E}_0|^2}{2m[(\omega^2-\omega_0^2) + i\gamma\omega]} + \text{c.c.},
\end{equation}
donde c.c. significa complejo conjugado.

Realizando las operaciones indicadas y tomando la parte real, obtenemos
\begin{equation}
\langle P_{\text{abs}} \rangle = \frac{q^2\gamma\omega^2|\vec{E}_0|^2}{2m[(\omega^2-\omega_0^2)^2 + \gamma^2\omega^2]}.
\end{equation}

Esta expresión muestra que la potencia absorbida alcanza un máximo cuando $\omega = \omega_0$ (resonancia). La potencia total reemitida en forma de radiación dispersada puede calcularse utilizando la fórmula de Larmor:
\begin{equation}
\langle P_{\text{rad}} \rangle = \frac{q^2\langle\dot{v}^2\rangle}{6\pi\epsilon_0c^3} = \frac{q^2\omega^4|\vec{r}|^2}{6\pi\epsilon_0c^3},
\end{equation}
donde hemos usado que $\langle\dot{v}^2\rangle = \omega^4\langle r^2 \rangle$ para movimiento armónico.

Sustituyendo la expresión para $\vec{r}(t)$, encontramos
\begin{equation}
\langle P_{\text{rad}} \rangle = \frac{q^4\omega^4|\vec{E}_0|^2}{12\pi\epsilon_0c^3m^2[(\omega^2-\omega_0^2)^2 + \gamma^2\omega^2]}.
\end{equation}

La diferencia entre la potencia absorbida y la potencia reemitida representa la energía disipada a través de mecanismos no radiativos, como por ejemplo el calentamiento del medio.

\section{Reacción de la radiación}

Hasta ahora hemos calculado el campo electromagnético producido por una partícula cargada en movimiento, pero hemos ignorado un hecho importante: el campo de radiación generado por la partícula actúa sobre la propia partícula que lo produce, modificando su movimiento. Este fenómeno se conoce como reacción de la radiación o fuerza de frenado por radiación.

Cuando una partícula cargada acelera, emite radiación y, por tanto, pierde energía. Esta pérdida de energía debe reflejarse en el movimiento de la partícula mediante una fuerza de reacción.

La fuerza de reacción por radiación puede derivarse a partir de consideraciones energéticas. Según la fórmula de Larmor, la potencia radiada por una partícula no relativista es
\begin{equation}
P = \frac{q^2\dot{v}^2}{6\pi\epsilon_0c^3}.
\end{equation}

La pérdida de energía cinética de la partícula debe ser igual a la energía radiada. Por tanto,
\begin{equation}
\frac{d}{dt}\left(\frac{1}{2}mv^2\right) = -\frac{q^2\dot{v}^2}{6\pi\epsilon_0c^3},
\end{equation}
o, equivalentemente,
\begin{equation}
m\vec{v} \cdot \dot{\vec{v}} = -\frac{q^2\dot{v}^2}{6\pi\epsilon_0c^3}.
\end{equation}

Esta ecuación sugiere que existe una fuerza de reacción por radiación $\vec{F}_{\text{rad}}$ tal que
\begin{equation}
\vec{F}_{\text{rad}} \cdot \vec{v} = -\frac{q^2\dot{v}^2}{6\pi\epsilon_0c^3}.
\end{equation}

La forma más simple de la fuerza que satisface esta condición es
\begin{equation}
\vec{F}_{\text{rad}} = -\frac{q^2}{6\pi\epsilon_0c^3}\ddot{\vec{v}}.
\end{equation}

Esta expresión, derivada por primera vez por Abraham y Lorentz, es conocida como la fuerza de Abraham-Lorentz. La ecuación de movimiento de la partícula, incluyendo esta fuerza de reacción, es
\begin{equation}
m\ddot{\vec{r}} = \vec{F}_{\text{ext}} + \vec{F}_{\text{rad}} = \vec{F}_{\text{ext}} - \frac{q^2}{6\pi\epsilon_0c^3}\dddot{\vec{r}},
\end{equation}
donde $\vec{F}_{\text{ext}}$ representa todas las fuerzas externas que actúan sobre la partícula, y $\dddot{\vec{r}}$ es la derivada tercera de la posición respecto al tiempo.

Esta ecuación diferencial de tercer orden presenta algunos problemas conceptuales. El más grave es que admite soluciones que violan el principio de causalidad: la partícula puede empezar a acelerar antes de que se aplique una fuerza externa (soluciones pre-aceleradas) o continuar acelerando después de que cese la fuerza externa (soluciones auto-aceleradas).

Para resolver estos problemas, podemos recurrir a un enfoque diferente. Consideremos un oscilador armónico amortiguado por radiación:
\begin{equation}
\ddot{\vec{r}} + \omega_0^2\vec{r} = -\frac{q^2}{6\pi\epsilon_0c^3m}\dddot{\vec{r}}.
\end{equation}

Definiendo el tiempo característico de radiación $\tau = \frac{q^2}{6\pi\epsilon_0c^3m}$, la ecuación se reescribe como
\begin{equation}
\ddot{\vec{r}} + \omega_0^2\vec{r} = -\tau\dddot{\vec{r}}.
\end{equation}

Para el electrón, $\tau \approx 6.24 \times 10^{-24}$ s, un tiempo extremadamente corto. Esto sugiere que, en la mayoría de las situaciones prácticas, podemos aproximar $\dddot{\vec{r}} \approx \omega_0^2\dot{\vec{r}}$, lo que conduce a
\begin{equation}
\ddot{\vec{r}} + \tau\omega_0^2\dot{\vec{r}} + \omega_0^2\vec{r} = 0.
\end{equation}

Esta es la ecuación de un oscilador armónico con amortiguamiento, donde el coeficiente de amortiguamiento $\gamma = \tau\omega_0^2$ surge como consecuencia de la reacción por radiación. Este enfoque evita los problemas de causalidad mencionados anteriormente.

La energía perdida por la partícula debido a la radiación se manifiesta como un amortiguamiento en su movimiento. Este amortiguamiento es especialmente relevante para electrones en aceleradores de partículas y en átomos sometidos a campos intensos.

\subsection{Función de distribución espectral}

La radiación emitida por una partícula acelerada no es, en general, monocromática, sino que abarca un espectro continuo de frecuencias. Para caracterizar esta distribución de energía en función de la frecuencia, introducimos la función de distribución espectral.

Si una partícula cargada se mueve de forma no relativista, la potencia total radiada está dada por la fórmula de Larmor:
\begin{equation}
P = \frac{q^2\dot{v}^2}{6\pi\epsilon_0c^3}.
\end{equation}

Pero esta fórmula no proporciona información sobre cómo se distribuye esta potencia entre las diferentes frecuencias. Para obtener esta distribución, necesitamos descomponer el movimiento de la partícula en sus componentes de Fourier.

La función de distribución espectral $I(\omega)$ se define de modo que $I(\omega)d\omega$ representa la potencia radiada en el intervalo de frecuencias $[\omega, \omega + d\omega]$. Así, la potencia total radiada puede expresarse como:
\begin{equation}
P = \int_0^{\infty} I(\omega) d\omega.
\end{equation}

Para calcular $I(\omega)$, consideramos la transformada de Fourier de la aceleración de la partícula:
\begin{equation}
\vec{a}(\omega) = \frac{1}{2\pi}\int_{-\infty}^{\infty}\dot{\vec{v}}(t)e^{i\omega t}dt.
\end{equation}

Puede demostrarse que la función de distribución espectral está relacionada con esta transformada mediante la expresión:
\begin{equation}
I(\omega) = \frac{q^2\omega^2}{3\pi\epsilon_0c^3}|\vec{a}(\omega)|^2.
\end{equation}

Esta expresión es general y válida para cualquier movimiento no relativista de la partícula cargada.

\subsection{Función de distribución espectral para el oscilador armónico simple}

Consideremos el caso de un oscilador armónico simple. La ecuación de movimiento es:
\begin{equation}
\ddot{\vec{r}} + \omega_0^2\vec{r} = 0,
\end{equation}
cuya solución puede escribirse como:
\begin{equation}
\vec{r}(t) = \vec{r}_0\cos(\omega_0t),
\end{equation}
donde $\vec{r}_0$ es la amplitud de oscilación.

La aceleración de la partícula es:
\begin{equation}
\dot{\vec{v}}(t) = -\omega_0^2\vec{r}_0\cos(\omega_0t).
\end{equation}

La transformada de Fourier de esta aceleración es:
\begin{equation}
\vec{a}(\omega) = -\frac{\omega_0^2\vec{r}_0}{2}\left[\delta(\omega - \omega_0) + \delta(\omega + \omega_0)\right],
\end{equation}
donde $\delta$ es la función delta de Dirac.

Sustituyendo en la expresión para $I(\omega)$, obtenemos:
\begin{equation}
I(\omega) = \frac{q^2\omega^2\omega_0^4r_0^2}{12\pi\epsilon_0c^3}\left[\delta(\omega - \omega_0) + \delta(\omega + \omega_0)\right]^2.
\end{equation}

Como estamos interesados solo en frecuencias positivas y teniendo en cuenta las propiedades de la función delta, podemos simplificar a:
\begin{equation}
I(\omega) = \frac{q^2\omega_0^6r_0^2}{6\pi\epsilon_0c^3}\delta(\omega - \omega_0) \quad \text{para} \quad \omega > 0.
\end{equation}

Esta expresión muestra que un oscilador armónico simple emite radiación a una única frecuencia: su frecuencia natural $\omega_0$. Es decir, la radiación es monocromática. La potencia total radiada es:
\begin{equation}
P = \int_0^{\infty}I(\omega)d\omega = \frac{q^2\omega_0^6r_0^2}{6\pi\epsilon_0c^3},
\end{equation}
que coincide con el resultado que se obtendría aplicando directamente la fórmula de Larmor al movimiento del oscilador.

\subsection{Función de distribución espectral para el oscilador amortiguado}

En la realidad, un oscilador no puede mantener sus oscilaciones indefinidamente debido a la pérdida de energía por radiación. Consideremos ahora un oscilador amortiguado cuya ecuación de movimiento es:
\begin{equation}
\ddot{\vec{r}} + \gamma\dot{\vec{r}} + \omega_0^2\vec{r} = 0,
\end{equation}
donde $\gamma$ es el coeficiente de amortiguamiento.

Para un amortiguamiento débil ($\gamma < 2\omega_0$), la solución es:
\begin{equation}
\vec{r}(t) = \vec{r}_0e^{-\gamma t/2}\cos(\omega't),
\end{equation}
donde $\omega' = \sqrt{\omega_0^2 - \gamma^2/4}$ es la frecuencia angular de las oscilaciones amortiguadas.

La aceleración es:
\begin{equation}
\dot{\vec{v}}(t) = \vec{r}_0e^{-\gamma t/2}\left[\frac{\gamma^2}{4}\cos(\omega't) - \gamma\omega'\sin(\omega't) - \omega'^2\cos(\omega't)\right].
\end{equation}

Para calcular la transformada de Fourier de esta aceleración, usamos:
\begin{equation}
\vec{a}(\omega) = \frac{1}{2\pi}\int_{-\infty}^{\infty}\dot{\vec{v}}(t)e^{i\omega t}dt.
\end{equation}

Después de realizar la integración y consideraciones algebraicas, obtenemos:
\begin{equation}
|\vec{a}(\omega)|^2 = \frac{\omega_0^4r_0^2}{4}\frac{\gamma^2\omega^2 + (\omega^2 - \omega_0^2)^2}{[(\omega^2 - \omega_0^2)^2 + \gamma^2\omega^2]^2}.
\end{equation}

Por lo tanto, la función de distribución espectral es:
\begin{equation}
I(\omega) = \frac{q^2\omega_0^4r_0^2}{12\pi\epsilon_0c^3}\frac{\omega^2[\gamma^2\omega^2 + (\omega^2 - \omega_0^2)^2]}{[(\omega^2 - \omega_0^2)^2 + \gamma^2\omega^2]^2}.
\end{equation}

A diferencia del oscilador armónico simple, que emite radiación a una única frecuencia, el oscilador amortiguado emite un espectro continuo de frecuencias. Sin embargo, la distribución muestra un pico pronunciado cerca de $\omega = \omega_0$. 

Para $\gamma \ll \omega_0$, podemos aproximar cerca de la resonancia:
\begin{equation}
I(\omega) \approx \frac{q^2\omega_0^6r_0^2}{12\pi\epsilon_0c^3}\frac{\gamma^2}{(\omega - \omega_0)^2 + \gamma^2/4} \quad \text{para} \quad \omega \approx \omega_0.
\end{equation}

Esta expresión corresponde a una distribución lorentziana centrada en $\omega_0$ con un ancho a media altura de $\gamma$. El coeficiente de amortiguamiento $\gamma$ determina el ancho del espectro de emisión: cuanto mayor es $\gamma$, más amplio es el rango de frecuencias emitidas.

En el límite $\gamma \to 0$, recuperamos el caso del oscilador armónico simple con radiación monocromática a la frecuencia $\omega_0$.

Para completar el análisis, podemos calcular la potencia total radiada integrando $I(\omega)$ sobre todas las frecuencias positivas:
\begin{equation}
P = \int_0^{\infty}I(\omega)d\omega = \frac{q^2\omega_0^4r_0^2\gamma}{12\pi\epsilon_0c^3}.
\end{equation}

Esta expresión muestra cómo la potencia radiada depende del amortiguamiento. A medida que $\gamma$ aumenta, la potencia radiada disminuye debido a la menor amplitud de las oscilaciones.

El amortiguamiento en este modelo puede tener múltiples orígenes. Uno de ellos es la propia reacción por radiación, que extrae energía del oscilador. Podemos establecer una relación entre el coeficiente de amortiguamiento $\gamma$ y la reacción por radiación:
\begin{equation}
\gamma_{\text{rad}} = \frac{q^2\omega_0^2}{6\pi\epsilon_0mc^3}.
\end{equation}

Este valor, conocido como amortiguamiento radiativo, representa la tasa mínima de amortiguamiento que tendría un oscilador debido únicamente a la pérdida de energía por radiación.

Para un electrón oscilando a frecuencias ópticas ($\omega_0 \sim 10^{15}$ Hz), el amortiguamiento radiativo es del orden de $\gamma_{\text{rad}} \sim 10^8$ Hz, lo que corresponde a un tiempo de vida radiativo de aproximadamente $10^{-8}$ s.

En sistemas reales, existen además otros mecanismos de amortiguamiento no radiativos, como colisiones o interacciones con el medio, que contribuyen al valor total de $\gamma$.

La función de distribución espectral para un oscilador amortiguado tiene importantes aplicaciones en la física atómica y molecular, donde los electrones ligados pueden modelarse como osciladores amortiguados. El espectro de emisión resultante explica características fundamentales de la radiación térmica y los espectros atómicos.

Examinemos ahora algunos casos límite:

1. Para frecuencias muy bajas ($\omega \ll \omega_0$), la función de distribución espectral se comporta como:
\begin{equation}
I(\omega) \approx \frac{q^2\omega^2r_0^2}{12\pi\epsilon_0c^3} \quad \text{para} \quad \omega \ll \omega_0.
\end{equation}
Es decir, $I(\omega) \propto \omega^2$ para bajas frecuencias.

2. Para frecuencias muy altas ($\omega \gg \omega_0$), la función de distribución espectral decae como:
\begin{equation}
I(\omega) \approx \frac{q^2\omega_0^4r_0^2}{12\pi\epsilon_0c^3}\frac{1}{\omega^2} \quad \text{para} \quad \omega \gg \omega_0.
\end{equation}
Es decir, $I(\omega) \propto \omega^{-2}$ para altas frecuencias.

Estos comportamientos asintóticos son características universales de la radiación emitida por sistemas oscilantes amortiguados y tienen importantes implicaciones en la física del espectro electromagnético y la interacción radiación-materia.

La distribución espectral también puede expresarse en términos de la energía total inicial del oscilador, $E_0 = \frac{1}{2}m\omega_0^2r_0^2$:
\begin{equation}
I(\omega) = \frac{q^2\omega_0^2}{6\pi\epsilon_0mc^3}E_0\frac{\omega^2[\gamma^2\omega^2 + (\omega^2 - \omega_0^2)^2]}{[(\omega^2 - \omega_0^2)^2 + \gamma^2\omega^2]^2}.
\end{equation}

Esta forma es útil para relacionar la emisión espectral con la energía almacenada en el sistema, lo que facilita comparaciones con mediciones experimentales.

\section{Desarrollo en multipolos para el campo de radiación}

En el estudio de la radiación electromagnética emitida por sistemas de cargas, es frecuentemente útil desarrollar el campo de radiación en términos de multipolos, similar a lo que hicimos para los campos electrostático y magnetostático. Este enfoque es particularmente valioso cuando la longitud de onda de la radiación es grande comparada con las dimensiones del sistema emisor.

Para desarrollar esta teoría, partimos de las expresiones para los potenciales retardados:
\begin{equation}
\phi(\vec{r}, t) = \frac{1}{4\pi\epsilon_0}\int\frac{\rho(\vec{r}', t-\frac{|\vec{r}-\vec{r}'|}{c})}{|\vec{r}-\vec{r}'|}dV'
\end{equation}
y
\begin{equation}
\vec{A}(\vec{r}, t) = \frac{\mu_0}{4\pi}\int\frac{\vec{J}(\vec{r}', t-\frac{|\vec{r}-\vec{r}'|}{c})}{|\vec{r}-\vec{r}'|}dV'.
\end{equation}

Consideremos un sistema de cargas confinado a una región pequeña del espacio, de modo que $|\vec{r}'| \ll |\vec{r}|$ para todos los puntos del sistema. Esto nos permite aproximar:
\begin{equation}
|\vec{r}-\vec{r}'| \approx r - \vec{n} \cdot \vec{r}',
\end{equation}
donde $\vec{n} = \vec{r}/r$ es un vector unitario en la dirección de observación.

Con esta aproximación, los potenciales retardados se convierten en:
\begin{equation}
\phi(\vec{r}, t) \approx \frac{1}{4\pi\epsilon_0r}\int\rho(\vec{r}', t_r + \frac{\vec{n} \cdot \vec{r}'}{c})dV'
\end{equation}
y
\begin{equation}
\vec{A}(\vec{r}, t) \approx \frac{\mu_0}{4\pi r}\int\vec{J}(\vec{r}', t_r + \frac{\vec{n} \cdot \vec{r}'}{c})dV',
\end{equation}
donde $t_r = t - \frac{r}{c}$ es el tiempo retardado.

Expandimos ahora las densidades retardadas en series de Taylor alrededor del tiempo $t_r$:
\begin{equation}
\rho(\vec{r}', t_r + \frac{\vec{n} \cdot \vec{r}'}{c}) = \rho(\vec{r}', t_r) + \frac{\vec{n} \cdot \vec{r}'}{c}\frac{\partial\rho(\vec{r}', t_r)}{\partial t_r} + \frac{1}{2}\left(\frac{\vec{n} \cdot \vec{r}'}{c}\right)^2\frac{\partial^2\rho(\vec{r}', t_r)}{\partial t_r^2} + \ldots
\end{equation}
y de manera similar para $\vec{J}$.

Sustituyendo estas expansiones en las expresiones de los potenciales y realizando algunas manipulaciones algebraicas, podemos expresar los potenciales en términos de los momentos multipolares del sistema:
\begin{equation}
\phi(\vec{r}, t) = \frac{1}{4\pi\epsilon_0r}\left[q(t_r) + \frac{\vec{n} \cdot \vec{p}(t_r)}{c} + \frac{1}{2c^2}\sum_{i,j}n_in_jQ_{ij}(t_r) + \ldots\right]
\end{equation}
y
\begin{equation}
\vec{A}(\vec{r}, t) = \frac{\mu_0}{4\pi r}\left[\frac{\dot{\vec{p}}(t_r)}{c} + \frac{\vec{n} \times \vec{m}(t_r)}{c} + \ldots\right],
\end{equation}
donde:
- $q(t) = \int\rho(\vec{r}', t)dV'$ es la carga total,
- $\vec{p}(t) = \int\vec{r}'\rho(\vec{r}', t)dV'$ es el momento dipolar eléctrico,
- $Q_{ij}(t) = \int(3r'_ir'_j - r'^2\delta_{ij})\rho(\vec{r}', t)dV'$ es el momento cuadrupolar eléctrico,
- $\vec{m}(t) = \frac{1}{2}\int\vec{r}' \times \vec{J}(\vec{r}', t)dV'$ es el momento dipolar magnético.

A partir de estos potenciales, podemos calcular los campos eléctrico y magnético utilizando:
\begin{equation}
\vec{E} = -\nabla\phi - \frac{\partial\vec{A}}{\partial t}
\end{equation}
y
\begin{equation}
\vec{B} = \nabla \times \vec{A}.
\end{equation}

En la región de radiación (campo lejano), los términos dominantes en los campos son aquellos que decaen como $1/r$, y corresponden a:
\begin{equation}
\vec{E}_{\text{rad}} = -\frac{1}{4\pi\epsilon_0c^2r}\left[\ddot{\vec{p}}(t_r) - \vec{n}(\vec{n} \cdot \ddot{\vec{p}}(t_r)) + \frac{1}{c}\vec{n} \times \dot{\vec{m}}(t_r) + \ldots\right]
\end{equation}
y
\begin{equation}
\vec{B}_{\text{rad}} = \frac{1}{c}\vec{n} \times \vec{E}_{\text{rad}}.
\end{equation}

Este desarrollo multipolar nos permite clasificar la radiación según su origen:
1. Radiación dipolar eléctrica: asociada a $\ddot{\vec{p}}$
2. Radiación dipolar magnética: asociada a $\dot{\vec{m}}$
3. Radiación cuadrupolar eléctrica: asociada a $\ddot{Q}_{ij}$
Y así sucesivamente.

En general, la contribución del multipolo de orden $l$ a la potencia radiada es proporcional a $(kr_0)^{2l}$, donde $k = \omega/c$ es el número de onda y $r_0$ es el tamaño característico del sistema radiante. Para la mayoría de los sistemas, $kr_0 \ll 1$, lo que significa que la radiación está dominada por el término de menor orden que no se anula.

Veamos ahora en detalle las principales contribuciones multipolares.
La ecuación anterior admite una interpretación física muy sencilla. Si en el instante $t'$ la partícula cargada emite una señal electromagnética en forma de luz, entonces esta señal arribará al punto de radiovector $\vec{r}$ en el instante $t$. El tiempo que demora la señal en propagarse es $\Delta t = t - t' = \frac{|\vec{r} - \vec{r}_0(t')|}{c}$.

Nos interesa saber cuál es la corriente $\vec{J}$ asociada a la carga. Para ello recordemos que $\vec{J} = \rho\vec{v}$. Por lo tanto
\begin{equation}
\vec{J}(\vec{r}, t) = q\vec{v}(t)\delta[\vec{r} - \vec{r}_0(t)],
\end{equation}
de donde
\begin{equation}
\vec{A}(\vec{r}, t) = \frac{\mu_0 q}{4\pi}\frac{\vec{v}(t')}{|\vec{r} - \vec{r}_0(t')|} = \frac{\mu_0 q}{4\pi}\frac{\dot{\vec{r}}_0(t')}{|\vec{r} - \vec{r}_0(t')|}.
\end{equation}

Definamos el vector $\vec{R}(t') = \vec{r} - \vec{r}_0(t')$. Dicho vector apunta desde la posición de la partícula en el instante $t'$ hasta el punto del espacio de radiovector $\vec{r}$ en el cual deseamos determinar los potenciales y los campos. El módulo de $\vec{R}(t')$ suele denotarse como $R(t') = |\vec{R}(t')|$. Así,
\begin{equation}
\phi(\vec{r}, t) = \frac{q}{4\pi\epsilon_0}\frac{1}{R(t')}
\end{equation}
y
\begin{equation}
\vec{A}(\vec{r}, t) = \frac{\mu_0 q}{4\pi}\frac{\vec{v}(t')}{R(t')}.
\end{equation}

Utilizando la expresión para la velocidad de la luz en función de las constantes eléctrica y magnética del vacío, es decir, $\mu_0\epsilon_0 = \frac{1}{c^2}$, la expresión (4.39) puede reescribirse como
\begin{equation}
\vec{A}(\vec{r}, t) = \frac{1}{c^2}\frac{\vec{v}(t')}{R(t')}\frac{q}{4\pi\epsilon_0} = \frac{\vec{v}(t')}{c^2}\phi(\vec{r}, t).
\end{equation}

Es interesante ver cómo se transforman los potenciales (4.38) y (4.39) en el caso en que la carga $q$ describa un movimiento relativista. Para ello introducimos el tiempo retardado $t' = t - \frac{R(t')}{c}$ y veamos cómo puede relacionarse una variación infinitesimal $dt'$ con la correspondiente variación infinitesimal $dt$.

Para ello diferenciamos la expresión para el tiempo retardado obteniendo
\begin{equation}
dt' = dt - \frac{1}{c}dR(t').
\end{equation}

Calculemos $dR(t')$. Tenemos que
\begin{equation}
R(t') = |\vec{R}(t')| = |\vec{r} - \vec{r}_0(t')| = \sqrt{[\vec{r} - \vec{r}_0(t')]^2}.
\end{equation}
Luego
\begin{equation}
dR(t') = \frac{d|\vec{R}(t')|}{dt'}dt' = \frac{d}{dt'}\sqrt{[\vec{r} - \vec{r}_0(t')]^2}\,dt' = \frac{1}{2}\frac{1}{\sqrt{[\vec{r} - \vec{r}_0(t')]^2}}\frac{d}{dt'}[\vec{r} - \vec{r}_0(t')]^2\,dt'.
\end{equation}
Pero
\begin{equation}
\frac{d}{dt'}[\vec{r} - \vec{r}_0(t')]^2 = \frac{d}{dt'}[(\vec{r} - \vec{r}_0(t')) \cdot (\vec{r} - \vec{r}_0(t'))] = -2(\vec{r} - \vec{r}_0(t')) \cdot \frac{d\vec{r}_0(t')}{dt'} = -2\vec{R}(t') \cdot \vec{v}(t').
\end{equation}
Por lo tanto
\begin{equation}
dR(t') = -\frac{\vec{R}(t') \cdot \vec{v}(t')}{|\vec{R}(t')|}\,dt' = -\frac{\vec{R}(t') \cdot \vec{v}(t')}{R(t')}\,dt'.
\end{equation}

Sustituyendo (4.42) en (4.41) encontramos
\begin{equation}
dt' = dt + \frac{1}{c}\frac{\vec{R}(t') \cdot \vec{v}(t')}{R(t')}\,dt' = dt + \frac{\vec{n}(t') \cdot \vec{v}(t')}{c}\,dt',
\end{equation}
de donde
\begin{equation}
dt' = \frac{dt}{1 - \frac{\vec{n}(t') \cdot \vec{v}(t')}{c}},
\end{equation}
siendo $\vec{n}(t') = \frac{\vec{R}(t')}{R(t')}$ un vector unitario en la dirección del vector $\vec{R}(t')$. La expresión (4.43) muestra que $dt' \neq dt$, salvo para el caso de una partícula en reposo ($\vec{v} = \vec{0}$).

Definamos la cantidad
\begin{equation}
\kappa = 1 - \frac{\vec{n}(t') \cdot \vec{v}(t')}{c} = 1 - \frac{\vec{R}(t') \cdot \vec{v}(t')}{R(t')c} = 1 - \beta\cos(\alpha),
\end{equation}
donde $\beta = \frac{v}{c}$, $v = |\vec{v}(t')|$ y $\alpha$ es el ángulo entre $\vec{R}(t')$ y $\vec{v}(t')$. Con ello (4.43) puede ser reescrita como
\begin{equation}
dt' = \frac{dt}{\kappa}.
\end{equation}

Veamos ahora cómo queda $\phi(\vec{r}, t)$ cuando lo escribimos en términos del tiempo no retardado. Para ello nos aprovecharemos del hecho de que $q\,dt' = q\,dt$. Entonces
\begin{equation}
\phi(\vec{r}, t) = \frac{q}{4\pi\epsilon_0}\frac{1}{R(t')} = \frac{q\,dt'}{4\pi\epsilon_0}\frac{1}{R(t')}\frac{1}{dt'} = \frac{q\,dt}{4\pi\epsilon_0}\frac{1}{R(t')}\frac{1}{dt'} = \frac{q\,dt}{4\pi\epsilon_0}\frac{1}{R(t')}\frac{\kappa}{dt} = \frac{q}{4\pi\epsilon_0}\frac{\kappa}{R(t')}.
\end{equation}

De manera análoga encontramos para el potencial vectorial
\begin{equation}
\vec{A}(\vec{r}, t) = \frac{\mu_0 q}{4\pi}\frac{\vec{v}(t')}{R(t')} = \frac{\mu_0 q\,dt'}{4\pi}\frac{\vec{v}(t')}{R(t')}\frac{1}{dt'} = \frac{\mu_0 q\,dt}{4\pi}\frac{\vec{v}(t')}{R(t')}\frac{1}{dt'} = \frac{\mu_0 q\,dt}{4\pi}\frac{\vec{v}(t')}{R(t')}\frac{\kappa}{dt} = \frac{\mu_0 q}{4\pi}\frac{\kappa\vec{v}(t')}{R(t')}.
\end{equation}

Las expresiones (4.46) y (4.47) son las expresiones relativísticamente correctas para los potenciales de Liénard-Wiechert. Haciendo uso de la relación $\mu_0\epsilon_0 = \frac{1}{c^2}$ podemos reescribir (4.47) como
\begin{equation}
\vec{A}(\vec{r}, t) = \frac{\mu_0 q}{4\pi}\frac{\kappa\vec{v}(t')}{R(t')} = \frac{1}{c^2}\frac{\kappa\vec{v}(t')}{R(t')}\frac{q}{4\pi\epsilon_0} = \frac{\vec{v}(t')}{c^2}\phi(\vec{r}, t).
\end{equation}

En lo sucesivo omitiremos la dependencia funcional explícita en $t'$. Así, escribiremos $\vec{R} = \vec{r} - \vec{r}_0$, $\vec{v} = \dot{\vec{r}}_0$, $\vec{n} = \frac{\vec{R}}{R}$, $\kappa = 1 - \frac{\vec{n} \cdot \vec{v}}{c}$, etcétera.

\subsection{Campos de Liénard-Wiechert}

Una vez determinados los potenciales del campo electromagnético (potenciales de Liénard-Wiechert) podemos proceder a calcular los campos $\vec{E}$ y $\vec{B}$ generados por una partícula cargada en movimiento arbitrario.

Del electromagnetismo sabemos que
\begin{equation}
\vec{E} = -\nabla\phi - \frac{\partial\vec{A}}{\partial t}
\end{equation}
y
\begin{equation}
\vec{B} = \nabla \times \vec{A}.
\end{equation}

Para calcular $\vec{E}$ debemos, en primer lugar, hallar $\nabla\phi$. De acuerdo con (4.46) tenemos
\begin{equation}
\phi(\vec{r}, t) = \frac{q}{4\pi\epsilon_0}\frac{\kappa}{R}.
\end{equation}
Entonces
\begin{equation}
\nabla\phi = \frac{q}{4\pi\epsilon_0}\nabla\left(\frac{\kappa}{R}\right) = \frac{q}{4\pi\epsilon_0}\left[\frac{\nabla\kappa}{R} - \frac{\kappa\nabla R}{R^2}\right].
\end{equation}

Para el cálculo de $\nabla\kappa$ tengamos en cuenta que $\kappa = 1 - \frac{\vec{n} \cdot \vec{v}}{c}$. Como
\begin{equation}
\vec{n} = \frac{\vec{R}}{R} = \frac{\vec{r} - \vec{r}_0}{|\vec{r} - \vec{r}_0|},
\end{equation}
podemos escribir
\begin{equation}
\nabla\kappa = -\frac{1}{c}\nabla(\vec{n} \cdot \vec{v}) = -\frac{1}{c}\nabla\left(\frac{\vec{R}}{R} \cdot \vec{v}\right) = -\frac{1}{c}\nabla\left(\frac{\vec{R} \cdot \vec{v}}{R}\right).
\end{equation}

Debemos tener presente que $\vec{r}_0$ y $\vec{v}$ son funciones del tiempo retardado $t'$, el cual a su vez depende del vector posición $\vec{r}$. Así, cuando calculamos $\nabla\kappa$ estamos derivando respecto a $\vec{r}$ y debemos hacer uso de la regla de la cadena.

Veamos primeramente cuál es la relación entre $\nabla$ y $\nabla'$, siendo $\nabla'$ el operador gradiente con respecto a $\vec{r}_0$. Sabemos que
\begin{equation}
\vec{R} = \vec{r} - \vec{r}_0.
\end{equation}
Luego
\begin{equation}
\nabla\vec{R} = \nabla\vec{r} - \nabla\vec{r}_0 = \mathbb{I} - \nabla\vec{r}_0,
\end{equation}
donde $\mathbb{I}$ es el tensor unitario ($\mathbb{I} = \sum_{i=1}^3 \vec{e}_i \otimes \vec{e}_i$, siendo $\vec{e}_i$, $i = 1, 2, 3$, los vectores de la base canónica en $\mathbb{R}^3$). 

De la ecuación que define al tiempo retardado
\begin{equation}
t' = t - \frac{R}{c}
\end{equation}
obtenemos
\begin{equation}
\nabla t' = \nabla t - \frac{\nabla R}{c} = -\frac{\nabla R}{c} = -\frac{\vec{R}}{Rc},
\end{equation}
ya que $\nabla R = \frac{\vec{R}}{R}$.

Por lo tanto,
\begin{equation}
\nabla\vec{r}_0 = \frac{d\vec{r}_0}{dt'}\nabla t' = \vec{v}\nabla t' = -\frac{\vec{v}\vec{R}}{Rc}
\end{equation}
y
\begin{equation}
\nabla\vec{R} = \mathbb{I} - \nabla\vec{r}_0 = \mathbb{I} + \frac{\vec{v}\vec{R}}{Rc}.
\end{equation}

El gradiente de un escalar $\varphi$ que depende de $\vec{r}_0$, es decir, $\varphi = \varphi(\vec{r}_0)$, puede calcularse como
\begin{equation}
\nabla\varphi = \nabla'\varphi \cdot \nabla\vec{r}_0 = -\frac{\vec{R}}{Rc}\nabla'\varphi \cdot \vec{v},
\end{equation}
donde usamos la expresión (4.56). Para una función vectorial $\vec{F} = \vec{F}(\vec{r}_0)$ se tiene que
\begin{equation}
\nabla\vec{F} = \nabla'\vec{F} \cdot \nabla\vec{r}_0 = -\frac{\vec{R}}{Rc}\nabla'\vec{F} \cdot \vec{v}.
\end{equation}

Continuemos ahora con el cálculo de $\nabla\kappa$. Se tiene que
\begin{align}
\nabla\kappa &= -\frac{1}{c}\nabla\left(\frac{\vec{R} \cdot \vec{v}}{R}\right) = -\frac{1}{c}\left[\frac{\nabla(\vec{R} \cdot \vec{v})}{R} - \frac{(\vec{R} \cdot \vec{v})\nabla R}{R^2}\right] \nonumber \\
&= -\frac{1}{c}\left[\frac{\nabla\vec{R} \cdot \vec{v} + \vec{R} \cdot \nabla\vec{v}}{R} - \frac{(\vec{R} \cdot \vec{v})\vec{R}}{R^3}\right].
\end{align}

De la expresión (4.57) obtenemos
\begin{equation}
\nabla\vec{R} \cdot \vec{v} = \left(\mathbb{I} + \frac{\vec{v}\vec{R}}{Rc}\right) \cdot \vec{v} = \vec{v} + \frac{v^2\vec{R}}{Rc} = \vec{v} + \frac{v^2\vec{n}}{c}.
\end{equation}

En virtud de (4.59) se tiene
\begin{equation}
\vec{R} \cdot \nabla\vec{v} = -\frac{\vec{R} \cdot \vec{R}}{Rc}\nabla'\vec{v} \cdot \vec{v} = -\frac{R^2}{Rc}\nabla'\vec{v} \cdot \vec{v} = -\frac{R}{c}\nabla'\vec{v} \cdot \vec{v} = -\frac{R}{c}\frac{d}{dt'}\left(\frac{v^2}{2}\right) = -\frac{R}{c}\vec{v} \cdot \dot{\vec{v}} = -\frac{R}{c}\vec{v} \cdot \vec{a},
\end{equation}
donde $\vec{a} = \dot{\vec{v}} = \ddot{\vec{r}}_0$ es la aceleración de la partícula.

Sustituyendo (4.61) y (4.62) en (4.60) encontramos que
\begin{align}
\nabla\kappa &= -\frac{1}{c}\left[\frac{\vec{v} + \frac{v^2\vec{n}}{c} - \frac{R}{c}\vec{v} \cdot \vec{a}}{R} - \frac{(\vec{R} \cdot \vec{v})\vec{R}}{R^3}\right] \nonumber \\
&= -\frac{1}{c}\left[\frac{\vec{v}}{R} + \frac{v^2\vec{n}}{Rc} - \frac{\vec{v} \cdot \vec{a}}{c} - \frac{(\vec{R} \cdot \vec{v})\vec{n}}{R^2}\right] \nonumber \\
&= -\frac{1}{c}\left[\frac{\vec{v}}{R} + \frac{v^2\vec{n}}{Rc} - \frac{\vec{v} \cdot \vec{a}}{c} - \frac{(\vec{n} \cdot \vec{v})\vec{n}}{R}\right] \nonumber \\
&= -\frac{1}{c}\left[\frac{\vec{v} - (\vec{n} \cdot \vec{v})\vec{n}}{R} + \frac{v^2\vec{n}}{Rc} - \frac{\vec{v} \cdot \vec{a}}{c}\right] \nonumber \\
&= -\frac{1}{c}\left[\frac{(\vec{v} \cdot \vec{n})\vec{n} - \vec{v}}{R} + \frac{v^2\vec{n}}{Rc} - \frac{\vec{v} \cdot \vec{a}}{c}\right].
\end{align}

Tengamos en cuenta que
\begin{equation}
\vec{v} - (\vec{n} \cdot \vec{v})\vec{n} = \vec{v} - \vec{n}(\vec{n} \cdot \vec{v})
\end{equation}
es la componente de $\vec{v}$ perpendicular a $\vec{n}$. Escribamos $\vec{v}_{\perp} = \vec{v} - (\vec{n} \cdot \vec{v})\vec{n}$. Entonces (4.63) adquiere la forma
\begin{equation}
\nabla\kappa = \frac{1}{c}\left[\frac{\vec{v}_{\perp}}{R} - \frac{v^2\vec{n}}{Rc} + \frac{\vec{v} \cdot \vec{a}}{c}\right].
\end{equation}

Ahora calculemos $\nabla R$:
\begin{equation}
\nabla R = \nabla|\vec{R}| = \frac{\vec{R}}{R} = \vec{n}.
\end{equation}

Sustituyendo (4.65) y (4.66) en (4.52) se obtiene
\begin{align}
\nabla\phi &= \frac{q}{4\pi\epsilon_0}\left[\frac{\nabla\kappa}{R} - \frac{\kappa\nabla R}{R^2}\right] = \frac{q}{4\pi\epsilon_0}\left[\frac{1}{R}\frac{1}{c}\left(\frac{\vec{v}_{\perp}}{R} - \frac{v^2\vec{n}}{Rc} + \frac{\vec{v} \cdot \vec{a}}{c}\right) - \frac{\kappa\vec{n}}{R^2}\right] \nonumber \\
&= \frac{q}{4\pi\epsilon_0}\left[\frac{1}{Rc}\left(\frac{\vec{v}_{\perp}}{R} - \frac{v^2\vec{n}}{Rc} + \frac{\vec{v} \cdot \vec{a}}{c}\right) - \frac{\kappa\vec{n}}{R^2}\right] \nonumber \\
&= \frac{q}{4\pi\epsilon_0}\left[\frac{\vec{v}_{\perp}}{R^2c} - \frac{v^2\vec{n}}{R^2c^2} + \frac{\vec{v} \cdot \vec{a}}{Rc^2} - \frac{\kappa\vec{n}}{R^2}\right].
\end{align}

Ahora necesitamos calcular el término $\frac{\partial\vec{A}}{\partial t}$. De la expresión (4.47) se tiene que
\begin{equation}
\vec{A}(\vec{r}, t) = \frac{\mu_0 q}{4\pi}\frac{\kappa\vec{v}}{R}.
\end{equation}
Por tanto,
\begin{equation}
\frac{\partial\vec{A}}{\partial t} = \frac{\mu_0 q}{4\pi}\frac{\partial}{\partial t}\left(\frac{\kappa\vec{v}}{R}\right) = \frac{\mu_0 q}{4\pi}\left[\frac{\frac{\partial\kappa}{\partial t}\vec{v} + \kappa\frac{\partial\vec{v}}{\partial t}}{R} - \frac{\kappa\vec{v}}{R^2}\frac{\partial R}{\partial t}\right].
\end{equation}

Para el cálculo de $\frac{\partial\kappa}{\partial t}$, $\frac{\partial\vec{v}}{\partial t}$ y $\frac{\partial R}{\partial t}$ haremos uso de la regla de la cadena y de la relación entre la derivada con respecto a $t$ y la derivada con respecto a $t'$:
\begin{equation}
\frac{\partial}{\partial t} = \frac{\partial t'}{\partial t}\frac{\partial}{\partial t'} = \frac{1}{\frac{\partial t}{\partial t'}}\frac{\partial}{\partial t'} = \frac{1}{\kappa}\frac{\partial}{\partial t'},
\end{equation}
donde en el último paso hemos usado (4.45).

La derivada de $\kappa$ con respecto a $t$ es
\begin{equation}
\frac{\partial\kappa}{\partial t} = \frac{1}{\kappa}\frac{\partial\kappa}{\partial t'} = \frac{1}{\kappa}\frac{\partial}{\partial t'}\left(1 - \frac{\vec{n} \cdot \vec{v}}{c}\right) = -\frac{1}{\kappa}\frac{1}{c}\frac{\partial}{\partial t'}(\vec{n} \cdot \vec{v}) = -\frac{1}{\kappa}\frac{1}{c}\left(\frac{\partial\vec{n}}{\partial t'} \cdot \vec{v} + \vec{n} \cdot \frac{\partial\vec{v}}{\partial t'}\right).
\end{equation}

Para calcular $\frac{\partial\vec{n}}{\partial t'}$ tengamos en cuenta que $\vec{n} = \frac{\vec{R}}{R}$. Luego
\begin{equation}
\frac{\partial\vec{n}}{\partial t'} = \frac{\partial}{\partial t'}\left(\frac{\vec{R}}{R}\right) = \frac{\frac{\partial\vec{R}}{\partial t'}}{R} - \frac{\vec{R}}{R^2}\frac{\partial R}{\partial t'}.
\end{equation}

Pero $\vec{R} = \vec{r} - \vec{r}_0$. Por tanto, $\frac{\partial\vec{R}}{\partial t'} = -\frac{\partial\vec{r}_0}{\partial t'} = -\vec{v}$. Además, $\frac{\partial R}{\partial t'} = \frac{\partial|\vec{R}|}{\partial t'} = \frac{\partial}{\partial t'}\sqrt{\vec{R} \cdot \vec{R}} = \frac{1}{2}\frac{1}{\sqrt{\vec{R} \cdot \vec{R}}}\frac{\partial}{\partial t'}(\vec{R} \cdot \vec{R}) = \frac{1}{2}\frac{1}{R}2\vec{R} \cdot \frac{\partial\vec{R}}{\partial t'} = \frac{\vec{R} \cdot \frac{\partial\vec{R}}{\partial t'}}{R} = \frac{\vec{R} \cdot (-\vec{v})}{R} = -\frac{\vec{R} \cdot \vec{v}}{R} = -\vec{n} \cdot \vec{v}$.

Por ende
\begin{equation}
\frac{\partial\vec{n}}{\partial t'} = \frac{-\vec{v}}{R} - \frac{\vec{R}}{R^2}(-\vec{n} \cdot \vec{v}) = -\frac{\vec{v}}{R} + \frac{\vec{n}(\vec{n} \cdot \vec{v})}{R} = \frac{\vec{n}(\vec{n} \cdot \vec{v}) - \vec{v}}{R} = -\frac{\vec{v}_{\perp}}{R}.
\end{equation}

Usando (4.73) en (4.71) obtenemos
\begin{equation}
\frac{\partial\kappa}{\partial t} = -\frac{1}{\kappa}\frac{1}{c}\left(-\frac{\vec{v}_{\perp}}{R} \cdot \vec{v} + \vec{n} \cdot \vec{a}\right) = \frac{1}{\kappa}\frac{1}{c}\left(\frac{\vec{v}_{\perp} \cdot \vec{v}}{R} - \vec{n} \cdot \vec{a}\right).
\end{equation}

Pero $\vec{v}_{\perp} \cdot \vec{v} = [\vec{v} - \vec{n}(\vec{n} \cdot \vec{v})] \cdot \vec{v} = \vec{v} \cdot \vec{v} - (\vec{n} \cdot \vec{v})(\vec{n} \cdot \vec{v}) = v^2 - (\vec{n} \cdot \vec{v})^2$. Así,
\begin{equation}
\frac{\partial\kappa}{\partial t} = \frac{1}{\kappa}\frac{1}{c}\left[\frac{v^2 - (\vec{n} \cdot \vec{v})^2}{R} - \vec{n} \cdot \vec{a}\right].
\end{equation}

Para $\frac{\partial\vec{v}}{\partial t}$ se tiene que
\begin{equation}
\frac{\partial\vec{v}}{\partial t} = \frac{1}{\kappa}\frac{\partial\vec{v}}{\partial t'} = \frac{\vec{a}}{\kappa}.
\end{equation}

En cuanto a $\frac{\partial R}{\partial t}$, podemos escribir
\begin{equation}
\frac{\partial R}{\partial t} = \frac{1}{\kappa}\frac{\partial R}{\partial t'} = \frac{1}{\kappa}(-\vec{n} \cdot \vec{v}) = -\frac{\vec{n} \cdot \vec{v}}{\kappa}.
\end{equation}

Sustituyendo (4.75), (4.76) y (4.77) en (4.69) hallamos
\begin{align}
\frac{\partial\vec{A}}{\partial t} &= \frac{\mu_0 q}{4\pi}\left[\frac{\frac{1}{\kappa}\frac{1}{c}\left[\frac{v^2 - (\vec{n} \cdot \vec{v})^2}{R} - \vec{n} \cdot \vec{a}\right]\vec{v} + \kappa\frac{\vec{a}}{\kappa}}{R} - \frac{\kappa\vec{v}}{R^2}\left(-\frac{\vec{n} \cdot \vec{v}}{\kappa}\right)\right] \nonumber \\
&= \frac{\mu_0 q}{4\pi}\left[\frac{\frac{1}{\kappa}\frac{1}{c}\left[\frac{v^2 - (\vec{n} \cdot \vec{v})^2}{R} - \vec{n} \cdot \vec{a}\right]\vec{v} + \vec{a}}{R} + \frac{\vec{v}(\vec{n} \cdot \vec{v})}{R^2}\right] \nonumber \\
&= \frac{\mu_0 q}{4\pi}\left[\frac{1}{Rc\kappa}[v^2 - (\vec{n} \cdot \vec{v})^2]\vec{v} - \frac{1}{Rc\kappa}(\vec{n} \cdot \vec{a})\vec{v} + \frac{\vec{a}}{R} + \frac{\vec{v}(\vec{n} \cdot \vec{v})}{R^2}\right].
\end{align}

Tengamos en cuenta que $\mu_0 = \frac{1}{\epsilon_0 c^2}$. En consecuencia,
\begin{align}
\frac{\partial\vec{A}}{\partial t} &= \frac{q}{4\pi\epsilon_0 c^2}\left[\frac{1}{Rc\kappa}[v^2 - (\vec{n} \cdot \vec{v})^2]\vec{v} - \frac{1}{Rc\kappa}(\vec{n} \cdot \vec{a})\vec{v} + \frac{\vec{a}}{R} + \frac{\vec{v}(\vec{n} \cdot \vec{v})}{R^2}\right] \nonumber \\
&= \frac{q}{4\pi\epsilon_0}\left[\frac{1}{R^2c^3\kappa}[v^2 - (\vec{n} \cdot \vec{v})^2]\vec{v} - \frac{1}{Rc^3\kappa}(\vec{n} \cdot \vec{a})\vec{v} + \frac{\vec{a}}{Rc^2} + \frac{\vec{v}(\vec{n} \cdot \vec{v})}{R^2c^2}\right].
\end{align}

Finalmente, podemos calcular el campo eléctrico:
\begin{align}
\vec{E} &= -\nabla\phi - \frac{\partial\vec{A}}{\partial t} \nonumber \\
&= -\frac{q}{4\pi\epsilon_0}\left[\frac{\vec{v}_{\perp}}{R^2c} - \frac{v^2\vec{n}}{R^2c^2} + \frac{\vec{v} \cdot \vec{a}}{Rc^2} - \frac{\kappa\vec{n}}{R^2}\right] \nonumber \\
&- \frac{q}{4\pi\epsilon_0}\left[\frac{1}{R^2c^3\kappa}[v^2 - (\vec{n} \cdot \vec{v})^2]\vec{v} - \frac{1}{Rc^3\kappa}(\vec{n} \cdot \vec{a})\vec{v} + \frac{\vec{a}}{Rc^2} + \frac{\vec{v}(\vec{n} \cdot \vec{v})}{R^2c^2}\right] \nonumber \\
&= \frac{q}{4\pi\epsilon_0}\left[-\frac{\vec{v}_{\perp}}{R^2c} + \frac{v^2\vec{n}}{R^2c^2} - \frac{\vec{v} \cdot \vec{a}}{Rc^2} + \frac{\kappa\vec{n}}{R^2} \right. \nonumber \\
&\left. -\frac{1}{R^2c^3\kappa}[v^2 - (\vec{n} \cdot \vec{v})^2]\vec{v} + \frac{1}{Rc^3\kappa}(\vec{n} \cdot \vec{a})\vec{v} - \frac{\vec{a}}{Rc^2} - \frac{\vec{v}(\vec{n} \cdot \vec{v})}{R^2c^2}\right].
\end{align}

Agrupando términos convenientemente obtenemos
\begin{align}
\vec{E} &= \frac{q}{4\pi\epsilon_0}\left[\frac{\kappa\vec{n}}{R^2} + \frac{v^2\vec{n}}{R^2c^2} - \frac{\vec{v}_{\perp}}{R^2c} - \frac{\vec{v}(\vec{n} \cdot \vec{v})}{R^2c^2} \right. \nonumber \\
&\left. -\frac{1}{R^2c^3\kappa}[v^2 - (\vec{n} \cdot \vec{v})^2]\vec{v} - \frac{\vec{v} \cdot \vec{a}}{Rc^2} + \frac{1}{Rc^3\kappa}(\vec{n} \cdot \vec{a})\vec{v} - \frac{\vec{a}}{Rc^2}\right].
\end{align}

\subsection{Aproximación dipolar eléctrica para el campo de radiación}

Si suponemos que la carga neta del sistema se conserva, es decir, $Q$ es constante en $\tau$, entonces el término de monopolo en la expresión (4.135) no contribuye al campo de radiación por ser independiente del tiempo. El potencial escalar del campo electromagnético en la aproximación dipolar eléctrica puede ser tomado como el segundo término del miembro derecho de (4.135), o sea,
\begin{equation}
\phi_{de}(\vec{r}, t) = \frac{\vec{n} \cdot \dot{\vec{p}}(\tau)}{4\pi\epsilon_0 cr}.
\end{equation}

De la misma manera, podemos elegir el primer término de la expresión (4.139) como el potencial vectorial del campo en la aproximación dipolar eléctrica. Así,
\begin{equation}
\vec{A}_{de}(\vec{r}, t) = \frac{\dot{\vec{p}}(\tau)}{4\pi\epsilon_0 c^2 r}.
\end{equation}

Calculemos el campo electromagnético asociado a estos potenciales. Para el vector inducción magnética tendremos que
\begin{align}
\vec{B}_{de} &= \nabla \times \vec{A}_{de} = \frac{1}{4\pi\epsilon_0 c^2} \nabla \times \frac{\dot{\vec{p}}}{r} = \frac{1}{4\pi\epsilon_0 c^2} \frac{\partial}{\partial r}\left[\frac{\dot{\vec{p}}}{r}\right] \times \vec{n} \nonumber \\
&= \frac{1}{4\pi\epsilon_0 c^2} \left[-\frac{\dot{\vec{p}}}{r^2} + \frac{\partial \dot{\vec{p}}/\partial r}{r}\right] \times \vec{n}.
\end{align}

Pero teniendo en cuenta la expresión (4.130),
\begin{equation}
\frac{\partial \dot{\vec{p}}}{\partial r} = \frac{\partial \dot{\vec{p}}}{\partial \tau}\frac{\partial \tau}{\partial r} = \ddot{\vec{p}} \cdot \left(-\frac{1}{c}\right).
\end{equation}

Luego
\begin{align}
\vec{B}_{de} = -\frac{1}{4\pi\epsilon_0 c^2} \left[\frac{\vec{n} \times \dot{\vec{p}}}{r^2} + \frac{\vec{n} \times \ddot{\vec{p}}}{cr}\right].
\end{align}

El término proporcional a $\frac{1}{r^2}$ puede ser despreciado, pues él pertenece a la parte estática del campo electromagnético y no contribuye al campo de radiación. Por lo tanto
\begin{equation}
\vec{B}_{de}(\vec{r}, t) = -\frac{\vec{n} \times \ddot{\vec{p}}(\tau)}{4\pi\epsilon_0c^3 r}.
\end{equation}

El vector intensidad del campo eléctrico asociado al campo electromagnético puede calcularse como
\begin{align}
\vec{E}_{de} &= -\nabla\phi_{de} - \frac{\partial \vec{A}_{de}}{\partial t} \nonumber \\
&= -\frac{1}{4\pi\epsilon_0 c} \nabla\left[\frac{\vec{n} \cdot \dot{\vec{p}}}{r}\right] - \frac{1}{4\pi\epsilon_0 c^2r}\ddot{\vec{p}}.
\end{align}

Pero
\begin{align}
\nabla\left[\frac{\vec{n} \cdot \dot{\vec{p}}}{r}\right] &= \nabla\left[\frac{\vec{r} \cdot \dot{\vec{p}}}{r^2}\right] \nonumber \\
&= \frac{\nabla(\vec{r} \cdot \dot{\vec{p}})}{r^2} + (\vec{r} \cdot \dot{\vec{p}})\nabla\left[\frac{1}{r^2}\right] \nonumber \\
&= \frac{\dot{\vec{p}}}{r^2} + \frac{\vec{r} \cdot (\vec{n} \partial \dot{\vec{p}}/\partial r)}{r^2} + \frac{\vec{r} \cdot \dot{\vec{p}}}{r^4}(-2\vec{r}) \nonumber \\
&= \frac{\dot{\vec{p}}}{r^2} - \frac{\vec{n}(\vec{n} \cdot \dot{\vec{p}})}{cr^2} - \frac{2(\vec{r} \cdot \dot{\vec{p}})\vec{n}}{r^3} \nonumber \\
&= \frac{\dot{\vec{p}}}{r^2} - \frac{\vec{n}(\vec{n} \cdot \ddot{\vec{p}})}{cr} - \frac{2(\vec{n} \cdot \dot{\vec{p}})\vec{n}}{r^2}.
\end{align}

En consecuencia,
\begin{align}
\vec{E}_{de} &= -\frac{1}{4\pi\epsilon_0 c}\left[\frac{\dot{\vec{p}}}{r^2} - \frac{\vec{n}(\vec{n} \cdot \ddot{\vec{p}})}{cr} - \frac{2(\vec{n} \cdot \dot{\vec{p}})\vec{n}}{r^2}\right] - \frac{1}{4\pi\epsilon_0 c^2r}\ddot{\vec{p}} \nonumber \\
&= \frac{1}{4\pi\epsilon_0 cr^2}[2(\vec{n} \cdot \dot{\vec{p}})\vec{n} - \dot{\vec{p}}] + \frac{1}{4\pi\epsilon_0 c^2r}[\vec{n}(\vec{n} \cdot \ddot{\vec{p}}) - \ddot{\vec{p}}].
\end{align}

Nuevamente podemos despreciar los términos proporcionales a $\frac{1}{r^2}$ que no contribuyen al campo de radiación, en virtud de lo cual
\begin{equation}
\vec{E}_{de} = \frac{1}{4\pi\epsilon_0 c^2r}[\vec{n}(\vec{n} \cdot \ddot{\vec{p}}) - \ddot{\vec{p}}],
\end{equation}
o bien
\begin{equation}
\vec{E}_{de}(\vec{r}, t) = \frac{\vec{n} \times [\vec{n} \times \ddot{\vec{p}}(\tau)]}{4\pi\epsilon_0 c^2 r}.
\end{equation}

Nótese que
\begin{equation}
\vec{E}_{de}(\vec{r}, t) = -c\vec{n} \times \vec{B}_{de}(\vec{r}, t).
\end{equation}

El vector de Poynting asociado al campo de radiación descrito por los vectores (4.142) y (4.143) tiene la forma
\begin{align}
\vec{S} &= \epsilon_0 c^2\vec{E}_{de} \times \vec{B}_{de} = -\epsilon_0 c^3 (\vec{n} \times \vec{B}_{de}) \times \vec{B}_{de} \nonumber \\
&= -\epsilon_0 c^3 [-\vec{n}B_{de}^2 + \vec{B}_{de}(\vec{n} \cdot \vec{B}_{de})] = \epsilon_0c^3 B_{de}^2\vec{n}.
\end{align}

Aquí se ha tenido en cuenta la ortogonalidad entre $\vec{B}_{de}$ y $\vec{n}$. De esta manera,
\begin{equation}
\vec{S}(\vec{r}, t) = \frac{|\vec{n} \times \ddot{\vec{p}}(\tau)|^2}{16\pi^2\epsilon_0 c^3 r^2}\vec{n},
\end{equation}
o bien,
\begin{equation}
\vec{S}(\vec{r}, t) = \frac{|\ddot{\vec{p}}(\tau)|^2\sin^2[\Theta(\tau)]}{16\pi^2\epsilon_0c^3 r^2}\vec{n},
\end{equation}
siendo $\Theta$ el ángulo formado entre los vectores $\ddot{\vec{p}}$ y $\vec{n}$.

Sea $S$ una superficie esférica de radio $r$ con centro en el origen de coordenadas, y elijamos el eje $z$ a lo largo de la dirección y en el sentido del vector $\ddot{\vec{p}}$. Nótese que el vector $\vec{n}$ es perpendicular a la superficie esférica en cualquier punto de la misma. El diferencial de potencia transferido a través del diferencial de área $dS$ en la unidad de tiempo es
\begin{equation}
dP_{de} = \vec{S} \cdot \vec{n} \, dS = \vec{S} \cdot \vec{n} \, r^2 d\Omega = \frac{dP_{de}}{d\Omega}d\Omega.
\end{equation}

Teniendo en cuenta (4.146) es evidente que
\begin{equation}
\frac{dP_{de}}{d\Omega}(t) = \frac{|\ddot{\vec{p}}(\tau)|^2}{16\pi^2\epsilon_0c^3}\sin^2[\Theta(\tau)].
\end{equation}

La expresión anterior nos permite calcular la distribución angular de potencia radiada en la aproximación dipolar eléctrica, y no es más que la fórmula diferencial de Larmor en dicha aproximación. Para obtener la potencia total radiada por el sistema en la aproximación dipolar eléctrica basta con integrar la expresión (4.147) sobre el ángulo sólido, con lo cual obtenemos
\begin{equation}
P_{de}(t) = \frac{|\ddot{\vec{p}}(\tau)|^2}{6\pi\epsilon_0c^3}.
\end{equation}

Nótese que $\tau \to t$ en el límite no relativista, así que podemos sustituir $\tau$ por $t$ en los miembros derechos de las expresiones (4.147) y (4.148).

\subsection{Aproximación dipolar magnética para el campo de radiación}

Para estudiar el campo de radiación en la aproximación dipolar magnética es suficiente tomar los potenciales del campo como
\begin{equation}
\phi_{dm}(\vec{r}, t) = 0
\end{equation}
y
\begin{equation}
\vec{A}_{dm}(\vec{r}, t) = \frac{\dot{\vec{m}}(\tau) \times \vec{n}}{4\pi\epsilon_0 c^3r}.
\end{equation}

El cálculo de los campos eléctrico y magnético es inmediato a partir de las expresiones anteriores. Así,
\begin{equation}
\vec{E}_{dm} = -\frac{\partial \vec{A}_{dm}}{\partial t},
\end{equation}
de donde
\begin{equation}
\vec{E}_{dm}(\vec{r}, t) = -\frac{\ddot{\vec{m}}(\tau) \times \vec{n}}{4\pi\epsilon_0c^3 r}.
\end{equation}

Por otra parte,
\begin{align}
\vec{B}_{dm} &= \nabla\times \vec{A}_{dm} = \frac{1}{4\pi\epsilon_0 c^3}\nabla\times\frac{\dot{\vec{m}} \times \vec{n}}{r} \nonumber \\
&= \frac{1}{4\pi\epsilon_0 c^3}\left\{(\nabla \cdot \vec{n})\frac{\dot{\vec{m}}}{r} - \vec{n}\nabla \cdot \left(\frac{\dot{\vec{m}}}{r}\right) + (\vec{n} \cdot \nabla)\frac{\dot{\vec{m}}}{r} - \left(\frac{\dot{\vec{m}}}{r}\right) \cdot \nabla\vec{n} \right\}.
\end{align}

Pero
\begin{align}
\nabla \cdot \vec{n} &= \nabla \cdot \frac{\vec{r}}{r} = \frac{\nabla \cdot \vec{r}}{r} + \vec{r} \cdot \nabla\frac{1}{r} \nonumber \\
&= \frac{3}{r} - \frac{\vec{r} \cdot \vec{r}}{r^3} = \frac{2}{r},
\end{align}

\begin{align}
\nabla \cdot \left(\frac{\dot{\vec{m}}}{r}\right) &= \frac{\nabla \cdot \dot{\vec{m}}}{r} + \dot{\vec{m}} \cdot \nabla\frac{1}{r} \nonumber \\
&= \frac{\vec{n} \cdot \partial\dot{\vec{m}}/\partial r}{r} - \frac{\dot{\vec{m}} \cdot \vec{r}}{r^3} \nonumber \\
&= -\frac{\vec{n} \cdot \ddot{\vec{m}}}{cr} - \frac{\dot{\vec{m}} \cdot \vec{n}}{r^2},
\end{align}

\begin{align}
(\vec{n} \cdot \nabla)\frac{\dot{\vec{m}}}{r} &= \vec{n} \cdot \vec{n}\frac{\partial}{\partial r}\frac{\dot{\vec{m}}}{r} \nonumber \\
&= \frac{\partial\dot{\vec{m}}/\partial r}{r} - \frac{\dot{\vec{m}}}{r^2} \nonumber \\
&= -\frac{\ddot{\vec{m}}}{cr} - \frac{\dot{\vec{m}}}{r^2}
\end{align}
y
\begin{align}
\left(\frac{\dot{\vec{m}}}{r}\right) \cdot \nabla\vec{n} &= \frac{\dot{\vec{m}}}{r} \cdot \nabla\frac{\vec{r}}{r} \nonumber \\
&= \frac{\dot{\vec{m}}}{r} \cdot \frac{\nabla\vec{r}}{r} + \frac{\dot{\vec{m}} \cdot \vec{r}}{r}\nabla\frac{1}{r} \nonumber \\
&= \frac{1}{r^2}(\dot{\vec{m}} \cdot \vec{n})\vec{n}.
\end{align}

En consecuencia,
\begin{align}
\vec{B}_{dm} &= \frac{1}{4\pi\epsilon_0 c^3}\left\{\frac{2\dot{\vec{m}}}{r^2} + \vec{n}\left[\frac{\vec{n} \cdot \ddot{\vec{m}}}{cr} + \frac{\dot{\vec{m}} \cdot \vec{n}}{r^2}\right] \right. \nonumber \\
&\left. -\frac{\ddot{\vec{m}}}{cr} - \frac{\dot{\vec{m}}}{r^2} - \frac{1}{r^2}(\dot{\vec{m}} \cdot \vec{n})\vec{n}\right\} \nonumber \\
&= \frac{(\dot{\vec{m}} \cdot \vec{n})\vec{n}}{2\pi\epsilon_0c^3 r^2} + \frac{1}{4\pi\epsilon_0c^4 r}[\vec{n}(\vec{n} \cdot \ddot{\vec{m}}) - \ddot{\vec{m}}].
\end{align}

Despreciando el término proporcional a $\frac{1}{r^2}$ obtenemos finalmente que
\begin{equation}
\vec{B}_{dm}(\vec{r}, t) = -\frac{\vec{n} \times [\vec{m}(\tau) \times \vec{n}]}{4\pi\epsilon_0 c^4 r}.
\end{equation}

Nótese que
\begin{equation}
\frac{1}{c}\vec{B}_{dm}(\vec{r}, t) = \vec{n} \times \vec{E}_{dm}(\vec{r}, t).
\end{equation}

Para calcular el vector de Poynting podemos proceder como en la sección anterior, es decir,
\begin{equation}
\vec{S} = \epsilon_0 c^2\vec{E}_{dm} \times \vec{B}_{dm} = \epsilon_0 cE_{dm}^2\vec{n},
\end{equation}
de donde
\begin{equation}
\vec{S}(\vec{r}, t) = \frac{|\ddot{\vec{m}}(\tau)|^2 \sin^2 [\Theta(\tau)]}{16\pi^2 \epsilon_0 c^5 r^2}\vec{n},
\end{equation}
siendo $\Theta$ el ángulo formado entre los vectores $\ddot{\vec{m}}$ y $\vec{n}$.

Para calcular la potencia radiada por unidad de ángulo sólido en la aproximación dipolar magnética podemos nuevamente tomar la superficie esférica $S$ de radio $r$ con centro en el origen de coordenadas y escoger el eje $z$ a lo largo de la dirección y en el sentido del vector $\ddot{\vec{m}}$. Procediendo como en la sección anterior es posible ver sin dificultades que
\begin{equation}
\frac{dP_{dm}}{d\Omega}(t) = \frac{|\ddot{\vec{m}}(\tau)|^2}{16\pi^2 \epsilon_0 c^5}\sin^2 [\Theta(\tau)]
\end{equation}
y que
\begin{equation}
P_{dm}(t) = \frac{|\ddot{\vec{m}}(\tau)|^2}{6\pi\epsilon_0 c^5}.
\end{equation}

Las expresiones (4.155) y (4.156) son la fórmula diferencial e integral de Larmor, respectivamente, correspondientes a la aproximación dipolar magnética.

\subsection{Campo de radiación en la aproximación cuadrupolar eléctrica}

Pasemos a estudiar la aproximación cuadrupolar eléctrica para el campo de radiación. Los potenciales del campo electromagnético en esta aproximación son
\begin{equation}
\phi_Q(\vec{r}, t) = 0
\end{equation}
y
\begin{equation}
\vec{A}_Q(\vec{r}, t) = \frac{1}{8\pi\epsilon_0 c^3 r}\frac{d}{dt}\int\{[\vec{n} \cdot \vec{r}^{\prime}]\vec{J}(\vec{r}^{\prime}, \tau) + \vec{r}^{\prime}[\vec{n} \cdot \vec{J}(\vec{r}^{\prime}, \tau)]\}dV^{\prime}.
\end{equation}

Por simplicidad trabajaremos aquí suponiendo que la densidad de corriente $\vec{J}$ es la definida para un sistema de partículas, es decir,
\begin{equation}
\vec{J} = \sum_a q_a \vec{v}_a \delta(\vec{r} - \vec{r}_a).
\end{equation}

En consecuencia,
\begin{equation}
\vec{A}_Q = \frac{1}{8\pi\epsilon_0 c^3 r}\frac{d}{dt}\sum_a q_a [(\vec{n} \cdot \vec{r}_a)\vec{v}_a + \vec{r}_a(\vec{n} \cdot \vec{v}_a)].
\end{equation}

Pero
\begin{equation}
\frac{d}{dt}(\vec{n} \cdot \vec{r}_a)\vec{r}_a = (\vec{n} \cdot \vec{r}_a)\vec{v}_a + \vec{r}_a(\vec{n} \cdot \vec{v}_a).
\end{equation}

Luego
\begin{equation}
\vec{A}_Q = \frac{1}{8\pi\epsilon_0 c^3 r}\frac{d^2}{dt^2}\sum_a q_a (\vec{n} \cdot \vec{r}_a)\vec{r}_a,
\end{equation}
o bien
\begin{equation}
\vec{A}_Q = \frac{1}{24\pi\epsilon_0 c^3 r}\frac{d^2}{dt^2}\sum_a q_a (\vec{n} \cdot \vec{r}_a)\vec{r}_a.
\end{equation}

Efectuemos la transformación de calibración
\begin{align}
\vec{A}_Q^T &= \vec{A}_Q + \nabla f,\\
\phi_Q^T &= \phi_Q - \frac{df}{dt} = -\frac{df}{dt},
\end{align}
y elijamos
\begin{equation}
f = -\frac{\ln(r)}{24\pi\epsilon_0 c^3}\frac{d^2}{dt^2}\sum_a q_a r_a^2.
\end{equation}

Nótese que
\begin{equation}
\nabla f = -\frac{\vec{n}}{24\pi\epsilon_0c^3 r}\frac{d^2}{dt^2}\sum_a q_a r_a^2.
\end{equation}

El potencial vectorial transformado del campo electromagnético adopta entonces la forma
\begin{equation}
\vec{A}_Q^T = \frac{1}{24\pi\epsilon_0 c^3 r}\frac{d^2}{dt^2}\sum_a q_a [3(\vec{n} \cdot \vec{r}_a)\vec{r}_a - r_a^2\vec{n}].
\end{equation}

La componente $i$ del vector $\vec{A}_Q^T$ es
\begin{align}
A_{Qi}^T &= \frac{1}{24\pi\epsilon_0 c^3 r}\frac{d^2}{dt^2}\sum_a q_a [3(n_j x_{aj})x_{ai} - r_a^2 n_i] \nonumber \\
&= \frac{1}{24\pi\epsilon_0 c^3 r}\frac{d^2}{dt^2}\sum_a q_a [3x_{ai}x_{aj} - r_a^2\delta_{ij}]n_j \nonumber \\
&= \frac{1}{24\pi\epsilon_0 c^3 r}\frac{d^2}{dt^2}D_{ij}n_j,
\end{align}
siendo
\begin{equation}
(D_{ij}) = \sum_a q_a [3x_{ai}x_{aj} - r_a^2\delta_{ij}]
\end{equation}
el momentum de cuadrupolo eléctrico del sistema de partículas. Definamos el vector $\vec{Q}$ cuyas componentes son
\begin{equation}
Q_i = D_{ij}n_j.
\end{equation}

Entonces
\begin{equation}
\vec{A}_Q^T = \frac{\ddot{\vec{Q}}}{24\pi\epsilon_0 c^3 r}.
\end{equation}

Para el potencial escalar transformado del campo electromagnético tendremos que
\begin{equation}
\phi_Q^T = \frac{\ln(r)}{24\pi\epsilon_0c^3}\frac{d^3}{dt^3}\sum_a q_a r_a^2.
\end{equation}

Los campos $\vec{E}_Q$ y $\vec{B}_Q$ en la aproximación cuadrupolar eléctrica pueden calcularse a partir del conocimiento de los potenciales escalar y vectorial del campo. Así, para la parte eléctrica del campo tendremos que
\begin{equation}
\vec{E}_Q = -\nabla\phi_Q^T - \frac{\partial \vec{A}_Q^T}{\partial t},
\end{equation}
de donde, teniendo en cuenta las expresiones (4.163) y (4.164), hallamos
\begin{equation}
\vec{E}_Q = -\frac{\vec{n}}{24\pi\epsilon_0 c^3 r}\frac{d^3}{dt^3}\sum_a q_a r_a^2 - \frac{\dddot{\vec{Q}}}{24\pi\epsilon_0 c^3r}.
\end{equation}

Para la parte magnética del campo
\begin{align}
\vec{B}_Q &= \nabla\times \vec{A}_Q^T = \frac{1}{24\pi\epsilon_0c^3}\nabla\times\frac{\ddot{\vec{Q}}}{r} \nonumber \\
&= \frac{1}{24\pi\epsilon_0 c^3}\left[\vec{n} \times \frac{\partial}{\partial r}\frac{\ddot{\vec{Q}}}{r}\right] \nonumber \\
&= \frac{1}{24\pi\epsilon_0 c^3}\left[\vec{n} \times \left(-\frac{\ddot{\vec{Q}}}{r^2} + \frac{\dddot{\vec{Q}}}{cr}\right)\right] \nonumber \\
&= -\frac{\vec{n} \times \ddot{\vec{Q}}}{24\pi\epsilon_0c^3 r^2} + \frac{\vec{n} \times \dddot{\vec{Q}}}{24\pi\epsilon_0c^4 r}.
\end{align}

Despreciando el término proporcional a $\frac{1}{r^2}$ hallamos
\begin{equation}
\vec{B}_Q = -\frac{\vec{n} \times \dddot{\vec{Q}}}{24\pi\epsilon_0c^4 r}.
\end{equation}

Nótese que
\begin{equation}
\frac{1}{c}\vec{B}_Q = \vec{n} \times \vec{E}_Q.
\end{equation}

Conviene ahora calcular la forma del vector de Poynting con vistas a obtener las fórmulas diferencial e integral de Larmor. Como se sabe,
\begin{equation}
\vec{S} = \epsilon_0 c^2 \vec{E}_Q \times \vec{B}_Q = \epsilon_0c \vec{E}_Q \times (\vec{n} \times \vec{E}_Q).
\end{equation}

Obviamente
\begin{align}
\vec{S} \cdot \vec{n} &= \epsilon_0 c \vec{n} \cdot [\vec{E}_Q \times (\vec{n} \times \vec{E}_Q)] \nonumber \\
&= \epsilon_0c (\vec{n} \times \vec{E}_Q) \cdot (\vec{n} \times \vec{E}_Q) \nonumber \\
&= \epsilon_0c |\vec{n} \times \vec{E}_Q|^2 = \frac{|\vec{n} \times \dddot{\vec{Q}}|^2}{576\pi^2\epsilon_0c^5 r^2}.
\end{align}

Teniendo en cuenta el resultado obtenido para el vector de Poynting y repitiendo el procedimiento ya conocido, la potencia radiada por unidad de ángulo sólido viene dada, en la aproximación cuadrupolar eléctrica, por la expresión
\begin{equation}
\frac{dP_Q}{d\Omega} = \frac{|\vec{n} \times \dddot{\vec{Q}}|^2}{576\pi^2 \epsilon_0 c^5}.
\end{equation}

La relación anterior puede ser modificada y puesta en términos de las componentes del momentum de cuadrupolo eléctrico. Puede verse que
\begin{align}
|\vec{n} \times \dddot{\vec{Q}}|^2 &= (\vec{n} \times \dddot{\vec{Q}}) \cdot (\vec{n} \times \dddot{\vec{Q}}) \nonumber \\
&= (\dddot{\vec{Q}} \times \vec{n}) \cdot (\vec{n} \times \dddot{\vec{Q}}) \nonumber \\
&= -\dddot{\vec{Q}}(\vec{n} \cdot \dddot{\vec{Q}}) - |\dddot{\vec{Q}}|^2\vec{n} \cdot \vec{n} \nonumber \\
&= |\dddot{\vec{Q}}|^2 - (\vec{n} \cdot \dddot{\vec{Q}})^2.
\end{align}

Por lo tanto,
\begin{equation}
|\vec{n} \times \dddot{\vec{Q}}|^2 = \dddot{D}_{ij}\dddot{D}_{ik}n_j n_k - \dddot{D}_{ij}\dddot{D}_{kl}n_i n_j n_k n_l.
\end{equation}

En la expresión anterior se entiende que estamos utilizando el convenio de suma de Einstein, es decir, siempre debemos sumar por los índices doblemente repetidos. Sustituyendo (4.169) en (4.168) hallamos
\begin{equation}
\frac{dP_Q}{d\Omega} = \frac{1}{576\pi^2 \epsilon_0 c^5}[\dddot{D}_{ij}\dddot{D}_{ik}n_j n_k - \dddot{D}_{ij}\dddot{D}_{kl}n_i n_j n_k n_l].
\end{equation}

Para obtener la fórmula integral de Larmor podemos integrar la expresión (4.170) por el ángulo sólido. Para ello hay que tener en cuenta que
\begin{equation}
\int n_j n_k d\Omega = \frac{4\pi}{3}\delta_{jk}
\end{equation}
y que
\begin{equation}
\int n_i n_j n_k n_l d\Omega = \frac{4\pi}{15}[\delta_{ij}\delta_{kl} + \delta_{ik}\delta_{jl} + \delta_{il}\delta_{jk}].
\end{equation}

Así obtenemos
\begin{equation}
P_Q = \frac{\dddot{D}_{ij}\dddot{D}_{ij}}{720\pi\epsilon_0 c^5},
\end{equation}
que es la buscada fórmula integral de Larmor en la aproximación cuadrupolar eléctrica.

\section{Problemas propuestos}

\begin{enumerate}
\item Considere un átomo de Hidrógeno descrito por el modelo clásico planetario de Rutherford. Suponga que el electrón gira en una órbita circular alrededor del núcleo que permanece en reposo. Aplicando la fórmula de Larmor calcule el tiempo que demora el electrón en caer sobre el núcleo producto de la continua pérdida de energía en forma de radiación.

\item Un protón se mueve en los campos homogéneos y constantes $\vec{E}$ y $\vec{B}$ perpendiculares entre sí. La velocidad inicial del protón es $\vec{v}_0$. Calcule la energía perdida por el protón al cabo de un tiempo $t$ en forma de radiación.

\item La partícula de carga $e$ y masa $m$ atraviesa el diámetro de una esfera cuyo radio es $a$, y dentro de la cual se distribuye uniformemente la carga $q$. Suponga que $e$ y $q$ poseen signos opuestos. Calcule la energía perdida por la partícula en forma de radiación al atravesar la esfera suponiendo que la energía cinética de entrada es $W_0$.

\item Un protón se mueve en una órbita perpendicular al campo magnético uniforme $\vec{B}$. En $t=0$ su energía cinética era $W_0$. Calcular la disminución de la energía cinética del protón como función del tiempo debido a las pérdidas por radiación.

\item El momentum magnético del neutrón es proporcional a su momentum angular, es decir, $\vec{m} = -\kappa\vec{L}$. El neutrón se coloca en el campo magnético uniforme y constante $\vec{B}$. Calcule la potencia radiada por el neutrón en aproximación dipolar magnética.

\item Un protón se mueve en el campo eléctrico uniforme $\vec{E}$. Calcule la energía radiada por el protón al cabo de un tiempo $t$ utilizando la aproximación dipolar eléctrica, la aproximación dipolar magnética y la aproximación cuadrupolar eléctrica. Compare los tres resultados.

\item Cierta antena es construida con un hilo conductor metálico muy fino al cual se le da la forma de un cuadrado de lado $d$. Por la antena circula la corriente eléctrica de intensidad $I = I_0 e^{-\alpha t^2}$. Calcule la potencia total radiada por la antena en la aproximación dipolar magnética.

\item Por la misma antena del problema anterior circula la corriente de intensidad $I = I_0 \cos(\omega t)$. Calcule la potencia radiada durante un período.

\item Una antena lineal es un hilo conductor recto de longitud $L$ por el cual circula la corriente $I = I_0 \cos(\omega t)$. Calcular el valor medio temporal (en un período) de la potencia radiada por la antena.

\item La partícula de carga $e$ y masa $m$ incide con velocidad inicial $\vec{v}_0$ sobre un átomo, siendo $L$ la distancia de aproximación mínima. El átomo es neutro y tiene momentum dipolar eléctrico nulo. El momentum cuadrupolar eléctrico del átomo es $D_{ij} = 0$ para $i \neq j$, mientras que $D_{11} = D_{22} = \frac{1}{2}D = const$. Suponiendo que el átomo no se polariza producto de su interacción con la partícula, calcule la energía radiada por la partícula durante todo el proceso.
\end{enumerate}


\end{document}
