\documentclass{beamer}
\usetheme{Singapore}
\useoutertheme{infolines}  % Información en el pie de página
\usepackage{graphicx}
\usepackage[spanish]{babel}
\usepackage[utf8]{inputenc}
\usepackage{multirow}

\title{Análisis de biospeckle en arándanos}
\author{Laura Velasquez, Juan Montoya}
\institute{FCEN@UdeA}
\date{\today}

\begin{document}

% Portada
\begin{frame}
    \titlepage
\end{frame}

% Introducción
\begin{frame}{Introducción}
    \begin{itemize}
        \item El biospeckle es un fenómeno óptico que ocurre cuando la luz láser incide sobre superficies biológicas activas, como frutas.
        \item Este proyecto analiza el biospeckle en arándanos para evaluar su calidad, distinguiendo entre arándanos óptimos y no óptimos para el consumo.
    \end{itemize}
\end{frame}


% Fundamentos Teóricos
\begin{frame}{Fundamentos Teóricos}
    \begin{itemize}
        \item El speckle es un patrón de interferencia generado por luz coherente reflejada en superficies rugosas.
        \item En tejidos biológicos, el speckle varía con el tiempo (biospeckle) debido al movimiento de las células.
        \\[0.5cm]
        \begin{figure}
            %\includegraphics[width=0.35\textwidth]{Superficie.png}
            \includegraphics[width=0.65\textwidth]{Observacion.png}
        \end{figure}
    \end{itemize}
\end{frame}

% Montaje Experimental
\begin{frame}{Montaje Experimental}
    \begin{figure}
        \includegraphics[width=0.4\textwidth]{Bananaforscale.png}
        \includegraphics[width=0.4\textwidth]{arandalep.png}
        \caption{Montaje experimental para el análisis de biospeckle en arándanos.}
    \end{figure}
\end{frame}

% Metodología
\begin{frame}{Metodología}
    \begin{enumerate}
        \item \textbf{Recolección de datos}: Imágenes de biospeckle de 18 arándanos (14 óptimos, 4 no óptimos).
        \item \textbf{Procesamiento}: Cálculo de MI, GdMean y GdStd por muestra.
        \item \textbf{Análisis estadístico}: 
        \begin{itemize}
            \item Pruebas de normalidad (Shapiro-Wilk) y varianza (Levene).
            \item Prueba t de Student para diferencias entre grupos.
        \end{itemize}
        \item \textbf{Modelado}: Regresión logística para clasificar arándanos con GdMean y GdStd.
    \end{enumerate}
\end{frame}

% Resultados
\begin{frame}{Resultados - Análisis Estadístico}
    \begin{table}
        \centering
        \begin{tabular}{|l|c|c|c|}
            \hline
            \textbf{Variable} & \textbf{p-valor (t-test)} & \textbf{Distinguibles?} & \textbf{IC 95\% óptimo} \\
            \hline
            GdMean & 0.0042 & Sí & [3.2, 4.6] \\
            GdStd & 0.00078 & Sí & [10, 15] \\
            MI & 0.064 & No & [3.3e+06, 7e+06] \\
            \hline
        \end{tabular}
        \caption{Resultados de la prueba t de Student.}
    \end{table}
    \begin{itemize}
        \item GdMean y GdStd diferencian significativamente entre grupos.
        \item MI no muestra diferencias relevantes (p $>$ 0.05).
    \end{itemize}
\end{frame}

\begin{frame}{Resultados - Modelo de Clasificación}
    \begin{itemize}
        \item Modelo de regresión logística con GdMean y GdStd.
        \item \textbf{Precisión}: 89\% (validación cruzada y prueba).
        \item \textbf{Matriz de confusión}:
        \centering
        \resizebox{\linewidth}{!}{
            \begin{tabular}{cc|c|c|}
                \cline{3-4}
                & & \multicolumn{2}{c|}{\textbf{Predicción}} \\
                \cline{3-4}
                & & no óptimo & óptimo \\
                \hline
                \multicolumn{1}{|c|}{\multirow{2}{*}{\textbf{Real}}} & no óptimo & 2 & 0 \\
                \cline{2-4}
                \multicolumn{1}{|c|}{} & óptimo & 1 & 6 \\
                \hline
            \end{tabular}
        }
    \end{itemize}
\end{frame}

% Conclusiones
\begin{frame}{Conclusiones}
    \begin{itemize}
        \item El biospeckle es efectivo para evaluar la calidad de arándanos.
        \item GdMean y GdStd son indicadores clave para clasificar arándanos.
        \item El modelo de regresión logística logra un 89\% de precisión.
        \item Sugerencia: Ampliar la muestra para robustecer el modelo.
    \end{itemize}
\end{frame}

% Agradecimientos
\begin{frame}
    \centering
    \Huge{¡Gracias!}
\end{frame}

\end{document}