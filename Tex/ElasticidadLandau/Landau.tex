\documentclass{article}
\usepackage{amsmath}
\usepackage{amssymb}
\usepackage{geometry}
\geometry{a4paper, margin=1in}
\usepackage{graphicx}

\begin{document}

\section*{Formulas relevantes}
$$ u_{ij} = \frac{1}{2}\left(\frac{\partial u_i}{\partial x_j} + \frac{\partial u_j}{\partial x_i}\right) \text{ Deformacion} $$
$$ \frac{\partial \sigma_{ij}}{\partial x_j}+f_i =0 \text{ Eq equilibrio, tensor de esfuerzo} $$

$$ \sigma_{ij} = \frac{E}{1+\sigma}(u_{ij}+\frac{\sigma}{1-2\sigma}u_{ll}\delta_{ij}) = \lambda u_{ll} \delta_{ij} + 2\mu u_{ij} \text{ Ley de Hooke}$$

$$ u_{ij} = \frac{1}{E}[(1+\sigma)\sigma_{ij} - \sigma \sigma_{ll}\delta_{ij}] \text{ Ley de Hooke reciproca}$$

Modulo de young $E = \frac{\mu (3\lambda+2\mu)}{\lambda + \mu} $

Coeficiente de Poisson $\nu = \frac{\lambda}{2(\lambda + \mu)} $

Modulo de compresibilidad $K = \lambda + \frac{2}{3}\mu $

\section*{Sección 7: Deformaciones en Sólidos}

\subsection*{Problema 1: Deformación de una Varilla Vertical}
\textbf{Enunciado.} Determinar la deformación de una larga varilla (longitud \textit{l}) vertical y en reposo en un campo gravitatorio

\textit{Solución.} Tomamos el eje $z$ en la dirección de la varilla, y el plano $x, y$ coincidente con el plano de su extremo inferior. Las ecuaciones de equilibrio son:

$$
\frac{\partial \sigma_{xi}}{\partial x_{i}} = \frac{\partial \sigma_{yi}}{\partial x_{i}} = 0, \quad \frac{\partial \sigma_{zi}}{\partial x_{i}} = \rho g.
$$

En la superficie lateral de la varilla, todas las componentes de $\sigma_{ik}$ deben anularse, excepto $\sigma_{zz}$. En el extremo superior ($z = l$), $\sigma_{xz} = \sigma_{yz} = \sigma_{zz} = 0$. La solución que satisface estas condiciones es $\sigma_{zz} = -\rho g(l - z)$, con las demás $\sigma_{ik}$ nulas.

Para $u_{ik}$, obtenemos:

$$
u_{xx} = u_{yy} = \frac{\sigma}{E}\rho g(l - z), \quad u_{zz} = -\frac{\rho g(l - z)}{E}, \quad u_{xy} = u_{xz} = u_{yz} = 0.
$$

Integrando, las componentes del vector de desplazamiento son:

$$
u_x = \frac{\sigma}{E}\rho g(l - z)x, \quad u_y = \frac{\sigma}{E}\rho g(l - z)y, \quad u_z = -\frac{\rho g}{2E} \{l^2 - (l - z)^2 - \sigma(x^2 + y^2)\}.
$$

La solución para $u_z$ satisface $u_z = 0$ solo en un punto del extremo inferior de la varilla, por lo que no es válida cerca de este extremo.

\subsection*{Problema 2: Deformación de una Esfera Hueca}
\textbf{Enunciado.} Hallar la deformación de una esfera hueca (de radios externo e interno $R_2$ y $R_1$, respectivamente), sometida a una presión interna $p_1$ y a una presión externa $p_2$.

\textit{Nota:} Equación 7.5 $$2(1-\sigma)\nabla(\nabla \cdot \vec{u}) - (1-2\sigma)\nabla \times \nabla \times \vec{u} = 0$$
Formulas 1.7 $$ u_{rr} = \frac{\partial u_r}{\partial r}, \quad u_{\theta \theta} = \frac{1}{r}\frac{\partial u_\theta}{\partial \theta} + \frac{u_r}{r}, \quad u_{\varphi \varphi} = \frac{1}{r\sin(\theta)}\frac{\partial u_\varphi}{\partial \varphi} + \frac{u_\theta}{r}\cot(\theta) + \frac{u_r}{r} $$
$$ 2u_{\theta \varphi} = \frac{1}{r}\left(\frac{\partial u_\varphi}{\partial \theta}- u_\varphi \cot(\theta)\right) + \frac{1}{r\sin(\theta)}\frac{\partial u_\varphi}{\partial \varphi}, \quad 2u_{r\theta} = \frac{\partial u_\theta }{\partial r}-\frac{u_\theta}{r} + \frac{1}{r}\frac{\partial u_r }{\partial \theta}$$
$$ 2u_{\varphi r}=\frac{1}{r\sin(\theta)}\frac{\partial u_r}{\partial \varphi}+\frac{\partial u_\varphi}{\partial r}-\frac{u_\varphi}{r} $$

\textit{Solución.} Usamos coordenadas esféricas con origen en el centro de la esfera. El desplazamiento $\vec{u}$ es radial y función de $r$ solamente. Así, $\nabla \times \vec{u} = 0$ y (eq 7.5)$\nabla \cdot \vec{u} = 0$. Resulta:

$$
\text{div } \vec{u} = \frac{1}{r^2} \frac{d(r^2 u)}{dr} = \text{const} \equiv 3a,
$$

donde $u = ar + \frac{b}{r^2}$. Las componentes del tensor de deformación son (Véase formulas 1.7) $u_{rr} = a - \frac{2b}{r^3}$, $u_{\theta\theta} = u_{\varphi\varphi} = a + \frac{b}{r^3}$. La tensión radial es:

$$
\sigma_{rr} = \frac{E}{(1+\sigma)(1-2\sigma)} \{(1-\sigma)u_{rr} + 2\sigma u_{\theta\theta}\} = \frac{E}{1-2\sigma} a - \frac{2E}{1+\sigma} \frac{b}{r^3}.
$$

Las constantes $a$ y $b$ se determinan por las condiciones de contorno: $\sigma_{rr} = -p_1$ para $r = R_1$ y $\sigma_{rr} = -p_2$ para $r = R_2$:

$$
a = \frac{p_1 R_1^3 - p_2 R_2^3}{R_2^3 - R_1^3} \cdot \frac{1-2\sigma}{E}, \quad b = \frac{R_1^3 R_2^3 (p_1 - p_2)}{R_2^3 - R_1^3} \cdot \frac{1+\sigma}{2E}.
$$

Para una capa esférica con $p = p_1$ y $p_2 = 0$:

$$
\sigma_{rr} = \frac{p R_2^3}{R_2^3 - R_1^3} \left(1 - \frac{R_2^3}{r^3}\right), \quad \sigma_{\theta\theta} = \sigma_{\varphi\varphi} = \frac{p R_2^3}{R_2^3 - R_1^3} \left(1 + \frac{R_1^3}{r^3}\right).
$$

Para una cáscara esférica delgada, de espesor $h= R_2-R_1 \llless R$ obtenemos aproximadamente:

$$
u = \frac{p R^2 (1-\sigma)}{2Eh}, \quad \sigma_{\theta\theta} = \sigma_{\varphi\varphi} = \frac{pR}{2h}, \quad \bar{\sigma}_{rr} = \frac{p}{2}.
$$

Donde $\bar{\sigma_{rr}}$ es el valor medio de la tensión radial promediada en el espesor de la cascara. Obtenemos la distribucion de tensiones en un medio elástico infinito con cavidad hueca (de radio R) sometido a compresión hidrostática, simplemente poniendo en las ecuaciones anteriores $R_1 = R$, $R_2 = \infty$, $p_1 = 0$ y $p_2 = p$:

$$
\sigma_{rr} = -p \left(1 - \frac{R^3}{r^3}\right), \quad \sigma_{\theta\theta} = \sigma_{\varphi\varphi} = -p \left(1 + \frac{R^3}{2r^3}\right).
$$

La tensión tangencial en la superficie es $\sigma_{\theta\theta} = \sigma_{\varphi\varphi} = -3p/2$.

\subsection*{Problema 3: Deformación de una Esfera Maciza}
\textbf{Enunciado.} Determinar la deformación de una esfera maciza (de radio $R$) bajo la acción de su propio campo gravitatorio.

\textit{Nota:} Equación 7.3: 
$$ \nabla (\nabla \cdot \vec{u}) - \frac{1-2\sigma}{2(1-\sigma)}\nabla \times \nabla \times \vec{u} = -\rho \vec{g}\frac{(1+\sigma)(1-2\sigma)}{E(1-\sigma)}$$

\textit{Solución.} La fuerza gravitatoria que actua sobre la unidad de masa en un cuerpo esférico es $-\frac{g\vec{r}}{R}$. Sustituyendo esta expresion en lugar de $g$ en la ecuación (7.3), obtenemos:

$$
\frac{E(1-\sigma)}{(1+\sigma)(1-2\sigma)} \frac{d}{dr} \left( \frac{1}{r^2} \frac{d(r^2 u)}{dr} \right) = \rho g \frac{r}{R}.
$$

La solución para $r = 0$ y $\sigma_{rr} = 0$ en $r = R$ es:

$$
u = -\frac{g\rho R(1-2\sigma)(1+\sigma)}{10E(1-\sigma)} r \left(\frac{3-\sigma}{1+\sigma} - \frac{r^2}{R^2}\right).
$$

La materia está comprimida ($u_{rr} < 0$) dentro de $R\sqrt{\frac{3-\sigma}{3(1+\sigma)}}$ y dilatada fuera ($u_{rr} > 0$). La presión en el centro es $\frac{3-\sigma}{10(1-\sigma)} g \rho R$.

\subsection*{Problema 4: Deformación de un Tubo Cilíndrico}
\textbf{Enunciado.} Hallar la deformación de un tubo cilíndrico de radios externo e interno $R_2$ y $R_1$, respectivamente, sometido unicamente a una presión interna $p$ (se supone que se mantiene constante la longitud del cilindro, de manera que no hay deformación longitudinal).

\textit{Nota:} Formulas 1.8: $$ u_{rr} = \frac{\partial u_r}{\partial r}, \quad u_{\varphi \varphi}=\frac{1}{r}\frac{\partial u_\varphi}{\partial \varphi} + \frac{u_r}{r}, \quad u_{zz} = \frac{\partial u_z}{\partial z} $$
$$ 2u_{\varphi z}=\frac{1}{r}\frac{\partial u_z}{\partial \varphi}+\frac{\partial u_\varphi}{\partial z}, \quad 2u_{rz}=\frac{\partial u_r}{\partial z} + \frac{\partial u_z}{\partial r}$$
$$ 2u_{r \varphi}= \frac{\partial u_\varphi}{\partial r}-\frac{u_\varphi}{r}+\frac{1}{r}\frac{\partial u}{\partial \varphi} $$

\textit{Solución.} Usamos coordenadas cilíndricas con el eje $z$ a lo largo del tubo. La presión uniforme causa un desplazamiento radial puro $u_r = u(r)$. Análogamente al problema 2:

$$
\text{div } \vec{u} = \frac{1}{r} \frac{d(ru)}{dr} = \text{const} \equiv 2a.
$$

Por lo tanto, $u = ar + \frac{b}{r}$. Las componentes no nulas del tensor de deformaciones son $u_{rr} = \frac{du}{dr} = a - \frac{b}{r^2}$, $u_{\varphi\varphi} = \frac{u}{r} = a + \frac{b}{r^2}$. De las condiciones $\sigma_{rr} = 0$ en $r = R_2$ y $\sigma_{rr} = -p$ en $r = R_1$:

$$
a = \frac{pR_1^2}{R_2^2 - R_1^2} \cdot \frac{(1+\sigma)(1-2\sigma)}{E}, \quad b = \frac{pR_1^2 R_2^2}{R_2^2 - R_1^2} \cdot \frac{1+\sigma}{E}.
$$

La distribución de tensiones en el espesor de las paredes del tubo esta dada por:

$$
\sigma_{rr} = \frac{pR_1^2}{R_2^2 - R_1^2} \left(1 - \frac{R_2^2}{r^2}\right), \quad \sigma_{\varphi\varphi} = \frac{pR_1^2}{R_2^2 - R_1^2} \left(1 + \frac{R_2^2}{r^2}\right), \quad \sigma_{zz} = 2\sigma \frac{pR_1^2}{R_2^2 - R_1^2}.
$$

\subsection*{Problema 5: Deformación de un Cilindro Giratorio}
\textbf{Enunciado.} Determinar la deformación de un cilindro que gira uniformemente alrededor de su eje.

\textit{Solución.} Reemplazamos la fuerza gravitatoria en (7.3) por la centrífuga $\rho \Omega^2 r$. La ecuación para $u_r = u(r)$ es:

$$
\frac{E(1-\sigma)}{(1+\sigma)(1-2\sigma)} \frac{d}{dr} \left( \frac{1}{r} \frac{d(ru_r)}{dr} \right) = -\rho \Omega^2 r.
$$

La solución para $r = 0$ y $\sigma_{rr} = 0$ en $r = R$ es:

$$
u = \frac{\rho \Omega^2 (1+\sigma)(1-2\sigma)}{8E(1-\sigma)} r [(3-2\sigma)R^2 - r^2].
$$

\section*{Sección 16: Rigidez a la Torsión}

\subsection*{Problema 1: Barra de Sección Circular}
\textbf{Enunciado.} Determinar la rigidez a la torsión de una barra de sección circular (de radio R).

\textit{Nota:}

De pag 97: Conviene notar también una analogía hidrodinámica: una equación de la forma $\nabla \chi = -1$ (16.11), determina la distribucion de velocidades $v(x,y)$ de una liquido viscoso en la sección de una tubo: la condición de contorno $\chi = 0$ (16.12), corresponde a la condición $v=0$ en las paredes inmóviles del tubo.

Equación 16.10: $$ \frac{\partial \psi}{\partial x} = y + 2\frac{\partial \chi}{\partial y}, \quad \frac{\partial \psi}{\partial y} = -x -2\frac{\partial \chi}{\partial x} $$

Equación 16.4: $$ u_z = \tau \psi(x,y) $$

\textit{Solución.} Las soluciones de los problemas 1-4 coinciden formalmente con las soluciones de los problemas del movimiento de un líquido viscoso en un tubo de sección correspondiente (nota de pag 97) a la cantidad de liquido $Q$ que pasa a través de una sección del tubo correspondiente aqui la cantidad $C$. Para una barra de sección circular, tenemos:

$$
\chi = \frac{1}{4} (R^2 - x^2 - y^2).
$$

La rigidez a la torsión es:

$$
C = \frac{\mu \pi R^4}{2}.
$$

Para la función $\psi$, (de 16.10) se deduce $\psi = \text{const}$. La constante (según 16.4) $\psi$ corresponde a un desplazamiento de toda la barra a lo largo del eje $z$, por lo que se puede considerar $\psi = 0$. Así, las secciones transversales permanecen planas.

\subsection*{Problema 2: Barra de Sección Elíptica}
\textbf{Enunciado.} Lo mismo para una barra de sección eliptica (de semiejes $a$ y $b$).

\textit{Solución.} La rigidez a la torsión es:

$$
C = \pi \mu \frac{a^3 b^3}{a^2 + b^2}.
$$

La distribución de desplazamientos longitudinales está dada por la función de torsión:

$$
\psi = \frac{b^2 - a^2}{b^2 + a^2} xy.
$$
Los ejes de coordenadas están dirigidos según los ejes de la elipse.

\subsection*{Problema 3: Barra de Sección Triangular}
\textbf{Enunciado.} Lo mismo para una barra cuya sección es un triángulo equilatero (de lado $a$).

\textit{Solución.} Rigidez a la torsión:

$$
C = \frac{\sqrt{3}}{80} \mu a^4.
$$

Función de torsión:

$$
\psi = \frac{y(x\sqrt{3} + y)(x\sqrt{3} - y)}{6a}.
$$

El origen de coordenadas se elige en el centro del triángulo, y el eje $x$ coincide con una de sus alturas.

\subsection*{Problema 4: Barra con Forma de Placa Delgada}
\textbf{Enunciado.} Ídem para una barra que tiene la forma de una larga placa delgada (de ancho $d$ y grueso $h\llless d$).

\textit{Solución.} El problema es equivalente al flujo de un líquido viscoso entre paredes planas y paralelas. El resultado es:

$$
C = \frac{\mu dh^3}{3}.
$$

\subsection*{Problema 5: Tubo Cilíndrico}
\textbf{Enunciado.} Lo mismo para un tubo cilíndrico (de radios interno y externo $R_1$ y $R_2$).

\textit{Nota:} Equación 16.13: $$ \oint \frac{\partial \chi}{\partial n}dl=-S $$
Equación 16.17: $$ C=4\mu \sum_{k}\chi_k S_k +4\mu \int \chi dx dy $$

\textit{Solución.} La función (en coordenadas polares):

$$
\chi = \frac{1}{4}(R^2 - r^2)
$$

satisface la condición (16.13) en ambos límites de la sección anular del tubo. De la fórmula (16.17) se sigue:

$$
C = \mu \pi \frac{R_2^4 - R_1^4}{4}.
$$

\subsection*{Problema 6: Tubo de Paredes Delgadas}
\textbf{Enunciado.} Ídem para un tubo de paredes delgada y sección arbitraria.

\textit{Solución.} Dado que la pared del tubo es delgada, se puede considerar que en ella la función $\chi$ varía linealmente desde cero, sobre una cara, hasta $\chi_1$, sobre la otra ($y$ es la coordenada a lo largo del espesor del la pared):

$$
\chi = \chi_1 \frac{y}{h}.
$$

La condición (16.13) da $\frac{\chi_1 L}{h} = S$, donde $L$ es la longitud del perímetro de la sección del tubo, y $S$ el área que encierra. En la expresion (16.17) el segundo término es pequeño comparado con el primero, y obtenemos:

$$
C = \frac{4hS^2 \mu}{L}.
$$

Si se corta el tubo a lo largo de una de sus generatrices, la rigidez a la torsión disminuye a (de acuerdo con el resultado del problema 4):

$$
C = \frac{\mu Lh^3}{3}.
$$

\section*{Sección 19: Problemas de Flexión}

\subsection*{Problema 1: Reducción a Cuadraturas}
\textbf{Enunciado.} Reducir a cuadraturas el problema de la determinación de la forma de una barra de sección circular (varilla elástica), fuertemente combada en un plano por fuerzas concentradas.

\textit{Nota:} Ecuacion 19.10: $$ EI\frac{d\vec{r}}{dl}\times \frac{d^{3}\vec{r}}{dl^{3}} = \vec{F}\times\frac{d\vec{r}}{dl}$$

\textit{Solución.} Consideremos una porción de la barra comprendida entre puntos de aplicación de las fuerzas; en tal región es $F = \text{const}$. Elijamos el plano de la flexión como plano $x, y$, con el eje $y$ paralelo a la fuerza $F$ e introduzcamos el ángulo $\theta$ entre la tangente a la línea de la barra y el eje $y$. Entonces:

$$
\frac{dx}{dl} = \sin \theta, \quad \frac{dy}{dl} = \cos \theta,
$$

donde $x, y$ son las coordenadas de los puntos de la barra. Desarrollando los productos vectoriales en (19.10), obtenemos una ecuación para $\theta$, considerando como función de la longitud de arco $l$:

$$
IE \frac{d^2 \theta}{dl^2} - F \sin \theta = 0.
$$

La primera integración da:

$$
\frac{IE}{2} \left( \frac{d\theta}{dl} \right)^2 + F \cos \theta = c_1,
$$

y de aquí:

$$
l = \pm \sqrt{\frac{IE}{2}} \int \frac{d\theta}{\sqrt{c_1 - F \cos \theta}} + c_2 \rightarrow (1)
$$

La función $\theta(l)$ puede expresarse mediante funciones elípticas. Para las coordenadas:

$$
x = \int \sin \theta \, dl, \quad y = \int \cos \theta \, dl,
$$

obtenemos:

$$
x = \pm \frac{1}{F} \sqrt{2IE} \sqrt{c_1 - F \cos \theta} + \text{const.}
$$

$$
y = \pm \sqrt{\frac{IE}{2}} \int \frac{\cos \theta \, d\theta}{\sqrt{c_1 - F \cos \theta}} + \text{const'.} \rightarrow (2)
$$

El momento $M$ está dirigido según el eje $z$ y su módulo vale $M = IE \frac{d\theta}{dl}$.

\subsection*{Problema 2: Barra Fuertemente Encorvada}
\textbf{Enunciado.} Hallar la forma de una barra fuertemente encorvada, uno de cuyos extremos está empotrado, y en el otro, que es libre, se aplica una fuerza $\vec{f}$: la dirección de $\vec{f}$ es perpendicular a la recta de la barra no deformada (fig. 15).
\begin{figure}[ht]
    \centering
    \includegraphics[width = 0.15\columnwidth]{Screenshot_20250227_180931.png}
    \caption{fig 15}
\end{figure}

\textit{Solución.} En toda longitud de la barra es $\vec{F} = \text{const} = \vec{f}$. En el extremo empotrado ($l = 0$) se tiene $\theta = \frac{\pi}{2}$, y en el libre ($l = L$, donde $L$ es la longitud de la barra), $M = 0$, esto es, $\theta' = 0$. Introduciendo la notación $\theta_0 = \theta(L)$, tenemos en (1) $c_1 =f\cos(\theta_0)$:

$$
l = \sqrt{\frac{IE}{2f}} \int_{\theta_0}^{\frac{\pi}{2}} \frac{d\theta}{\sqrt{\cos \theta - \cos \theta}}.
$$

De aquí se deduce la ecuación que determina $\theta_0$:

$$
L = \sqrt{\frac{IE}{2f}} \int_{\theta_0}^{\frac{\pi}{2}} \frac{d\theta}{\sqrt{\cos \theta_0 - \cos \theta}}.
$$

La forma de la barra se halla mediante las fórmulas:

$$
x = \sqrt{\frac{2IE}{f}} \left( \sqrt{\cos \theta_0} - \sqrt{\cos \theta_0 - \cos \theta} \right),
$$

$$
y = \sqrt{\frac{IE}{2f}} \int_{\theta}^{\frac{\pi}{2}} \frac{\cos \theta \, d\theta}{\sqrt{\cos \theta_0 - \cos \theta}}.
$$

\subsection*{Problema 3: Fuerza Aplicada en el Extremo Libre}
\textbf{Enunciado.} Ídem, si la fuerza $\vec{f}$, aplicada en el extremo libre, está dirigida paralelamente a la linea de la barra no deformada.
\begin{figure}[h]
    \centering
    \includegraphics[width=0.15\columnwidth]
    {Screenshot_20250227_182326.png}
    \caption{fig. 16}
\end{figure}

\textit{Solución.} Tenemos $\vec{F} = -\vec{f}$. Condiciones de contorno: $\theta = 0$ en $l = 0$, $\theta' = 0$ en $l = L$. Tenemos:

$$
l = \sqrt{\frac{IE}{2f}} \int_{0}^{\theta} \frac{d\theta}{\sqrt{\cos \theta - \cos \theta_0}},
$$

donde $\theta_0 = \theta(L)$ se determina por:

$$
L = \sqrt{\frac{IE}{2f}} \int_{0}^{\theta_0} \frac{d\theta}{\sqrt{\cos \theta - \cos \theta_0}}.
$$

Para $x$ e $y$ obtenemos:

$$
x = \sqrt{\frac{2IE}{f}} \left( \sqrt{1 - \cos \theta_0} - \sqrt{\cos \theta - \cos \theta_0} \right),
$$

$$
y = \sqrt{\frac{IE}{2f}} \int_{0}^{\theta} \frac{\cos \theta \, d\theta}{\sqrt{\cos \theta - \cos \theta_0}}.
$$

En una flexión pequeña, $\theta_0 \ll 1$ y se puede escribir:

$$
L \approx \sqrt{\frac{IE}{f}} \int_{0}^{\theta_0} \frac{d\theta}{\sqrt{\theta_0^2 - \theta^2}} = \frac{\pi}{2} \sqrt{\frac{IE}{f}}.
$$

Esto es, $\theta_0$ desaparece de esta relación. Indicando que la solución existe solo cuando $f \geq \frac{\pi^2 IE}{4L^2}$.

\subsection*{Problema 4: Barra con Ambos Extremos Apoyados}
\textbf{Enunciado.} Ídem, si ambos extremos de la barra están apoyados y en su punto medio se aplica una fuerza $\vec{f}$: la distancia entre los puntos de apoyo es $L_0$.

\begin{figure}[h]
    \centering
    \includegraphics[width=0.2\columnwidth]
    {Screenshot_20250227_183218.png}
    \caption{fig. 17}
\end{figure}

\textit{Solución.} La fuerza $\vec{F}$ es constante en cada una de las porciones $AB$ y $BC$, y en cada una de ellas es perpendicular a la linea de la barra en los puntos de apoyo $A$ y $C$, respectivamente. La diferencia entre los valores de $\vec{F}$ en $AB$ y $BC$ es igual a $\vec{f}$, de donde se deduce que en $AB$ es $F \sin \theta_0 = -\frac{f}{2}$, donde $\theta_0$ es el ángulo entre el eje $y$ y la línea $AC$. En el punto $A$ ($l = 0$) tenemos $\theta = \frac{\pi}{2}$ y $M = 0$, es decir, $\theta' = 0$, de modo que en $AB$:

$$
l = \sqrt{\frac{IE}{f}} \sqrt{\sin \theta_0} \int_{\theta}^{\frac{\pi}{2}} \frac{d\theta}{\sqrt{\cos \theta}} \quad
x = 2 \sqrt{\frac{IE \sin \theta_0}{f}} \sqrt{\cos \theta},
$$

$$
y = \sqrt{\frac{IE \sin \theta_0}{f}} \int_{\theta}^{\frac{\pi}{2}} \sqrt{\cos \theta}  d\theta.
$$

El ángulo $\theta_0$ se determina mediante la condición que expresa que la proyección del arco $AB$ sobre la recta $AC$ debe ser igual a $\frac{L_0}{2}$, de donde:

$$
\frac{L_0}{2} = \sqrt{\frac{IE \sin \theta_0}{f}} \int_{\theta_0}^{\frac{\pi}{2}} \frac{\cos (\theta - \theta_0)}{\sqrt{\sin \theta}} \, d\theta.
$$

Para determinado valor de $\theta_0$, entre $0$ y $\frac{\pi}{2}$ la derivada $\frac{df}{d\theta_0}$ se anula y pasa a ser positiva, indicando que la solución se hace inestable.

\section*{Sección 20: Formas de Barras}

\subsection*{Problema 1: Barra Combadura por su Propio Peso}
\textbf{Enunciado.} Hallar la forma de una barra (de longuitud $l$), combada por la acciónde su propio peso, para diferentes manera de fijar sus extremos.

\textit{Solución.} La forma buscada se obtiene como solución de la ecuación $\zeta^{\prime\prime\prime\prime} = \frac{q}{EI}$ ($q$ es el peso de la barra por unidad de longitud) con condiciones de contorno en sus extremos. Para distintos apoyos de los extremos de la barra se obtienen las formas de flexión y los desplazamientos máximos (el origen de coordenadas coincide simepre con uno de los extremos de la barra):

\begin{enumerate}
    \item[a)] Ambos extremos empotrados:
    $$
    \zeta = \frac{q}{24EI} z^2(z-l)^2, \quad \zeta\left(\frac{l}{2}\right) = \frac{1}{384} \frac{ql^4}{EI}.
    $$

    \item[b)] Ambos extremos apoyados:
    $$
    \zeta = \frac{q}{24EI} z(z^3 - 2lz^2 + l^3), \quad \zeta\left(\frac{l}{2}\right) = \frac{5}{384} \frac{ql^4}{EI}.
    $$

    \item[c)] Un extremo ($z = l$) empotrado, y el otro ($z = 0$) apoyado:
    $$
    \zeta = \frac{q}{48EI} z(2z^3 - 3lz^2 + l^3), \quad \zeta(0.42l) = 0.0054 \frac{ql^4}{EI}.
    $$

    \item[d)] Un extremo ($z = 0$) empotrado, y el otro ($z = l$) libre:
    $$
    \zeta = \frac{q}{24EI} z^2(z^2 - 4lz + 6l^2), \quad \zeta(l) = \frac{1}{8} \frac{ql^4}{EI}.
    $$
\end{enumerate}

\subsection*{Problema 2: Barra Encorvada por una Fuerza Concentrada}
\textbf{Enunciado.} Determinar la forma de una barra encorvada por la acción de una fuerza concentrada $f$, aplicada en su punto medio.

\textit{Solución.} En todas partes, menos en el punto $z = l/2$, tenemos la ecuación $\zeta^{\prime\prime\prime\prime} = 0$. Las condiciones de contorno en los extremos de la barra determinan el modo de fijación. En el punto $z = l/2$ deben ser continuas $\zeta$, $\zeta'$, $\zeta''$, pero la diferencia de las fuerzas de corte $F=-EI\zeta^{\prime\prime\prime}$ a uno y otro lado debe ser igual a la fuerza $f$.
La forma de la barra (en la porción $0\ll z\ll \frac{l}{2}$) y la flecha de la linea elástica vienen dadas por las siguientes fórmulas:

\begin{enumerate}
    \item[a)] Ambos extremos empotrados:
    $$
    \zeta = \frac{f}{48EI} z^2(3l - 4z), \quad \zeta\left(\frac{l}{2}\right) = \frac{fl^3}{192EI}.
    $$

    \item[b)] Ambos extremos apoyados:
    $$
    \zeta = \frac{f}{48EI} z(3l^2 - 4z^2), \quad \zeta\left(\frac{l}{2}\right) = \frac{fl^3}{48EI}.
    $$
\end{enumerate}

La forma de la barra es simétrica respecto a su punto medio, de modo que la función $\zeta (z)$ en la porción $\frac{l}{2}\ll z\ll l$ se obtiene cambiando simplemente $z$ por $l-z$ en estas expresiones.

\subsection*{Problema 3: Barra con un Extremo Empotrado y el Otro Libre}
\textbf{Enunciado.} Ídem, para una barra con uno de sus extremos ($z=0$) empotrado y el otro ($z=l$) libre: en este último esta aplicada una fuerza concentrada $f$.

\textit{Solución.} A lo largo de toda la barra es $F = \text{const} = f$, de modo que $\zeta^{\prime\prime\prime} = -\frac{f}{EI}$. Con las condiciones $\zeta = 0$, $\zeta' = 0$ para $z = 0$ y $\zeta'' = 0$ para $z = l$ obtenemos:

$$
\zeta = \frac{f}{6EI} z^2(3l - z), \quad \zeta(l) = \frac{fl^3}{3EI}.
$$

\subsection*{Problema 4: Barra con Extremos Fijos y Par de Fuerzas}
\textbf{Enunciado.} Hallar la forma de una barra con extremos fijos, combada por la acción de un par de fuerzas concentrado aplicado en su punto medio.

\textit{Solución.} A lo largo de toda la barra es $\zeta^{\prime\prime\prime\prime} = 0$, pero en el punto $z = l/2$ el momento $M = EI\zeta''$ experimenta un salto igual al momento $m$ del par concentrado, aplicado en él. Con condiciones apropiadas en los extremos:

\begin{enumerate}
    \item[a)] Ambos extremos empotrados:
    $$
    \zeta = \frac{m}{24EIl} z^2(l + 2z) \quad \text{cuando } 0 \leq z \leq l/2,
    $$
    $$
    \zeta = -\frac{m}{24EIl} (l - z)^2[l + 2(l - z)] \quad \text{cuando } l/2 \leq z \leq l.
    $$

    \item[b)] Ambos extremos articulados:
    $$
    \zeta = \frac{m}{24EIl} z(l^2 - 4z^2) \quad \text{cuando } 0 \leq z \leq l/2,
    $$
    $$
    \zeta = -\frac{m}{24EIl} (l - z)[l^2 - 4(l - z)^2] \quad \text{cuando } l/2 \leq z \leq l.
    $$
\end{enumerate}
La barra se flexiona en sentidos opuestos a uno y a otro lado en el punto $x=l/2$.

\subsection*{Problema 5: Par Concentrado en el Extremo Libre}
\textbf{Enunciado.} Ídem, si el par concentrado está aplicado en el extremo libre de una barra, cuyo otro extremo se ecuentra empotrado.

\textit{Solución.} A lo largo de toda la barra tenemos $M = EI\zeta'' = m$, y en el punto $z = 0$ es $\zeta = 0$, $\zeta' = 0$. La forma de la flexión viene dada por la fórmula:

$$
\zeta = \frac{m}{2EI} z^2.
$$

\section*{Fluidos}
\subsection*{Notación}
\begin{itemize}
    \item $\rho $ densidad del fluido.
    \item $P$ presión.
    \item $T$ temperatura.
    \item $s$ entropía por unidad de masa.
    \item $\epsilon$ energía interna por unidad de masa.
    \item $w = \epsilon + P/\rho$ funcion de calor (entalpia).
    \item $\gamma = c_{P}/c_{v}$ ratio de calores específicos a una presión constante y a un volumen constante.
    \item $\eta$ viscosidad dinámica.
    \item $v= \eta / \rho$ viscosidad cinemática.
    \item $k$ conductividad térmica.
    \item $\chi =k/\rho c_{p}$ coeficiente de difusión térmica.
    \item $R$ Numero de Reynolds.
    \item $c$ velocidad del sonido.
    \item $M$ Ratio de la velocidad del fluido a la velocidad del sonido. (Mach number).
\end{itemize}
\section*{Seccion 8: Conservacion de la circulación}
\subsection*{Problema 1}

\textbf{Enunciado.} Show that, in flow which is not isentropic, any moving particle carries with it a constant value of the product $(1/\rho) \, \nabla \, s \cdot \nabla \times \, \mathbf{v}$ (H. Ertel 1942).

\textit{Solución.}

When the flow is not isentropic, the right-hand side of Euler's equation (2.3) cannot be replaced by $-\nabla \, w$, and (2.11) becomes

$$
\frac{\partial \boldsymbol{\omega}}{\partial t} = \nabla \times \, (\mathbf{v} \times \boldsymbol{\omega}) + \left( \frac{1}{\rho^2} \right) \nabla \, \rho \times \nabla \, p,
$$

where for brevity $\boldsymbol{\omega} = \nabla \times \, \mathbf{v}$. We multiply scalarly by $\nabla \, s$; since $s = s(p, \rho)$, $\nabla \, s$ is a linear function of $\nabla \, p$ and $\nabla \, \rho$, and $\nabla \, s \cdot (\nabla \, \rho \times \nabla \, p) = 0$. The expression on the right-hand side can then be transformed as follows:

$$
\nabla \, s \cdot \frac{\partial \boldsymbol{\omega}}{\partial t} = \nabla \, s \cdot \nabla \times \, (\mathbf{v} \times \boldsymbol{\omega})
$$

$$
= -\nabla \cdot \, [\nabla \, s \times (\mathbf{v} \times \boldsymbol{\omega})]
$$

$$
= -\nabla \cdot \, [\mathbf{v} (\boldsymbol{\omega} \cdot \nabla \, s)] + \nabla \cdot \, [\boldsymbol{\omega} (\mathbf{v} \cdot \nabla \, s)]
$$

$$
= -(\boldsymbol{\omega} \cdot \nabla \, s) \, \nabla \cdot \, \mathbf{v} - \mathbf{v} \cdot \nabla \, (\boldsymbol{\omega} \cdot \nabla \, s) + \boldsymbol{\omega} \cdot \nabla \, (\mathbf{v} \cdot \nabla \, s).
$$

From (2.6), $\mathbf{v} \cdot \nabla \, s = -\partial s/\partial t$, and therefore

$$
\frac{\partial}{\partial t} (\boldsymbol{\omega} \cdot \nabla \, s) + \mathbf{v} \cdot \nabla \, (\boldsymbol{\omega} \cdot \nabla \, s) + (\boldsymbol{\omega} \cdot \nabla \, s) \, \nabla \cdot \, \mathbf{v} = 0.
$$

The first two terms can be combined as $d(\boldsymbol{\omega} \cdot \nabla \, s)/dt$, where $d/dt = \partial/\partial t + \mathbf{v} \cdot \nabla$; in the last term, we put from (1.3) $\rho \, \nabla \cdot \, \mathbf{v} = -d\rho/dt$. The result is

$$
\frac{d}{dt} \left( \frac{\boldsymbol{\omega} \cdot \nabla \, s}{\rho} \right) = 0,
$$

which gives the required conservation law.

\section*{Seccion 10: Fluidos incompresibles}
\subsection*{Problem 1}

\textbf{Enunciado.} Determine the shape of the surface of an incompressible fluid subject to a gravitational field, contained in a cylindrical vessel which rotates about its (vertical) axis with a constant angular velocity $\Omega$.

\textit{Solution.} Let us take the axis of the cylinder as the $z$-axis. Then $v_x = -y\Omega$, $v_y = x\Omega$, $v_z = 0$. The equation of continuity is satisfied identically, and Euler's equation (10.1) gives

$$
x\Omega^2 = \frac{1}{\rho} \frac{\partial p}{\partial x}, \quad y\Omega^2 = \frac{1}{\rho} \frac{\partial p}{\partial y}, \quad \frac{1}{\rho} \frac{\partial p}{\partial z} + g = 0.
$$

The general integral of these equations is

$$
p/\rho = \frac{1}{2} \Omega^2 (x^2 + y^2) - gz + \text{constant}.
$$

At the free surface $p = \text{constant}$, so that the surface is a paraboloid:

$$
z = \frac{1}{2} \Omega^2 (x^2 + y^2)/g,
$$

the origin being taken at the lowest point of the surface.

\subsection*{Problem 2}

\textbf{Enunciado.}A sphere, with radius $R$, moves with velocity $\mathbf{u}$ in an incompressible ideal fluid. Determine the potential flow of the fluid past the sphere.

\textit{Solution.} The fluid velocity must vanish at infinity. The solutions of Laplace's equation $\Delta \phi = 0$ which vanish at infinity are well known to be $1/r$ and the derivatives, of various orders, of $1/r$ with respect to the coordinates (the origin is taken at the centre of the sphere). On account of the complete symmetry of the sphere, only one constant vector, the velocity $\mathbf{u}$, can appear in the solution, and, on account of the linearity of both Laplace's equation and the boundary condition, $\phi$ must involve $\mathbf{u}$ linearly. The only scalar which can be formed from $\mathbf{u}$ and the derivatives of $1/r$ is the scalar product $\mathbf{u} \cdot \nabla (1/r)$. We therefore seek $\phi$ in the form

$$
\phi = \mathbf{A} \cdot \nabla (1/r) = -(\mathbf{A} \cdot \mathbf{n})/r^2,
$$

where $\mathbf{n}$ is a unit vector in the direction of $\mathbf{r}$. The constant $\mathbf{A}$ is determined from the condition that the normal components of the velocities $\mathbf{v}$ and $\mathbf{u}$ must be equal at the surface at the sphere, i.e. $\mathbf{v} \cdot \mathbf{n} = \mathbf{u} \cdot \mathbf{n}$ for $r = R$. This condition gives $\mathbf{A} = \frac{1}{2} \mathbf{u} R^3$, so that

$$
\phi = -\frac{R^3}{2r^2} \mathbf{u} \cdot \mathbf{n}, \quad \mathbf{v} = \frac{R^3}{2r^3} [3\mathbf{n}(\mathbf{u} \cdot \mathbf{n}) - \mathbf{u}].
$$

The pressure distribution is given by equation (10.7):

$$
p = p_0 - \frac{1}{2} \rho v^2 - \rho \frac{\partial \phi}{\partial t},
$$

where $p_0$ is the pressure at infinity. To calculate the derivative $\partial \phi/\partial t$, we must bear in mind that the origin (which we have taken at the centre of the sphere) moves with velocity $\mathbf{u}$. Hence

$$
\frac{\partial \phi}{\partial t} = (\partial \phi/\partial \mathbf{u}) \cdot \dot{\mathbf{u}} - \mathbf{u} \cdot \nabla \phi.
$$

The pressure distribution over the surface of the sphere is given by the formula

$$
p = p_0 + \frac{1}{8} \rho u^2 (9 \cos^2 \theta - 5) + \frac{1}{2} \rho R n \cdot \dot{\mathbf{u}},
$$

where $\theta$ is the angle between $\mathbf{n}$ and $\mathbf{u}$.

\subsection*{Problem 3}

\textbf{Enunciado.} The same as Problem 2, but for an infinite cylinder moving perpendicular to its axis.

\textit{Solution.} The flow is independent of the axial coordinate, so that we have to solve Laplace's equation in two dimensions. The solutions which vanish at infinity are the first and higher derivatives of $\log r$ with respect to the coordinates, where $r$ is the radius vector perpendicular to the axis of the cylinder. We seek a solution in the form

$$
\phi = \mathbf{A} \cdot \nabla \log r = \mathbf{A} \cdot \mathbf{n}/r,
$$

and from the boundary conditions we obtain $\mathbf{A} = -R^2 \mathbf{u}$, so that

$$
\phi = -\frac{R^2}{r} \mathbf{u} \cdot \mathbf{n}, \quad \mathbf{v} = \frac{R^2}{r^2} [2\mathbf{n}(\mathbf{u} \cdot \mathbf{n}) - \mathbf{u}].
$$

The pressure at the surface of the cylinder is given by

$$
p = p_0 + \frac{1}{2} \rho u^2 (4 \cos^2 \theta - 3) + \rho R n \cdot \frac{d\mathbf{u}}{dt}.
$$

\subsection*{Problem 4}

\textbf{Enunciado.} Determine the potential flow of an incompressible ideal fluid in an ellipsoidal vessel rotating about a principal axis with angular velocity $\Omega$, and determine the total angular momentum of the fluid.

\textit{Solution.} We take Cartesian coordinates $x, y, z$ along the axes of the ellipsoid at a given instant, the $z$-axis being the axis of rotation. The velocity of points in the vessel wall is

$$
\mathbf{u} = \Omega \mathbf{x} \mathbf{r},
$$

so that the boundary condition $v_n = \partial \phi/\partial n = u_n$ is

$$
\frac{\partial \phi}{\partial n} = \Omega (x_n, -y_n),
$$

or, using the equation of the ellipsoid $x^2/a^2 + y^2/b^2 + z^2/c^2 = 1$,

$$
\frac{x}{a^2} \frac{\partial \phi}{\partial x} + \frac{y}{b^2} \frac{\partial \phi}{\partial y} + \frac{z}{c^2} \frac{\partial \phi}{\partial z} = xy \Omega \left( \frac{1}{b^2} - \frac{1}{a^2} \right).
$$

The solution of Laplace's equation which satisfies this boundary condition is

$$
\phi = \Omega \frac{a^2 - b^2}{a^2 + b^2} xy. {1}
$$

The angular momentum of the fluid in the vessel is

$$
M = \rho \int (xy - yx) \, dV.
$$

Integrating over the volume $V$ of the ellipsoid, we have

$$
M = \frac{\Omega \rho V (a^2 - b^2)^2}{5(a^2 + b^2)}.
$$

Formula (1) gives the absolute motion of the fluid relative to the instantaneous position of the axes $x, y, z$ which are fixed to the rotating vessel. The motion relative to the vessel is found by subtracting the velocity $\Omega \mathbf{x} \mathbf{r}$ from the absolute velocity; denoting the relative velocity of the fluid by $\mathbf{v}'$, we have

$$
v'_x = \frac{\partial \phi}{\partial x} + y \Omega = \frac{2 \Omega a^2}{a^2 + b^2} y, \quad v'_y = -\frac{2 \Omega b^2}{a^2 + b^2} x, \quad v'_z = 0.
$$

The paths of the relative motion are found by integrating the equations $\dot{x} = v'_x$, $\dot{y} = v'_y$, and are the ellipses $x^2/a^2 + y^2/b^2 = \text{constant}$, which are similar to the boundary ellipse.

\subsection*{Problem 5}

\textbf{Enunciado.} Determine the flow near a stagnation point (Fig. 2).

\textit{Solution.} A small part of the surface of the body near the stagnation point may be regarded as plane. Let us take it as the $xy$-plane. Expanding $\phi$ for $x, y, z$ small, we have as far as the second-order terms

$$
\phi = ax + by + cz + Ax^2 + By^2 + Cz^2 + Dxy + Eyz + Fzx;
$$

a constant term in $\phi$ is immaterial. The constant coefficients are determined so that $\phi$ satisfies the equation $\Delta \phi = 0$ and the boundary conditions $v_z = \partial \phi/\partial z = 0$ for $z = 0$ and all $x, y$, $\partial \phi/\partial x = \partial \phi/\partial y = 0$ for $x = y = z = 0$ (the stagnation point). This gives $a = b = c = 0; C = -A - B, E = F = 0$. The term $Dxy$ can always be removed by an appropriate rotation of the $x$ and $y$ axes. We then have

$$
\phi = Ax^2 + By^2 - (A + B)z^2. {1}
$$

If the flow is axially symmetrical about the $z$-axis (symmetrical flow past a solid of revolution), we must have $A = B$, so that

$$
\phi = A(x^2 + y^2 - 2z^2).
$$

The velocity components are $v_x = 2Ax, v_y = 2Ay, v_z = -4Az$. The streamlines are given by equations (5.2), from which we find $x^2z = c_1, y^2z = c_2$, i.e. the streamlines are cubical hyperbolae.

If the flow is uniform in the $y$-direction (e.g. flow in the $z$-direction past a cylinder with its axis in the $y$-direction), we must have $B = 0$ in (1), so that

$$
\phi = A(x^2 - z^2).
$$

The streamlines are the hyperbolae $xz = \text{constant}$.

\subsection*{Problem 6}

\textbf{Enunciado.} Determine the potential flow near an angle formed by two intersecting planes.

\textit{Solution.} Let us take polar coordinates $r, \theta$ in the cross-sectional plane (perpendicular to the line of intersection), with the origin at the vertex of the angle; $\theta$ is measured from one of the arms of the angle. Let the angle be $\alpha$ radians; for $\alpha < \pi$ the flow takes place within the angle, for $\alpha > \pi$ outside it. The boundary condition that the normal velocity component vanish means that $\partial \phi/\partial \theta = 0$ for $\theta = 0$ and $\theta = \alpha$. The solution of Laplace's equation satisfying these conditions can be written

$$
\phi = Ar^n \cos n\theta, \quad n = \pi/\alpha,
$$

so that

$$
v_r = nAr^{n-1} \cos n\theta, \quad v_\theta = -nAr^{n-1} \sin n\theta.
$$

For $n < 1$ (flow outside an angle; Fig. 3), $v_r$ becomes infinite as $1/r^{1-n}$ at the origin. For $n > 1$ (flow inside an angle; Fig. 4), $v_r$ becomes zero for $r = 0$.

The stream function, which gives the form of the streamlines, is $\psi = Ar^n \sin n\theta$. The expressions obtained for $\phi$ and $\psi$ are the real and imaginary parts of the complex potential $w = Az^n$.

\subsection*{Problem 7}

\textbf{Enunciado.} A spherical hole with radius $a$ is suddenly formed in an incompressible fluid filling all space. Determine the time taken for the hole to be filled with fluid (Besant 1859; Rayleigh 1917).

\subsection*{Solution}

The flow after the formation of the hole will be spherically symmetrical, the velocity at every point being directed to the centre of the hole. For the radial velocity $v_r \equiv v < 0$ we have Euler's equation in spherical polar coordinates:

$$
\frac{\partial v}{\partial t} + v \frac{\partial v}{\partial r} = -\frac{1}{\rho} \frac{\partial p}{\partial r}. {1}
$$

The equation of continuity gives

$$
r^2 v = F(t), {2}
$$

where $F(t)$ is an arbitrary function of time; this equation expresses the fact that, since the fluid is incompressible, the volume flowing through any spherical surface is independent of the radius of that surface.

Substituting $v$ from (2) in (1), we have

$$
\frac{F'(t)}{r^2} + v \frac{\partial v}{\partial r} = -\frac{1}{\rho} \frac{\partial p}{\partial r}.
$$

Integrating this equation over $r$ from the instantaneous radius $R = R(t) \leq a$ of the hole to infinity, we obtain

$$
-\frac{F'(t)}{R} + \frac{1}{2} v^2 = \frac{p_0}{\rho}, {3}
$$

where $\dot{R} = dR(t)/dt$ is the rate of change of the radius of the hole, and $p_0$ is the pressure at infinity; the fluid velocity at infinity is zero, and so is the pressure at the surface of the hole. From equation (2) for points on the surface of the hole we find

$$
F(t) = R^2(t) \dot{R}(t),
$$

and, substituting this expression for $F(t)$ in (3), we obtain the equation

$$
\frac{3\dot{R}^2}{2} - \frac{1}{2} R \frac{d\dot{R}^2}{dR} = \frac{p_0}{\rho}. {4}
$$

The variables are separable; integrating with the boundary condition $\dot{R} = 0$ for $R = a$ (the fluid being initially at rest), we have

$$
\dot{R} \equiv \frac{dR}{dt} = -\sqrt{\frac{2p_0}{3\rho} \left( a^3/R^3 - 1 \right)}.
$$

Hence we have for the required total time for the hole to be filled

$$
\tau = \sqrt{\frac{3\rho}{2p_0}} \int_0^a \frac{dR}{\sqrt{[(a/R)^3 - 1]}}.
$$

This integral reduces to a beta function, and we have finally

$$
\tau = \sqrt{\frac{3a^2 \rho \pi \Gamma(5/6)}{2p_0 \Gamma(1/3)}} = 0.915a \sqrt{\frac{\rho}{p_0}}.
$$

\subsection*{Problem 8}

\textbf{Enunciado.} A sphere immersed in an incompressible fluid expands according to a given law $R = R(t)$. Determine the fluid pressure at the surface of the sphere.

\subsection*{Solution}

Let the required pressure be $P(t)$. Calculations exactly similar to those of Problem 7, except that the pressure at $r = R$ is $P(t)$ and not zero, give instead of (3) the equation

$$
-\frac{F'(t)}{R(t)} + \frac{1}{2} V^2 = \frac{p_0}{\rho} - \frac{P(t)}{\rho}
$$

and accordingly instead of (4) the equation

$$
\frac{p_0 - P(t)}{\rho} = -\frac{3V^2}{2} - R \frac{V' dV}{dR}.
$$

Bearing in mind the fact that $V = dR/dt$, we can write the expression for $P(t)$ in the form

$$
P(t) = p_0 + \frac{1}{2} \left[ \frac{d^2(R^2)}{dt^2} + \left( \frac{dR}{dt} \right)^2 \right].
$$

\section*{Seccion 17: Flujo en una tuberia}

\subsection*{Problem 1}

Determine the flow in a pipe of annular cross-section, the internal and external radii being $R_1, R_2$.

\textit{Solution.} Determining the constants $a$ and $b$ in the general solution (17.8) from the conditions that $v = 0$ for $r = R_1$ and $r = R_2$, we find

$$
v = \frac{\Delta p}{4\eta l} \left[ R_2^2 - r^2 + \frac{R_2^2 - R_1^2}{\log(R_2/R_1)} \log \frac{r}{R_2} \right]
$$

The discharge is

$$
Q = \frac{\pi \Delta p}{8\eta l} \left[ R_2^4 - R_1^4 - \frac{(R_2^2 - R_1^2)^2}{\log(R_2/R_1)} \right].
$$

\subsection*{Problem 2}

The same as Problem 1, but for a pipe of elliptical cross-section.

\textit{Solution.} We seek a solution of equation (17.7) in the form $v = Ay^2 + Bz^2 + C$. The constants $A, B, C$ are determined from the requirement that this expression must satisfy the boundary condition $v = 0$ on the circumference of the ellipse (i.e. $Ay^2 + Bz^2 + C = 0$ must be the same as the equation $y^2/a^2 + z^2/b^2 = 1$, where $a$ and $b$ are the semi-axes of the ellipse). The result is

$$
v = \frac{\Delta p}{2\eta l} \frac{a^2 b^2}{a^2 + b^2} \left( 1 - \frac{y^2}{a^2} - \frac{z^2}{b^2} \right).
$$

The discharge is

$$
Q = \frac{\pi \Delta p}{4\eta l} \frac{a^3 b^3}{a^2 + b^2}.
$$

\subsection*{Problem 3}

The same as Problem 1, but for a pipe whose cross-section is an equilateral triangle with side $a$.

\textit{Solution.} The solution of equation (17.7) which vanishes on the bounding triangle is

$$
v = \frac{\Delta p}{l} \frac{2}{3a\eta} h_1 h_2 h_3,
$$

where $h_1, h_2, h_3$ are the lengths of the perpendiculars from a given point in the triangle to its three sides. For each of the expressions $\Delta h_1, \Delta h_2, \Delta h_3$ (where $\Delta = \partial^2/\partial y^2 + \partial^2/\partial z^2$) is zero; this is seen at once from the fact that each of the perpendiculars $h_1, h_2, h_3$ may be taken as the axis of $y$ or $z$, and the result of applying the Laplacian to a coordinate is zero. We therefore have

$$
\Delta (h_1 h_2 h_3) = 2(h_1 \, \nabla \, h_2 \cdot \nabla \, h_3 + h_2 \, \nabla \, h_3 \cdot \nabla \, h_1 + h_3 \, \nabla \, h_1 \cdot \nabla \, h_2)
$$

But $\nabla \, h_1 = \mathbf{n}_1, \nabla \, h_2 = \mathbf{n}_2, \nabla \, h_3 = \mathbf{n}_3$, where $\mathbf{n}_1, \mathbf{n}_2, \mathbf{n}_3$ are unit vectors along the perpendiculars $h_1, h_2, h_3$. Any two of $\mathbf{n}_1, \mathbf{n}_2, \mathbf{n}_3$ are at an angle $2\pi/3$, so that $\nabla \, h_1 \cdot \nabla \, h_2 = \mathbf{n}_1 \cdot \mathbf{n}_2 = \cos(2\pi/3) = -\frac{1}{2}$, and so on. We thus obtain the relation

$$
\Delta (h_1 h_2 h_3) = -(h_1 + h_2 + h_3) = -\frac{1}{2} \sqrt{3} a,
$$

and we see that equation (17.7) is satisfied. The discharge is

$$
Q = \frac{\sqrt{3} a^4 \Delta p}{320 \eta l}.
$$

\subsection*{Problem 4}

A cylinder with radius $R_1$ moves parallel to its axis with velocity $u$ inside a coaxial cylinder with radius $R_2$. Determine the motion of a fluid occupying the space between the cylinders.

\textit{Solution.} We take cylindrical polar coordinates, with the $z$-axis along the axis of the cylinders. The velocity is everywhere along the $z$-axis and depends only on $r$ (as does the pressure); $v_z = v(r)$. We obtain for $v$ the equation

$$
\Delta v = -\frac{1}{r} \frac{d}{dr} \left( r \frac{dv}{dr} \right) = 0;
$$

the term $(\mathbf{v} \cdot \nabla)w = v \, \partial v/\partial z$ vanishes identically. Using the boundary conditions $v = u$ for $r = R_1$ and $v = 0$ for $r = R_2$, we find

$$
v = \frac{u \log(r/R_2)}{\log(R_1/R_2)}.
$$

The frictional force per unit length of either cylinder is $2\pi \eta u/\log(R_2/R_1)$.

\subsection*{Problem 5}

A layer of fluid with thickness $h$ is bounded above by a free surface and below by a fixed plane inclined at an angle $\alpha$ to the horizontal. Determine the flow due to gravity.

\textit{Solution.} We take the fixed plane as the $xy$-plane, with the $x$-axis in the direction of flow (Fig. 6). We seek a solution depending only on $z$. The Navier–Stokes equations with $v_x = v(z)$ in a gravitational field are

$$
\eta \frac{d^2 v}{dz^2} + \rho g \sin \alpha = 0, \quad \frac{dp}{dz} + \rho g \cos \alpha = 0.
$$

At the free surface ($z = h$) we must have $\sigma_{xz} = \eta dv/dz = 0$, $\sigma_{zz} = -p = -p_0$ ($p_0$ being the atmospheric pressure). For $z = 0$ we must have $v = 0$. The solution satisfying these conditions is

$$
p = p_0 + \rho g (h - z) \cos \alpha, \quad v = \frac{\rho g \sin \alpha}{2\eta} z(2h - z).
$$

The discharge, per unit length in the $y$-direction, is

$$
Q = \rho \int_0^h v \, dz = \frac{\rho g h^3 \sin \alpha}{3\eta}.
$$

\section*{Seccion 20: Flujo con pequeños numueros de Reynolds}

\subsection*{Problem 1}

Determine the motion of a fluid occupying the space between two concentric spheres with radii $R_1, R_2$ ($R_2 > R_1$), rotating uniformly about different diameters with angular velocities $\Omega_1, \Omega_2$; the Reynolds numbers $\Omega_1 R_1^2/\nu, \Omega_2 R_2^2/\nu$ are small compared with unity.

\textit{Solution.} On account of the linearity of the equations, the motion between two rotating spheres may be regarded as a superposition of the two motions obtained when one sphere is at rest and the other rotates. We first put $\Omega_2 = 0$, i.e. only the inner sphere is rotating. It is reasonable to suppose that the fluid velocity at every point is along the tangent to a circle in a plane perpendicular to the axis of rotation with its centre on the axis. On account of the axial symmetry, the pressure gradient in this direction is zero. Hence the equation of motion (20.1) becomes $\Delta v = 0$. The angular velocity vector $\Omega_1$ is an axial vector. Arguments similar to those given previously show that the velocity can be written as

$$
\mathbf{v} = \nabla \times [f(r) \Omega_1] = \nabla f \times \Omega_1.
$$

The equation of motion then gives $\nabla \Delta f \times \Omega_1 = 0$. Since the vector $\nabla \Delta f$ is parallel to the position vector, and the vector product $\mathbf{r} \times \Omega_1$ cannot be zero for given $\Omega_1$ and arbitrary $\mathbf{r}$, we must have $\nabla \Delta f = 0$, so that $\Delta f = \text{constant}$.

Integrating, we find

$$
f = ar^2 + \frac{b}{r}, \quad \mathbf{v} = \left( \frac{b}{r^3} - 2a \right) \Omega_1 \times \mathbf{r}.
$$

The constants $a$ and $b$ are found from the conditions that $v = 0$ for $r = R_2$ and $v = u$ for $r = R_1$, where $u = \Omega_1 \times \mathbf{r}$ is the velocity of points on the rotating sphere. The result is

$$
\mathbf{v} = \frac{R_1^3 R_2^3}{R_2^3 - R_1^3} \left( \frac{1}{r^3} - \frac{1}{R_2^3} \right) \Omega_1 \times \mathbf{r}.
$$

The fluid pressure is constant ($p = p_0$). Similarly, we have for the case where the outer sphere rotates and the inner one is at rest ($\Omega_1 = 0$)

$$
\mathbf{v} = \frac{R_1^3 R_2^3}{R_2^3 - R_1^3} \left( \frac{1}{R_1^3} - \frac{1}{r^3} \right) \Omega_2 \times \mathbf{r}.
$$

In the general case where both spheres rotate, we have

$$
\mathbf{v} = \frac{R_1^3 R_2^3}{R_2^3 - R_1^3} \left( \left( \frac{1}{r^3} - \frac{1}{R_2^3} \right) \Omega_1 \times \mathbf{r} + \left( \frac{1}{R_1^3} - \frac{1}{r^3} \right) \Omega_2 \times \mathbf{r} \right).
$$

If the outer sphere is absent ($R_2 = \infty, \Omega_2 = 0$), i.e. we have simply a sphere with radius $R$ rotating in an infinite fluid, then

$$
\mathbf{v} = (R^3/r^3) \Omega \times \mathbf{r}.
$$

Let us calculate the moment of the frictional forces acting on the sphere in this case. If we take spherical polar coordinates with the polar axis parallel to $\Omega$, we have $v_r = v_\theta = 0, v_\phi = (R^2 \Omega/r^2) \sin \theta$. The frictional force on unit area of the sphere is

$$
\sigma_{r\phi} = \eta \left( \frac{\partial v_\phi}{\partial r} - \frac{v_\phi}{r} \right)_{r=R} = -3\eta \Omega \sin \theta.
$$

The total moment on the sphere is

$$
M = \int_0^\pi \sigma_{r\phi} R \sin \theta \cdot 2\pi R^2 \sin \theta \, d\theta,
$$

whence we find

$$
M = -8\pi \eta R^3 \Omega.
$$

If the inner sphere is absent, $\mathbf{v} = \Omega \times \mathbf{r}$, i.e. the fluid simply rotates rigidly with the sphere surrounding it.

\subsection*{Problem 3}

Two parallel plane circular disks (with radius $R$) lie one above the other a small distance apart; the space between them is filled with fluid. The disks approach at a constant velocity $u$, displacing the fluid. Determine the resistance to their motion (O. Reynolds).

\textit{Solution.} We take cylindrical polar coordinates, with the origin at the centre of the lower disk, which we suppose fixed. The flow is axially symmetric and, since the fluid layer is thin, predominantly radial: $v_z \ll v_r$, and also $\partial v_r/\partial r \ll \partial v_r/\partial z$. Hence the equations of motion become

$$
\eta \frac{\partial^2 v_r}{\partial z^2} = \frac{\partial p}{\partial r}, \quad \frac{\partial p}{\partial z} = 0, {1}
$$

$$
\frac{1}{r} \frac{\partial (rv_r)}{\partial r} + \frac{\partial v_z}{\partial z} = 0, {2}
$$

with the boundary conditions

$$
\text{at } z = 0: \quad v_r = v_z = 0;
$$

$$
\text{at } z = h: \quad v_r = 0, \quad v_z = -u;
$$

$$
\text{at } r = R: \quad p = p_0,
$$

where $h$ is the distance between the disks, and $p_0$ the external pressure. From equations (1) we find

$$
v_r = \frac{1}{2\eta} \frac{dp}{dr} z(z - h).
$$

Integrating equation (2) with respect to $z$, we obtain

$$
u = \frac{1}{r} \frac{d}{dr} \int_0^h rv_r \, dz = -\frac{h^3}{12\eta r} \frac{d}{dr} \left( r \frac{dp}{dr} \right),
$$

whence

$$
p = p_0 + \frac{3\eta u}{h^3} (R^2 - r^2).
$$

The total resistance to the moving disk is

$$
F = \frac{3\pi \eta u R^4}{2h^3}.
$$

\end{document}