\documentclass{article}
\usepackage{amsmath}
\usepackage{amssymb}
\usepackage{geometry}
\geometry{a4paper, margin=1in}
\usepackage{graphicx}

\begin{document}

\section*{Formulas relevantes}
$$ u_{ij} = \frac{1}{2}\left(\frac{\partial u_i}{\partial x_j} + \frac{\partial u_j}{\partial x_i}\right) \text{ Deformacion} $$
$$ \frac{\partial \sigma_{ij}}{\partial x_j}+f_i =0 \text{ Eq equilibrio, tensor de esfuerzo} $$

$$ \sigma_{ij} = \frac{E}{1+\sigma}(u_{ij}+\frac{\sigma}{1-2\sigma}u_{ll}\delta_{ij}) = \lambda u_{ll} \delta_{ij} + 2\mu u_{ij} \text{ Ley de Hooke}$$

$$ u_{ij} = \frac{1}{E}[(1+\sigma)\sigma_{ij} - \sigma \sigma_{ll}\delta_{ij}] \text{ Ley de Hooke reciproca}$$

Modulo de young $E = \frac{\mu (3\lambda+2\mu)}{\lambda + \mu} $

Coeficiente de Poisson $\nu = \frac{\lambda}{2(\lambda + \mu)} $

Modulo de compresibilidad $K = \lambda + \frac{2}{3}\mu $

\section*{Sección 7: Deformaciones en Sólidos}

\subsection*{Problema 1: Deformación de una Varilla Vertical}
\textbf{Enunciado.} Determinar la deformación de una larga varilla (longitud \textit{l}) vertical y en reposo en un campo gravitatorio

\textit{Solución.} Tomamos el eje $z$ en la dirección de la varilla, y el plano $x, y$ coincidente con el plano de su extremo inferior. Las ecuaciones de equilibrio son:

$$
\frac{\partial \sigma_{xi}}{\partial x_{i}} = \frac{\partial \sigma_{yi}}{\partial x_{i}} = 0, \quad \frac{\partial \sigma_{zi}}{\partial x_{i}} = \rho g.
$$

En la superficie lateral de la varilla, todas las componentes de $\sigma_{ik}$ deben anularse, excepto $\sigma_{zz}$. En el extremo superior ($z = l$), $\sigma_{xz} = \sigma_{yz} = \sigma_{zz} = 0$. La solución que satisface estas condiciones es $\sigma_{zz} = -\rho g(l - z)$, con las demás $\sigma_{ik}$ nulas.

Para $u_{ik}$, obtenemos:

$$
u_{xx} = u_{yy} = \frac{\sigma}{E}\rho g(l - z), \quad u_{zz} = -\frac{\rho g(l - z)}{E}, \quad u_{xy} = u_{xz} = u_{yz} = 0.
$$

Integrando, las componentes del vector de desplazamiento son:

$$
u_x = \frac{\sigma}{E}\rho g(l - z)x, \quad u_y = \frac{\sigma}{E}\rho g(l - z)y, \quad u_z = -\frac{\rho g}{2E} \{l^2 - (l - z)^2 - \sigma(x^2 + y^2)\}.
$$

La solución para $u_z$ satisface $u_z = 0$ solo en un punto del extremo inferior de la varilla, por lo que no es válida cerca de este extremo.

\subsection*{Problema 2: Deformación de una Esfera Hueca}
\textbf{Enunciado.} Hallar la deformación de una esfera hueca (de radios externo e interno $R_2$ y $R_1$, respectivamente), sometida a una presión interna $p_1$ y a una presión externa $p_2$.

\textit{Nota:} Equación 7.5 $$2(1-\sigma)\nabla(\nabla \cdot \vec{u}) - (1-2\sigma)\nabla \times \nabla \times \vec{u} = 0$$
Formulas 1.7 $$ u_{rr} = \frac{\partial u_r}{\partial r}, \quad u_{\theta \theta} = \frac{1}{r}\frac{\partial u_\theta}{\partial \theta} + \frac{u_r}{r}, \quad u_{\varphi \varphi} = \frac{1}{r\sin(\theta)}\frac{\partial u_\varphi}{\partial \varphi} + \frac{u_\theta}{r}\cot(\theta) + \frac{u_r}{r} $$
$$ 2u_{\theta \varphi} = \frac{1}{r}\left(\frac{\partial u_\varphi}{\partial \theta}- u_\varphi \cot(\theta)\right) + \frac{1}{r\sin(\theta)}\frac{\partial u_\varphi}{\partial \varphi}, \quad 2u_{r\theta} = \frac{\partial u_\theta }{\partial r}-\frac{u_\theta}{r} + \frac{1}{r}\frac{\partial u_r }{\partial \theta}$$
$$ 2u_{\varphi r}=\frac{1}{r\sin(\theta)}\frac{\partial u_r}{\partial \varphi}+\frac{\partial u_\varphi}{\partial r}-\frac{u_\varphi}{r} $$

\textit{Solución.} Usamos coordenadas esféricas con origen en el centro de la esfera. El desplazamiento $\vec{u}$ es radial y función de $r$ solamente. Así, $\nabla \times \vec{u} = 0$ y (eq 7.5)$\nabla \cdot \vec{u} = 0$. Resulta:

$$
\text{div } \vec{u} = \frac{1}{r^2} \frac{d(r^2 u)}{dr} = \text{const} \equiv 3a,
$$

donde $u = ar + \frac{b}{r^2}$. Las componentes del tensor de deformación son (Véase formulas 1.7) $u_{rr} = a - \frac{2b}{r^3}$, $u_{\theta\theta} = u_{\varphi\varphi} = a + \frac{b}{r^3}$. La tensión radial es:

$$
\sigma_{rr} = \frac{E}{(1+\sigma)(1-2\sigma)} \{(1-\sigma)u_{rr} + 2\sigma u_{\theta\theta}\} = \frac{E}{1-2\sigma} a - \frac{2E}{1+\sigma} \frac{b}{r^3}.
$$

Las constantes $a$ y $b$ se determinan por las condiciones de contorno: $\sigma_{rr} = -p_1$ para $r = R_1$ y $\sigma_{rr} = -p_2$ para $r = R_2$:

$$
a = \frac{p_1 R_1^3 - p_2 R_2^3}{R_2^3 - R_1^3} \cdot \frac{1-2\sigma}{E}, \quad b = \frac{R_1^3 R_2^3 (p_1 - p_2)}{R_2^3 - R_1^3} \cdot \frac{1+\sigma}{2E}.
$$

Para una capa esférica con $p = p_1$ y $p_2 = 0$:

$$
\sigma_{rr} = \frac{p R_2^3}{R_2^3 - R_1^3} \left(1 - \frac{R_2^3}{r^3}\right), \quad \sigma_{\theta\theta} = \sigma_{\varphi\varphi} = \frac{p R_2^3}{R_2^3 - R_1^3} \left(1 + \frac{R_1^3}{r^3}\right).
$$

Para una cáscara esférica delgada, de espesor $h= R_2-R_1 \llless R$ obtenemos aproximadamente:

$$
u = \frac{p R^2 (1-\sigma)}{2Eh}, \quad \sigma_{\theta\theta} = \sigma_{\varphi\varphi} = \frac{pR}{2h}, \quad \bar{\sigma}_{rr} = \frac{p}{2}.
$$

Donde $\bar{\sigma_{rr}}$ es el valor medio de la tensión radial promediada en el espesor de la cascara. Obtenemos la distribucion de tensiones en un medio elástico infinito con cavidad hueca (de radio R) sometido a compresión hidrostática, simplemente poniendo en las ecuaciones anteriores $R_1 = R$, $R_2 = \infty$, $p_1 = 0$ y $p_2 = p$:

$$
\sigma_{rr} = -p \left(1 - \frac{R^3}{r^3}\right), \quad \sigma_{\theta\theta} = \sigma_{\varphi\varphi} = -p \left(1 + \frac{R^3}{2r^3}\right).
$$

La tensión tangencial en la superficie es $\sigma_{\theta\theta} = \sigma_{\varphi\varphi} = -3p/2$.

\subsection*{Problema 3: Deformación de una Esfera Maciza}
\textbf{Enunciado.} Determinar la deformación de una esfera maciza (de radio $R$) bajo la acción de su propio campo gravitatorio.

\textit{Nota:} Equación 7.3: 
$$ \nabla (\nabla \cdot \vec{u}) - \frac{1-2\sigma}{2(1-\sigma)}\nabla \times \nabla \times \vec{u} = -\rho \vec{g}\frac{(1+\sigma)(1-2\sigma)}{E(1-\sigma)}$$

\textit{Solución.} La fuerza gravitatoria que actua sobre la unidad de masa en un cuerpo esférico es $-\frac{g\vec{r}}{R}$. Sustituyendo esta expresion en lugar de $g$ en la ecuación (7.3), obtenemos:

$$
\frac{E(1-\sigma)}{(1+\sigma)(1-2\sigma)} \frac{d}{dr} \left( \frac{1}{r^2} \frac{d(r^2 u)}{dr} \right) = \rho g \frac{r}{R}.
$$

La solución para $r = 0$ y $\sigma_{rr} = 0$ en $r = R$ es:

$$
u = -\frac{g\rho R(1-2\sigma)(1+\sigma)}{10E(1-\sigma)} r \left(\frac{3-\sigma}{1+\sigma} - \frac{r^2}{R^2}\right).
$$

La materia está comprimida ($u_{rr} < 0$) dentro de $R\sqrt{\frac{3-\sigma}{3(1+\sigma)}}$ y dilatada fuera ($u_{rr} > 0$). La presión en el centro es $\frac{3-\sigma}{10(1-\sigma)} g \rho R$.

\subsection*{Problema 4: Deformación de un Tubo Cilíndrico}
\textbf{Enunciado.} Hallar la deformación de un tubo cilíndrico de radios externo e interno $R_2$ y $R_1$, respectivamente, sometido unicamente a una presión interna $p$ (se supone que se mantiene constante la longitud del cilindro, de manera que no hay deformación longitudinal).

\textit{Nota:} Formulas 1.8: $$ u_{rr} = \frac{\partial u_r}{\partial r}, \quad u_{\varphi \varphi}=\frac{1}{r}\frac{\partial u_\varphi}{\partial \varphi} + \frac{u_r}{r}, \quad u_{zz} = \frac{\partial u_z}{\partial z} $$
$$ 2u_{\varphi z}=\frac{1}{r}\frac{\partial u_z}{\partial \varphi}+\frac{\partial u_\varphi}{\partial z}, \quad 2u_{rz}=\frac{\partial u_r}{\partial z} + \frac{\partial u_z}{\partial r}$$
$$ 2u_{r \varphi}= \frac{\partial u_\varphi}{\partial r}-\frac{u_\varphi}{r}+\frac{1}{r}\frac{\partial u}{\partial \varphi} $$

\textit{Solución.} Usamos coordenadas cilíndricas con el eje $z$ a lo largo del tubo. La presión uniforme causa un desplazamiento radial puro $u_r = u(r)$. Análogamente al problema 2:

$$
\text{div } \vec{u} = \frac{1}{r} \frac{d(ru)}{dr} = \text{const} \equiv 2a.
$$

Por lo tanto, $u = ar + \frac{b}{r}$. Las componentes no nulas del tensor de deformaciones son $u_{rr} = \frac{du}{dr} = a - \frac{b}{r^2}$, $u_{\varphi\varphi} = \frac{u}{r} = a + \frac{b}{r^2}$. De las condiciones $\sigma_{rr} = 0$ en $r = R_2$ y $\sigma_{rr} = -p$ en $r = R_1$:

$$
a = \frac{pR_1^2}{R_2^2 - R_1^2} \cdot \frac{(1+\sigma)(1-2\sigma)}{E}, \quad b = \frac{pR_1^2 R_2^2}{R_2^2 - R_1^2} \cdot \frac{1+\sigma}{E}.
$$

La distribución de tensiones en el espesor de las paredes del tubo esta dada por:

$$
\sigma_{rr} = \frac{pR_1^2}{R_2^2 - R_1^2} \left(1 - \frac{R_2^2}{r^2}\right), \quad \sigma_{\varphi\varphi} = \frac{pR_1^2}{R_2^2 - R_1^2} \left(1 + \frac{R_2^2}{r^2}\right), \quad \sigma_{zz} = 2\sigma \frac{pR_1^2}{R_2^2 - R_1^2}.
$$

\subsection*{Problema 5: Deformación de un Cilindro Giratorio}
\textbf{Enunciado.} Determinar la deformación de un cilindro que gira uniformemente alrededor de su eje.

\textit{Solución.} Reemplazamos la fuerza gravitatoria en (7.3) por la centrífuga $\rho \Omega^2 r$. La ecuación para $u_r = u(r)$ es:

$$
\frac{E(1-\sigma)}{(1+\sigma)(1-2\sigma)} \frac{d}{dr} \left( \frac{1}{r} \frac{d(ru_r)}{dr} \right) = -\rho \Omega^2 r.
$$

La solución para $r = 0$ y $\sigma_{rr} = 0$ en $r = R$ es:

$$
u = \frac{\rho \Omega^2 (1+\sigma)(1-2\sigma)}{8E(1-\sigma)} r [(3-2\sigma)R^2 - r^2].
$$

\section*{Sección 16: Rigidez a la Torsión}

\subsection*{Problema 1: Barra de Sección Circular}
\textbf{Enunciado.} Determinar la rigidez a la torsión de una barra de sección circular (de radio R).

\textit{Nota:}

De pag 97: Conviene notar también una analogía hidrodinámica: una equación de la forma $\nabla \chi = -1$ (16.11), determina la distribucion de velocidades $v(x,y)$ de una liquido viscoso en la sección de una tubo: la condición de contorno $\chi = 0$ (16.12), corresponde a la condición $v=0$ en las paredes inmóviles del tubo.

Equación 16.10: $$ \frac{\partial \psi}{\partial x} = y + 2\frac{\partial \chi}{\partial y}, \quad \frac{\partial \psi}{\partial y} = -x -2\frac{\partial \chi}{\partial x} $$

Equación 16.4: $$ u_z = \tau \psi(x,y) $$

\textit{Solución.} Las soluciones de los problemas 1-4 coinciden formalmente con las soluciones de los problemas del movimiento de un líquido viscoso en un tubo de sección correspondiente (nota de pag 97) a la cantidad de liquido $Q$ que pasa a través de una sección del tubo correspondiente aqui la cantidad $C$. Para una barra de sección circular, tenemos:

$$
\chi = \frac{1}{4} (R^2 - x^2 - y^2).
$$

La rigidez a la torsión es:

$$
C = \frac{\mu \pi R^4}{2}.
$$

Para la función $\psi$, (de 16.10) se deduce $\psi = \text{const}$. La constante (según 16.4) $\psi$ corresponde a un desplazamiento de toda la barra a lo largo del eje $z$, por lo que se puede considerar $\psi = 0$. Así, las secciones transversales permanecen planas.

\subsection*{Problema 2: Barra de Sección Elíptica}
\textbf{Enunciado.} Lo mismo para una barra de sección eliptica (de semiejes $a$ y $b$).

\textit{Solución.} La rigidez a la torsión es:

$$
C = \pi \mu \frac{a^3 b^3}{a^2 + b^2}.
$$

La distribución de desplazamientos longitudinales está dada por la función de torsión:

$$
\psi = \frac{b^2 - a^2}{b^2 + a^2} xy.
$$
Los ejes de coordenadas están dirigidos según los ejes de la elipse.

\subsection*{Problema 3: Barra de Sección Triangular}
\textbf{Enunciado.} Lo mismo para una barra cuya sección es un triángulo equilatero (de lado $a$).

\textit{Solución.} Rigidez a la torsión:

$$
C = \frac{\sqrt{3}}{80} \mu a^4.
$$

Función de torsión:

$$
\psi = \frac{y(x\sqrt{3} + y)(x\sqrt{3} - y)}{6a}.
$$

El origen de coordenadas se elige en el centro del triángulo, y el eje $x$ coincide con una de sus alturas.

\subsection*{Problema 4: Barra con Forma de Placa Delgada}
\textbf{Enunciado.} Ídem para una barra que tiene la forma de una larga placa delgada (de ancho $d$ y grueso $h\llless d$).

\textit{Solución.} El problema es equivalente al flujo de un líquido viscoso entre paredes planas y paralelas. El resultado es:

$$
C = \frac{\mu dh^3}{3}.
$$

\subsection*{Problema 5: Tubo Cilíndrico}
\textbf{Enunciado.} Lo mismo para un tubo cilíndrico (de radios interno y externo $R_1$ y $R_2$).

\textit{Nota:} Equación 16.13: $$ \oint \frac{\partial \chi}{\partial n}dl=-S $$
Equación 16.17: $$ C=4\mu \sum_{k}\chi_k S_k +4\mu \int \chi dx dy $$

\textit{Solución.} La función (en coordenadas polares):

$$
\chi = \frac{1}{4}(R^2 - r^2)
$$

satisface la condición (16.13) en ambos límites de la sección anular del tubo. De la fórmula (16.17) se sigue:

$$
C = \mu \pi \frac{R_2^4 - R_1^4}{4}.
$$

\subsection*{Problema 6: Tubo de Paredes Delgadas}
\textbf{Enunciado.} Ídem para un tubo de paredes delgada y sección arbitraria.

\textit{Solución.} Dado que la pared del tubo es delgada, se puede considerar que en ella la función $\chi$ varía linealmente desde cero, sobre una cara, hasta $\chi_1$, sobre la otra ($y$ es la coordenada a lo largo del espesor del la pared):

$$
\chi = \chi_1 \frac{y}{h}.
$$

La condición (16.13) da $\frac{\chi_1 L}{h} = S$, donde $L$ es la longitud del perímetro de la sección del tubo, y $S$ el área que encierra. En la expresion (16.17) el segundo término es pequeño comparado con el primero, y obtenemos:

$$
C = \frac{4hS^2 \mu}{L}.
$$

Si se corta el tubo a lo largo de una de sus generatrices, la rigidez a la torsión disminuye a (de acuerdo con el resultado del problema 4):

$$
C = \frac{\mu Lh^3}{3}.
$$

\section*{Sección 19: Problemas de Flexión}

\subsection*{Problema 1: Reducción a Cuadraturas}
\textbf{Enunciado.} Reducir a cuadraturas el problema de la determinación de la forma de una barra de sección circular (varilla elástica), fuertemente combada en un plano por fuerzas concentradas.

\textit{Nota:} Ecuacion 19.10: $$ EI\frac{d\vec{r}}{dl}\times \frac{d^{3}\vec{r}}{dl^{3}} = \vec{F}\times\frac{d\vec{r}}{dl}$$

\textit{Solución.} Consideremos una porción de la barra comprendida entre puntos de aplicación de las fuerzas; en tal región es $F = \text{const}$. Elijamos el plano de la flexión como plano $x, y$, con el eje $y$ paralelo a la fuerza $F$ e introduzcamos el ángulo $\theta$ entre la tangente a la línea de la barra y el eje $y$. Entonces:

$$
\frac{dx}{dl} = \sin \theta, \quad \frac{dy}{dl} = \cos \theta,
$$

donde $x, y$ son las coordenadas de los puntos de la barra. Desarrollando los productos vectoriales en (19.10), obtenemos una ecuación para $\theta$, considerando como función de la longitud de arco $l$:

$$
IE \frac{d^2 \theta}{dl^2} - F \sin \theta = 0.
$$

La primera integración da:

$$
\frac{IE}{2} \left( \frac{d\theta}{dl} \right)^2 + F \cos \theta = c_1,
$$

y de aquí:

$$
l = \pm \sqrt{\frac{IE}{2}} \int \frac{d\theta}{\sqrt{c_1 - F \cos \theta}} + c_2 \rightarrow (1)
$$

La función $\theta(l)$ puede expresarse mediante funciones elípticas. Para las coordenadas:

$$
x = \int \sin \theta \, dl, \quad y = \int \cos \theta \, dl,
$$

obtenemos:

$$
x = \pm \frac{1}{F} \sqrt{2IE} \sqrt{c_1 - F \cos \theta} + \text{const.}
$$

$$
y = \pm \sqrt{\frac{IE}{2}} \int \frac{\cos \theta \, d\theta}{\sqrt{c_1 - F \cos \theta}} + \text{const'.} \rightarrow (2)
$$

El momento $M$ está dirigido según el eje $z$ y su módulo vale $M = IE \frac{d\theta}{dl}$.

\subsection*{Problema 2: Barra Fuertemente Encorvada}
\textbf{Enunciado.} Hallar la forma de una barra fuertemente encorvada, uno de cuyos extremos está empotrado, y en el otro, que es libre, se aplica una fuerza $\vec{f}$: la dirección de $\vec{f}$ es perpendicular a la recta de la barra no deformada (fig. 15).
\begin{figure}[ht]
    \centering
    \includegraphics[width = 0.15\columnwidth]{Screenshot_20250227_180931.png}
    \caption{fig 15}
\end{figure}

\textit{Solución.} En toda longitud de la barra es $\vec{F} = \text{const} = \vec{f}$. En el extremo empotrado ($l = 0$) se tiene $\theta = \frac{\pi}{2}$, y en el libre ($l = L$, donde $L$ es la longitud de la barra), $M = 0$, esto es, $\theta' = 0$. Introduciendo la notación $\theta_0 = \theta(L)$, tenemos en (1) $c_1 =f\cos(\theta_0)$:

$$
l = \sqrt{\frac{IE}{2f}} \int_{\theta_0}^{\frac{\pi}{2}} \frac{d\theta}{\sqrt{\cos \theta - \cos \theta}}.
$$

De aquí se deduce la ecuación que determina $\theta_0$:

$$
L = \sqrt{\frac{IE}{2f}} \int_{\theta_0}^{\frac{\pi}{2}} \frac{d\theta}{\sqrt{\cos \theta_0 - \cos \theta}}.
$$

La forma de la barra se halla mediante las fórmulas:

$$
x = \sqrt{\frac{2IE}{f}} \left( \sqrt{\cos \theta_0} - \sqrt{\cos \theta_0 - \cos \theta} \right),
$$

$$
y = \sqrt{\frac{IE}{2f}} \int_{\theta}^{\frac{\pi}{2}} \frac{\cos \theta \, d\theta}{\sqrt{\cos \theta_0 - \cos \theta}}.
$$

\subsection*{Problema 3: Fuerza Aplicada en el Extremo Libre}
\textbf{Enunciado.} Ídem, si la fuerza $\vec{f}$, aplicada en el extremo libre, está dirigida paralelamente a la linea de la barra no deformada.
\begin{figure}[h]
    \centering
    \includegraphics[width=0.15\columnwidth]
    {Screenshot_20250227_182326.png}
    \caption{fig. 16}
\end{figure}

\textit{Solución.} Tenemos $\vec{F} = -\vec{f}$. Condiciones de contorno: $\theta = 0$ en $l = 0$, $\theta' = 0$ en $l = L$. Tenemos:

$$
l = \sqrt{\frac{IE}{2f}} \int_{0}^{\theta} \frac{d\theta}{\sqrt{\cos \theta - \cos \theta_0}},
$$

donde $\theta_0 = \theta(L)$ se determina por:

$$
L = \sqrt{\frac{IE}{2f}} \int_{0}^{\theta_0} \frac{d\theta}{\sqrt{\cos \theta - \cos \theta_0}}.
$$

Para $x$ e $y$ obtenemos:

$$
x = \sqrt{\frac{2IE}{f}} \left( \sqrt{1 - \cos \theta_0} - \sqrt{\cos \theta - \cos \theta_0} \right),
$$

$$
y = \sqrt{\frac{IE}{2f}} \int_{0}^{\theta} \frac{\cos \theta \, d\theta}{\sqrt{\cos \theta - \cos \theta_0}}.
$$

En una flexión pequeña, $\theta_0 \ll 1$ y se puede escribir:

$$
L \approx \sqrt{\frac{IE}{f}} \int_{0}^{\theta_0} \frac{d\theta}{\sqrt{\theta_0^2 - \theta^2}} = \frac{\pi}{2} \sqrt{\frac{IE}{f}}.
$$

Esto es, $\theta_0$ desaparece de esta relación. Indicando que la solución existe solo cuando $f \geq \frac{\pi^2 IE}{4L^2}$.

\subsection*{Problema 4: Barra con Ambos Extremos Apoyados}
\textbf{Enunciado.} Ídem, si ambos extremos de la barra están apoyados y en su punto medio se aplica una fuerza $\vec{f}$: la distancia entre los puntos de apoyo es $L_0$.

\begin{figure}[h]
    \centering
    \includegraphics[width=0.2\columnwidth]
    {Screenshot_20250227_183218.png}
    \caption{fig. 17}
\end{figure}

\textit{Solución.} La fuerza $\vec{F}$ es constante en cada una de las porciones $AB$ y $BC$, y en cada una de ellas es perpendicular a la linea de la barra en los puntos de apoyo $A$ y $C$, respectivamente. La diferencia entre los valores de $\vec{F}$ en $AB$ y $BC$ es igual a $\vec{f}$, de donde se deduce que en $AB$ es $F \sin \theta_0 = -\frac{f}{2}$, donde $\theta_0$ es el ángulo entre el eje $y$ y la línea $AC$. En el punto $A$ ($l = 0$) tenemos $\theta = \frac{\pi}{2}$ y $M = 0$, es decir, $\theta' = 0$, de modo que en $AB$:

$$
l = \sqrt{\frac{IE}{f}} \sqrt{\sin \theta_0} \int_{\theta}^{\frac{\pi}{2}} \frac{d\theta}{\sqrt{\cos \theta}} \quad
x = 2 \sqrt{\frac{IE \sin \theta_0}{f}} \sqrt{\cos \theta},
$$

$$
y = \sqrt{\frac{IE \sin \theta_0}{f}} \int_{\theta}^{\frac{\pi}{2}} \sqrt{\cos \theta}  d\theta.
$$

El ángulo $\theta_0$ se determina mediante la condición que expresa que la proyección del arco $AB$ sobre la recta $AC$ debe ser igual a $\frac{L_0}{2}$, de donde:

$$
\frac{L_0}{2} = \sqrt{\frac{IE \sin \theta_0}{f}} \int_{\theta_0}^{\frac{\pi}{2}} \frac{\cos (\theta - \theta_0)}{\sqrt{\sin \theta}} \, d\theta.
$$

Para determinado valor de $\theta_0$, entre $0$ y $\frac{\pi}{2}$ la derivada $\frac{df}{d\theta_0}$ se anula y pasa a ser positiva, indicando que la solución se hace inestable.

\section*{Sección 20: Formas de Barras}

\subsection*{Problema 1: Barra Combadura por su Propio Peso}
\textbf{Enunciado.} Hallar la forma de una barra (de longuitud $l$), combada por la acciónde su propio peso, para diferentes manera de fijar sus extremos.

\textit{Solución.} La forma buscada se obtiene como solución de la ecuación $\zeta^{\prime\prime\prime\prime} = \frac{q}{EI}$ ($q$ es el peso de la barra por unidad de longitud) con condiciones de contorno en sus extremos. Para distintos apoyos de los extremos de la barra se obtienen las formas de flexión y los desplazamientos máximos (el origen de coordenadas coincide simepre con uno de los extremos de la barra):

\begin{enumerate}
    \item[a)] Ambos extremos empotrados:
    $$
    \zeta = \frac{q}{24EI} z^2(z-l)^2, \quad \zeta\left(\frac{l}{2}\right) = \frac{1}{384} \frac{ql^4}{EI}.
    $$

    \item[b)] Ambos extremos apoyados:
    $$
    \zeta = \frac{q}{24EI} z(z^3 - 2lz^2 + l^3), \quad \zeta\left(\frac{l}{2}\right) = \frac{5}{384} \frac{ql^4}{EI}.
    $$

    \item[c)] Un extremo ($z = l$) empotrado, y el otro ($z = 0$) apoyado:
    $$
    \zeta = \frac{q}{48EI} z(2z^3 - 3lz^2 + l^3), \quad \zeta(0.42l) = 0.0054 \frac{ql^4}{EI}.
    $$

    \item[d)] Un extremo ($z = 0$) empotrado, y el otro ($z = l$) libre:
    $$
    \zeta = \frac{q}{24EI} z^2(z^2 - 4lz + 6l^2), \quad \zeta(l) = \frac{1}{8} \frac{ql^4}{EI}.
    $$
\end{enumerate}

\subsection*{Problema 2: Barra Encorvada por una Fuerza Concentrada}
\textbf{Enunciado.} Determinar la forma de una barra encorvada por la acción de una fuerza concentrada $f$, aplicada en su punto medio.

\textit{Solución.} En todas partes, menos en el punto $z = l/2$, tenemos la ecuación $\zeta^{\prime\prime\prime\prime} = 0$. Las condiciones de contorno en los extremos de la barra determinan el modo de fijación. En el punto $z = l/2$ deben ser continuas $\zeta$, $\zeta'$, $\zeta''$, pero la diferencia de las fuerzas de corte $F=-EI\zeta^{\prime\prime\prime}$ a uno y otro lado debe ser igual a la fuerza $f$.
La forma de la barra (en la porción $0\ll z\ll \frac{l}{2}$) y la flecha de la linea elástica vienen dadas por las siguientes fórmulas:

\begin{enumerate}
    \item[a)] Ambos extremos empotrados:
    $$
    \zeta = \frac{f}{48EI} z^2(3l - 4z), \quad \zeta\left(\frac{l}{2}\right) = \frac{fl^3}{192EI}.
    $$

    \item[b)] Ambos extremos apoyados:
    $$
    \zeta = \frac{f}{48EI} z(3l^2 - 4z^2), \quad \zeta\left(\frac{l}{2}\right) = \frac{fl^3}{48EI}.
    $$
\end{enumerate}

La forma de la barra es simétrica respecto a su punto medio, de modo que la función $\zeta (z)$ en la porción $\frac{l}{2}\ll z\ll l$ se obtiene cambiando simplemente $z$ por $l-z$ en estas expresiones.

\subsection*{Problema 3: Barra con un Extremo Empotrado y el Otro Libre}
\textbf{Enunciado.} Ídem, para una barra con uno de sus extremos ($z=0$) empotrado y el otro ($z=l$) libre: en este último esta aplicada una fuerza concentrada $f$.

\textit{Solución.} A lo largo de toda la barra es $F = \text{const} = f$, de modo que $\zeta^{\prime\prime\prime} = -\frac{f}{EI}$. Con las condiciones $\zeta = 0$, $\zeta' = 0$ para $z = 0$ y $\zeta'' = 0$ para $z = l$ obtenemos:

$$
\zeta = \frac{f}{6EI} z^2(3l - z), \quad \zeta(l) = \frac{fl^3}{3EI}.
$$

\subsection*{Problema 4: Barra con Extremos Fijos y Par de Fuerzas}
\textbf{Enunciado.} Hallar la forma de una barra con extremos fijos, combada por la acción de un par de fuerzas concentrado aplicado en su punto medio.

\textit{Solución.} A lo largo de toda la barra es $\zeta^{\prime\prime\prime\prime} = 0$, pero en el punto $z = l/2$ el momento $M = EI\zeta''$ experimenta un salto igual al momento $m$ del par concentrado, aplicado en él. Con condiciones apropiadas en los extremos:

\begin{enumerate}
    \item[a)] Ambos extremos empotrados:
    $$
    \zeta = \frac{m}{24EIl} z^2(l + 2z) \quad \text{cuando } 0 \leq z \leq l/2,
    $$
    $$
    \zeta = -\frac{m}{24EIl} (l - z)^2[l + 2(l - z)] \quad \text{cuando } l/2 \leq z \leq l.
    $$

    \item[b)] Ambos extremos articulados:
    $$
    \zeta = \frac{m}{24EIl} z(l^2 - 4z^2) \quad \text{cuando } 0 \leq z \leq l/2,
    $$
    $$
    \zeta = -\frac{m}{24EIl} (l - z)[l^2 - 4(l - z)^2] \quad \text{cuando } l/2 \leq z \leq l.
    $$
\end{enumerate}
La barra se flexiona en sentidos opuestos a uno y a otro lado en el punto $x=l/2$.

\subsection*{Problema 5: Par Concentrado en el Extremo Libre}
\textbf{Enunciado.} Ídem, si el par concentrado está aplicado en el extremo libre de una barra, cuyo otro extremo se ecuentra empotrado.

\textit{Solución.} A lo largo de toda la barra tenemos $M = EI\zeta'' = m$, y en el punto $z = 0$ es $\zeta = 0$, $\zeta' = 0$. La forma de la flexión viene dada por la fórmula:

$$
\zeta = \frac{m}{2EI} z^2.
$$

\section*{Fluidos}
\subsection*{Notación}
\begin{itemize}
    \item $\rho $ densidad del fluido.
    \item $P$ presión.
    \item $T$ temperatura.
    \item $s$ entropía por unidad de masa.
    \item $\epsilon$ energía interna por unidad de masa.
    \item $w = \epsilon + P/\rho$ funcion de calor (entalpia).
    \item $\gamma = c_{P}/c_{v}$ ratio de calores específicos a una presión constante y a un volumen constante.
    \item $\eta$ viscosidad dinámica.
    \item $v= \eta / \rho$ viscosidad cinemática.
    \item $k$ conductividad térmica.
    \item $\chi =k/\rho c_{p}$ coeficiente de difusión térmica.
    \item $R$ Numero de Reynolds.
    \item $c$ velocidad del sonido.
    \item $M$ Ratio de la velocidad del fluido a la velocidad del sonido. (Mach number).
\end{itemize}
\section*{Sección 8: Conservación de la circulación}
\subsection*{Problema 1}

\textbf{Enunciado.} Muestra que, en un flujo que no es isoentrópico, cualquier partícula en movimiento lleva consigo un valor constante del producto $(1/\rho) , \nabla , s \cdot \nabla \times , \mathbf{v}$ (H. Ertel 1942).

\textit{Solución.}

Cuando el flujo no es isoentrópico, el lado derecho de la ecuación de Euler (2.3) no se puede reemplazar por $-\nabla , w$, y (2.11) se convierte en

$$
\frac{\partial \boldsymbol{\omega}}{\partial t} = \nabla \times , (\mathbf{v} \times \boldsymbol{\omega}) + \left( \frac{1}{\rho^2} \right) \nabla , \rho \times \nabla , p,
$$

donde, para abreviar, $\boldsymbol{\omega} = \nabla \times , \mathbf{v}$. Multiplicamos escalarmente por $\nabla , s$; dado que $s = s(p, \rho)$, $\nabla , s$ es una función lineal de $\nabla , p$ y $\nabla , \rho$, y $\nabla , s \cdot (\nabla , \rho \times \nabla , p) = 0$. La expresión en el lado derecho se puede transformar de la siguiente manera:

$$
\nabla , s \cdot \frac{\partial \boldsymbol{\omega}}{\partial t} = \nabla , s \cdot \nabla \times , (\mathbf{v} \times \boldsymbol{\omega})
$$

$$
= -\nabla \cdot , [\nabla , s \times (\mathbf{v} \times \boldsymbol{\omega})]
$$

$$
= -\nabla \cdot , [\mathbf{v} (\boldsymbol{\omega} \cdot \nabla , s)] + \nabla \cdot , [\boldsymbol{\omega} (\mathbf{v} \cdot \nabla , s)]
$$

$$
= -(\boldsymbol{\omega} \cdot \nabla , s) , \nabla \cdot , \mathbf{v} - \mathbf{v} \cdot \nabla , (\boldsymbol{\omega} \cdot \nabla , s) + \boldsymbol{\omega} \cdot \nabla , (\mathbf{v} \cdot \nabla , s).
$$

De (2.6), $\mathbf{v} \cdot \nabla , s = -\partial s/\partial t$, y por lo tanto

$$
\frac{\partial}{\partial t} (\boldsymbol{\omega} \cdot \nabla , s) + \mathbf{v} \cdot \nabla , (\boldsymbol{\omega} \cdot \nabla , s) + (\boldsymbol{\omega} \cdot \nabla , s) , \nabla \cdot , \mathbf{v} = 0.
$$

Los primeros dos términos se pueden combinar como $d(\boldsymbol{\omega} \cdot \nabla , s)/dt$, donde $d/dt = \partial/\partial t + \mathbf{v} \cdot \nabla$; en el último término, ponemos de (1.3) $\rho , \nabla \cdot , \mathbf{v} = -d\rho/dt$. El resultado es

$$
\frac{d}{dt} \left( \frac{\boldsymbol{\omega} \cdot \nabla , s}{\rho} \right) = 0,
$$

lo que da la ley de conservación requerida.

\section*{Sección 10: Fluidos incompresibles}
\subsection*{Problema 1}

\textbf{Enunciado.} Determina la forma de la superficie de un fluido incompresible sujeto a un campo gravitacional, contenido en un recipiente cilíndrico que gira alrededor de su eje (vertical) con una velocidad angular constante $\Omega$.

\textit{Solución.} Tomemos el eje del cilindro como el eje $z$. Entonces $v_x = -y\Omega$, $v_y = x\Omega$, $v_z = 0$. La ecuación de continuidad se satisface idénticamente, y la ecuación de Euler (10.1) da

$$
x\Omega^2 = \frac{1}{\rho} \frac{\partial p}{\partial x}, \quad y\Omega^2 = \frac{1}{\rho} \frac{\partial p}{\partial y}, \quad \frac{1}{\rho} \frac{\partial p}{\partial z} + g = 0.
$$

La integral general de estas ecuaciones es

$$
p/\rho = \frac{1}{2} \Omega^2 (x^2 + y^2) - gz + \text{constante}.
$$

En la superficie libre $p = \text{constante}$, de modo que la superficie es un paraboloide:

$$
z = \frac{1}{2} \Omega^2 (x^2 + y^2)/g,
$$

tomando el origen en el punto más bajo de la superficie.

\subsection*{Problema 2}

\textbf{Enunciado.} Una esfera, con radio $R$, se mueve con velocidad $\mathbf{u}$ en un fluido incompresible ideal. Determina el flujo potencial del fluido alrededor de la esfera.

\textit{Solución.} La velocidad del fluido debe anularse en el infinito. Las soluciones de la ecuación de Laplace $\Delta \phi = 0$ que se anulan en el infinito son bien conocidas: $1/r$ y las derivadas, de varios órdenes, de $1/r$ con respecto a las coordenadas (el origen se toma en el centro de la esfera). Debido a la completa simetría de la esfera, solo un vector constante, la velocidad $\mathbf{u}$, puede aparecer en la solución, y, debido a la linealidad tanto de la ecuación de Laplace como de la condición de contorno, $\phi$ debe involucrar a $\mathbf{u}$ linealmente. El único escalar que se puede formar a partir de $\mathbf{u}$ y las derivadas de $1/r$ es el producto escalar $\mathbf{u} \cdot \nabla (1/r)$. Por lo tanto, buscamos $\phi$ en la forma

$$
\phi = \mathbf{A} \cdot \nabla (1/r) = -(\mathbf{A} \cdot \mathbf{n})/r^2,
$$

donde $\mathbf{n}$ es un vector unitario en la dirección de $\mathbf{r}$. La constante $\mathbf{A}$ se determina a partir de la condición de que los componentes normales de las velocidades $\mathbf{v}$ y $\mathbf{u}$ deben ser iguales en la superficie de la esfera, es decir, $\mathbf{v} \cdot \mathbf{n} = \mathbf{u} \cdot \mathbf{n}$ para $r = R$. Esta condición da $\mathbf{A} = \frac{1}{2} \mathbf{u} R^3$, de modo que

$$
\phi = -\frac{R^3}{2r^2} \mathbf{u} \cdot \mathbf{n}, \quad \mathbf{v} = \frac{R^3}{2r^3} [3\mathbf{n}(\mathbf{u} \cdot \mathbf{n}) - \mathbf{u}].
$$

La distribución de presión está dada por la ecuación (10.7):

$$
p = p_0 - \frac{1}{2} \rho v^2 - \rho \frac{\partial \phi}{\partial t},
$$

donde $p_0$ es la presión en el infinito. Para calcular la derivada $\partial \phi/\partial t$, debemos tener en cuenta que el origen (que hemos tomado en el centro de la esfera) se mueve con velocidad $\mathbf{u}$. Por lo tanto,

$$
\frac{\partial \phi}{\partial t} = (\partial \phi/\partial \mathbf{u}) \cdot \dot{\mathbf{u}} - \mathbf{u} \cdot \nabla \phi.
$$

La distribución de presión sobre la superficie de la esfera está dada por la fórmula

$$
p = p_0 + \frac{1}{8} \rho u^2 (9 \cos^2 \theta - 5) + \frac{1}{2} \rho R n \cdot \dot{\mathbf{u}},
$$

donde $\theta$ es el ángulo entre $\mathbf{n}$ y $\mathbf{u}$.

\subsection*{Problema 3}

\textbf{Enunciado.} Lo mismo que el Problema 2, pero para un cilindro infinito que se mueve perpendicularmente a su eje.

\textit{Solución.} El flujo es independiente de la coordenada axial, por lo que tenemos que resolver la ecuación de Laplace en dos dimensiones. Las soluciones que se anulan en el infinito son la primera y las derivadas superiores de $\log r$ con respecto a las coordenadas, donde $r$ es el vector radial perpendicular al eje del cilindro. Buscamos una solución en la forma

$$
\phi = \mathbf{A} \cdot \nabla \log r = \mathbf{A} \cdot \mathbf{n}/r,
$$

y de las condiciones de contorno obtenemos $\mathbf{A} = -R^2 \mathbf{u}$, de modo que

$$
\phi = -\frac{R^2}{r} \mathbf{u} \cdot \mathbf{n}, \quad \mathbf{v} = \frac{R^2}{r^2} [2\mathbf{n}(\mathbf{u} \cdot \mathbf{n}) - \mathbf{u}].
$$

La presión en la superficie del cilindro está dada por

$$
p = p_0 + \frac{1}{2} \rho u^2 (4 \cos^2 \theta - 3) + \rho R n \cdot \frac{d\mathbf{u}}{dt}.
$$

\subsection*{Problema 4}

\textbf{Enunciado.} Determina el flujo potencial de un fluido incompresible ideal en un recipiente elipsoidal que gira alrededor de un eje principal con velocidad angular $\Omega$, y determina el momento angular total del fluido.

\textit{Solución.} Tomamos coordenadas cartesianas $x, y, z$ a lo largo de los ejes del elipsoide en un instante dado, siendo el eje $z$ el eje de rotación. La velocidad de los puntos en la pared del recipiente es

$$
\mathbf{u} = \Omega \mathbf{x} \mathbf{r},
$$

por lo que la condición de contorno $v_n = \partial \phi/\partial n = u_n$ es

$$
\frac{\partial \phi}{\partial n} = \Omega (x_n, -y_n),
$$

o, usando la ecuación del elipsoide $x^2/a^2 + y^2/b^2 + z^2/c^2 = 1$,

$$
\frac{x}{a^2} \frac{\partial \phi}{\partial x} + \frac{y}{b^2} \frac{\partial \phi}{\partial y} + \frac{z}{c^2} \frac{\partial \phi}{\partial z} = xy \Omega \left( \frac{1}{b^2} - \frac{1}{a^2} \right).
$$

La solución de la ecuación de Laplace que satisface esta condición de contorno es

$$
\phi = \Omega \frac{a^2 - b^2}{a^2 + b^2} xy. {1}
$$

El momento angular del fluido en el recipiente es

$$
M = \rho \int (xy - yx) , dV.
$$

Integrando sobre el volumen $V$ del elipsoide, tenemos

$$
M = \frac{\Omega \rho V (a^2 - b^2)^2}{5(a^2 + b^2)}.
$$

La fórmula (1) da el movimiento absoluto del fluido relativo a la posición instantánea de los ejes $x, y, z$ que están fijos al recipiente en rotación. El movimiento relativo al recipiente se encuentra restando la velocidad $\Omega \mathbf{x} \mathbf{r}$ de la velocidad absoluta; denotando la velocidad relativa del fluido por $\mathbf{v}'$, tenemos

$$
v'_x = \frac{\partial \phi}{\partial x} + y \Omega = \frac{2 \Omega a^2}{a^2 + b^2} y, \quad v'_y = -\frac{2 \Omega b^2}{a^2 + b^2} x, \quad v'_z = 0.
$$

Las trayectorias del movimiento relativo se encuentran integrando las ecuaciones $\dot{x} = v'_x$, $\dot{y} = v'_y$, y son las elipses $x^2/a^2 + y^2/b^2 = \text{constante}$, que son similares a la elipse límite.

\subsection*{Problema 5}

\textbf{Enunciado.} Determina el flujo cerca de un punto de estancamiento (Fig. 2).

\textit{Solución.} Una pequeña parte de la superficie del cuerpo cerca del punto de estancamiento puede considerarse plana. Tomémosla como el plano $xy$. Expandiendo $\phi$ para $x, y, z$ pequeños, tenemos hasta los términos de segundo orden

$$
\phi = ax + by + cz + Ax^2 + By^2 + Cz^2 + Dxy + Eyz + Fzx;
$$

un término constante en $\phi$ es irrelevante. Los coeficientes constantes se determinan de modo que $\phi$ satisfaga la ecuación $\Delta \phi = 0$ y las condiciones de contorno $v_z = \partial \phi/\partial z = 0$ para $z = 0$ y todos $x, y$, $\partial \phi/\partial x = \partial \phi/\partial y = 0$ para $x = y = z = 0$ (el punto de estancamiento). Esto da $a = b = c = 0; C = -A - B, E = F = 0$. El término $Dxy$ siempre se puede eliminar mediante una rotación adecuada de los ejes $x$ e $y$. Entonces tenemos

$$
\phi = Ax^2 + By^2 - (A + B)z^2. {1}
$$

Si el flujo es axialmente simétrico alrededor del eje $z$ (flujo simétrico alrededor de un sólido de revolución), debemos tener $A = B$, de modo que

$$
\phi = A(x^2 + y^2 - 2z^2).
$$

Los componentes de la velocidad son $v_x = 2Ax, v_y = 2Ay, v_z = -4Az$. Las líneas de corriente están dadas por las ecuaciones (5.2), de las cuales encontramos $x^2z = c_1, y^2z = c_2$, es decir, las líneas de corriente son hipérbolas cúbicas.

Si el flujo es uniforme en la dirección $y$ (por ejemplo, flujo en la dirección $z$ alrededor de un cilindro con su eje en la dirección $y$), debemos tener $B = 0$ en (1), de modo que

$$
\phi = A(x^2 - z^2).
$$

Las líneas de corriente son las hipérbolas $xz = \text{constante}$.

\subsection*{Problema 6}

\textbf{Enunciado.} Determina el flujo potencial cerca de un ángulo formado por dos planos que se intersectan.

\textit{Solución.} Tomemos coordenadas polares $r, \theta$ en el plano transversal (perpendicular a la línea de intersección), con el origen en el vértice del ángulo; $\theta$ se mide desde uno de los brazos del ángulo. Sea el ángulo $\alpha$ radianes; para $\alpha < \pi$ el flujo tiene lugar dentro del ángulo, para $\alpha > \pi$ fuera de él. La condición de contorno de que el componente normal de la velocidad se anule significa que $\partial \phi/\partial \theta = 0$ para $\theta = 0$ y $\theta = \alpha$. La solución de la ecuación de Laplace que satisface estas condiciones se puede escribir como

$$
\phi = Ar^n \cos n\theta, \quad n = \pi/\alpha,
$$

de modo que

$$
v_r = nAr^{n-1} \cos n\theta, \quad v_\theta = -nAr^{n-1} \sin n\theta.
$$

Para $n < 1$ (flujo fuera de un ángulo; Fig. 3), $v_r$ se hace infinita como $1/r^{1-n}$ en el origen. Para $n > 1$ (flujo dentro de un ángulo; Fig. 4), $v_r$ se hace cero para $r = 0$.

La función de corriente, que da la forma de las líneas de corriente, es $\psi = Ar^n \sin n\theta$. Las expresiones obtenidas para $\phi$ y $\psi$ son las partes real e imaginaria del potencial complejo $w = Az^n$.

\subsection*{Problema 7}

\textbf{Enunciado.} Un agujero esférico con radio $a$ se forma repentinamente en un fluido incompresible que llena todo el espacio. Determina el tiempo que tarda el agujero en llenarse de fluido (Besant 1859; Rayleigh 1917).

\subsection*{Solución}

El flujo después de la formación del agujero será esféricamente simétrico, con la velocidad en cada punto dirigida hacia el centro del agujero. Para la velocidad radial $v_r \equiv v < 0$ tenemos la ecuación de Euler en coordenadas polares esféricas:

$$
\frac{\partial v}{\partial t} + v \frac{\partial v}{\partial r} = -\frac{1}{\rho} \frac{\partial p}{\partial r}. {1}
$$

La ecuación de continuidad da

$$
r^2 v = F(t), {2}
$$

donde $F(t)$ es una función arbitraria del tiempo; esta ecuación expresa el hecho de que, dado que el fluido es incompresible, el volumen que fluye a través de cualquier superficie esférica es independiente del radio de esa superficie.

Sustituyendo $v$ de (2) en (1), tenemos

$$
\frac{F'(t)}{r^2} + v \frac{\partial v}{\partial r} = -\frac{1}{\rho} \frac{\partial p}{\partial r}.
$$

Integrando esta ecuación sobre $r$ desde el radio instantáneo $R = R(t) \leq a$ del agujero hasta el infinito, obtenemos

$$
-\frac{F'(t)}{R} + \frac{1}{2} v^2 = \frac{p_0}{\rho}, {3}
$$

donde $\dot{R} = dR(t)/dt$ es la tasa de cambio del radio del agujero, y $p_0$ es la presión en el infinito; la velocidad del fluido en el infinito es cero, y también lo es la presión en la superficie del agujero. De la ecuación (2) para puntos en la superficie del agujero encontramos

$$
F(t) = R^2(t) \dot{R}(t),
$$

y, sustituyendo esta expresión para $F(t)$ en (3), obtenemos la ecuación

$$
\frac{3\dot{R}^2}{2} - \frac{1}{2} R \frac{d\dot{R}^2}{dR} = \frac{p_0}{\rho}. {4}
$$

Las variables son separables; integrando con la condición de contorno $\dot{R} = 0$ para $R = a$ (el fluido está inicialmente en reposo), tenemos

$$
\dot{R} \equiv \frac{dR}{dt} = -\sqrt{\frac{2p_0}{3\rho} \left( a^3/R^3 - 1 \right)}.
$$

Por lo tanto, tenemos para el tiempo total requerido para que el agujero se llene

$$
\tau = \sqrt{\frac{3\rho}{2p_0}} \int_0^a \frac{dR}{\sqrt{[(a/R)^3 - 1]}}.
$$

Esta integral se reduce a una función beta, y finalmente tenemos

$$
\tau = \sqrt{\frac{3a^2 \rho \pi \Gamma(5/6)}{2p_0 \Gamma(1/3)}} = 0.915a \sqrt{\frac{\rho}{p_0}}.
$$

\subsection*{Problema 8}

\textbf{Enunciado.} Una esfera sumergida en un fluido incompresible se expande según una ley dada $R = R(t)$. Determina la presión del fluido en la superficie de la esfera.

\subsection*{Solución}

Sea la presión requerida $P(t)$. Cálculos exactamente similares a los del Problema 7, excepto que la presión en $r = R$ es $P(t)$ y no cero, dan en lugar de (3) la ecuación

$$
-\frac{F'(t)}{R(t)} + \frac{1}{2} V^2 = \frac{p_0}{\rho} - \frac{P(t)}{\rho}
$$

y, en consecuencia, en lugar de (4), la ecuación

$$
\frac{p_0 - P(t)}{\rho} = -\frac{3V^2}{2} - R \frac{V' dV}{dR}.
$$

Teniendo en cuenta que $V = dR/dt$, podemos escribir la expresión para $P(t)$ en la forma

$$
P(t) = p_0 + \frac{1}{2} \left[ \frac{d^2(R^2)}{dt^2} + \left( \frac{dR}{dt} \right)^2 \right].
$$

\section*{Sección 17: Flujo en una tubería}

\subsection*{Problema 1}

Determina el flujo en una tubería de sección transversal anular, siendo los radios interno y externo $R_1, R_2$.

\textit{Solución.} Determinando las constantes $a$ y $b$ en la solución general (17.8) a partir de las condiciones de que $v = 0$ para $r = R_1$ y $r = R_2$, encontramos

$$
v = \frac{\Delta p}{4\eta l} \left[ R_2^2 - r^2 + \frac{R_2^2 - R_1^2}{\log(R_2/R_1)} \log \frac{r}{R_2} \right]
$$

El caudal es

$$
Q = \frac{\pi \Delta p}{8\eta l} \left[ R_2^4 - R_1^4 - \frac{(R_2^2 - R_1^2)^2}{\log(R_2/R_1)} \right].
$$

\subsection*{Problema 2}

Lo mismo que el Problema 1, pero para una tubería de sección transversal elíptica.

\textit{Solución.} Buscamos una solución de la ecuación (17.7) en la forma $v = Ay^2 + Bz^2 + C$. Las constantes $A, B, C$ se determinan a partir del requisito de que esta expresión debe satisfacer la condición de contorno $v = 0$ en la circunferencia de la elipse (es decir, $Ay^2 + Bz^2 + C = 0$ debe ser lo mismo que la ecuación $y^2/a^2 + z^2/b^2 = 1$, donde $a$ y $b$ son los semiejes de la elipse). El resultado es

$$
v = \frac{\Delta p}{2\eta l} \frac{a^2 b^2}{a^2 + b^2} \left( 1 - \frac{y^2}{a^2} - \frac{z^2}{b^2} \right).
$$

El caudal es

$$
Q = \frac{\pi \Delta p}{4\eta l} \frac{a^3 b^3}{a^2 + b^2}.
$$

\subsection*{Problema 3}

Lo mismo que el Problema 1, pero para una tubería cuya sección transversal es un triángulo equilátero con lado $a$.

\textit{Solución.} La solución de la ecuación (17.7) que se anula en el triángulo limítrofe es

$$
v = \frac{\Delta p}{l} \frac{2}{3a\eta} h_1 h_2 h_3,
$$

donde $h_1, h_2, h_3$ son las longitudes de las perpendiculares desde un punto dado en el triángulo a sus tres lados. Para cada una de las expresiones $\Delta h_1, \Delta h_2, \Delta h_3$ (donde $\Delta = \partial^2/\partial y^2 + \partial^2/\partial z^2$) es cero; esto se ve inmediatamente del hecho de que cada una de las perpendiculares $h_1, h_2, h_3$ puede tomarse como el eje de $y$ o $z$, y el resultado de aplicar el Laplaciano a una coordenada es cero. Por lo tanto, tenemos

$$
\Delta (h_1 h_2 h_3) = 2(h_1 , \nabla , h_2 \cdot \nabla , h_3 + h_2 , \nabla , h_3 \cdot \nabla , h_1 + h_3 , \nabla , h_1 \cdot \nabla , h_2)
$$

Pero $\nabla , h_1 = \mathbf{n}_1, \nabla , h_2 = \mathbf{n}_2, \nabla , h_3 = \mathbf{n}_3$, donde $\mathbf{n}_1, \mathbf{n}_2, \mathbf{n}_3$ son vectores unitarios a lo largo de las perpendiculares $h_1, h_2, h_3$. Cualesquiera dos de $\mathbf{n}_1, \mathbf{n}_2, \mathbf{n}_3$ están en un ángulo $2\pi/3$, de modo que $\nabla , h_1 \cdot \nabla , h_2 = \mathbf{n}_1 \cdot \mathbf{n}_2 = \cos(2\pi/3) = -\frac{1}{2}$, y así sucesivamente. Así, obtenemos la relación

$$
\Delta (h_1 h_2 h_3) = -(h_1 + h_2 + h_3) = -\frac{1}{2} \sqrt{3} a,
$$

y vemos que la ecuación (17.7) se satisface. El caudal es

$$
Q = \frac{\sqrt{3} a^4 \Delta p}{320 \eta l}.
$$

\subsection*{Problema 4}

Un cilindro con radio $R_1$ se mueve paralelo a su eje con velocidad $u$ dentro de un cilindro coaxial con radio $R_2$. Determina el movimiento de un fluido que ocupa el espacio entre los cilindros.

\textit{Solución.} Tomamos coordenadas polares cilíndricas, con el eje $z$ a lo largo del eje de los cilindros. La velocidad está en todas partes a lo largo del eje $z$ y depende solo de $r$ (al igual que la presión); $v_z = v(r)$. Obtenemos para $v$ la ecuación

$$
\Delta v = -\frac{1}{r} \frac{d}{dr} \left( r \frac{dv}{dr} \right) = 0;
$$

el término $(\mathbf{v} \cdot \nabla)w = v , \partial v/\partial z$ se anula idénticamente. Usando las condiciones de contorno $v = u$ para $r = R_1$ y $v = 0$ para $r = R_2$, encontramos

$$
v = \frac{u \log(r/R_2)}{\log(R_1/R_2)}.
$$

La fuerza de fricción por unidad de longitud de cualquiera de los cilindros es $2\pi \eta u/\log(R_2/R_1)$.

\subsection*{Problema 5}

Una capa de fluido con espesor $h$ está limitada arriba por una superficie libre y abajo por un plano fijo inclinado en un ángulo $\alpha$ con respecto a la horizontal. Determina el flujo debido a la gravedad.

\textit{Solución.} Tomamos el plano fijo como el plano $xy$, con el eje $x$ en la dirección del flujo (Fig. 6). Buscamos una solución que dependa solo de $z$. Las ecuaciones de Navier-Stokes con $v_x = v(z)$ en un campo gravitacional son

$$
\eta \frac{d^2 v}{dz^2} + \rho g \sin \alpha = 0, \quad \frac{dp}{dz} + \rho g \cos \alpha = 0.
$$

En la superficie libre ($z = h$) debemos tener $\sigma_{xz} = \eta dv/dz = 0$, $\sigma_{zz} = -p = -p_0$ ($p_0$ siendo la presión atmosférica). Para $z = 0$ debemos tener $v = 0$. La solución que satisface estas condiciones es

$$
p = p_0 + \rho g (h - z) \cos \alpha, \quad v = \frac{\rho g \sin \alpha}{2\eta} z(2h - z).
$$

El caudal, por unidad de longitud en la dirección $y$, es

$$
Q = \rho \int_0^h v , dz = \frac{\rho g h^3 \sin \alpha}{3\eta}.
$$

\section*{Sección 20: Flujo con pequeños números de Reynolds}

\subsection*{Problema 1}

Determina el movimiento de un fluido que ocupa el espacio entre dos esferas concéntricas con radios $R_1, R_2$ ($R_2 > R_1$), que giran uniformemente alrededor de diferentes diámetros con velocidades angulares $\Omega_1, \Omega_2$; los números de Reynolds $\Omega_1 R_1^2/\nu, \Omega_2 R_2^2/\nu$ son pequeños en comparación con la unidad.

\textit{Solución.} Debido a la linealidad de las ecuaciones, el movimiento entre dos esferas giratorias puede considerarse como una superposición de los dos movimientos obtenidos cuando una esfera está en reposo y la otra gira. Primero ponemos $\Omega_2 = 0$, es decir, solo la esfera interior está girando. Es razonable suponer que la velocidad del fluido en cada punto está a lo largo de la tangente a un círculo en un plano perpendicular al eje de rotación con su centro en el eje. Debido a la simetría axial, el gradiente de presión en esta dirección es cero. Por lo tanto, la ecuación de movimiento (20.1) se convierte en $\Delta v = 0$. El vector de velocidad angular $\Omega_1$ es un vector axial. Argumentos similares a los dados anteriormente muestran que la velocidad se puede escribir como

$$
\mathbf{v} = \nabla \times [f(r) \Omega_1] = \nabla f \times \Omega_1.
$$

La ecuación de movimiento entonces da $\nabla \Delta f \times \Omega_1 = 0$. Dado que el vector $\nabla \Delta f$ es paralelo al vector de posición, y el producto vectorial $\mathbf{r} \times \Omega_1$ no puede ser cero para un $\Omega_1$ dado y un $\mathbf{r}$ arbitrario, debemos tener $\nabla \Delta f = 0$, de modo que $\Delta f = \text{constante}$.

Integrando, encontramos

$$
f = ar^2 + \frac{b}{r}, \quad \mathbf{v} = \left( \frac{b}{r^3} - 2a \right) \Omega_1 \times \mathbf{r}.
$$

Las constantes $a$ y $b$ se encuentran a partir de las condiciones de que $v = 0$ para $r = R_2$ y $v = u$ para $r = R_1$, donde $u = \Omega_1 \times \mathbf{r}$ es la velocidad de los puntos en la esfera giratoria. El resultado es

$$
\mathbf{v} = \frac{R_1^3 R_2^3}{R_2^3 - R_1^3} \left( \frac{1}{r^3} - \frac{1}{R_2^3} \right) \Omega_1 \times \mathbf{r}.
$$

La presión del fluido es constante ($p = p_0$). De manera similar, tenemos para el caso en que la esfera exterior gira y la interior está en reposo ($\Omega_1 = 0$)

$$
\mathbf{v} = \frac{R_1^3 R_2^3}{R_2^3 - R_1^3} \left( \frac{1}{R_1^3} - \frac{1}{r^3} \right) \Omega_2 \times \mathbf{r}.
$$

En el caso general en que ambas esferas giran, tenemos

$$
\mathbf{v} = \frac{R_1^3 R_2^3}{R_2^3 - R_1^3} \left( \left( \frac{1}{r^3} - \frac{1}{R_2^3} \right) \Omega_1 \times \mathbf{r} + \left( \frac{1}{R_1^3} - \frac{1}{r^3} \right) \Omega_2 \times \mathbf{r} \right).
$$

Si la esfera exterior está ausente ($R_2 = \infty, \Omega_2 = 0$), es decir, simplemente tenemos una esfera con radio $R$ girando en un fluido infinito, entonces

$$
\mathbf{v} = (R^3/r^3) \Omega \times \mathbf{r}.
$$

Calculemos el momento de las fuerzas de fricción que actúan sobre la esfera en este caso. Si tomamos coordenadas polares esféricas con el eje polar paralelo a $\Omega$, tenemos $v_r = v_\theta = 0, v_\phi = (R^2 \Omega/r^2) \sin \theta$. La fuerza de fricción por unidad de área de la esfera es

$$
\sigma_{r\phi} = \eta \left( \frac{\partial v_\phi}{\partial r} - \frac{v_\phi}{r} \right)_{r=R} = -3\eta \Omega \sin \theta.
$$

El momento total sobre la esfera es

$$
M = \int_0^\pi \sigma_{r\phi} R \sin \theta \cdot 2\pi R^2 \sin \theta , d\theta,
$$

de donde encontramos

$$
M = -8\pi \eta R^3 \Omega.
$$

Si la esfera interior está ausente, $\mathbf{v} = \Omega \times \mathbf{r}$, es decir, el fluido simplemente gira rígidamente con la esfera que lo rodea.

\subsection*{Problema 3}

Dos discos planos circulares (con radio $R$) se encuentran uno sobre el otro a una pequeña distancia; el espacio entre ellos está lleno de fluido. Los discos se acercan a una velocidad constante $u$, desplazando el fluido. Determina la resistencia a su movimiento (O. Reynolds).

\textit{Solución.} Tomamos coordenadas polares cilíndricas, con el origen en el centro del disco inferior, que suponemos fijo. El flujo es axialmente simétrico y, dado que la capa de fluido es delgada, predominantemente radial: $v_z \ll v_r$, y también $\partial v_r/\partial r \ll \partial v_r/\partial z$. Por lo tanto, las ecuaciones de movimiento se convierten en

$$
\eta \frac{\partial^2 v_r}{\partial z^2} = \frac{\partial p}{\partial r}, \quad \frac{\partial p}{\partial z} = 0, {1}
$$

$$
\frac{1}{r} \frac{\partial (rv_r)}{\partial r} + \frac{\partial v_z}{\partial z} = 0, {2}
$$

con las condiciones de contorno

$$
\text{en } z = 0: \quad v_r = v_z = 0;
$$

$$
\text{en } z = h: \quad v_r = 0, \quad v_z = -u;
$$

$$
\text{en } r = R: \quad p = p_0,
$$

donde $h$ es la distancia entre los discos, y $p_0$ la presión externa. De las ecuaciones (1) encontramos

$$
v_r = \frac{1}{2\eta} \frac{dp}{dr} z(z - h).
$$

Integrando la ecuación (2) con respecto a $z$, obtenemos

$$
u = \frac{1}{r} \frac{d}{dr} \int_0^h rv_r , dz = -\frac{h^3}{12\eta r} \frac{d}{dr} \left( r \frac{dp}{dr} \right),
$$

de donde

$$
p = p_0 + \frac{3\eta u}{h^3} (R^2 - r^2).
$$

La resistencia total al disco en movimiento es

$$
F = \frac{3\pi \eta u R^4}{2h^3}.
$$
\end{document}