\documentclass[12pt]{article}
\usepackage[utf8]{inputenc}
\usepackage[T1]{fontenc}
\usepackage[spanish]{babel}
\usepackage{amsmath,amssymb,amsfonts}
\usepackage{lmodern}
\usepackage{geometry}
\geometry{a4paper, margin=1in}

\begin{document}
\textbf{Ejercicios de Mecánica y Teoría Clásica de Campos (Landau)}

\section{Introducción}
Este documento es una recopilación de ejercicios y ejemplos de mecánica clásica y teoría clásica de campos del Landau.

\section{Mecánica Clásica}
Encontrar la lagrangiana de los siguientes sistemas, colocados en un campo gravitatorio (aceleración de la gravedad: $g$).
\subsection{Ejercicio 1: Péndulo doble coplanario}
\textbf{Solución:} Tomemos como coordenadas los ángulos $\phi_1$ y $\phi_2$ que forman los hilos $l_1$ y $l_2$ con la vertical. Tenemos entonces para la partícula $m_1$:
$$
T_1 = \frac{1}{2} m_1 l_1^2 \dot{\phi}_1^2, \quad U = -m_1 g l_1 \cos \phi_1
$$
Para hallar la energía cinética de la segunda partícula, expresamos sus coordenadas cartesianas $x_2, y_2$ (origen de coordenadas en el punto de suspensión, eje $y$ dirigido verticalmente hacia abajo) en función de $\phi_1$ y $\phi_2$:

$$
x_2 = l_1 \sin \phi_1 + l_2 \sin \phi_2, \quad y_2 = l_1 \cos \phi_1 + l_2 \cos \phi_2
$$

Obtenemos entonces:

$$
T_2 = \frac{1}{2} m_2 (\dot{x}_2^2 + \dot{y}_2^2)
$$

$$
= \frac{1}{2} m_2 [l_1^2 \dot{\phi}_1^2 + l_2^2 \dot{\phi}_2^2 + 2 l_1 l_2 \cos(\phi_1 - \phi_2) \dot{\phi}_1 \dot{\phi}_2]
$$

Y finalmente,

$$
L = \frac{1}{2} (m_1 + m_2) l_1^2 \dot{\phi}_1^2 + \frac{1}{2} m_2 l_2^2 \dot{\phi}_2^2 + m_2 l_1 l_2 \dot{\phi}_1 \dot{\phi}_2 \cos(\phi_1 - \phi_2) + (m_1 + m_2) g l_1 \cos \phi_1 + m_2 g l_2 \cos \phi_2
$$

\subsection{Ejercicio 2: Pendulo plano}
\textbf{Enunciado.}  Péndulo plano de masa $m_2$, cuyo punto de suspensión (de masa $m_1$) puede desplazarse en el mismo plano sobre una recta horizontal (fig. 2).

\textbf{Solución:} Usando la coordenada $x$ del punto $m_1$ y el ángulo $\phi$ entre el hilo del péndulo y la vertical, tenemos:

$$
L = \frac{1}{2} (m_1 + m_2) \dot{x}^2 + \frac{1}{2} m_2 (l^2 \dot{\phi}^2 + 2 l \dot{x} \dot{\phi} \cos \phi) + m_2 g l \cos \phi
$$

\subsection{Ejercicio 3: Péndulo plano, cuyo punto de suspensión:}
\textbf{a)}  se desplaza uniformemente sobre una circunferencia vertical con una frecuencia constante $\gamma$ (fig. 3);  

\textbf{b)} oscila horizontalmente en el plano del péndulo según la ley $x = a \cos \gamma t$;

\textbf{c)} oscila verticalmente según la ley $y = a \cos \gamma t$.

\textbf{Solución:} a) coordenadas del punto $m$: 
$$ x = a \cos \gamma t + l \sin \phi \text{,       } \text{        } y = -a \sin \gamma t + l \cos \phi $$.
Lagrangiana:

$$
L = \frac{1}{2} m l^2 \dot{\phi}^2 + m l a \gamma \sin(\phi - \gamma t) + m g l \cos \phi;
$$

se han omitido los términos que sólo dependen del tiempo y eliminado la derivada total con respecto al tiempo de $ m l a \gamma \cos(\phi - \gamma t) $.

b) Coordenadas del punto $m$:

$$ x = a \cos \gamma t + l \sin \phi \text{,   }\text{} y = l \cos \phi $$.

Lagrangiana (omitiendo las derivadas totales con respecto al tiempo):

$$
L = \frac{1}{2} m l^2 \dot{\phi}^2 + m l a \gamma^2 \cos \gamma t \sin \phi + m g l \cos \phi.
$$

c) De la misma manera:

$$
L = \frac{1}{2} m l^2 \dot{\phi}^2 + m l a \gamma^2 \cos \gamma t \cos \phi + m g l \cos \phi.
$$

\subsection{Ejercicio 4:}
\textbf{Enunciado.} En el sistema representado en la figura 4, el punto $m_2$ se mueve sobre el eje vertical, y todo el sistema gira con velocidad angular constante $\Omega$ alrededor de este eje.

\textbf{Solución:} Sea $\theta$ el ángulo formado por el segmento $a$ y la vertical, y $\phi$ el ángulo de rotación del sistema alrededor del eje; $\dot{\phi} = \Omega$. Para cada partícula $m_1$, un desplazamiento elemental $dl_1^2 = a^2 d\theta^2 + a^2 \sin^2 \theta d\phi^2$. La distancia de $m_2$ al punto de suspensión $A$ es $2a \cos \theta$ y así, $dl_2 = -2a \sin \theta d\theta$. La lagrangiana:

$$
L = m_1 a^2 (\dot{\theta}^2 + \Omega^2 \sin^2 \theta) + 2 m_2 a^2 \dot{\theta}^2 \sin^2 \theta + 2 (m_1 + m_2) g a \cos \theta.
$$

\subsection{Ejercicio 5: Anillo oscilante con masa puntual}
\textbf{Enunciado.} Considere un anillo de radio $R$ y masa $M$ que se cuelga de uno de sus puntos y oscila en su propio plano. Se comporta como un péndulo físico cuyo centro de gravedad está ubicado a una distancia $R$ del punto de suspensión. En el anillo se coloca una masa puntual $M$ que se puede deslizar sin fricción. Halle el lagrangiano del sistema. Realice la aproximación de pequeñas oscilaciones. Resuelva el problema de vectores propios y valores propios implicado. Analice los resultados.

\subsection{Ejercicio 6: Oscilaciones de molécula triatómica lineal}
\textit{Ejercicio 1 secc 24: Frecuencia de vibraciones de molécula triatómica lineal}
1. Determinar la frecuencia de las vibraciones de una molécula simétrica lineal triatómica $ABA$ (fig. 28). Se supone que la energía potencial de la molécula depende solamente de las distancias $AB$ y $BA$ y del ángulo $ABA$.

\textbf{Solución:} Los desplazamientos longitudinales $x_1, x_2, x_3$ de los átomos están relacionados, según (24.1), por:
\[
m_A(x_1 + x_3) + m_B x_2 = 0.
\]
Con la ayuda de esta ecuación, eliminamos $x_2$ de la lagrangiana del movimiento longitudinal de la molécula:
\[
L = \frac{1}{2}m_A(\dot{x}_1^2 + \dot{x}_3^2) + \frac{1}{2}m_B \dot{x}_2^2 - \frac{1}{2}k_1[(x_1 - x_2)^2 + (x_3 - x_2)^2],
\]
y utilizando las nuevas coordenadas:
\[
Q_a = x_1 + x_3, \quad Q_s = x_1 - x_3,
\]
obtenemos:
\[
L = \frac{\mu}{4m_B} \dot{Q}_a^2 + \frac{m_A}{4} \dot{Q}_s^2 - \frac{k_1 l^2}{4m_B^2} Q_a^2 - \frac{k_1}{4} Q_s^2.
\]

Es evidente que $Q_a$ y $Q_s$ son coordenadas normales (todavía no normalizadas). La coordenada $Q_a$ corresponde a una vibración antisimétrica alrededor del centro de la molécula ($x_1 = x_3$; fig. 28, a) y de frecuencia:
\[
\omega_a = \sqrt{\frac{k_1 \mu}{m_A m_B}}.
\]
La coordenada $Q_s$ corresponde a una vibración simétrica ($x_1 = x_3$; fig. 28, b) de frecuencia:
\[
\omega_{s1} = \sqrt{\frac{k_1}{m_A}}.
\]
Los desplazamientos transversales de los átomos $y_1, y_2, y_3$ están relacionados, según (24.1) y (24.2), por:
\[
m_A(y_1 + y_2) + m_B y_2 = 0, \quad y_1 = y_3.
\]

(vibraciones simétricas de curvatura de la molécula; fig. 28, c). Sea $\frac{1}{2}k_2 l^2 \delta^2$ la energía potencial de curvatura de la molécula, donde $\delta$ es la desviación del ángulo $ABA$ con respecto a $\pi$; su expresión en función de los desplazamientos es:
\[
\delta = \frac{[(y_1 - y_2) + (y_3 - y_2)]}{l}.
\]
Expresando $y_1, y_2, y_3$ en función de $\delta$, se obtiene la lagrangiana del movimiento transversal de la molécula:
\[
L = \frac{1}{2}m_A(\dot{y}_1^2 + \dot{y}_3^2) + \frac{1}{2}m_B\dot{y}_2^2 - \frac{1}{2}k_2 l^2 \delta^2
\]
\[
= \frac{m_A m_B l^2 \dot{\delta}^2}{4\mu} - \frac{1}{2}k_2 l^2 \delta^2,
\]
de donde la frecuencia:
\[
\omega_{s2} = \sqrt{\frac{2k_2 \mu}{m_A m_B}}.
\]

\textbf{Enunciado.} Plantee y resuelva el problema de las oscilaciones de la molécula triatómica lineal, descrito en la Fig.28 (que corresponde al Pro.1 de la sección §24), cuando se considera que cada átomo está sometido a fricción y a una forzante periódica.

\subsection{Ejercicio 7: Oscilador forzado y amortiguado}
\textbf{Enunciado.} Considere el oscilador forzado y amortiguado descrito por la ecuación (26.1). Halle la amplitud de la oscilación de estado estacionario de la velocidad correspondiente a la $B$ de (26.2). Halle el valor $\gamma_R$ de $\gamma$ para el cual dicha amplitud es máxima y compárelo con la frecuencia de las oscilaciones amortiguadas libres. Se define el factor de calidad del oscilador:
\[
Q = \frac{\gamma_R}{2\lambda}.
\]
Demuestre que si el amortiguamiento es pequeño y la frecuencia de la forzante es cercana a la resonancia:
\[
Q \approx 2\pi \frac{\text{Energía total}}{\text{Energía perdida en un período}} \approx \frac{\omega_0}{\Delta \omega},
\]
donde $\Delta \omega$ representa el intervalo de frecuencia entre los puntos en los cuales la amplitud alcanza $1/\sqrt{2}$ de su valor máximo. Ilustre con gráficas cuando $Q = 5$ y cuando $Q = 100$.

\subsection{Ejercicio 8: Sistema con fuerzas periódicas}
\textbf{Enunciado.} Considere el sistema mostrado en la Fig.2, que corresponde al Pro.2 de la sección §5. Suponga que las masas $m_1$ y $m_2$ están sometidas a fuerzas externas periódicas horizontales de la forma $f_i \cos(\gamma t + \alpha_i)$ para $i = 1, 2$. Realice la aproximación de pequeñas oscilaciones. Lleve el problema a coordenadas normales. Analice los resultados.

\section{2da unidad teorica 2"}
\subsection*{§ 38. Cuerpos rígidos en contacto}

Las ecuaciones del movimiento (34.1) y (34.3) muestran que las \textit{condiciones de equilibrio} de un cuerpo rígido se pueden formular anulando la resultante general y el momento resultante de las fuerzas que actúan sobre él.
\[
\mathbf{F} = \sum \mathbf{f} = 0, \qquad \mathbf{K} = \sum \mathbf{r} \times \mathbf{f} = 0. \tag{38.1}
\]
La suma se extiende a todas las fuerzas exteriores aplicadas al cuerpo, y $\mathbf{r}$ es el radio vector del ``punto de aplicación'' de estas fuerzas; el origen, con respecto al cual se definen los momentos, puede escogerse arbitrariamente, ya que si $\mathbf{F} = 0$ el valor de $\mathbf{K}$ no depende de esta elección [véase (34.5)].

La condición impuesta en el movimiento de rodadura es que las velocidades de los puntos de contacto deben ser iguales; por ejemplo, cuando un cuerpo rueda sobre una superficie fija, la velocidad del punto de contacto debe ser nula. En el caso general, esta condición está expresada por \textit{ecuaciones de ligadura} del tipo
\[
\sum_i c_{\alpha i} \dot{q}_i = 0, \tag{38.2}
\]
donde las $c_{\alpha i}$ son funciones de las coordenadas solamente (el índice $\alpha$ numera las ecuaciones de ligadura). Si los primeros miembros de estas ecuaciones no son derivadas totales con respecto al tiempo de funciones de las coordenadas, estas ecuaciones no pueden ser integradas. En otras palabras, no pueden reducirse a la forma $f(q_1, \ldots, q_n, t) = \text{const}$, y se llaman \textit{ligaduras no holónomas}.

Como de costumbre, llamaremos $\mathbf{V}$ a la velocidad de traslación (velocidad del centro de la esfera), y $\mathbf{\Omega}$ a la velocidad angular de rotación. La velocidad del punto de contacto con el plano se obtiene poniendo $\mathbf{r} = -a\mathbf{n}$ en la fórmula general $\mathbf{v} = \mathbf{V} + \mathbf{\Omega} \times \mathbf{r}$ ($a$ es el radio de la esfera y $\mathbf{n}$ el vector unitario de la normal al plano en el punto de contacto). La ligadura buscada está dada por la condición de que no haya deslizamiento en el punto de contacto, es decir,
\[
\mathbf{V} - a \mathbf{\Omega} \times \mathbf{n} = 0. \tag{38.3}
\]
Esta ecuación no es integrable: aunque la velocidad $\mathbf{V}$ es la derivada total respecto al tiempo del radio vector del centro de la esfera, la velocidad angular no es en general la derivada total de una coordenada respecto al tiempo, de modo que la ligadura (38.3) es no holónoma\footnote{Observemos que la ligadura análoga en la rodadura de un cilindro sería holónoma. En este caso, el eje de rotación tiene una dirección fija en el espacio, y, por lo tanto, $\Omega = d\phi/dt$ es una derivada total del ángulo $\phi$ de rotación del cilindro alrededor de su eje. La condición (38.3) puede entonces integrarse y da una relación entre el ángulo $\phi$ y la coordenada del centro de masa.}.

La presencia de ligaduras del tipo (38.2) impone ciertas restricciones en los valores posibles de las variaciones de las coordenadas: multiplicando las ecuaciones (38.2) por $\delta t$, se encuentra que las variaciones $\delta q_i$ no son independientes, sino están relacionadas por
\[
\sum_i c_{\alpha i} \delta q_i = 0. \tag{38.4}
\]
Esto debe tenerse en cuenta cuando se varía la acción. Según el método de Lagrange para hallar extremales condicionados, deben añadirse al integrando de la variación de la acción
\[
\delta S = \int \left[ \sum_i \delta q_i \left( \frac{\partial L}{\partial q_i} - \frac{d}{dt} \frac{\partial L}{\partial \dot{q}_i} \right) \right] dt
\]
los términos de la izquierda de las ecuaciones (38.4) multiplicados por coeficientes indeterminados\footnote{Denominados \textit{multiplicadores de Lagrange}.} $\lambda_\alpha$ (funciones de las coordenadas), y después igualar el integrando a cero. Entonces pueden considerarse todas las variaciones $\delta q_i$ como independientes, y se obtienen las ecuaciones
\[
\frac{d}{dt} \left( \frac{\partial L}{\partial \dot{q}_i} \right) - \frac{\partial L}{\partial q_i} = \sum_\alpha \lambda_\alpha c_{\alpha i}. \tag{38.5}
\]

Existe sin embargo otro método para establecer las ecuaciones del movimiento de cuerpos en contacto, en el cual se introducen explícitamente las reacciones. Este método, que constituye el \textit{principio de d'Alembert}, consiste esencialmente en escribir para cada uno de los cuerpos en contacto las ecuaciones
\[
\frac{d\mathbf{P}}{dt} = \sum \mathbf{f}, \qquad \frac{d\mathbf{M}}{dt} = \sum \mathbf{r} \times \mathbf{f}, \tag{38.6}
\]
estando comprendidas las fuerzas de reacción en el conjunto de las fuerzas $\mathbf{f}$ que actúan sobre el cuerpo; estas fuerzas son desconocidas \textit{a priori} y se determinarán, a la vez que el movimiento del cuerpo, resolviendo las ecuaciones.

\subsection{Ejecicios}
\textbf{1. Utilizando el principio de d'Alembert, hallar las ecuaciones del movimiento de una esfera homogénea que rueda sobre un plano bajo la acción de una fuerza exterior $\mathbf{F}$ y de un momento $\mathbf{K}$.}

\textit{Solución:} La ecuación de ligadura es (38.3). Llamando a $\mathbf{R}$ la fuerza de reacción en el punto de contacto entre la esfera y el plano, se escriben las ecuaciones (38.6):
\begin{align*}
m \frac{d\mathbf{V}}{dt} &= \mathbf{F} + \mathbf{R}, \tag{1} \\
I \frac{d\mathbf{\Omega}}{dt} &= \mathbf{K} - a \mathbf{n} \times \mathbf{R}, \tag{2}
\end{align*}
(teniendo en cuenta que $\mathbf{P} = m\mathbf{V}$ y que para una peonza esférica $M = I\Omega$). Derivando con respecto al tiempo la ecuación de ligadura (38.3), se obtiene:
\[
\dot{\mathbf{V}} = a \dot{\mathbf{\Omega}} \times \mathbf{n}.
\]
Sustituyendo en (1) y eliminando $\dot{\mathbf{\Omega}}$ por medio de (2), se encuentra,
\[
(I/ma)\mathbf{F} + \mathbf{R} = \mathbf{K} \times \mathbf{n} - a \mathbf{R} + a \mathbf{n}(\mathbf{n} \cdot \mathbf{R}),
\]
que relaciona la fuerza de reacción con $\mathbf{F}$ y $\mathbf{K}$. Escribiendo esta ecuación en componentes y sustituyendo $I = \frac{2}{5}ma^2$ (véase problema 2 b, §32), se tiene
\[
R_{x} = \frac{5}{7a}K_y - \frac{2}{7}F_x, \qquad R_{y} = -\frac{5}{7a}K_x - \frac{2}{7}F_y, \qquad R_z = -F_z,
\]
(se ha tomado el plano como plano $xy$). Finalmente, sustituyendo estas expresiones en (1), se obtienen las ecuaciones del movimiento conteniendo solamente los datos, fuerza exterior y momento:
\[
m \frac{dV_x}{dt} = \frac{5}{7} \left( F_x + \frac{K_y}{a} \right), \qquad m \frac{dV_y}{dt} = \frac{5}{7} \left( F_y - \frac{K_x}{a} \right).
\]
Las componentes $\Omega_x$ y $\Omega_y$ de la velocidad angular se expresan en función de $V_y$, $V_x$ por la ecuación de ligadura (38.3), y para $\Omega_z$ se tiene,
\[
I \frac{d\Omega_z}{dt} = K_z,
\]
donde $I$ es el momento de inercia respecto al eje $z$.


\textbf{2. Una varilla homogénea $BD$ de peso $P$ y longitud $l$ está apoyada contra una pared (fig. 52); su extremo inferior $B$ está mantenido por un hilo $AB$. Calcular la reacción de la pared y la tensión del hilo.}

\textit{Solución:} El peso de la varilla puede representarse por una fuerza $P$ aplicada en su punto medio y dirigida verticalmente hacia abajo. Las reacciones $R_B$ y $R_C$ están respectivamente dirigidas verticalmente hacia arriba y perpendicularmente a la varilla; la tensión $T$ del hilo está dirigida de $B$ hacia $A$. La resolución de las ecuaciones de equilibrio da:
\[
R_C = \frac{Pl}{4h} \sen 2\alpha, \qquad R_B = P - R_C \sen \alpha, \qquad T = R_C \cos \alpha,
\]
donde $h$ es la altura y $\alpha$ el ángulo indicado en la figura.

\vspace{1cm}

\textbf{3. Una varilla $AB$ de peso $P$ tiene un extremo sobre un plano horizontal y el otro en un plano vertical, y se mantiene en esta posición por dos hilos horizontales $AD$ y $BC$; el hilo $BC$ se halla en el mismo plano vertical que la varilla. Determinar las reacciones de los planos y las tensiones en los hilos.}

\textit{Solución:} Las tensiones de los hilos $T_A$ y $T_B$ están dirigidas de $A$ hacia $D$ y de $B$ a $C$, respectivamente. Las reacciones $R_A$ y $R_B$ son perpendiculares a los planos correspondientes. La resolución de las ecuaciones de equilibrio da:
\[
R_B = P, \qquad T_B = \frac{1}{2}P \cotg \alpha, \qquad R_A = T_B \sen \beta, \qquad T_A = T_B \cos \beta.
\]

\vspace{1cm}

\textbf{4. Dos varillas de longitud $l$ y peso despreciable están unidas por una articulación, y sus extremos se apoyan en un plano conectados por un hilo $AB$ (fig. 54). En el centro de una de las varillas se aplica una fuerza $F$. Determinar las reacciones.}

\textit{Solución:} La tensión $T$ del hilo actúa en el punto $A$ de $A$ hacia $B$, y en el punto $B$ de $B$ hacia $A$. Las reacciones $R_A$ y $R_B$ son perpendiculares al plano. Sea $R_C$ la reacción sobre la varilla $AC$ en la articulación; entonces, sobre la varilla $BC$ actúa una reacción $-R_C$. La condición de que la suma de los momentos de las fuerzas $R_B$, $T$ y $-R_C$ sobre la varilla $BC$ sea nula muestra que $R_C$ actúa a lo largo de $BC$. Las otras condiciones de equilibrio (para las dos varillas por separado) dan:
\[
R_A = \frac{3}{4}F, \qquad R_B = \frac{1}{4}F, \qquad R_C = \frac{1}{4}F \csc \alpha, \qquad T = \frac{1}{4}F \cotg \alpha,
\]
donde $\alpha$ es el ángulo $CAB$.

\vspace{1cm}

\subsection*{§ 39. Movimiento en un sistema de referencia no inercial}

Hasta aquí, siempre se han utilizado sistemas de referencia inerciales al discutir el movimiento de los sistemas mecánicos. Por ejemplo, la lagrangiana de una partícula en un campo exterior
\begin{equation}
L_0 = \tfrac{1}{2} m v_0^2 - U,
\tag{39.1}
\end{equation}
y la ecuación del movimiento correspondiente
\begin{equation}
m \, d\mathbf{v}_0/dt = -\partial U/\partial \mathbf{r},
\end{equation}
son válidas solamente en un sistema inercial. (En esta sección se designará con el índice 0 las magnitudes referidas a un sistema inercial.)

Veamos ahora qué forma toman las ecuaciones del movimiento de una partícula en un sistema no inercial. El punto de partida para resolver este problema es otra vez el principio de la mínima acción, cuya validez no depende del sistema de referencia elegido. Las ecuaciones de Lagrange
\begin{equation}
\frac{d}{dt} \left( \frac{\partial L}{\partial \mathbf{v}} \right) = \frac{\partial L}{\partial \mathbf{r}}
\tag{39.2}
\end{equation}
son igualmente válidas. Sin embargo, la lagrangiana no toma la forma (39.1), y para calcularla debe transformarse adecuadamente la función $L_0$.

Esta transformación se realiza en dos etapas. Consideremos, primero un sistema de referencia $K'$ que se mueve con una velocidad de traslación $\mathbf{V}(t)$ con respecto al sistema de referencia inercial $K_0$. Las velocidades $\mathbf{v}_0$ y $\mathbf{v}'$ de una partícula, en los sistemas $K_0$ y $K'$ respectivamente, están relacionadas por
\begin{equation}
\mathbf{v}_0 = \mathbf{v}' + \mathbf{V}(t).
\tag{39.3}
\end{equation}
Sustituyendo en (39.1) se obtiene la lagrangiana en el sistema $K'$:
\[
L' = \tfrac{1}{2} m v'^2 + m \mathbf{v}' \cdot \mathbf{V} + \tfrac{1}{2} m V^2 - U.
\]

\vspace{1em}

Introduzcamos ahora otro sistema de referencia $K$, cuyo origen coincide con el de $K'$, pero que gira con relación a $K'$ con velocidad angular $\mathbf{\Omega}(t)$; con respecto al sistema inercial $K_0$, el sistema $K$ efectúa a la vez una traslación y una rotación.

La velocidad $\mathbf{v'}$ de la partícula con relación al sistema $K'$ se expresa en función de su velocidad $\mathbf{v}$ relativa al sistema $K$ y de la velocidad $\mathbf{\Omega} \times \mathbf{r}$ y del movimiento de su rotación con $K$:
\[
\mathbf{v'} = \mathbf{v} + \mathbf{\Omega} \times \mathbf{r}
\]
(los vectores de posición $\mathbf{r}$ y $\mathbf{r}'$ de la partícula en los sistemas $K$ y $K'$ coinciden).
Sustituyendo en la lagrangiana (39.4), se obtiene:
\begin{align}
L &= \tfrac{1}{2} m v^2 + m \mathbf{v} \cdot \mathbf{\Omega} \times \mathbf{r} + \tfrac{1}{2} m (\mathbf{\Omega} \times \mathbf{r})^2 - m \mathbf{w} \cdot \mathbf{r} - U.
\tag{39.6}
\end{align}

Esta es la forma general de la lagrangiana de una partícula en un sistema de referencia arbitrario, no necesariamente inercial. Observamos que la rotación del sistema de referencia hace aparecer en la lagrangiana un término lineal con respecto a la velocidad de la partícula.

Para calcular las derivadas que entran en la ecuación de Lagrange, escribimos la diferencial total:
\begin{align*}
dL &= m \mathbf{v} \cdot d\mathbf{v} + m d\mathbf{v} \cdot \mathbf{\Omega} \times \mathbf{r} + m \mathbf{v} \cdot \mathbf{\Omega} \times d\mathbf{r} \\
&\quad + \tfrac{1}{2} m (\mathbf{\Omega} \times \mathbf{r}) \cdot d(\mathbf{\Omega} \times \mathbf{r}) - m d\mathbf{w} \cdot \mathbf{r} - (\partial U/\partial \mathbf{r}) \cdot d\mathbf{r}.
\end{align*}
Reuniendo por separado los términos que contienen $d\mathbf{v}$ y $d\mathbf{r}$, se tiene,
\[
\frac{\partial L}{\partial \mathbf{v}} = m \mathbf{v} + m \mathbf{\Omega} \times \mathbf{r}, \qquad
\frac{\partial L}{\partial \mathbf{r}} = m \mathbf{v} \times \mathbf{\Omega} + m (\mathbf{\Omega} \times \mathbf{r}) \times \mathbf{\Omega} - m \mathbf{w} - \frac{\partial U}{\partial \mathbf{r}}.
\]
Sustituidas estas expresiones en (39.2), nos dan la ecuación del movimiento buscada:
\begin{equation}
m \frac{d\mathbf{v}}{dt} = -\frac{\partial U}{\partial \mathbf{r}} - m \mathbf{w} + m \mathbf{r} \times \mathbf{\dot{\Omega}} + 2m \mathbf{v} \times \mathbf{\Omega} + m \mathbf{\Omega} \times (\mathbf{r} \times \mathbf{\Omega}).
\tag{39.7}
\end{equation}

\vspace{1em}

Haciendo en (39.6) y (39.7) $\mathbf{\Omega} = \text{cte.}$, $\mathbf{w} = 0$, se obtiene la lagrangiana
\begin{equation}
L = \tfrac{1}{2} m v^2 + m \mathbf{v} \cdot \mathbf{\Omega} \times \mathbf{r} + \tfrac{1}{2} m (\mathbf{\Omega} \times \mathbf{r})^2 - U
\tag{39.8}
\end{equation}
y la ecuación del movimiento
\begin{equation}
m \frac{d\mathbf{v}}{dt} = -\frac{\partial U}{\partial \mathbf{r}} + 2m \mathbf{v} \times \mathbf{\Omega} + m \mathbf{\Omega} \times (\mathbf{r} \times \mathbf{\Omega}).
\tag{39.9}
\end{equation}

La energía de la partícula en este caso se obtiene sustituyendo
\[
\mathbf{p} = \frac{\partial L}{\partial \mathbf{v}} = m \mathbf{v} + m \mathbf{\Omega} \times \mathbf{r}
\]
en $E = \mathbf{p} \cdot \mathbf{v} - L$, obteniéndose,
\begin{equation}
E = \tfrac{1}{2} m v^2 - \tfrac{1}{2} m (\mathbf{\Omega} \times \mathbf{r})^2 + U.
\tag{39.11}
\end{equation}

Observemos que la expresión de la energía no contiene término lineal en la velocidad. La rotación del sistema añade simplemente a la energía un término que depende solamente de las coordenadas de la partícula y es proporcional al cuadrado de la velocidad angular. Este término adicional $-\tfrac{1}{2} m (\mathbf{\Omega} \times \mathbf{r})^2$ se llama \textit{energía potencial centrífuga}.

La velocidad $\mathbf{v}$ de la partícula con respecto al sistema que gira uniformemente está relacionada con su velocidad $\mathbf{v}_0$ con respecto al sistema inercial $K_0$ por
\begin{equation}
\mathbf{v}_0 = \mathbf{v} + \mathbf{\Omega} \times \mathbf{r}.
\tag{39.12}
\end{equation}

El ímpetu $\mathbf{p}$ (39.10) de la partícula en el sistema $K$ coincide por lo tanto con su ímpetu $\mathbf{p}_0 = m \mathbf{v}_0$ en el sistema $K_0$. Los momentos angulares $\mathbf{M} = \mathbf{r} \times \mathbf{p}$ y $\mathbf{M}_0 = \mathbf{r} \times \mathbf{p}_0$ son también iguales. Sin embargo, las energías de la partícula en los sistemas $K$ y $K_0$ son diferentes. Sustituyendo $\mathbf{v'}$ de (39.12) en (39.11), se obtiene
\[
E = \tfrac{1}{2} m v_0^2 - m \mathbf{v}_0 \cdot \mathbf{\Omega} \times \mathbf{r} + U = \tfrac{1}{2} m v_0^2 + U - m \mathbf{r} \times \mathbf{v}_0 \cdot \mathbf{\Omega}.
\]
Los dos primeros términos son la energía $E_0$ en el sistema $K_0$. Utilizando el momento angular, se tiene
\begin{equation}
E = E_0 - \mathbf{M} \cdot \mathbf{\Omega}.
\tag{39.13}
\end{equation}

Esta fórmula define la ley de transformación de la energía cuando se pasa a un sistema de coordenadas animado de una rotación uniforme.

\subsection*{PROBLEMAS}

\textbf{1. Encontrar la separación con respecto a la vertical, provocada por la rotación de la Tierra, de un cuerpo que cae libremente. (La velocidad angular de rotación se considera pequeña.)}

\textit{Solución:} En el campo de la gravedad $U = -mg\mathbf{r}$ donde $g$ es el vector aceleración de la gravedad; despreciando en la ecuación (39.9) la fuerza centrífuga que contiene el cuadrado de $\Omega$, se tiene la ecuación del movimiento
\begin{equation}
\dot{\mathbf{v}} = 2\mathbf{v} \times \mathbf{\Omega} + \mathbf{g}. \tag{1}
\end{equation}

Esta ecuación puede resolverse por aproximaciones sucesivas. Para ello ponemos $\mathbf{v} = \mathbf{v}_1 + \mathbf{v}_2$, donde $\mathbf{v}_1$ es la solución de la ecuación $\dot{\mathbf{v}}_1 = \mathbf{g}$, es decir, $\mathbf{v}_1 = g t + \mathbf{v}_0$ (siendo $\mathbf{v}_0$ la velocidad inicial). Sustituyendo $\mathbf{v} = \mathbf{v}_1 + \mathbf{v}_2$ en (1) y conservando solamente $\mathbf{v}_1$ en el segundo miembro, se obtiene para $\mathbf{v}_2$ la ecuación
\[
\dot{\mathbf{v}}_2 = 2\mathbf{v}_1 \times \mathbf{\Omega} = 2t\mathbf{g} \times \mathbf{\Omega} + 2\mathbf{v}_0 \times \mathbf{\Omega}.
\]

La integración da
\begin{equation}
\mathbf{r} = \mathbf{h} + \mathbf{v}_0 t + \tfrac{1}{2} g t^2 + \tfrac{1}{3} t^3 \mathbf{g} \times \mathbf{\Omega} + t^2 \mathbf{v}_0 \times \mathbf{\Omega},
\tag{2}
\end{equation}
donde $\mathbf{h}$ es el vector de posición inicial de la partícula.

Tomemos el eje $z$ verticalmente hacia arriba y el eje $x$ hacia el polo; entonces
\[
g_x = g_y = 0, \quad g_z = -g; \quad \Omega_x = \Omega \cos \lambda, \quad \Omega_y = 0, \quad \Omega_z = \Omega \sin \lambda,
\]
donde $\lambda$ es la latitud (que tomamos norte para fijar ideas). Haciendo $\mathbf{v}_0 = 0$ (en $z$), resulta,
\[
x = 0, \quad y = -\tfrac{1}{3} t^3 g \Omega \cos \lambda.
\]

Sustituyendo el tiempo de caída $t \approx \sqrt{2h/g}$, encontramos finalmente,
\[
x = 0, \quad y = -\tfrac{1}{3} (2h/g)^{3/2} g \Omega \cos \lambda,
\]
(el signo menos indica un desplazamiento hacia el este).

\vspace{1em}

\textbf{2. Determinar la separación de la trayectoria de un cuerpo lanzado desde la superficie de la Tierra con velocidad $v_0$, respecto del plano inicial.}

\textit{Solución:} Sea el plano $xz$ tal que contenga la velocidad $v_0$. La altura inicial es $h = 0$. La desviación lateral dada por la ecuación (2) del problema 1 es:
\[
y = -\tfrac{1}{3} t^2 g \Omega_z + t^2 ( \Omega_x v_{0z} - \Omega_z v_{0x} )
\]
o, sustituyendo la duración de la trayectoria $t \approx 2v_{0z}/g$:
\[
y = \tfrac{4v_{0z}^2}{g^2} (\frac{1}{3} \Omega_x v_{0z} - \Omega_z v_{0x} ).
\]

\vspace{1em}

\textbf{3. Determinar la influencia de la rotación de la Tierra en las pequeñas oscilaciones de un péndulo (problema del \textit{péndulo de Foucault}).}

\textit{Solución:} Despreciando el desplazamiento vertical del péndulo, como infinitésimo de segundo orden, puede considerarse que el movimiento tiene lugar en el plano horizontal $xy$. Omitiendo los términos que contienen $\Omega^2$, se tienen las ecuaciones del movimiento
\[
\ddot{x} + \omega^2 x = 2\Omega_z \dot{y}, \qquad \ddot{y} + \omega^2 y = -2\Omega_z \dot{x},
\]
donde $\omega$ es la frecuencia de oscilación del péndulo si no se tuviese en cuenta la rotación de la Tierra. Multiplicando la segunda ecuación por $i$ y sumando, se obtiene la ecuación única
\[
\ddot{\xi} + 2i\Omega_z \dot{\xi} + \omega^2 \xi = 0
\]
en la magnitud compleja $\xi = x + i y$. Para $\Omega_z \ll \omega$ la solución de esta ecuación es:
\[
\xi = \exp(-i\Omega_z t) [A_1 \exp(i\omega t) + A_2 \exp(-i\omega t)]
\]
o
\[
x + i y = (x_0 + i y_0) \exp(-i\Omega_z t),
\]
donde las funciones $x_0(t)$ e $y_0(t)$ dan la trayectoria del péndulo cuando se desprecia la rotación de la Tierra. El efecto de esta rotación es, por lo tanto, un giro de la trayectoria alrededor de la vertical con una

\section{Ecuaciones canonicas}

\paragraph*{§ 40. Ecuaciones de Hamilton}

La formulación de las leyes de la Mecánica con la ayuda de la lagrangiana (y de las ecuaciones de Lagrange que de ella se deducen) presupone que el estado mecánico del sistema está determinado dando sus coordenadas y velocidades generalizadas. Sin embargo, éste no es el único método posible; la descripción del estado de un sistema en función de sus coordenadas e ímpetus generalizados presenta un cierto número de ventajas, especialmente en el estudio de diferentes problemas generales de Mecánica. Entonces se deben deducir las ecuaciones del movimiento correspondientes a esta formulación.

El paso de un conjunto de variables independientes a otro puede realizarse mediante lo que se llama en matemáticas transformación de Legendre. En el presente caso esta transformación toma la forma siguiente. La diferencial total de la lagrangiana como función de las coordenadas y de las velocidades es:
\[
dL
= \sum_i \frac{\partial L}{\partial q_i}\,dq_i
+ \sum_i \frac{\partial L}{\partial \dot q_i}\,d\dot q_i.
\]
Esta expresión puede escribirse
\begin{equation}
d\biggl(\sum_i p_i\,\dot q_i - L\biggr)
= -\sum_i \dot p_i\,dq_i
+ \sum_i \dot q_i\,dp_i
\tag{40.1}
\end{equation}

La cantidad bajo el signo diferencial es la energía del sistema (véase § 6); expresada en función de las coordenadas y de los ímpetus, se llama función de Hamilton o hamiltoniana del sistema:
\begin{equation}
H(p,q,t)
= \sum_i p_i\,\dot q_i - L
\tag{40.2}
\end{equation}

De la ecuación
\begin{equation}
dH
= -\sum_i \dot p_i\,dq_i
+ \sum_i \dot q_i\,dp_i
\tag{40.3}
\end{equation}
en la cual las variables independientes son las coordenadas y los ímpetus, se obtienen las ecuaciones
\begin{equation}
\dot q_i = \frac{\partial H}{\partial p_i},
\qquad
\dot p_i = -\frac{\partial H}{\partial q_i}
\tag{40.4}
\end{equation}
Estas son las deseadas ecuaciones del movimiento en las variables \(p\) y \(q\), y se llaman ecuaciones de Hamilton. Constituyen un conjunto de \(2s\) ecuaciones diferenciales de primer orden, entre las \(2s\) funciones incógnitas \(p_i(t)\) y \(q_i(t)\), que sustituyen a las ecuaciones de segundo orden obtenidas por medio de Lagrange. A causa de su sencillez y simetría de forma se les llama ecuaciones canónicas.

La derivada total con respecto al tiempo de la hamiltoniana es:
\[
\frac{dH}{dt}
= \frac{\partial H}{\partial t}
+ \sum_i \frac{\partial H}{\partial q_i}\,\dot q_i
+ \sum_i \frac{\partial H}{\partial p_i}\,\dot p_i.
\]
Sustituyendo \(\dot q_i\) y \(\dot p_i\) de las ecuaciones (40.4) los dos últimos términos se eliminan, de modo que
\begin{equation}
\frac{dH}{dt}
= \frac{\partial H}{\partial t}
\tag{40.5}
\end{equation}
En particular, si la función de Hamilton no depende explícitamente del tiempo, entonces \(\tfrac{dH}{dt}=0\), y se tiene la ley de la conservación de la energía.

Además de las variables dinámicas \(q\) y \(p\), las funciones de Lagrange y de Hamilton contienen varios parámetros, los cuales se refieren bien a las propiedades del sistema mecánico o bien al campo exterior que actúa sobre él. Sea \(\lambda\) uno de estos parámetros. Considerándolo como una variable, se tiene en lugar de (40.1):
\[
dL
= \sum_i p_i\,d\dot q_i
+ \sum_i \dot p_i\,dq_i
+ \frac{\partial L}{\partial \lambda}\,d\lambda,
\]
y en lugar de (40.3):
\[
dH
= -\sum_i \dot p_i\,dq_i
+ \sum_i \dot q_i\,dp_i
- \frac{\partial L}{\partial \lambda}\,d\lambda,
\]
de donde
\begin{equation}
\Bigl(\frac{\partial H}{\partial \lambda}\Bigr)_{p,q}
= -\Bigl(\frac{\partial L}{\partial \lambda}\Bigr)_{\dot q,q}
\tag{40.6}
\end{equation}

\paragraph*{Este resultado puede expresarse de otro modo.}
Sea la lagrangiana 
\[
L = L_0 + L',
\]
siendo \(L'\) una pequeña corrección a la función fundamental \(L_0\). La correspondiente adición \(H'\) en la hamiltoniana 
\[
H = H_0 + H'
\]
está relacionada con \(L'\) por
\begin{equation}
(H')_{p,q}
= -\,\bigl(L'\bigr)_{\dot q,q}
\tag{40.7}
\end{equation}

Puede observarse que en la transformación de (40.1) en (40.3) no se ha escrito un término en \(dt\), que tendría en cuenta la posible dependencia explícita del tiempo de la lagrangiana; el tiempo representaría en este caso sólo el papel de un parámetro que nada tiene que ver con la transformación. Las derivadas parciales con respecto al tiempo de \(L\) y de \(H\) están ligadas por la relación
\begin{equation}
\Bigl(\tfrac{\partial H}{\partial t}\Bigr)_{p,q}
= -\,\Bigl(\tfrac{\partial L}{\partial t}\Bigr)_{\dot q,q}
\tag{40.8}
\end{equation}

\section*{Problemas}

1. Hallar la función de Hamilton de una partícula, en coordenadas cartesianas, cilíndricas y esféricas.  
Solución.  
En coordenadas cartesianas \((x,y,z)\):
\[
H=\frac{1}{2m}\bigl(p_x^2+p_y^2+p_z^2\bigr)+U(x,y,z).
\]
En coordenadas cilíndricas \((r,\phi,z)\):
\[
H=\frac{1}{2m}\Bigl(p_r^2+\frac{p_\phi^2}{r^2}+p_z^2\Bigr)+U(r,\phi,z).
\]
En coordenadas esféricas \((r,\theta,\phi)\):
\[
H=\frac{1}{2m}\Bigl(p_r^2+\frac{p_\theta^2}{r^2}+\frac{p_\phi^2}{r^2\sin^2\theta}\Bigr)
+U(r,\theta,\phi).
\]

2. Hallar la hamiltoniana de una partícula en un sistema de referencia animado de un movimiento de rotación uniforme.  
Solución.  
Usando la expresión de la energía en función de \(p\) se obtiene
\[
H=\frac{p^2}{2m}-\boldsymbol\Omega\!\cdot(\mathbf r\times\mathbf p)+U.
\]

3. Hallar la función de Hamilton de una partícula de masa \(M\) y \(n\) partículas de masa \(m\), excluido el movimiento del centro de masa (véase el problema en § 13).  
Solución.  
Partiendo de la lagrangiana del § 13 (cambiando el signo de \(U\)), los ímpetus generalizados son
\[
p_a=\frac{\partial L}{\partial \dot v_a}
= m\,v_a-\frac{m}{M}\sum_{b}v_b.
\]
De aquí se sigue
\[
v_a=\frac{p_a}{m}+\frac{1}{M}\sum_{b}p_b.
\]

\section*{§ 41. Función de Routh}

En ciertos casos es conveniente, cuando se pasa a las nuevas variables, no sustituir todas las velocidades generalizadas por los ímpetus, sino sólo algunas de ellas. La transformación correspondiente es análoga a la del § 40.

Para simplificar, supongamos dos coordenadas \(q\) y \(\xi\). Pasamos de \((q,\xi,\dot q,\dot\xi)\) a \((q,\xi,p,\dot\xi)\), siendo 
\[
p=\frac{\partial L}{\partial \dot q}
\]
el ímpetu generalizado de \(q\). La diferencial de \(L(q,\xi,\dot q,\dot\xi)\) es
\[
dL
=(\partial_qL)\,dq+(\partial_{\dot q}L)\,d\dot q
+(\partial_\xi L)\,d\xi+(\partial_{\dot\xi}L)\,d\dot\xi
=p\,d\dot q+(\partial_qL)\,dq+(\partial_\xi L)\,d\xi+(\partial_{\dot\xi}L)\,d\dot\xi,
\]
por tanto
\[
d\bigl(L-p\,\dot q\bigr)
=-p\,dq+q\,dp+(\partial_\xi L)\,d\xi+(\partial_{\dot\xi}L)\,d\dot\xi.
\]
Si se define la función de Routh como
\begin{equation}
R(q,p,\xi,\dot\xi)=p\,\dot q-L,
\tag{41.1}
\end{equation}
en la cual \(\dot q\) se expresa en función de \(p\), su diferencial es
\begin{equation}
dR=-\dot q\,dq+q\,dp-(\partial_\xi L)\,d\xi-(\partial_{\dot\xi}L)\,d\dot\xi.
\tag{41.2}
\end{equation}
De aquí se deduce
\begin{equation}
\frac{\partial R}{\partial p}=\dot q,\quad
\dot p=-\frac{\partial R}{\partial q},
\tag{41.3}
\end{equation}
\begin{equation}
\frac{\partial R}{\partial \xi}=-\,\frac{\partial L}{\partial \xi},\quad
\frac{\partial R}{\partial \dot\xi}=-\,\frac{\partial L}{\partial \dot\xi}.
\tag{41.4}
\end{equation}
Sustituyendo en las ecuaciones de Lagrange para \(\xi\) se obtiene
\begin{equation}
\frac{d}{dt}\!\Bigl(\partial_{\dot\xi}R\Bigr)
-\partial_\xi R=0.
\tag{41.5}
\end{equation}

Entonces la función de Routh es una hamiltoniana con respecto a \(q\) (ec.\,(41.3)) y una lagrangiana para \(\xi\) (ec.\,(41.5)).
De acuerdo con la definición general, la energía del sistema es:
\[
E = \dot q\,\frac{\partial L}{\partial \dot q}
  + \dot\xi\,\frac{\partial L}{\partial \dot\xi}
  - L
  = p\dot q + \xi\,\frac{\partial L}{\partial \dot\xi} - L.
\]
Se expresa mediante la función de Routh sustituyendo (41.1) y (41.4):
\begin{equation}
E = R - \xi\,\frac{\partial R}{\partial \dot\xi}.
\tag{41.6}
\end{equation}

La generalización de estas fórmulas para el caso de varias coordenadas \(q\) y \(\xi\) es inmediata.  
En particular, puede convenir usar la función de Routh cuando algunas coordenadas son cíclicas. Si \(q\) es cíclica, no aparece en la lagrangiana ni en \(R\), que sólo depende entonces de \(p,\xi,\dot\xi\). Los ímpetus \(p\) relativos a las coordenadas cíclicas son constantes (véase la segunda ecuación de (41.3)). Sustituyendo estos valores constantes en (41.5)
\[
\frac{d}{dt}\!\Bigl(\frac{\partial R}{\partial \dot\xi}\Bigr)
= \frac{\partial R}{\partial \xi},
\]
se obtienen ecuaciones sólo en \(\xi(t)\). Una vez resuelta \(\xi(t)\), las ecuaciones
\[
\dot q = \frac{\partial R}{\partial p}
\]
dan \(q(t)\) por integración directa.

\paragraph*{Problema}  
Encontrar la función de Routh de una peonza simétrica en un campo exterior \(U(\theta,\varphi)\), eliminando la coordenada cíclica \(\psi\) (\(\varphi,\psi,\theta\) son los ángulos de Euler).

\paragraph*{Solución}  
La lagrangiana es
\[
L = \frac12 I\bigl(\dot\varphi^2\sin^2\theta + \dot\theta^2\bigr)
  + \frac12 I_3\bigl(\dot\psi + \dot\varphi\cos\theta\bigr)^2
  - U(\theta,\varphi).
\]
(véase problema 1, § 35). La función de Routh se define por
\[
R = p_\psi\,\dot\psi - L
= p_\psi\,\dot\psi
- \frac12 I\bigl(\dot\varphi^2\sin^2\theta + \dot\theta^2\bigr)
- \frac12 I_3\bigl(\dot\psi + \dot\varphi\cos\theta\bigr)^2
+ U(\theta,\varphi).
\]
El primer término es constante y puede omitirse.

\section*{§ 42. Paréntesis de Poisson}

Sea \(f(q,p,t)\) una función de coordenadas, ímpetus y tiempo. Su derivada total es
\[
\frac{df}{dt}
= \frac{\partial f}{\partial t}
+ \sum_k\Bigl(\frac{\partial f}{\partial q_k}\dot q_k
           + \frac{\partial f}{\partial p_k}\dot p_k\Bigr).
\]
Sustituyendo \(\dot q_k,\dot p_k\) de las ecuaciones de Hamilton (40.4), resulta
\begin{equation}
\frac{df}{dt} = \frac{\partial f}{\partial t} + [H,f],
\tag{42.1}
\end{equation}
donde
\begin{equation}
[H,f]
= \sum_k\Bigl(
    \frac{\partial H}{\partial p_k}\frac{\partial f}{\partial q_k}
  - \frac{\partial H}{\partial q_k}\frac{\partial f}{\partial p_k}
  \Bigr).
\tag{42.2}
\end{equation}
La expresión (42.2) es el paréntesis de Poisson de \(H\) y \(f\).

Recordando que las integrales del movimiento son funciones constantes en la evolución, de (42.1) se deduce la condición
\begin{equation}
\frac{\partial f}{\partial t} + [H,f] = 0.
\tag{42.3}
\end{equation}
Si \(f\) no depende explícitamente del tiempo,
\begin{equation}
[H,f] = 0.
\tag{42.4}
\end{equation}

Para dos magnitudes \(f,g\), el paréntesis de Poisson se define como
\begin{equation}
[f,g]
= \sum_k\Bigl(
    \frac{\partial f}{\partial p_k}\frac{\partial g}{\partial q_k}
  - \frac{\partial f}{\partial q_k}\frac{\partial g}{\partial p_k}
  \Bigr).
\tag{42.5}
\end{equation}
Sus propiedades son:
\[
[f,g] = -[g,f],
\tag{42.6}
\]
\[
[f,c] = 0,
\tag{42.7}
\]
\[
[f_1+f_2,g] = [f_1,g] + [f_2,g],
\tag{42.8}
\]
\[
[f_1f_2,g] = f_1\,[f_2,g] + [f_1,g]\,f_2.
\tag{42.9}
\]
Derivando (42.5) respecto al tiempo:
\[
\frac{\partial}{\partial t}[f,g]
= \Bigl[\frac{\partial f}{\partial t},g\Bigr]
+ \Bigl[f,\frac{\partial g}{\partial t}\Bigr].
\tag{42.10}
\]
Si una de las funciones coincide con una coordenada o un ímpetu,
\[
[f,q_j] = \frac{\partial f}{\partial p_j},
\tag{42.11}
\]
\[
[f,p_j] = -\frac{\partial f}{\partial q_j}.
\tag{42.12}
\]

La fórmula (42.11), por ejemplo, se obtiene haciendo \(g=q_k\) en (42.5); la suma se reduce a un solo término, puesto que \(\partial q_l/\partial q_k=\delta_{lk}\) y \(\partial q_l/\partial p_k=0\). Haciendo \(g=p_k\) en (42.5) y (42.12), y la función \(f\) igual a \(q_j\) y \(p_j\), se tiene, en particular,
\[
[q_i,q_k]=0,\quad [p_i,p_k]=0,\quad [p_k,q_i]=\delta_{ik}.
\tag{42.13}
\]

Entre los paréntesis de Poisson formados por tres funciones existe la relación
\begin{equation}
[[f,g],h]+[h,[f,g]]+[g,[h,f]]=0,
\tag{42.14}
\end{equation}
llamada identidad de Jacobi. Para demostrarla, observemos que, según la definición (42.5), los paréntesis de Poisson \([f,g]\) son funciones homogéneas bilineales de las primeras derivadas de \(f\) y \(g\). Entonces el paréntesis \([h,[f,g]]\), por ejemplo, será una función lineal homogénea de las segundas derivadas de \(f\), de \(g\) y de \(h\). Agrupemos en (42.14) los términos que contienen las derivadas segundas de \(f\). El primer paréntesis no contiene tales derivadas, ya que sólo contiene las primeras derivadas de \(f\). La suma del segundo y tercer paréntesis puede escribirse en forma simbólica utilizando dos operadores diferenciales lineales \(D_1,D_2\) definidos por
\[
D_1=\sum_k\xi_k\frac{\partial}{\partial x_k},
\qquad
D_2=\sum_k\eta_k\frac{\partial}{\partial x_k},
\]
donde \(\xi_k\) y \(\eta_k\) son funciones arbitrarias de las variables \(x_1,x_2,\dots\). Entonces,
\[
D_1D_2\phi
=\sum_{k,l}\xi_k\frac{\partial\eta_l}{\partial x_k}\,
\frac{\partial\phi}{\partial x_l}
+\sum_{k,l}\xi_k\eta_l\frac{\partial^2\phi}{\partial x_k\partial x_l},
\]
\[
D_2D_1\phi
=\sum_{k,l}\eta_k\frac{\partial\xi_l}{\partial x_k}\,
\frac{\partial\phi}{\partial x_l}
+\sum_{k,l}\eta_k\xi_l\frac{\partial^2\phi}{\partial x_k\partial x_l},
\]
y su diferencia es
\[
(D_1D_2-D_2D_1)\phi
=\sum_{l}\Bigl(\sum_k\xi_k\frac{\partial\eta_l}{\partial x_k}
            -\eta_k\frac{\partial\xi_l}{\partial x_k}\Bigr)
\frac{\partial\phi}{\partial x_l},
\]
que sólo contiene derivadas primeras. Procediendo análogamente con los otros términos de (42.14), todos los segundos derivados se cancelan, y la identidad de Jacobi queda demostrada.

De nuevo, un operador que sólo contiene derivadas primeras. Así, en el primer miembro de la ecuación (42.14), todos los términos con derivadas segundas se anulan recíprocamente, y puesto que ocurre análogamente con las funciones \(g\) y \(h\), la expresión entera es idénticamente nula.

Una propiedad importante de los paréntesis de Poisson es que, si \(f\) y \(g\) son integrales del movimiento, su paréntesis de Poisson es también una integral del movimiento:
\begin{equation}
[f,g]=\mathrm{cte}.
\tag{42.15}
\end{equation}
Este es el teorema de Poisson. La demostración es muy sencilla si \(f\) y \(g\) no dependen explícitamente del tiempo. Haciendo \(h=H\) en la identidad de Jacobi (42.14), se tiene
\[
[[H,f],g]+[g,[H,f]]+[f,[g,H]]=0.
\]
Como \( [H,g]=0\) y \([H,f]=0\), se deduce \([H,[f,g]]=0\), que es justo lo que se quería demostrar.

Si las integrales \(f\) y \(g\) dependen explícitamente del tiempo, partiendo de (42.1) escribimos
\[
\frac{d}{dt}[f,g]
=\frac{\partial}{\partial t}[f,g]+[H,[f,g]].
\]
Utilizando la fórmula (42.10) y expresando \([H,[f,g]]\) por medio de la identidad de Jacobi, se obtiene
\[
\begin{aligned}
\frac{d}{dt}[f,g]
&=\Bigl[\tfrac{\partial f}{\partial t},g\Bigr]
 +\Bigl[f,\tfrac{\partial g}{\partial t}\Bigr]
 +[H,[f,g]]
 -\bigl[[f,H],g\bigr]
 -\bigl[[g,H],f\bigr]\\
&=\bigl[\tfrac{df}{dt},\,g\bigr]
 +\bigl[f,\,\tfrac{dg}{dt}\bigr].
\end{aligned}
\tag{42.16}
\]
Esto demuestra el teorema de Poisson en el caso general.

Por supuesto, aplicando el teorema de Poisson no siempre se obtendrán nuevas integrales del movimiento, pues su número es limitado (\(2s-1\), siendo \(s\) el número de grados de libertad). En algunos casos el paréntesis es constante; en otros, la nueva integral resulta depender sólo de las originales. Sólo cuando no ocurre ninguno de estos casos, \([f,g]\) es una integral nueva.

\section*{Problemas}

\begin{enumerate}
\item Determinar los paréntesis de Poisson formados por las componentes cartesianas del ímpetu $\mathbf p$ y del momento angular $\mathbf M=\mathbf r\times\mathbf p$ de una partícula.

Solución. Por medio de la fórmula (42.12) se tiene
\[
[M_x,p_x]
= -\frac{\partial M_x}{\partial y}
= -\frac{\partial(y\,p_z - z\,p_y)}{\partial y}
= -\,p_z,
\quad
[M_x,p_y]=0,
\quad
[M_x,p_z]=p_y.
\]
Los restantes paréntesis se obtienen por permutación circular de los índices $x,y,z$.

\item Determinar los paréntesis de Poisson formados por las componentes de $\mathbf M$.

Solución. Un cálculo inmediato de la fórmula (42.5) da
\[
[M_x,M_y]=M_z,\quad
[M_y,M_z]=M_x,\quad
[M_z,M_x]=M_y.
\]
Puesto que los ímpetus y las coordenadas de distintas partículas son variables independientes, las fórmulas de los problemas 1 y 2 valen también para el momento angular total de un sistema.

\item Demostrar que
\[
[f,\mathbf M]=0,
\]
siendo $f$ una función escalar de las coordenadas $\mathbf r$ y del ímpetu $\mathbf p$.

Solución. Una función escalar sólo puede depender de $\mathbf r,\mathbf p$ a través de $r^2$, $p^2$ y $\mathbf r\cdot\mathbf p$. Luego
\[
\frac{\partial f}{\partial\mathbf r}
=2\frac{\partial f}{\partial(r^2)}\,\mathbf r
+\frac{\partial f}{\partial(\mathbf r\cdot\mathbf p)}\,\mathbf p,
\quad
\frac{\partial f}{\partial\mathbf p}
=2\frac{\partial f}{\partial(p^2)}\,\mathbf p
+\frac{\partial f}{\partial(\mathbf r\cdot\mathbf p)}\,\mathbf r,
\]
y la relación se verifica inmediatamente con (42.5).

\item Demostrar que
\[
[\boldsymbol\varepsilon,\,M_z]
=\mathbf n\times\boldsymbol\varepsilon,
\]
donde $\boldsymbol\varepsilon$ es una función vectorial de $\mathbf r$ y $\mathbf p$ y $\mathbf n$ el versor del eje $z$.

Solución. Todo vector $\boldsymbol\varepsilon(\mathbf r,\mathbf p)$ puede escribirse
\[
\boldsymbol\varepsilon
=\mathbf r\,\phi+\mathbf p\,\psi+(\mathbf r\times\mathbf p)\,\chi,
\]
con $\phi,\psi,\chi$ escalares. La relación se verifica directamente con (42.9), (42.11), (42.12) y el resultado del problema 3.
\end{enumerate}

\section*{\S\,43. La acción como una función de las coordenadas}

Al formular el principio de la mínima acción, consideramos la integral
\begin{equation}
S=\int_{t_1}^{t_2}L\,dt
\tag{43.1}
\end{equation}
tomada a lo largo de una trayectoria entre dos posiciones dadas $q^{(1)}$ y $q^{(2)}$ en los instantes $t_1$ y $t_2$. Al variar la acción, se comparan los valores de la integral para trayectorias vecinas con los mismos extremos; sólo aquella que hace $S$ mínimo es el movimiento real.

Ahora tratamos $S$ como función de la posición final $q(t_2)=q$, con $q(t_1)=q^{(1)}$ fijo. La variación de la acción al pasar a una trayectoria vecina está dada (si hay un grado de libertad) por
\[
\delta S
=\biggl[\frac{\partial L}{\partial\dot q}\,\delta q\biggr]_{t_1}^{t_2}
+\int_{t_1}^{t_2}\Bigl(\frac{\partial L}{\partial q}
-\frac{d}{dt}\frac{\partial L}{\partial\dot q}\Bigr)\delta q\,dt.
\]
Como la trayectoria real satisface las ecuaciones de Lagrange, la integral se anula. En el primer término ponemos $\partial L/\partial\dot q=p$ y designamos $q(t_2)=q$, resultando
\begin{equation}
\delta S = p\,\delta q,
\tag{43.2}
\end{equation}
y en el caso general
\begin{equation}
\delta S = \sum_i p_i\,\delta q_i.
\tag{43.2}
\end{equation}
De aquí se deduce que las derivadas parciales de la acción respecto de las coordenadas finales son
\begin{equation}
\frac{\partial S}{\partial q_i}=p_i.
\tag{43.3}
\end{equation}

De la definición de acción se sigue que su derivada total respecto al tiempo a lo largo de la trayectoria es:
\[
\frac{dS}{dt}=L.
\tag{43.4}
\]

Por otra parte, considerando \(S\) como función de las coordinadas y del tiempo, y aplicando la fórmula (43.3) se tiene:
\[
\frac{dS}{dt}
=\frac{\partial S}{\partial t}
+\sum_i\frac{\partial S}{\partial q_i}\,\dot q_i
=\frac{\partial S}{\partial t}
+\sum_i p_i\,\dot q_i.
\]
Comparando las dos expresiones, se deduce
\[
\frac{\partial S}{\partial t}
= L - \sum_i p_i\,\dot q_i,
\quad\text{o sea,}\quad
\frac{\partial S}{\partial t} = -H.
\tag{43.5}
\]

Las fórmulas (43.3) y (43.5) pueden reunirse en la expresión
\[
dS = \sum_i p_i\,dq_i - H\,dt.
\tag{43.6}
\]

Supongamos ahora que las coordenadas y el tiempo varían tanto en el instante inicial \(t_1\) como en el final \(t_2\). La variación de la acción vendrá dada por la diferencia de las expresiones (43.3) y (43.5) en los extremos de la trayectoria:
\[
dS
=\sum_i p_i\,\delta q_i\big\vert_{t_2}
 - H\,\delta t_2
-\Bigl(\sum_i p_i\,\delta q_i\big\vert_{t_1}
 - H\,\delta t_1\Bigr).
\tag{43.7}
\]

De aquí se deduce que, independientemente de la forma particular de \(L\), la condición de mínima acción equivale a requerir que la integral
\[
S=-\int\Bigl(\sum_i p_i\,dq_i - H\,dt\Bigr)
\tag{43.8}
\]
sea estacionaria.

\section*{\S\,44. Principio de Maupertuis}

Si \(L\) y, por tanto, \(H\) no dependen explícitamente del tiempo, la energía del sistema se conserva:
\[
H(p,q)=E=\mathrm{cte}.
\]
De acuerdo con el principio de mínima acción, variando el tiempo final y manteniendo fijas las coordenadas inicial y final, la variación de la acción es
\[
\delta S = -H\,\delta t.
\tag{44.1}
\]
Como \(H=E\) es constante,
\[
\delta S + E\,\delta t = 0.
\tag{44.2}
\]
Escribiendo la acción en la forma (43.8) y reemplazando \(H\) por \(E\), resulta
\[
S = \int \sum_i p_i\,dq_i \;-\; E\,(t_2 - t_1).
\tag{44.3}
\]

\paragraph*{El primer término en esta expresión}
\[
S_0 \;=\;\int \sum_i p_i\,dq_i
\tag{44.4}
\]
se denomina acción abreviada. Sustituyendo (44.3) en (44.2) se encuentra que
\[
\delta S_0 = 0.
\tag{44.5}
\]

Para aplicar este principio variacional, los ímpetus y todo el integrando de (44.2) deben expresarse en función de las coordenadas \(q\) y de sus diferenciales \(dq\); para ello empleamos

\begin{equation}
p_i = \frac{\partial L}{\partial \dot q_i}\bigl(q,\dot q\bigr)\,,
\tag{44.6}
\end{equation}

\begin{equation}
E\bigl(q,\dot q\bigr) = E\,.
\tag{44.7}
\end{equation}

Cuando la lagrangiana tiene la forma habitual  
\[
L=\frac12\sum_{i,k}a_{ik}(q)\,\dot q_i\dot q_k - U(q),
\]
los ímpetus y la energía se escriben
\[
p_i=\sum_k a_{ik}(q)\,\dot q_k,\qquad
E=\frac12\sum_{i,k}a_{ik}(q)\,\dot q_i\dot q_k+U(q).
\]
De aquí se deduce
\begin{equation}
dt = \sqrt{\frac{\sum_{i,k}a_{ik}\,dq_i\,dq_k}{2\bigl(E-U\bigr)}}.
\tag{44.8}
\end{equation}
Sustituyendo en \(\sum_i p_i\,dq_i\) se obtiene la acción abreviada
\begin{equation}
S_0 = \int \sqrt{2\bigl(E-U\bigr)\sum_{i,k}a_{ik}\,dq_i\,dq_k}\,.
\tag{44.9}
\end{equation}

En particular, para una sola partícula la energía cinética es  
\[
T=\tfrac12m\bigl(dl/dt\bigr)^2,
\]
donde \(m\) es la masa y \(dl\) un elemento de longitud de la trayectoria; el principio variacional es
\[
\delta\int\sqrt{2m\bigl(E-U\bigr)}\,dl = 0.
\tag{44.10}
\]

Volviendo a la acción abreviada \(S_0\), su derivada con respecto a \(E\) da
\begin{equation}
\frac{\partial S_0}{\partial E} = t - t_0.
\tag{44.11}
\end{equation}
Cuando \(S_0\) toma la forma (44.9), esta igualdad conduce a
\begin{equation}
\int\sqrt{\frac{\sum_{i,k}a_{ik}\,dq_i\,dq_k}{2\bigl(E-U\bigr)}} \;=\; t - t_0.
\tag{44.12}
\end{equation}

\paragraph*{Problema}
Deducir la ecuación de la trayectoria del principio variacional (44.10).

\paragraph*{Solución}
Efectuando la variación se halla
\[
\delta\int\sqrt{E-U}\,dl
=-\int\frac{\partial U/\partial r}{2\sqrt{E-U}}\,\delta r\,dl
-\int\sqrt{E-U}\,\delta(dl).
\]
Como \(dl^2=dr^2\), se usa \(dl\,\delta(dl)=dr\,\delta r\); integrando por partes el segundo término y anulando el coeficiente de \(\delta r\), se obtiene la ecuación diferencial
\[
2\bigl(E-U\bigr)\,\frac{d}{dl}\Bigl(\sqrt{E-U}\,\frac{dr}{dl}\Bigr)
=-\frac{\partial U}{\partial r}\,. 
\] 
\section*{\S\,45. Transformaciones canónicas}

La elección de las coordenadas generalizadas \(q\) no está limitada por ninguna condición; pueden ser \(s\) magnitudes cualesquiera que definan unívocamente la posición del sistema en el espacio. El aspecto formal de las ecuaciones de Lagrange no depende de esta elección, y en este sentido puede decirse que las ecuaciones de Lagrange son invariantes respecto a una transformación de las coordenadas \(q_i\), \(q_i'\), … en otras magnitudes independientes \(Q_i\), \(Q_i'\), …. Las nuevas coordenadas \(Q_i\) son funciones de las antiguas \(q_i\), y admitimos que pueden depender explícitamente del tiempo, es decir, que la transformación es de la forma
\[
Q_i = Q_i(q,t)
\tag{45.1}
\]
(denominadas a veces transformaciones puntuales).

Puesto que las ecuaciones de Lagrange son invariantes por la transformación (45.1), también son invariantes las ecuaciones de Hamilton (40.4). Sin embargo, estas últimas ecuaciones admiten en realidad un margen mucho más amplio de transformaciones. Esto es, por supuesto, porque en el método de Hamilton los ímpetus \(p\) son variables independientes con igual categoría que las coordenadas \(q\), y, por tanto, puede ampliarse la transformación para incluir las \(2s\) variables independientes \(p\) y \(q\) que pasarán a \(P\) y \(Q\) según las fórmulas
\[
Q_i = Q_i(q,p,t), 
\quad
P_i = P_i(q,p,t).
\tag{45.2}
\]

Esta ampliación de las posibles transformaciones constituye una de las ventajas esenciales del método de Hamilton en Mecánica. Sin embargo, las ecuaciones del movimiento no conservan su forma canónica para toda transformación del tipo (45.2). Por ello vamos a deducir ahora las condiciones que deben satisfacerse para que las ecuaciones del movimiento en las nuevas variables \(P,Q\) sean de la forma
\[
\dot Q_i = +\frac{\partial H'}{\partial P_i}, 
\quad
\dot P_i = -\frac{\partial H'}{\partial Q_i},
\tag{45.3}
\]
con una nueva hamiltoniana \(H'(P,Q)\). Cuando esto ocurre, la transformación se dice que es canónica.

Las fórmulas para las transformaciones canónicas pueden obtenerse de modo siguiente. Se ha demostrado al final del § 43 que las ecuaciones de Hamilton pueden deducirse del principio de la mínima acción en la forma
\[
\delta\!\bigl(\sum_i p_i\,dq_i - H\,dt\bigr) = 0,
\tag{45.4}
\]
en la cual la variación se aplica a todas las coordenadas y los ímpetus independientemente. Si las nuevas variables \(P\) y \(Q\) satisfacen también a las ecuaciones de Hamilton, deben verificar igualmente el principio de la mínima acción:
\[
\delta\!\bigl(\sum_i P_i\,dQ_i - H'\,dt\bigr) = 0.
\tag{45.5}
\]

Las dos formas (45.4) y (45.5) son equivalentes solamente si sus integrandos difieren en la diferencial total de una función arbitraria \(F\) de las coordenadas, de los ímpetus y del tiempo; la diferencia entre las dos integrales (diferencia de los valores de \(F\) en los límites de integración) será entonces una constante cuya variación es nula. Consecuentemente se debe tener:
\[
\sum_i p_i\,dq_i - H\,dt
= \sum_i P_i\,dQ_i - H'\,dt + dF.
\tag{45.6}
\]

Toda transformación canónica está caracterizada por su función \(F\), denominada función generatriz de la transformación. Escribiendo la relación anterior en la forma
\[
dF = \sum_i p_i\,dq_i - \sum_i P_i\,dQ_i + (H' - H)\,dt,
\tag{45.6$'$}
\]
se tiene, comparando coeficientes,
\[
p_i = \frac{\partial F}{\partial q_i},
\quad
P_i = -\,\frac{\partial F}{\partial Q_i},
\quad
H' = H + \frac{\partial F}{\partial t}.
\tag{45.7}
\] 
\noindent El argumento de la diferencial del primer miembro, expresada en función de las variables \(q\) y \(P\), es una nueva función generatriz. Denominémosla \(\Phi(q,P,t)\), y se tiene:
\begin{equation}
p_i \;=\;\frac{\partial\Phi}{\partial q_i},\quad
Q_i \;=\;\frac{\partial\Phi}{\partial P_i},\quad
H' \;=\;H + \frac{\partial\Phi}{\partial t}\tag{45.8}
\end{equation}

De modo análogo se pueden obtener las fórmulas para las transformaciones canónicas que encierran funciones generatrices dependientes de las variables \(p\) y \(Q\), o de las \(p\) y \(P\).

La relación entre la nueva y la antigua hamiltoniana es siempre de la misma forma; la diferencia \(H'-H\) es la derivada parcial de la función generatriz respecto al tiempo. En particular, si la función generatriz es independiente del tiempo, entonces \(H'=H\); es decir, en este caso la nueva hamiltoniana se obtiene sustituyendo en \(H\) las magnitudes \(p\) y \(q\) por sus valores en función de las nuevas variables \(P\) y \(Q\).

El amplio alcance de las transformaciones canónicas en el método de Hamilton se mantiene incluso cuando parte de su significado inicial. Puesto que las transformaciones (45.2) relacionan cada una de las magnitudes \(P_i, Q_i\) tanto a las coordenadas \(q\) como a los antiguos \(p\), no se pueden considerar las variables \(Q\) como coordenadas estrictamente espaciales, y la diferencia entre los dos grupos de variables \(p\) y \(Q\) se convierte esencialmente en cuestión de nomenclatura. Esto se ve muy claramente en la transformación
\[
Q_1 = p_1,\quad P_1 = -q_1,
\]
que evidentemente no cambia la forma canónica de las ecuaciones, y se reduce simplemente al intercambio de los nombres de las coordenadas y de los ímpetus.

Teniendo en cuenta esta arbitrariedad de terminología, a las variables \(p\) y \(q\) en el método de Hamilton se les denomina habitualmente magnitudes canónicamente conjugadas. Las condiciones para que las variables sean canónicamente conjugadas pueden expresarse con ayuda de los paréntesis de Poisson. Demostremos primero un teorema general sobre la invarianza de los paréntesis de Poisson con respecto a las transformaciones canónicas.

Sea \([f,g]_{p,q}\) el paréntesis de Poisson de las magnitudes \(f\) y \(g\), en el cual la diferenciación se hace respecto a las variables \(p\) y \(q\), y \([f,g]_{P,Q}\) el paréntesis de Poisson de las mismas magnitudes diferenciadas con respecto a las variables canónicas \(P\) y \(Q\). Entonces
\begin{equation}
[f,g]_{p,q} \;=\;[f,g]_{P,Q} \tag{45.9}
\end{equation}
Esta relación puede demostrarse por cálculo directo, empleando las fórmulas de las transformaciones canónicas; sin embargo, pueden evitarse estos cálculos mediante el razonamiento siguiente:

Empecemos por observar que en las transformaciones canónicas (45.7) y (45.8), el tiempo aparece como un parámetro; por lo tanto, es suficiente demostrar el teorema (45.9) para magnitudes que no dependan explícitamente del tiempo. Consideremos ahora, de modo puramente formal, a la magnitud \(H\) como la hamiltoniana de un sistema ficticio; entonces, de la fórmula (42.1)
\[
\dot f \;=\;[f,H].
\]
La derivada \(\dot f/dt\) sólo puede depender de las propiedades del movimiento del sistema ficticio, y de la elección particular de las variables. De aquí que el paréntesis de Poisson \([\,,\,]\) sea inalterable por el paso de un conjunto de variables canónicas a otro.

De las fórmulas (42.13) y del teorema (45.9) se tiene:
\begin{equation}
\{Q_i,Q_j\}_{p,q} = 0,\quad
\{P_i,P_j\}_{p,q} = 0,\quad
\{P_i,Q_j\}_{p,q} = \delta_{ij}
\tag{45.10}
\end{equation}
Estas son las condiciones, expresadas en función de los paréntesis de Poisson, que deben satisfacer las nuevas variables para que la transformación \(q,p\to Q,P\) sea canónica.

Es interesante observar que la variación de las magnitudes \(p,q\) durante el movimiento del sistema puede ser considerada por sí misma como una serie de transformaciones canónicas. El significado de esta afirmación es el siguiente: sean \(q_i,p_i\) dos variables canónicas conjugadas en el instante \(t\); sean también \(q_i',p_i'\) sus valores en el instante \(t+\tau\). Estas últimas son funciones de las primeras (y del valor \(t\) del intervalo como parámetro):
\[
q_i' = q_i(q,p,t),\quad
p_i' = p_i(q,p,t).
\]
Si se consideran estas fórmulas como una transformación de las variables \(q_i,p_i\to q_i',p_i'\), entonces esta transformación será canónica; esto es evidente si se considera la expresión de la diferencia de la acción \(S\):
\[
\Delta S = \sum_i p_i\,dq_i - \sum_i p_i'\,dq_i',
\]
calculada a lo largo de la trayectoria real que pasa por los puntos \((q,p)\) en los instantes dados \(t\) y \(t+\tau\); cf. (43.7). La comparación de esta fórmula con (45.6) prueba que \(-S\) es la función generatriz de la transformación.

\section*{\S\,46. Teorema de Liouville}
\section*{\S\,46. Teorema de Liouville}

\noindent Esta integral tiene la propiedad de ser invariante con respecto a las transformaciones canónicas, es decir, si las variables \(p,q\) se transforman canónicamente en las variables \(P,Q\), los volúmenes de las regiones correspondientes de los espacios \(p,q\) y \(P,Q\) serán los mismos:
\begin{equation}
\int\cdots\int dq_1\ldots dq_n\,dp_1\ldots dp_n
=
\int\cdots\int dQ_1\ldots dQ_n\,dP_1\ldots dP_n
\tag{46.1}
\end{equation}

\noindent Como es sabido, la transformación de variables en una integral múltiple se realiza por la fórmula
\[
\int\cdots\int f(q_1,\dots,q_n,p_1,\dots,p_n)\,dq_1\ldots dq_n\,dp_1\ldots dp_n
=
\int\cdots\int D\,dq_1\ldots dq_n\,dp_1\ldots dp_n,
\]
en la que
\begin{equation}
D = \frac{\partial(Q_1,\dots,Q_n,P_1,\dots,P_n)}{\partial(q_1,\dots,q_n,p_1,\dots,p_n)}
\tag{46.2}
\end{equation}
es el jacobiano de la transformación. Consecuentemente, la demostración del teorema (46.1) queda reducida a probar que el jacobiano de toda transformación canónica es igual a la unidad:
\begin{equation}
D = 1
\tag{46.3}
\end{equation}

\noindent Haremos uso de la conocida propiedad de los jacobianos por la cual pueden ser tratados como si fuesen fracciones. Dividiendo numerador y denominador por
\(\partial(q_1,\dots,q_n,P_1,\dots,P_n)\), se obtiene:
\begin{equation}
D
=
\frac{\partial(Q_1,\dots,Q_n,P_1,\dots,P_n)}
     {\partial(q_1,\dots,q_n,P_1,\dots,P_n)}
\;
\frac{\partial(q_1,\dots,q_n,P_1,\dots,P_n)}
     {\partial(q_1,\dots,q_n,p_1,\dots,p_n)}
\tag{46.4}
\end{equation}

\noindent Otra propiedad de los jacobianos es que cuando las mismas magnitudes aparecen en el numerador y denominador, el jacobiano se reduce a otro con menor número de variables, y en el cual las magnitudes repetidas se consideran como constantes y salen fuera de los símbolos de derivación. De aquí que
\[
D
=
\left[\frac{\partial(Q_1,\dots,Q_n)}{\partial(q_1,\dots,q_n)}\right]_{P=\mathrm{cte}}
\;
\left[\frac{\partial(P_1,\dots,P_n)}{\partial(P_1,\dots,P_n)}\right]_{q=\mathrm{cte}}.
\]

\noindent El jacobiano del numerador es, por definición, un determinante de orden \(n\) cuyo elemento de la fila \(i\) y columna \(k\) es \(\partial Q_i/\partial q_k\). Representando la transformación canónica por la función generatriz \(\Phi(q,P,t)\) en la forma (45.8), se obtiene:
\[
\frac{\partial Q_i}{\partial q_k}
=
\frac{\partial^2\Phi}{\partial P_i\,\partial q_k}
=
\frac{\partial^2\Phi}{\partial q_k\,\partial P_i}
=
\frac{\partial p_k}{\partial P_i}.
\]
Análogamente se encuentra que el elemento de la fila \(i\) y columna \(k\) del determinante del denominador es \(\partial^2\Phi/\partial q_i\,\partial P_k\). Esto quiere decir que ambos determinantes sólo difieren en el intercambio de filas y columnas; por tanto serán iguales, de modo que el cociente (46.4) es igual a la unidad, como queríamos demostrar.

\noindent Supongamos ahora que cada punto en la región considerada del espacio físico se mueve en el curso del tiempo de acuerdo con las ecuaciones de movimiento del sistema mecánico; la región también se moverá como un todo, sin que cambie su volumen:
\begin{equation}
\int d\Gamma = \mathrm{cte}
\tag{46.5}
\end{equation}

\noindent Este resultado, conocido como teorema de Liouville, es consecuencia inmediata de la invariancia del volumen del espacio físico en las transformaciones canónicas, y del hecho de que la variación de \(p\) y \(q\) durante el movimiento puede ser considerada como una transformación canónica (como se ha demostrado al final de \S45).

\noindent De modo análogo puede demostrarse la invariancia de las integrales
\[
\iint\sum_i q_i\,dp_i,\quad
\iiint\sum_i q_i\,dq_i\,dp_i,\dots,
\]
en las que la integración se extiende a variedades bidimensionales, cuadrimensionales, etc., del espacio físico.

\section*{\S\,47. Ecuación de Hamilton–Jacobi}

\noindent Al considerar en \S43 la acción \(S\) como función de las coordenadas y del tiempo se ha demostrado que la derivada parcial respecto al tiempo de esta función \(S(q,t)\) está relacionada con la hamiltoniana por la expresión
\[
\frac{\partial S}{\partial t}
+
H\Bigl(q_1,\dots,q_n;\tfrac{\partial S}{\partial q_1},\dots,\tfrac{\partial S}{\partial q_n};t\Bigr)
=0,
\]
y que sus derivadas parciales con respecto a las coordenadas son los ímpetus. Sustituyendo entonces en la hamiltoniana los ímpetus \(p\) por las derivadas \(\partial S/\partial q\), se obtiene la ecuación:
\begin{equation}
\frac{\partial S}{\partial t}
+
H\Bigl(q_1,\dots,q_n;\tfrac{\partial S}{\partial q_1},\dots,\tfrac{\partial S}{\partial q_n};t\Bigr)
=0
\tag{47.1}
\end{equation}
\section*{\S\,47. Ecuación de Hamilton–Jacobi}

\noindent En un sistema con \(s\) grados de libertad, una integral completa de la ecuación de Hamilton–Jacobi debe contener \(s+1\) constantes arbitrarias. Como la función \(S\) sólo interviene en la ecuación por sus derivadas, una de dichas constantes será aditiva. Así, una integral completa tiene la forma
\begin{equation}
S = f\bigl(t,q_1,\dots,q_s;\alpha_1,\dots,\alpha_s\bigr)\,+\,A
\tag{47.2}
\end{equation}
siendo \(\alpha_1,\dots,\alpha_s\) y \(A\) constantes arbitrarias.

\noindent Para relacionar esta integral con la solución de las ecuaciones del movimiento, efectuemos una transformación canónica de las variables \(p,q\) a las nuevas variables \(\alpha,\beta\), tomando como función generatriz \(f(t,q;\alpha)\). Las nuevas coordenadas las llamamos \(\alpha_i\) y los nuevos ímpetus \(\beta_i\). Según las fórmulas (45.8),
\[
p_i = \frac{\partial f}{\partial q_i},\quad
\beta_i = -\frac{\partial f}{\partial \alpha_i},\quad
H' = H + \frac{\partial f}{\partial t}.
\]
Como \(f\) satisface la ecuación de Hamilton–Jacobi, \(\partial f/\partial t + H = 0\), resulta
\[
H' = H + \frac{\partial f}{\partial t} = 0.
\]
Por tanto las ecuaciones canónicas en las nuevas variables son
\(\dot\alpha_i=0\), \(\dot\beta_i=0\), de modo que
\begin{equation}
\alpha_i = \text{cte.},\quad \beta_i = \text{cte.}
\tag{47.3}
\end{equation}

\noindent Además, las \(s\) ecuaciones
\begin{equation}
\frac{\partial S}{\partial \alpha_i} = \beta_i
\tag{47.4}
\end{equation}
permiten expresar las coordenadas \(q_i\) como funciones de \(t\), \(\alpha\) y \(\beta\). A su vez,
\[
p_i = \frac{\partial S}{\partial q_i}
\]
da los ímpetus en función del tiempo y de las constantes.

\noindent Si se dispone sólo de una integral incompleta dependiente de menos de \(s\) constantes, no podrá obtenerse la solución general, pero sí simplificarse el problema. Por ejemplo, si \(S\) contiene una constante arbitraria \(a\), la relación
\[
\frac{\partial S}{\partial a} = \text{cte}
\]
genera una ecuación entre \(q_1,\dots,q_s\) y \(t\).

\noindent En el caso conservativo (\(H\) no depende explícitamente de \(t\)), la acción se escribe
\begin{equation}
S = S_0(q_1,\dots,q_s)\;-\;E\,t
\tag{47.5}
\end{equation}
y la ecuación (47.1) pasa a
\begin{equation}
H\bigl(q_1,\dots,q_s;\,\tfrac{\partial S_0}{\partial q_1},\dots,\tfrac{\partial S_0}{\partial q_s}\bigr)
=E.
\tag{47.6}
\end{equation}
\section*{\S\,48. Separación de variables}

\noindent En este caso se busca una solución en la forma de una suma
\begin{equation}
S = S'(q_{1},t) \;+\; S_{1}(q_{1})
\tag{48.2}
\end{equation}
que sustituida en la ecuación (48.1) da
\begin{equation}
\phi\!\bigl(q_{1},t,\tfrac{\partial S'}{\partial q_{1}},
\tfrac{\partial S'}{\partial t};\,\tfrac{dS_{1}}{dq_{1}}\bigr)
=0.
\tag{48.3}
\end{equation}
Supongamos que ha sido hallada la solución (48.2). Al sustituirla en (48.3), ésta se convierte en una identidad para todo \(q_{1}\). Para que así sea, deben cumplirse las dos ecuaciones
\begin{align}
\phi\!\bigl(q_{1},\tfrac{dS_{1}}{dq_{1}}\bigr)&=\alpha_{1},
\tag{48.4}\\
\phi\!\bigl(q_{1},t,\tfrac{\partial S'}{\partial q_{1}},
\tfrac{\partial S'}{\partial t};\alpha_{1}\bigr)&=0,
\tag{48.5}
\end{align}
donde \(\alpha_{1}\) es constante arbitraria. La primera de ellas permite obtener \(S_{1}(q_{1})\) por simple integración; la segunda es una EDP en menos variables.

Si así se separan sucesivamente las \(s\) coordenadas y el tiempo, la integral completa de la ecuación de Hamilton–Jacobi se reduce a cuadraturas. Para un sistema conservativo, basta separar las \(s\) coordenadas en la ecuación (47.6) y se obtiene
\begin{equation}
S \;=\;\sum_{k=1}^{s}S_{k}\bigl(q_{k};\alpha_{1},\dots,\alpha_{s}\bigr)\;-\;E\,t
\tag{48.6}
\end{equation}
donde cada \(S_{k}\) depende sólo de \(q_{k}\), y \(E\) se determina al sustituir \(S\) en (47.6).

Un caso particular es la separación de una variable cíclica \(q_{j}\). Entonces  
\(\phi(q_{j},\partial S/\partial q_{j})=\partial S/\partial q_{j}\)  
y de (48.4) se obtiene  
\begin{equation}
S_{j}=\alpha_{j}\,q_{j},
\tag{48.7}
\end{equation}
siendo \(\alpha_{j}=p_{j}\) constante. Obsérvese que el término \(-E\,t\) en (48.6) equivale a la separación de la variable cíclica \(t\).

El método de separación engloba así tanto el uso de variables cíclicas como los casos en que, aun sin serlo, pueden separarse las coordenadas. A continuación, un ejemplo clásico:

\paragraph{1. Coordenadas esféricas.}
En \((r,\theta,\varphi)\) la hamiltoniana es
\[
H=\frac{1}{2m}\Bigl(p_{r}^{2}+\frac{p_{\theta}^{2}}{r^{2}}
+\frac{p_{\varphi}^{2}}{r^{2}\sin^{2}\theta}\Bigr)
+U(r,\theta,\varphi),
\]
y puede separarse si
\[
U(r,\theta,\varphi)=a(r)+\frac{b(\theta)}{r^{2}}
+\frac{c(\varphi)}{r^{2}\sin^{2}\theta}.
\]
Físicamente basta tomar
\begin{equation}
U(r,\theta)=a(r)+\frac{b(\theta)}{r^{2}}\,.
\tag{48.8}
\end{equation}
La ecuación de Hamilton–Jacobi para \(S_{0}(r,\theta,\varphi)\) es
\[
\frac{1}{2m}\Bigl(\frac{\partial S_{0}}{\partial r}\Bigr)^{2}
+a(r)
+\frac{1}{2m\,r^{2}}\Bigl(\frac{\partial S_{0}}{\partial\theta}\Bigr)^{2}
+\frac{1}{2m\,r^{2}\sin^{2}\theta}\Bigl(\frac{\partial S_{0}}{\partial\varphi}\Bigr)^{2}
=E.
\]
Como \(\varphi\) es cíclica, buscamos
\[
S_{0}=p_{\varphi}\,\varphi+S_{\theta}(\theta)+S_{r}(r),
\]
y obtenemos las ecuaciones separadas
\[
\Bigl(\frac{dS_{r}}{dr}\Bigr)^{2}
=2m\bigl[E-a(r)\bigr]-\beta,
\quad
\Bigl(\frac{dS_{\theta}}{d\theta}\Bigr)^{2}
=\beta-2m\,b(\theta)-\frac{p_{\varphi}^{2}}{\sin^{2}\theta},
\]
donde \(\beta\) es constante de separación. Su integración da finalmente
\begin{equation}
S=-E\,t
+ p_{\varphi}\,\varphi
+\int\sqrt{2m\bigl[E-a(r)\bigr]-\beta}\,dr
+\int\sqrt{\beta-2m\,b(\theta)-\frac{p_{\varphi}^{2}}{\sin^{2}\theta}}\,
d\theta.
\tag{48.9}
\end{equation}
Aquí las constantes libres son \(p_{\varphi}\) y \(\beta\); derivando respecto a ellas y fijando el resultado como nuevas constantes, se obtiene la solución general del movimiento.
\subsection*{2. Coordenadas parabólicas}

Se pasa de las coordenadas cilíndricas \((\rho,\varphi,z)\) a las parabólicas \(\xi,\eta,\varphi\) mediante las fórmulas:
\begin{equation}
z = \xi(\xi - \eta),\quad
\rho = \sqrt{\xi\eta}
\tag{48.10}
\end{equation}
Las coordenadas \(\xi\) y \(\eta\) pueden tomar valores de \(0\) a \(\infty\); las superficies \(\xi=\text{cte}\) y \(\eta=\text{cte}\) son dos familias de paraboloides de revolución (eje de simetría \(z\)). De (48.10) sigue
\begin{equation}
r=\sqrt{z^2+\rho^2}=\xi+\eta,
\tag{48.11}
\end{equation}
de donde
\begin{equation}
\xi = r+z,\quad
\eta = r - z.
\tag{48.12}
\end{equation}

Derivando (48.10) respecto al tiempo y sustituyendo en
\[
L = \tfrac12 m\bigl(\dot\rho^2 + \rho^2\dot\varphi^2 + \dot z^2\bigr)-U(\rho,\varphi,z),
\]
se obtiene
\begin{equation}
L = \tfrac12 m\bigl((\xi+\eta)(\dot\xi^2+\dot\eta^2)+\xi\eta\,\dot\varphi^2\bigr)-U(\xi,\eta,\varphi).
\tag{48.13}
\end{equation}
Los ímpetus son
\[
p_\xi = m(\xi+\eta)\dot\xi,\quad
p_\eta = m(\xi+\eta)\dot\eta,\quad
p_\varphi = m\,\xi\eta\,\dot\varphi,
\]
y la hamiltoniana resulta
\begin{equation}
H = \frac{p_\xi^2+p_\eta^2}{2m(\xi+\eta)}
+\frac{p_\varphi^2}{2m\,\xi\eta}
+U(\xi,\eta,\varphi).
\tag{48.14}
\end{equation}

Los casos de interés para separación de variables aparecen con un potencial de la forma
\begin{equation}
U(\xi,\eta)=\frac{a(\xi)+b(\eta)}{\xi+\eta}.
\tag{48.15}
\end{equation}
La ecuación de Hamilton–Jacobi
\[
\frac{1}{2m}\Bigl(\frac{(\partial S/\partial\xi)^2+(\partial S/\partial\eta)^2}{\xi+\eta}
+\frac{1}{\xi\eta}\bigl(\partial S/\partial\varphi\bigr)^2\Bigr)
+\frac{a(\xi)+b(\eta)}{\xi+\eta}
=E
\]
con \(\partial S/\partial\varphi=p_\varphi\) cíclica. Multiplicando por \(2m(\xi+\eta)\) y reagrupando:
\[
2\xi\Bigl(\frac{\partial S}{\partial\xi}\Bigr)^2 + m\,a(\xi)-mE\,\xi + \frac{p_\varphi^2}{\xi}
\;+\;
2\eta\Bigl(\frac{\partial S}{\partial\eta}\Bigr)^2 + m\,b(\eta)-mE\,\eta + \frac{p_\varphi^2}{\eta}
=0.
\]
Al separar las variables quedan dos ecuaciones
\[
2\xi\Bigl(\frac{dS_\xi}{d\xi}\Bigr)^2 + m\,a(\xi)-mE\,\xi + \frac{p_\varphi^2}{\xi} = \beta,
\quad
2\eta\Bigl(\frac{dS_\eta}{d\eta}\Bigr)^2 + m\,b(\eta)-mE\,\eta + \frac{p_\varphi^2}{\eta} = -\beta.
\]
Su integración da finalmente
\begin{equation}
S = -E\,t + p_\varphi\,\varphi
+\int \sqrt{\frac{\beta - m\,a(\xi) + mE\,\xi - \tfrac{p_\varphi^2}{\xi}}{2\,\xi}}\;d\xi
+\int \sqrt{\frac{-\beta - m\,b(\eta) + mE\,\eta - \tfrac{p_\varphi^2}{\eta}}{2\,\eta}}\;d\eta.
\tag{48.16}
\end{equation}
\noindent La hamiltoniana es
\begin{equation}
H = \frac{1}{2m\sigma^{2}(\xi^{2}-\eta^{2})}
\Bigl[
(\xi^{2}-1)\,p_{\xi}^{2} + (1-\eta^{2})\,p_{\eta}^{2}
+ \Bigl(\frac{1}{\xi^{2}-1} + \frac{1}{1-\eta^{2}}\Bigr)\,p_{\varphi}^{2}
\Bigr]
+ U(\xi,\eta,\varphi)
\tag{48.20}
\end{equation}

\noindent Los casos físicamente interesantes de separación de variables corresponden a un potencial de la forma
\begin{equation}
U(\xi,\eta)
= \frac{a(\xi)+b(\eta)}{\xi^{2}-\eta^{2}}
= \frac{\sigma^{2}}{\xi^{2}-\eta^{2}}
\Bigl[
a\!\bigl(\tfrac{\xi+\eta}{2\sigma}\bigr)
+ b\!\bigl(\tfrac{\xi-\eta}{2\sigma}\bigr)
\Bigr]
\tag{48.21}
\end{equation}

\noindent De este modo, la separación de variables en la ecuación de Hamilton–Jacobi conduce a
\begin{equation}
S = -E\,t + p_{\varphi}\,\varphi
+ \int 
\sqrt{
\frac{2m\sigma^{2}E + \beta - 2m\sigma^{2}a(\xi)}{\xi^{2}-1}
- \frac{p_{\varphi}^{2}}{(\xi^{2}-1)^{2}}
}
\,d\xi
+ \int 
\sqrt{
\frac{2m\sigma^{2}E - \beta + 2m\sigma^{2}b(\eta)}{1-\eta^{2}}
- \frac{p_{\varphi}^{2}}{(1-\eta^{2})^{2}}
}
\,d\eta.
\tag{48.22}
\end{equation}

\paragraph*{Problemas}

1. Hallar una integral completa de la ecuación de Hamilton–Jacobi para el movimiento de una partícula en un campo  
\[
U = -\frac{\alpha}{r}\;-\;F\,z
\]
(superposición de un campo coulombiano y de un campo uniforme).\noindent La hamiltoniana es
\begin{equation}
H = \frac{1}{2m\sigma^{2}(\xi^{2}-\eta^{2})}
\Bigl[
(\xi^{2}-1)\,p_{\xi}^{2} + (1-\eta^{2})\,p_{\eta}^{2}
+ \Bigl(\frac{1}{\xi^{2}-1} + \frac{1}{1-\eta^{2}}\Bigr)\,p_{\varphi}^{2}
\Bigr]
+ U(\xi,\eta,\varphi)
\tag{48.20}
\end{equation}

\noindent Los casos físicamente interesantes de separación de variables corresponden a un potencial de la forma
\begin{equation}
U(\xi,\eta)
= \frac{a(\xi)+b(\eta)}{\xi^{2}-\eta^{2}}
= \frac{\sigma^{2}}{\xi^{2}-\eta^{2}}
\Bigl[
a\!\bigl(\tfrac{\xi+\eta}{2\sigma}\bigr)
+ b\!\bigl(\tfrac{\xi-\eta}{2\sigma}\bigr)
\Bigr]
\tag{48.21}
\end{equation}

\noindent De este modo, la separación de variables en la ecuación de Hamilton–Jacobi conduce a
\begin{equation}
S = -E\,t + p_{\varphi}\,\varphi
+ \int 
\sqrt{
\frac{2m\sigma^{2}E + \beta - 2m\sigma^{2}a(\xi)}{\xi^{2}-1}
- \frac{p_{\varphi}^{2}}{(\xi^{2}-1)^{2}}
}
\,d\xi
+ \int 
\sqrt{
\frac{2m\sigma^{2}E - \beta + 2m\sigma^{2}b(\eta)}{1-\eta^{2}}
- \frac{p_{\varphi}^{2}}{(1-\eta^{2})^{2}}
}
\,d\eta.
\tag{48.22}
\end{equation}

\section*{\S\,49. Invariantes adiabáticos}

\noindent Sea \(H(p,q;\lambda)\) la hamiltoniana del sistema que depende del parámetro \(\lambda\). De acuerdo con (40.5), la derivada total de la energía con respecto al tiempo es
\[
\frac{dE}{dt}
=
\frac{\partial H}{\partial t}
=
\frac{\partial H}{\partial \lambda}\,\frac{d\lambda}{dt}.
\]
Tomemos el valor medio de esta ecuación durante un período del movimiento; dado que \(\lambda\) (y por consiguiente \(\dot\lambda\)) varía lentamente, no es necesario promediar \(\dot\lambda\):
\[
\frac{d\overline E}{dt}
=
\frac{d\lambda}{dt}\Bigl\langle\frac{\partial H}{\partial \lambda}\Bigr\rangle.
\]
En la función \(\partial H/\partial\lambda\) que se promedia podemos considerar como variables únicamente \(p\) y \(q\), es decir, como si el movimiento tuviese lugar con \(\lambda\) constante.

Puesto que \(\dot q=\partial H/\partial p\), se tiene
\[
dt=\frac{dq}{\dot q}
=\frac{dq}{\partial H/\partial p},
\]
y el período puede expresarse como
\[
T=\int_{0}^{T}dt
=\oint\frac{dq}{\partial H/\partial p}.
\]
Por tanto,
\begin{equation}
\frac{d\overline E}{dt}
=
\frac{d\lambda}{dt}\,
\frac{\displaystyle\oint\frac{\partial H}{\partial\lambda}\,dq}
     {\displaystyle\oint\frac{\partial H}{\partial p}\,dq}
\tag{49.2}
\end{equation}

\noindent A lo largo de la trayectoria, \(H=E\) es constante, y \(p=p(q;E,\lambda)\). Derivando \(H(p,q;\lambda)=E\) respecto de \(\lambda\) se obtiene
\[
\frac{\partial H}{\partial \lambda}
+
\frac{\partial H}{\partial p}\,\frac{\partial p}{\partial \lambda}
=0.
\]
Sustituyendo en el numerador de (49.2) y escribiendo el denominador como \(\oint(\partial p/\partial E)\,dq\), resulta
\[
\frac{d\overline E}{dt}
=
\frac{d\lambda}{dt}\,
\frac{-\displaystyle\oint\frac{\partial H}{\partial p}\,
                  \frac{\partial p}{\partial \lambda}\,dq}
     {\displaystyle\oint\frac{\partial p}{\partial E}\,dq},
\]
o equivalentemente
\[
\oint\Bigl(\frac{\partial p}{\partial E}\,\frac{d\overline E}{dt}
          -\frac{\partial p}{\partial \lambda}\,\frac{d\lambda}{dt}\Bigr)\,dq
=0.
\]
Finalmente, esta igualdad puede escribirse
\begin{equation}
\frac{dI}{dt}=0,
\tag{49.4}
\end{equation}
siendo
\begin{equation}
I=\frac{1}{2\pi}\oint p\,dq.
\tag{49.5}
\end{equation}
La magnitud \(I\) permanece constante cuando varía lentamente \(\lambda\), por lo que se denomina invariante adiabático. Además,
\begin{equation}
2\pi\frac{\partial I}{\partial E}
=\oint\frac{\partial p}{\partial E}\,dq
=T.
\tag{49.6}
\end{equation}
% Transcripción a LaTeX de §49 y §50 con numeración original

\section*{Invariante adiabático para un oscilador lineal}

Definimos
\[
I = \frac{1}{2\pi}\oint p\,dq.
\]
Como ejemplo, determinemos el invariante adiabático para un oscilador lineal. Su hamiltoniana es
\[
H = \frac{p^2}{2m} + \frac{1}{2}\,m\omega^2 q^2.
\]
Siendo \(\omega\) la frecuencia propia del oscilador. La ecuación de la trayectoria física está dada por la ley de conservación de la energía \(H(p,q)=E\); es una elipse de semiejes \(\sqrt{2mE}\) y \(\sqrt{2E/(m\omega^2)}\), y su área, dividida por \(2\pi\), es
\[
I = \frac{E}{\omega}
\tag{49.7}
\]
La invariancia adiabática de esta magnitud significa que, cuando los parámetros del oscilador varían lentamente, la energía es proporcional a la frecuencia. Las ecuaciones del movimiento de un sistema cerrado con parámetros constantes pueden volver a formularse en función de \(I\). Efectuemos una transformación canónica de las variables \(p,q\), tomando \(I\) como nueva “impulsa”; la función generatriz es la acción abreviada \(S_0\), expresada como función de \(q\) e \(I\). Puesto que \(S_0\) está definida para una energía dada del sistema, y en un sistema cerrado \(I\) sólo es función de la energía, \(S_0\) puede escribirse también como una función \(S(q,I)\). La derivada parcial \(\bigl(\partial S_0/\partial q\bigr)_E\) coincide con la derivada \(\bigl(\partial S/\partial q\bigr)_I\). En consecuencia,
\[
p = \frac{\partial S_0(q,E)}{\partial q}
  = \frac{\partial S(q,I)}{\partial q}
\tag{49.8}
\]
lo que corresponde a la primera de las fórmulas para una transformación canónica. La segunda nos da la nueva “coordenada” que designamos por \(\omega\):
\[
\omega = \frac{\partial S_0(q,E)}{\partial E}
        = \frac{\partial S(q,I)}{\partial I}
\tag{49.9}
\]
Las variables \(I\) y \(\omega\) se llaman variables canónicas: \(I\) es la variable acción y \(\omega\) la variable angular.

Puesto que la función generatriz \(S(q,I)\) no depende explícitamente del tiempo, la nueva hamiltoniana \(H'\) coincide con la antigua \(H\) expresada en las nuevas variables. En otras palabras, \(H'\) representa la energía \(E(I)\) en función de la acción. Así, las ecuaciones de Hamilton son
\[
\dot I = 0,\quad
\dot \omega = \frac{dE}{dI}
\tag{49.10}
\]
La primera da \(I=\mathrm{cte.}\). De la segunda se deduce que la variable angular es lineal en \(t\):
\[
\omega = \frac{dE}{dI}\,t + \mathrm{cte.}
\tag{49.11}
\]
La acción \(S(q,I,t)\) es multiforme en las coordenadas. Tras cada período aumenta en
\[
\Delta S_0 = 2\pi I
\tag{49.12}
\]
y, al mismo tiempo, la variable angular se incrementa en
\[
\Delta\omega
= \omega\,\Delta\bigl(\partial S_0/\partial I\bigr)
= \Delta\bigl(\Delta S_0/\Delta I\bigr)
= 2\pi
\tag{49.13}
\]
Cualquier función uniforme \(F(p,q)\) expresada en variables canónicas es periódica en \(\omega\) de período \(2\pi\).

\section*{§50 Propiedades generales del movimiento en el espacio}

Consideremos un sistema con varios grados de libertad y movimiento finito en cada coordenada. Si el problema admite separación completa por Hamilton–Jacobi, la acción reducida se descompone:
\[
S_0 = \sum_i S_i(q_i)
\tag{50.1}
\]
cada término depende de una sola coordenada. Los ímpetus generalizados son
\[
p_i = \frac{\partial S_0}{\partial q_i}
    = \frac{dS_i}{dq_i}.
\]
% Continuación de §50

Estas funciones no son uniformes. Como el movimiento del sistema es finito, cada coordenada sólo puede describir un intervalo finito, y cuando las \(q_i\) varían “ida y vuelta” en ese intervalo, la acción aumenta en
\[
\Delta S_i = \Delta S_{0,i} = 2\pi I_i
\tag{50.3}
\]
siendo
\[
I_i = \frac{1}{2\pi}\oint p_i\,dq_i
\tag{50.4}
\]
extendida la integral a la mencionada variación de \(q_i\).

Realicemos ahora una transformación canónica análoga a la considerada en §49 para el caso de un solo grado de libertad. Las nuevas variables son las variables de acción \(I_i\) y las variables angulares
\[
w_i = \frac{\partial S(q,I)}{\partial I_i}
\tag{50.5}
\]
donde la función generatriz \(S(q,I)\) es la acción abreviada expresada en función de las \(q_i\) y los \(I_i\). Las ecuaciones del movimiento en estas variables dan
\[
\dot I_i = 0,\quad
\dot w_i = \frac{\partial E(I)}{\partial I_i}
\]
luego,
\[
I_i = \text{cte.}
\tag{50.6}
\]
\[
w_i = \frac{\partial E(I)}{\partial I_i}\,t + \text{cte.}
\tag{50.7}
\]
También encontramos, análogamente a (49.13), que una variación completa (“ida y vuelta”) de la coordenada \(q_i\) corresponde a un cambio de \(2\pi\) en \(w_i\):
\[
\Delta w_i = 2\pi
\tag{50.8}
\]

De aquí se deduce que cualquier función uniforme \(F(p,q)\) del estado del sistema, cuando se expresa en variables canónicas, es una función periódica de las variables angulares de período \(2\pi\) respecto de cada una de ellas. Puede, por lo tanto, ser desarrollada en serie múltiple de Fourier
\[
F = \sum_{l_1=-\infty}^{\infty}\cdots\sum_{l_n=-\infty}^{\infty}
    A_{l_1,\dots,l_n}(I)\,\exp\bigl[i(l_1w_1+\dots+l_nw_n)\bigr]
\tag{50.9}
\]
(donde \(l_1,\dots,l_n\) son números enteros). Sustituyendo las variables angulares por sus expresiones en función del tiempo se obtiene
\[
F = \sum_{l_1,\dots,l_n}
    A_{l_1,\dots,l_n}(I)\,
    \exp\!\Bigl[i\bigl(l_1\dot w_1+\dots+l_n\dot w_n\bigr)t\Bigr]
  = \sum_{l_1,\dots,l_n}
    A_{l_1,\dots,l_n}(I)\,
    \exp\!\Bigl[i(l_1\partial E/\partial I_1+\dots+l_n\partial E/\partial I_n)\,t\Bigr]
\tag{50.10}
\]
Cada término de esta suma es una función periódica del tiempo de frecuencia
\[
\omega_{l} = \bigl|l_1\partial E/\partial I_1+\dots+l_n\partial E/\partial I_n\bigr|
\tag{50.11}
\]
Como en general estas frecuencias no son conmensurables, la suma no es periódica estricta, ni lo son las coordenadas \(q\) ni los ímpetus \(p\). El sistema es en general casi periódico (condicionalmente periódico).

En casos particulares, si dos (o más) de las frecuencias fundamentales
\(\omega_i=\partial E/\partial I_i\)
son conmensurables para valores arbitrarios de \(I\), hablamos de degeneración. Si las \(n\) frecuencias son conmensurables el movimiento es completamente degenerado y las trayectorias son cerradas. La existencia de degeneración implica una reducción del número de magnitudes independientes \(I_i\). Por ejemplo, si
\[
m_i\,\frac{\partial E}{\partial I_i}
  = m_j\,\frac{\partial E}{\partial I_j},
\tag{50.12}
\]
con \(m_i,m_j\in\mathbb{Z}\), entonces hay sólo \(n-1\) acciones independientes.
% Continuación de §50

Donde \(n_1\) y \(n_2\) son números enteros, se deduce que \(I_1\) e \(I_2\) sólo aparecen en la energía en forma de la suma \(n_1I_1 + n_2I_2\).

Una particularidad muy importante del movimiento degenerado es el aumento del número de integrales uniformes respecto al caso no degenerado (con igual número de grados de libertad). En éste, de las \(n-1\) integrales sólo \(s\) son uniformes —por ejemplo, las magnitudes \(I_\nu\)— y las \(s-1\) restantes pueden escribirse como las diferencias
\[
\omega_{s2}\,\frac{\partial E}{\partial I_1}
- \omega_{s1}\,\frac{\partial E}{\partial I_2}
\tag{50.13}
\]
La constancia de estas cantidades se sigue de (50.7), pero no son uniformes porque las variables angulares no lo son.

En los casos de degeneración, la situación cambia. Según (50.12), aunque la combinación
\[
\omega_{2}l_2 \;=\;\omega_{2}l_1
\tag{50.14}
\]
no sea uniforme, su no uniformidad sólo consiste en sumar un múltiplo arbitrario de \(2\pi\). Basta entonces tomar una función trigonométrica de dicha magnitud para obtener una nueva integral uniforme.

Como ejemplo, en el campo \(U=-e/r\) aparece, además de las dos integrales usuales (momento angular \(M\) y energía \(E\)), una nueva integral uniforme característica de este potencial central.

Si ahora introducimos un parámetro \(\lambda\) y hacemos que la función generatriz \(S(q,I;\lambda)\) dependa de \(\lambda\) (y, si \(\lambda=\lambda(t)\), explícitamente del tiempo), la nueva hamiltoniana
\[
H' \;=\;E(I)\;+\;\dot\lambda\,\Lambda,
\qquad
\Lambda = \frac{\partial S}{\partial \lambda}
\]
no coincide con \(H\). De las ecuaciones de Hamilton resulta
\[
\dot I_i \;=\; -\frac{\partial H'}{\partial w_i}
\;=\; -\frac{\partial\Lambda}{\partial w_i}\,\dot\lambda.
\tag{50.15}
\]
Si tomamos el valor medio sobre muchos períodos fundamentales, \(\langle\partial\Lambda/\partial w_i\rangle=0\) y, por tanto, \(\dot I_i=0\). Esto demuestra la invariancia adiabática de las \(I_i\).

Con esto concluimos las propiedades generales del movimiento casi periódico en sistemas finitos de \(s\) grados de libertad.````% Continuación de §50

Donde \(n_1\) y \(n_2\) son números enteros, se deduce que \(I_1\) e \(I_2\) sólo aparecen en la energía en forma de la suma \(n_1I_1 + n_2I_2\).

Una particularidad muy importante del movimiento degenerado es el aumento del número de integrales uniformes respecto al caso no degenerado (con igual número de grados de libertad). En éste, de las \(n-1\) integrales sólo \(s\) son uniformes —por ejemplo, las magnitudes \(I_\nu\)— y las \(s-1\) restantes pueden escribirse como las diferencias
\[
\omega_{s2}\,\frac{\partial E}{\partial I_1}
- \omega_{s1}\,\frac{\partial E}{\partial I_2}
\tag{50.13}
\]
La constancia de estas cantidades se sigue de (50.7), pero no son uniformes porque las variables angulares no lo son.

En los casos de degeneración, la situación cambia. Según (50.12), aunque la combinación
\[
\omega_{2}l_2 \;=\;\omega_{2}l_1
\tag{50.14}
\]
no sea uniforme, su no uniformidad sólo consiste en sumar un múltiplo arbitrario de \(2\pi\). Basta entonces tomar una función trigonométrica de dicha magnitud para obtener una nueva integral uniforme.

Como ejemplo, en el campo \(U=-e/r\) aparece, además de las dos integrales usuales (momento angular \(M\) y energía \(E\)), una nueva integral uniforme característica de este potencial central.

Si ahora introducimos un parámetro \(\lambda\) y hacemos que la función generatriz \(S(q,I;\lambda)\) dependa de \(\lambda\) (y, si \(\lambda=\lambda(t)\), explícitamente del tiempo), la nueva hamiltoniana
\[
H' \;=\;E(I)\;+\;\dot\lambda\,\Lambda,
\qquad
\Lambda = \frac{\partial S}{\partial \lambda}
\]
no coincide con \(H\). De las ecuaciones de Hamilton resulta
\[
\dot I_i \;=\; -\frac{\partial H'}{\partial w_i}
\;=\; -\frac{\partial\Lambda}{\partial w_i}\,\dot\lambda.
\tag{50.15}
\]
Si tomamos el valor medio sobre muchos períodos fundamentales, \(\langle\partial\Lambda/\partial w_i\rangle=0\) y, por tanto, \(\dot I_i=0\). Esto demuestra la invariancia adiabática de las \(I_i\).

Con esto concluimos las propiedades generales del movimiento casi periódico en sistemas finitos de \(s\) grados de libertad.

\subsection*{Problema}

Calcular las variables acción para un movimiento elíptico en un campo \(U=-\alpha/r\).

Solución. En coordenadas polares \((r,\varphi)\) en el plano del movimiento se tiene

\[
I_{\varphi}
= \frac{1}{2\pi}\oint p_{\varphi}\,d\varphi
= M
\tag{50.16}
\]

\[
I_{r}
= \frac{1}{2\pi}\oint p_{r}\,dr
= \frac{1}{\pi}\int_{r_{\min}}^{r_{\max}}
    \sqrt{2m\Bigl(E+\frac{\alpha}{r}\Bigr)-\frac{M^{2}}{r^{2}}}\;dr
= -M + \alpha\sqrt{\frac{m}{2E}}
\tag{50.17}
\]

De donde resulta que la energía expresada en función de las variables acción es

\[
E = -\frac{m\,\alpha^{2}}{2\,(I_{r}+I_{\varphi})^{2}}
\tag{50.18}
\]

Depende solamente de la suma \(I = I_{r}+I_{\varphi}\), lo que significa que el movimiento es degenerado: las dos frecuencias fundamentales (respecto a \(r\) y \(\varphi\)) coinciden.

Los parámetros \(\rho\) y \(e\) de la órbita (véase (15.4)) se expresan en función de \(I_{r}\) e \(I_{\varphi}\) por

\[
\rho = \frac{I_{\varphi}^{2}}{m\,\alpha},
\quad
e^{2} = 1 - \frac{I_{\varphi}}{I_{r}+I_{\varphi}}
\tag{50.19}
\]

Como consecuencia de la invariancia adiabática de \(I_{r}\) e \(I_{\varphi}\), la excentricidad de la órbita permanece constante cuando \(\alpha\) o \(m\) varían lentamente, y sus dimensiones varían en proporción inversa a \(m\) y \(\alpha\).

\subsection*{Problemas generales}
\begin{enumerate}
  \item Demuestre la identidad de Jacobi (42.14).
  \item Use la identidad de Jacobi y otras propiedades de los corchetes de Poisson para probar que si \(p_y\) y \(l_z\) son constantes, también lo es \(p_x\); y que si \(l_x\) y \(l_y\) son constantes, también lo es \(l_z\).

  \textbf{Solución:}

  \textbf{Parte 1:} Si \(p_y\) y \(l_z\) son constantes, probaremos que \(p_x\) es constante.

  Recordemos que una cantidad es constante del movimiento cuando su corchete de Poisson con el hamiltoniano es cero:
  \[
  \{p_y,H\} = 0 \quad \text{y} \quad \{l_z,H\} = 0
  \]

  Aplicando la identidad de Jacobi a \(l_z\), \(p_y\) y \(H\):
  \[
  \{l_z,\{p_y,H\}\} + \{p_y,\{H,l_z\}\} + \{H,\{l_z,p_y\}\} = 0
  \]

  Los dos primeros términos son cero pues \(p_y\) y \(l_z\) son constantes, entonces:
  \[
  \{H,\{l_z,p_y\}\} = 0
  \]

  Calculemos \(\{l_z,p_y\}\). Sabemos que \(l_z = xp_y - yp_x\), por lo tanto:
  \[
  \{l_z,p_y\} = \{xp_y - yp_x, p_y\} = \{xp_y,p_y\} - \{yp_x,p_y\}
  \]

  El primer término es cero porque \(\{p_y,p_y\} = 0\). Para el segundo:
  \begin{align*}
  \{yp_x,p_y\} &= y\{p_x,p_y\} + p_x\{y,p_y\} \\
  &= y \cdot 0 + p_x \cdot (-1) \\
  &= -p_x
  \end{align*}

  Por lo tanto, \(\{l_z,p_y\} = p_x\), lo que implica:
  \[
  \{H,p_x\} = 0
  \]

  Esto demuestra que \(p_x\) es constante del movimiento.

  \textbf{Parte 2:} Si \(l_x\) y \(l_y\) son constantes, probaremos que \(l_z\) es constante.

  Dado que \(l_x\) y \(l_y\) son constantes:
  \[
  \{l_x,H\} = 0 \quad \text{y} \quad \{l_y,H\} = 0
  \]

  Aplicamos la identidad de Jacobi a \(l_x\), \(l_y\) y \(H\):
  \[
  \{l_x,\{l_y,H\}\} + \{l_y,\{H,l_x\}\} + \{H,\{l_x,l_y\}\} = 0
  \]

  Los dos primeros términos son cero, por lo que:
  \[
  \{H,\{l_x,l_y\}\} = 0
  \]

  De las propiedades del momento angular sabemos que \(\{l_x,l_y\} = l_z\), entonces:
  \[
  \{H,l_z\} = 0
  \]

  Lo que demuestra que \(l_z\) es constante del movimiento.
  
  \item Deduza la fórmula
    \[
      \{f,\;l_i\} \;=\;\hat u_i\times f,
    \]
    donde \(l_i\) (para \(i=x,y,z\)) son las componentes del momento angular y \(f\) una variable dinámica vectorial.

  \textbf{Solución:}
  Consideremos una partícula con coordenadas $(x,y,z)$ y momentos $(p_x,p_y,p_z)$, donde el momento angular está definido por $\mathbf{l} = \mathbf{r} \times \mathbf{p}$. Sus componentes son:

  \begin{align}
  l_x &= yp_z - zp_y\\
  l_y &= zp_x - xp_z\\
  l_z &= xp_y - yp_x
  \end{align}

  Sea $\mathbf{f} = (f_1, f_2, f_3)$ un vector dinámico, cuyas componentes son funciones de las coordenadas y momentos. Calculemos explícitamente el corchete de Poisson $\{f_j, l_i\}$ para el caso $i=z$:

  \begin{align}
  \{f_j, l_z\} &= \{f_j, xp_y - yp_x\}\\
  &= \{f_j, xp_y\} - \{f_j, yp_x\}\\
  &= x\{f_j, p_y\} + p_y\{f_j, x\} - y\{f_j, p_x\} - p_x\{f_j, y\}
  \end{align}

  Usando las propiedades de los corchetes básicos:
  \begin{align}
  \{f_j, p_y\} &= \frac{\partial f_j}{\partial y}\\
  \{f_j, x\} &= -\frac{\partial f_j}{\partial p_x}\\
  \{f_j, p_x\} &= \frac{\partial f_j}{\partial x}\\
  \{f_j, y\} &= -\frac{\partial f_j}{\partial p_y}
  \end{align}

  Sustituyendo:
  \begin{align}
  \{f_j, l_z\} &= x\frac{\partial f_j}{\partial y} - p_y\frac{\partial f_j}{\partial p_x} - y\frac{\partial f_j}{\partial x} + p_x\frac{\partial f_j}{\partial p_y}
  \end{align}

  Para $j=1$ (componente $x$):
  \begin{align}
  \{f_1, l_z\} = -f_2
  \end{align}

  Para $j=2$ (componente $y$):
  \begin{align}
  \{f_2, l_z\} = f_1
  \end{align}

  Para $j=3$ (componente $z$):
  \begin{align}
  \{f_3, l_z\} = 0
  \end{align}

  Por lo tanto, vectorialmente:
  \begin{align}
  \{\mathbf{f}, l_z\} = (f_2, -f_1, 0) = \hat{u}_z \times \mathbf{f}
  \end{align}

  El mismo procedimiento aplicado a los demás componentes del momento angular confirma la relación general:
  \begin{align}
  \{\mathbf{f}, l_x\} &= \hat{u}_x \times \mathbf{f} = (0, f_3, -f_2)\\
  \{\mathbf{f}, l_y\} &= \hat{u}_y \times \mathbf{f} = (-f_3, 0, f_1)
  \end{align}

  Esta fórmula muestra que el corchete de Poisson entre una variable dinámica vectorial y una componente del momento angular corresponde a la rotación infinitesimal de dicho vector alrededor del eje respectivo.

  \item Para un vector constante \(\mathbf b\) y las componentes \(r_i,p_i,l_i\) de posición, momento y momento angular, calcule
    \[
      \{\mathbf l,\;\mathbf r\cdot\mathbf p\},\quad
      \{\mathbf p,\;r^n\},\quad
      \{\mathbf p,\;(\mathbf b\cdot\mathbf r)^2\}.
    \]

  \textbf{Solución:} Calculemos cada corchete de Poisson por separado.

  \paragraph{1. Cálculo de $\{\mathbf l,\;\mathbf r\cdot\mathbf p\}$}

  Para este corchete necesitamos evaluar $\{l_i, \mathbf{r}\cdot\mathbf{p}\}$ para cada componente $i=x,y,z$. Tomemos como ejemplo $\{l_z, \mathbf{r}\cdot\mathbf{p}\}$:

  \begin{align}
  \{l_z, \mathbf{r}\cdot\mathbf{p}\} &= \{xp_y - yp_x, xp_x + yp_y + zp_z\} \\
  &= \{xp_y, \mathbf{r}\cdot\mathbf{p}\} - \{yp_x, \mathbf{r}\cdot\mathbf{p}\}
  \end{align}

  Aplicando la regla del producto para corchetes:
  \begin{align}
  \{xp_y, \mathbf{r}\cdot\mathbf{p}\} &= x\{p_y, \mathbf{r}\cdot\mathbf{p}\} + \{x, \mathbf{r}\cdot\mathbf{p}\}p_y \\
  \{yp_x, \mathbf{r}\cdot\mathbf{p}\} &= y\{p_x, \mathbf{r}\cdot\mathbf{p}\} + \{y, \mathbf{r}\cdot\mathbf{p}\}p_x
  \end{align}

  Calculemos cada término:
  \begin{align}
  \{p_y, \mathbf{r}\cdot\mathbf{p}\} &= \sum_i\left(\frac{\partial p_y}{\partial r_i}\frac{\partial(\mathbf{r}\cdot\mathbf{p})}{\partial p_i} - \frac{\partial p_y}{\partial p_i}\frac{\partial(\mathbf{r}\cdot\mathbf{p})}{\partial r_i}\right) \\
  &= -\sum_i \delta_{yi}p_i = -p_y
  \end{align}

  \begin{align}
  \{x, \mathbf{r}\cdot\mathbf{p}\} &= \sum_i\left(\frac{\partial x}{\partial r_i}\frac{\partial(\mathbf{r}\cdot\mathbf{p})}{\partial p_i} - \frac{\partial x}{\partial p_i}\frac{\partial(\mathbf{r}\cdot\mathbf{p})}{\partial r_i}\right) \\
  &= \sum_i \delta_{xi}r_i = x
  \end{align}

  Por tanto:
  \begin{align}
  \{xp_y, \mathbf{r}\cdot\mathbf{p}\} &= x(-p_y) + xp_y = 0 \\
  \{yp_x, \mathbf{r}\cdot\mathbf{p}\} &= y(-p_x) + yp_x = 0
  \end{align}

  Y finalmente:
  \begin{align}
  \{l_z, \mathbf{r}\cdot\mathbf{p}\} = 0
  \end{align}

  De manera similar se puede demostrar que $\{l_x, \mathbf{r}\cdot\mathbf{p}\} = 0$ y $\{l_y, \mathbf{r}\cdot\mathbf{p}\} = 0$. Por lo tanto:
  \[
  \{\mathbf{l}, \mathbf{r}\cdot\mathbf{p}\} = \mathbf{0}
  \]

  Este resultado refleja el hecho físico de que el momento angular y el generador de dilataciones conmutan.

  \paragraph{2. Cálculo de $\{\mathbf p,\;r^n\}$}

  Donde $r = |\mathbf{r}| = \sqrt{x^2 + y^2 + z^2}$. Para cada componente:
  \begin{align}
  \{p_i, r^n\} &= \sum_j\left(\frac{\partial p_i}{\partial r_j}\frac{\partial r^n}{\partial p_j} - \frac{\partial p_i}{\partial p_j}\frac{\partial r^n}{\partial r_j}\right) \\
  &= -\sum_j \delta_{ij}\frac{\partial r^n}{\partial r_j} \\
  &= -\frac{\partial r^n}{\partial r_i}
  \end{align}

  La derivada de $r^n$ respecto a $r_i$ es:
  \begin{align}
  \frac{\partial r^n}{\partial r_i} &= \frac{\partial}{\partial r_i}(r^2)^{n/2} \\
  &= \frac{n}{2}(r^2)^{n/2-1}\frac{\partial r^2}{\partial r_i} \\
  &= \frac{n}{2}r^{n-2}\cdot 2r_i \\
  &= nr^{n-2}r_i
  \end{align}

  Por lo tanto:
  \[
  \{\mathbf{p}, r^n\} = -n r^{n-2} \mathbf{r}
  \]

  \paragraph{3. Cálculo de $\{\mathbf p,\;(\mathbf b\cdot\mathbf r)^2\}$}

  Para cada componente:
  \begin{align}
  \{p_i, (\mathbf{b}\cdot\mathbf{r})^2\} &= \sum_j\left(\frac{\partial p_i}{\partial r_j}\frac{\partial(\mathbf{b}\cdot\mathbf{r})^2}{\partial p_j} - \frac{\partial p_i}{\partial p_j}\frac{\partial(\mathbf{b}\cdot\mathbf{r})^2}{\partial r_j}\right) \\
  &= -\sum_j \delta_{ij}\frac{\partial(\mathbf{b}\cdot\mathbf{r})^2}{\partial r_j} \\
  &= -\frac{\partial(\mathbf{b}\cdot\mathbf{r})^2}{\partial r_i}
  \end{align}

  Calculando la derivada:
  \begin{align}
  \frac{\partial(\mathbf{b}\cdot\mathbf{r})^2}{\partial r_i} &= \frac{\partial}{\partial r_i}(\mathbf{b}\cdot\mathbf{r})^2 \\
  &= 2(\mathbf{b}\cdot\mathbf{r})\frac{\partial(\mathbf{b}\cdot\mathbf{r})}{\partial r_i} \\
  &= 2(\mathbf{b}\cdot\mathbf{r})b_i
  \end{align}

  Por lo tanto:
  \[
  \{\mathbf{p}, (\mathbf{b}\cdot\mathbf{r})^2\} = -2(\mathbf{b}\cdot\mathbf{r})\mathbf{b}
  \]

  \item Considere el vector de Laplace–Runge–Lenz
    \[
      \mathbf A = \mathbf l\times\mathbf v \;+\;\alpha\,\frac{\mathbf r}{r},
    \]
    constante de movimiento del problema de Kepler. Resuelva el problema 10.26 de Serbo–Kotkin relativo a este vector.

  \textbf{Solución:}
  El problema 10.26 de Serbo-Kotkin pide demostrar que el vector de Laplace-Runge-Lenz $\mathbf{A}$ es una constante de movimiento, calcular los corchetes de Poisson con el momento angular y demostrar que determina la orientación de la órbita en el espacio.

  \paragraph{1. Demostración de que $\mathbf{A}$ es constante de movimiento}

  Para el problema de Kepler con potencial $U(r) = -\alpha/r$, las ecuaciones de movimiento son:
  \begin{align}
  \dot{\mathbf{r}} &= \mathbf{v} \\
  \dot{\mathbf{v}} &= -\frac{\alpha}{r^3}\mathbf{r}
  \end{align}

  Calculamos la derivada temporal de $\mathbf{A}$:
  \begin{align}
  \frac{d\mathbf{A}}{dt} &= \frac{d}{dt}(\mathbf{l} \times \mathbf{v}) + \alpha\frac{d}{dt}\left(\frac{\mathbf{r}}{r}\right) \\
  &= \dot{\mathbf{l}} \times \mathbf{v} + \mathbf{l} \times \dot{\mathbf{v}} + \alpha\frac{d}{dt}\left(\frac{\mathbf{r}}{r}\right)
  \end{align}

  Sabemos que $\dot{\mathbf{l}} = \mathbf{r} \times \dot{\mathbf{v}} = \mathbf{r} \times (-\alpha\mathbf{r}/r^3) = \mathbf{0}$, ya que $\mathbf{r} \times \mathbf{r} = \mathbf{0}$.

  Para el segundo término:
  \begin{align}
  \mathbf{l} \times \dot{\mathbf{v}} &= \mathbf{l} \times \left(-\frac{\alpha}{r^3}\mathbf{r}\right) \\
  &= -\frac{\alpha}{r^3}(\mathbf{l} \times \mathbf{r})
  \end{align}

  Para el tercer término:
  \begin{align}
  \frac{d}{dt}\left(\frac{\mathbf{r}}{r}\right) &= \frac{\dot{\mathbf{r}}}{r} - \frac{\mathbf{r}}{r^2}\dot{r} \\
  &= \frac{\mathbf{v}}{r} - \frac{\mathbf{r}}{r^2}\frac{\mathbf{r}\cdot\mathbf{v}}{r} \\
  &= \frac{1}{r}\left(\mathbf{v} - \frac{\mathbf{r}(\mathbf{r}\cdot\mathbf{v})}{r^2}\right)
  \end{align}

  El término entre paréntesis es la componente de $\mathbf{v}$ perpendicular a $\mathbf{r}$, que podemos expresar como:
  \begin{align}
  \mathbf{v}_{\perp} &= \mathbf{v} - \frac{\mathbf{r}(\mathbf{r}\cdot\mathbf{v})}{r^2} \\
  &= \frac{\mathbf{r}\times(\mathbf{r}\times\mathbf{v})}{r^2} \\
  &= \frac{\mathbf{r}\times\mathbf{l}}{r^2}
  \end{align}

  Por lo tanto:
  \begin{align}
  \frac{d}{dt}\left(\frac{\mathbf{r}}{r}\right) &= \frac{1}{r}\frac{\mathbf{r}\times\mathbf{l}}{r^2} = \frac{\mathbf{r}\times\mathbf{l}}{r^3}
  \end{align}

  Combinando todos los términos:
  \begin{align}
  \frac{d\mathbf{A}}{dt} &= -\frac{\alpha}{r^3}(\mathbf{l} \times \mathbf{r}) + \alpha\frac{\mathbf{r}\times\mathbf{l}}{r^3} \\
  &= -\frac{\alpha}{r^3}(\mathbf{l} \times \mathbf{r}) - \frac{\alpha}{r^3}(\mathbf{l} \times \mathbf{r}) \\
  &= \mathbf{0}
  \end{align}

  Donde hemos utilizado que $\mathbf{r}\times\mathbf{l} = -\mathbf{l}\times\mathbf{r}$.

  \paragraph{2. Cálculo de los corchetes de Poisson}

  Los corchetes de Poisson de $\mathbf{A}$ con el momento angular $\mathbf{l}$ son:
  \begin{align}
  \{A_i, l_j\} = \epsilon_{ijk}A_k
  \end{align}

  donde $\epsilon_{ijk}$ es el símbolo de Levi-Civita.

  Los corchetes de Poisson entre las componentes de $\mathbf{A}$ son:
  \begin{align}
  \{A_i, A_j\} = -2H\epsilon_{ijk}l_k
  \end{align}

  donde $H = \frac{v^2}{2} - \frac{\alpha}{r}$ es la energía total por unidad de masa.

  \paragraph{3. Propiedades geométricas}

  El vector $\mathbf{A}$ apunta desde el foco hacia el pericentro de la órbita, y su magnitud está relacionada con la excentricidad $e$ de la órbita:
  \begin{align}
  |\mathbf{A}| = \alpha e
  \end{align}

  Para verificar esta relación, podemos tomar el producto escalar de $\mathbf{A}$ con $\mathbf{r}$:
  \begin{align}
  \mathbf{A}\cdot\mathbf{r} &= (\mathbf{l}\times\mathbf{v})\cdot\mathbf{r} + \alpha\frac{\mathbf{r}\cdot\mathbf{r}}{r} \\
  &= \mathbf{l}\cdot(\mathbf{v}\times\mathbf{r}) + \alpha r \\
  &= -\mathbf{l}\cdot\mathbf{l} + \alpha r \\
  &= -l^2 + \alpha r
  \end{align}

  Esta relación demuestra que $\mathbf{r}$ traza una cónica con un foco en el origen, ecuación que puede escribirse como:
  \begin{align}
  r = \frac{l^2/\alpha}{1 + (|\mathbf{A}|/\alpha)\cos\theta}
  \end{align}

  donde $\theta$ es el ángulo entre $\mathbf{r}$ y $\mathbf{A}$, lo que confirma que $|\mathbf{A}|/\alpha = e$ es la excentricidad de la órbita.

  Los vectores $\mathbf{A}$ y $\mathbf{l}$ son perpendiculares entre sí, lo que se puede demostrar calculando su producto escalar:
  \begin{align}
  \mathbf{A}\cdot\mathbf{l} &= (\mathbf{l}\times\mathbf{v})\cdot\mathbf{l} + \alpha\frac{\mathbf{r}}{r}\cdot\mathbf{l} \\
  &= \mathbf{l}\cdot(\mathbf{v}\times\mathbf{l}) + \alpha\frac{\mathbf{r}\cdot\mathbf{l}}{r} \\
  &= 0 + \alpha\frac{\mathbf{r}\cdot(\mathbf{r}\times\mathbf{v})}{r} \\
  &= 0
  \end{align}

  Por tanto, $\mathbf{A}$ y $\mathbf{l}$ determinan completamente la orientación de la órbita en el espacio.

  \item Considere una partícula con coordenadas \(\mathbf r=(x,y,z)\) en un sistema de referencia espacial. Halle la ecuación de Hamilton–Jacobi en un sistema de coordenadas rotante alrededor del eje \(z\).

  \textbf{Solución:}  
  En el marco que gira con velocidad angular \(\boldsymbol\Omega=\Omega\,\hat e_z\) la hamiltoniana es
  \[
    H(\mathbf r,\mathbf p)
    = \frac{1}{2m}(p_x^2+p_y^2+p_z^2)
      \;-\;\boldsymbol\Omega\cdot(\mathbf r\times\mathbf p)
      \;+\;U(\mathbf r)
    = \frac{1}{2m}\mathbf p^2 - \Omega\,L_z + U(x,y,z).
  \]
  La ecuación de Hamilton–Jacobi
  \[
    \frac{\partial S}{\partial t}
    + H\!\Bigl(x,y,z;\partial_xS,\partial_yS,\partial_zS\Bigr)
    = 0
  \]
  se expresa como
  \[
    \frac{\partial S}{\partial t}
    + \frac{1}{2m}\Bigl[(\partial_xS)^2+(\partial_yS)^2+(\partial_zS)^2\Bigr]
    - \Omega\bigl(x\,\partial_yS - y\,\partial_xS\bigr)
    + U(x,y,z)
    = 0.
  \]
  \item Halle y resuelva la ecuación de Hamilton–Jacobi para un oscilador armónico tridimensional con frecuencias \(\omega_x,\omega_y,\omega_z\) diferentes.
  
  \textbf{Solución:} 
  El hamiltoniano del sistema es:
  \[
  H = \frac{1}{2m}(p_x^2 + p_y^2 + p_z^2) + \frac{m}{2}(\omega_x^2 x^2 + \omega_y^2 y^2 + \omega_z^2 z^2)
  \]

  La ecuación de Hamilton-Jacobi correspondiente es:
  \[
  \frac{\partial S}{\partial t} + \frac{1}{2m}\left[\left(\frac{\partial S}{\partial x}\right)^2 + \left(\frac{\partial S}{\partial y}\right)^2 + \left(\frac{\partial S}{\partial z}\right)^2\right] + \frac{m}{2}(\omega_x^2 x^2 + \omega_y^2 y^2 + \omega_z^2 z^2) = 0
  \]

  Como el sistema es conservativo, podemos separar la dependencia temporal:
  \[
  S(x,y,z,t) = W(x,y,z) - Et
  \]

  Sustituyendo en la ecuación de Hamilton-Jacobi:
  \[
  -E + \frac{1}{2m}\left[\left(\frac{\partial W}{\partial x}\right)^2 + \left(\frac{\partial W}{\partial y}\right)^2 + \left(\frac{\partial W}{\partial z}\right)^2\right] + \frac{m}{2}(\omega_x^2 x^2 + \omega_y^2 y^2 + \omega_z^2 z^2) = 0
  \]

  Esta ecuación es separable. Proponemos:
  \[
  W(x,y,z) = W_x(x) + W_y(y) + W_z(z)
  \]

  Obtenemos tres ecuaciones independientes:
  \begin{align}
  \frac{1}{2m}\left(\frac{dW_x}{dx}\right)^2 + \frac{m\omega_x^2}{2}x^2 &= E_x\\
  \frac{1}{2m}\left(\frac{dW_y}{dy}\right)^2 + \frac{m\omega_y^2}{2}y^2 &= E_y\\
  \frac{1}{2m}\left(\frac{dW_z}{dz}\right)^2 + \frac{m\omega_z^2}{2}z^2 &= E_z
  \end{align}

  donde $E = E_x + E_y + E_z$, siendo $E_x$, $E_y$ y $E_z$ constantes de separación.

  Para cada coordenada:
  \[
  \frac{dW_i}{dq_i} = \pm\sqrt{2m\left(E_i - \frac{m\omega_i^2}{2}q_i^2\right)}
  \]

  Integrando:
  \[
  W_i(q_i) = \int\sqrt{2m\left(E_i - \frac{m\omega_i^2}{2}q_i^2\right)}\,dq_i = \frac{E_i}{\omega_i}\arcsin\left(q_i\sqrt{\frac{m\omega_i^2}{2E_i}}\right) + \text{const.}
  \]

  La solución completa de la ecuación de Hamilton-Jacobi es:
  \[
  S(x,y,z,t) = \sum_{i=x,y,z}\left[\frac{E_i}{\omega_i}\arcsin\left(q_i\sqrt{\frac{m\omega_i^2}{2E_i}}\right) + \alpha_i\right] - Et
  \]

  donde $\alpha_i$ son constantes de integración.

  Para obtener las ecuaciones de movimiento, aplicamos:
  \[
  \frac{\partial S}{\partial E_i} = \beta_i \quad \text{(constantes)}
  \]

  Esto nos lleva a:
  \[
  \frac{1}{\omega_i}\arcsin\left(q_i\sqrt{\frac{m\omega_i^2}{2E_i}}\right) - t = \beta_i
  \]

  Despejando:
  \[
  q_i(t) = \sqrt{\frac{2E_i}{m\omega_i^2}}\sin[\omega_i(t + \beta_i)]
  \]

  Que son precisamente las ecuaciones de movimiento de tres osciladores armónicos independientes con frecuencias $\omega_x$, $\omega_y$ y $\omega_z$. Los momentos conjugados se obtienen de:
  \[
  p_i = \frac{\partial S}{\partial q_i} = \sqrt{2mE_i}\cos[\omega_i(t + \beta_i)]
  \]

  \item Halle y resuelva la ecuación de Hamilton–Jacobi para un oscilador armónico tridimensional en coordenadas esféricas. Exprese la solución \(r(t)\), \(\theta(t)\), \(\varphi(t)\) y los momentos conjugados usando variables acción–ángulo.
  
  \textbf{Solución}

  Consideremos un oscilador armónico tridimensional isotrópico con un potencial dado por \( V = \frac{1}{2} k r^2 \), donde \( r \) es la coordenada radial en coordenadas esféricas \( (r, \theta, \varphi) \), \( k \) es la constante de fuerza y \( m \) es la masa de la partícula. El objetivo es hallar y resolver la ecuación de Hamilton-Jacobi en estas coordenadas y expresar la solución en términos de variables acción-ángulo.

  \subsection*{Paso 1: Hamiltoniano en coordenadas esféricas}

  En coordenadas esféricas, las coordenadas generalizadas son \( r \), \( \theta \) y \( \varphi \), con momentos conjugados \( p_r \), \( p_\theta \) y \( p_\varphi \). La energía cinética \( T \) se expresa como:

  \[
  T = \frac{1}{2} m \left( \dot{r}^2 + r^2 \dot{\theta}^2 + r^2 \sin^2 \theta \dot{\varphi}^2 \right).
  \]

  Los momentos conjugados se derivan del lagrangiano \( L = T - V \):

  \[
  p_r = \frac{\partial L}{\partial \dot{r}} = m \dot{r}, \quad p_\theta = \frac{\partial L}{\partial \dot{\theta}} = m r^2 \dot{\theta}, \quad p_\varphi = \frac{\partial L}{\partial \dot{\varphi}} = m r^2 \sin^2 \theta \dot{\varphi}.
  \]

  Resolviendo para las velocidades:

  \[
  \dot{r} = \frac{p_r}{m}, \quad \dot{\theta} = \frac{p_\theta}{m r^2}, \quad \dot{\varphi} = \frac{p_\varphi}{m r^2 \sin^2 \theta}.
  \]

  Sustituyendo en \( T \), el Hamiltoniano \( H = T + V \) queda:

  \[
  H = \frac{p_r^2}{2m} + \frac{p_\theta^2}{2m r^2} + \frac{p_\varphi^2}{2m r^2 \sin^2 \theta} + \frac{1}{2} k r^2.
  \]

  \subsection*{Paso 2: Ecuación de Hamilton-Jacobi}

  La ecuación de Hamilton-Jacobi para un sistema conservativo es:

  \[
  \frac{\partial S}{\partial t} + H \left( r, \theta, \varphi, \frac{\partial S}{\partial r}, \frac{\partial S}{\partial \theta}, \frac{\partial S}{\partial \varphi} \right) = 0,
  \]

  donde \( S \) es la función de acción. Dado que \( H \) no depende explícitamente del tiempo, podemos separar variables asumiendo \( S = W(r, \theta, \varphi) - E t \), donde \( E \) es la energía total. Entonces:

  \[
  H \left( r, \theta, \varphi, \frac{\partial W}{\partial r}, \frac{\partial W}{\partial \theta}, \frac{\partial W}{\partial \varphi} \right) = E.
  \]

  Sustituyendo el Hamiltoniano:

  \[
  \frac{1}{2m} \left( \frac{\partial W}{\partial r} \right)^2 + \frac{1}{2m r^2} \left( \frac{\partial W}{\partial \theta} \right)^2 + \frac{1}{2m r^2 \sin^2 \theta} \left( \frac{\partial W}{\partial \varphi} \right)^2 + \frac{1}{2} k r^2 = E.
  \]

  \subsection*{Paso 3: Separación de variables}

  Dado que \( \varphi \) es una coordenada cíclica (no aparece explícitamente en \( H \)), el momento \( p_\varphi = \frac{\partial W}{\partial \varphi} \) es constante. Así, \( W \) se puede escribir como:

  \[
  W = W_r(r) + W_\theta(\theta) + p_\varphi \varphi.
  \]

  Sustituyendo \( \frac{\partial W}{\partial \varphi} = p_\varphi \):

  \[
  \frac{1}{2m} \left( \frac{d W_r}{dr} \right)^2 + \frac{1}{2m r^2} \left( \frac{d W_\theta}{d \theta} \right)^2 + \frac{p_\varphi^2}{2m r^2 \sin^2 \theta} + \frac{1}{2} k r^2 = E.
  \]

  Para separar las variables \( r \) y \( \theta \), multiplicamos ambos lados por \( 2m r^2 \):

  \[
  r^2 \left( \frac{d W_r}{dr} \right)^2 + m k r^4 + \left( \frac{d W_\theta}{d \theta} \right)^2 + \frac{p_\varphi^2}{\sin^2 \theta} = 2m E r^2.
  \]

  Los términos \( \left( \frac{d W_\theta}{d \theta} \right)^2 + \frac{p_\varphi^2}{\sin^2 \theta} \) dependen solo de \( \theta \), mientras que el resto depende de \( r \). Esto sugiere una constante de separación \( \alpha^2 \):

  \[
  \left( \frac{d W_\theta}{d \theta} \right)^2 + \frac{p_\varphi^2}{\sin^2 \theta} = \alpha^2,
  \]

  \[
  \frac{1}{2m} \left( \frac{d W_r}{dr} \right)^2 + \frac{\alpha^2}{2m r^2} + \frac{1}{2} k r^2 = E.
  \]

  Aquí, \( \alpha^2 \) está relacionada con el cuadrado del momento angular total, \( L^2 \), pero por ahora lo dejamos como una constante.

  Resolviendo:

  \[
  \frac{d W_\theta}{d \theta} = \sqrt{ \alpha^2 - \frac{p_\varphi^2}{\sin^2 \theta} },
  \]

  \[
  W_\theta = \int \sqrt{ \alpha^2 - \frac{p_\varphi^2}{\sin^2 \theta} } \, d\theta,
  \]

  \[
  \frac{d W_r}{dr} = \sqrt{ 2m E - \frac{\alpha^2}{r^2} - m k r^2 },
  \]

  \[
  W_r = \int \sqrt{ 2m E - \frac{\alpha^2}{r^2} - m k r^2 } \, dr.
  \]

  Entonces, la acción es:

  \[
  S = W_r(r) + W_\theta(\theta) + p_\varphi \varphi - E t.
  \]

  \subsection*{Paso 4: Variables acción-ángulo}

  Para expresar \( r(t) \), \( \theta(t) \), \( \varphi(t) \) y los momentos conjugados en términos de variables acción-ángulo, notamos que el oscilador armónico tridimensional isotrópico es superintegrable, y es más sencillo resolverlo primero en coordenadas cartesianas, donde se separa naturalmente en tres osciladores unidimensionales, y luego transformar a coordenadas esféricas.

  En coordenadas cartesianas, el Hamiltoniano es:

  \[
  H = \frac{p_x^2}{2m} + \frac{1}{2} m \omega^2 x^2 + \frac{p_y^2}{2m} + \frac{1}{2} m \omega^2 y^2 + \frac{p_z^2}{2m} + \frac{1}{2} m \omega^2 z^2,
  \]

  con \( \omega = \sqrt{k/m} \). La ecuación de Hamilton-Jacobi se separa en:

  \[
  \frac{1}{2m} \left( \frac{d W_x}{dx} \right)^2 + \frac{1}{2} m \omega^2 x^2 = E_x,
  \]

  y análogamente para \( y \) y \( z \), con \( E = E_x + E_y + E_z \).

  Para un oscilador 1D, la variable acción se define como \( I_x = \frac{1}{2\pi} \oint p_x \, dx \). Sabemos que \( E_x = \omega I_x \), y la solución es:

  \[
  x = \sqrt{\frac{2 I_x}{m \omega}} \sin w_x, \quad p_x = \sqrt{2 m \omega I_x} \cos w_x,
  \]

  donde \( w_x = \omega t + \delta_x \) es la variable ángulo. Similarmente:

  \[
  y = \sqrt{\frac{2 I_y}{m \omega}} \sin w_y, \quad p_y = \sqrt{2 m \omega I_y} \cos w_y,
  \]

  \[
  z = \sqrt{\frac{2 I_z}{m \omega}} \sin w_z, \quad p_z = \sqrt{2 m \omega I_z} \cos w_z,
  \]

  con \( w_y = \omega t + \delta_y \), \( w_z = \omega t + \delta_z \), y la energía total es:

  \[
  E = \omega (I_x + I_y + I_z).
  \]

  \subsection*{Paso 5: Transformación a coordenadas esféricas}

  En coordenadas esféricas:

  \[
  r = \sqrt{x^2 + y^2 + z^2}, \quad \theta = \arccos \left( \frac{z}{r} \right), \quad \varphi = \arctan \left( \frac{y}{x} \right).
  \]

  Sustituyendo:

  \[
  r(t) = \sqrt{ \left( \sqrt{\frac{2 I_x}{m \omega}} \sin (\omega t + \delta_x) \right)^2 + \left( \sqrt{\frac{2 I_y}{m \omega}} \sin (\omega t + \delta_y) \right)^2 + \left( \sqrt{\frac{2 I_z}{m \omega}} \sin (\omega t + \delta_z) \right)^2 },
  \]

  \[
  \theta(t) = \arccos \left( \frac{ \sqrt{\frac{2 I_z}{m \omega}} \sin (\omega t + \delta_z) }{ r(t) } \right),
  \]

  \[
  \varphi(t) = \arctan \left( \frac{ \sqrt{\frac{2 I_y}{m \omega}} \sin (\omega t + \delta_y) }{ \sqrt{\frac{2 I_x}{m \omega}} \sin (\omega t + \delta_x) } \right).
  \]

  Los momentos conjugados \( p_r \), \( p_\theta \), \( p_\varphi \) se pueden obtener transformando \( p_x \), \( p_y \), \( p_z \) a coordenadas esféricas usando las relaciones estándar, pero su forma explícita es complicada y depende de \( r \), \( \theta \), \( \varphi \). Por simplicidad, notamos que \( p_\varphi \) está relacionado con el momento angular conservado \( L_z \).

  \subsection*{Conclusión}

  Hemos resuelto la ecuación de Hamilton-Jacobi separando variables en coordenadas esféricas y expresado la solución en términos de variables acción-ángulo aprovechando la separabilidad en coordenadas cartesianas. Las expresiones para \( r(t) \), \( \theta(t) \), \( \varphi(t) \) reflejan la naturaleza oscilatoria del sistema, con \( I_x \), \( I_y \), \( I_z \) como variables acción y \( w_x \), \( w_y \), \( w_z \) como variables ángulo que avanzan linealmente con el tiempo a la frecuencia \( \omega \).

  \item Calcule la acción como función de las coordenadas para una partícula libre en tres dimensiones. Compruebe que coincide con la función generatriz de la transformación canónica de evolución temporal y escriba dicha transformación.
  
  \textbf{Solución:}  
  Para una partícula libre de masa \(m\) en \(\mathbb{R}^3\)
  \[
    L = \tfrac12 m\bigl(\dot x^2+\dot y^2+\dot z^2\bigr)\,,
    \quad H=\frac{p_x^2+p_y^2+p_z^2}{2m}\,.
  \]
  La trayectoria de mínima acción va en línea recta con \(\dot{\mathbf r}=\mathbf v\) constante,
  \(\mathbf r(t)=(x,y,z)\), \(\mathbf r(0)=\mathbf R\). Entonces
  \[
    S(\mathbf r,t;\mathbf R,0)
    =\int_{0}^{t}L\,dt'
    =\frac12m v^2\,t
    =\frac{m}{2t}\,\bigl|\mathbf r-\mathbf R\bigr|^2.
  \]
  Esta función coincide (hasta elegir \(\mathbf R\) como variable canónica “inicial”)
  con la función generatriz de tipo 1 que produce la evolución temporal en tiempo \(t\):
  \[
    F(\mathbf r,\mathbf R,t)
    =\frac m{2t}\,\bigl(\mathbf r-\mathbf R\bigr)^2.
  \]
  De ahí
  \[
    \mathbf p
    =\frac{\partial F}{\partial \mathbf r}
    =\frac{m}{t}\,(\mathbf r-\mathbf R),
    \qquad
    \mathbf P
    =-\,\frac{\partial F}{\partial \mathbf R}
    =\frac{m}{t}\,(\mathbf r-\mathbf R).
  \]
  Como \(\mathbf p=\mathbf P\), se conserva el momento, y la segunda ecuación da
  \[
    \mathbf R
    =\mathbf r - \frac{\mathbf p}{m}\,t,
  \]
  es decir, la transformación canónica de evolución temporal
  \[
    (\mathbf R,\mathbf P)\;\longmapsto\;(\mathbf r,\mathbf p)
    \quad\text{con}\quad
    \mathbf r=\mathbf R+\frac{\mathbf P}{m}\,t,\quad
    \mathbf p=\mathbf P.
  \]
  \item Calcule la acción para una partícula en caída libre como función de las coordenadas. Compruebe que coincide con el generador de la transformación canónica de evolución temporal e implemente dicha transformación.
  
  \textbf{Solución:}
  Consideremos una partícula de masa $m$ que se mueve en un campo gravitacional uniforme con aceleración $g$ en dirección vertical. Tomamos el eje $y$ positivo hacia arriba.

  \paragraph{1. Cálculo de la acción:}

  El lagrangiano del sistema es:
  \[
  L = \frac{1}{2}m\dot{x}^2 + \frac{1}{2}m\dot{y}^2 - mgy
  \]

  La ecuación de movimiento para la coordenada $x$ da $\ddot{x}=0$, lo que implica $x(t) = x_0 + v_x(t-t_0)$, mientras que para $y$ obtenemos $\ddot{y}=-g$, por lo que $y(t) = y_0 + v_y(t-t_0) - \frac{1}{2}g(t-t_0)^2$.

  Calculemos la acción entre los instantes $t_0$ y $t$:
  \begin{align}
  S(x,y,t;x_0,y_0,t_0) &= \int_{t_0}^{t} L\,d\tau \\
  &= \int_{t_0}^{t} \left[\frac{1}{2}m\dot{x}^2 + \frac{1}{2}m\dot{y}^2 - mgy\right]\,d\tau
  \end{align}

  Sustituyendo las velocidades:
  \begin{align}
  \dot{x} &= v_x = \frac{x-x_0}{t-t_0} \\
  \dot{y} &= v_y - g(t-\tau) = \frac{y-y_0}{t-t_0} + \frac{1}{2}g(t-t_0) - g(t-\tau)
  \end{align}

  Después de integrar y simplificar, obtenemos:
  \begin{align}
  S(x,y,t;x_0,y_0,t_0) = \frac{m}{2(t-t_0)}[(x-x_0)^2 + (y-y_0)^2] + \frac{mg}{2}(y+y_0)(t-t_0) - \frac{mg^2}{24}(t-t_0)^3
  \end{align}

  \paragraph{2. Verificación como generador de transformación canónica:}

  Para comprobar que esta acción coincide con el generador de una transformación canónica de evolución temporal, calculamos:
  \begin{align}
  p_x &= \frac{\partial S}{\partial x} = \frac{m(x-x_0)}{t-t_0} \\
  p_y &= \frac{\partial S}{\partial y} = \frac{m(y-y_0)}{t-t_0} + \frac{mg(t-t_0)}{2}
  \end{align}

  Estas expresiones coinciden con los momentos físicos: $p_x = m\dot{x}$ y $p_y = m\dot{y}$.

  Además, verificamos:
  \begin{align}
  -\frac{\partial S}{\partial t} &= \frac{m[(x-x_0)^2+(y-y_0)^2]}{2(t-t_0)^2} - \frac{mg(y+y_0)}{2} + \frac{mg^2(t-t_0)^2}{8} \\
  &= \frac{p_x^2+p_y^2}{2m} + mgy = H
  \end{align}

  Esto confirma que $-\partial S/\partial t$ es el hamiltoniano.

  \paragraph{3. Implementación de la transformación canónica:}

  La acción $S$ genera una transformación canónica de $(x_0,y_0,p_{x0},p_{y0})$ a $(x,y,p_x,p_y)$ mediante:
  \begin{align}
  p_{x0} &= -\frac{\partial S}{\partial x_0} = \frac{m(x-x_0)}{t-t_0} = p_x \\
  p_{y0} &= -\frac{\partial S}{\partial y_0} = \frac{m(y-y_0)}{t-t_0} - \frac{mg(t-t_0)}{2} = p_y - mg(t-t_0)
  \end{align}

  La transformación canónica queda:
  \begin{align}
  x &= x_0 + \frac{p_{x0}}{m}(t-t_0) \\
  y &= y_0 + \frac{p_{y0}}{m}(t-t_0) - \frac{1}{2}g(t-t_0)^2 \\
  p_x &= p_{x0} \\
  p_y &= p_{y0} - mg(t-t_0)
  \end{align}

  Esta transformación describe la evolución del sistema desde el estado inicial \\
   $(x_0,y_0,p_{x0},p_{y0})$ al instante $t_0$ hasta el estado $(x,y,p_x,p_y)$ al instante $t$. La conservación de $p_x$ y la variación lineal de $p_y$ con el tiempo concuerdan con las propiedades físicas del movimiento bajo gravedad constante.

  \item Considere el movimiento de “tiro parabólico” desde el suelo y desde un ascensor que se mueve hacia arriba con aceleración \(g\). Realice la transformación de coordenadas entre ambos sistemas y halle los hamiltonianos correspondientes.
  
  \textbf{Solución:}  
  Estudiamos el movimiento de una partícula de masa $m$ en un campo gravitatorio uniforme desde dos sistemas de referencia:

  \begin{itemize}
  \item Sistema $S$ (inercial): ligado al suelo, con coordenadas $(x,y)$ y momentos conjugados $(p_x,p_y)$.
  \item Sistema $S'$ (no inercial): ligado al ascensor que asciende con aceleración constante $g$, con coordenadas $(X,Y)$ y momentos $(P_X,P_Y)$.
  \end{itemize}

  \paragraph{Hamiltoniano en el sistema inercial $S$:}
  En el sistema ligado al suelo, la hamiltoniana incluye la energía cinética y el potencial gravitatorio:
  \[
  H(x,y;p_x,p_y) = \frac{p_x^2+p_y^2}{2m} + mgy
  \]

  \paragraph{Transformación de coordenadas:}
  La relación entre las coordenadas de ambos sistemas es:
  \[
  x = X, \quad y = Y + \frac{1}{2}gt^2
  \]

  donde el término $\frac{1}{2}gt^2$ representa el desplazamiento del ascensor respecto al suelo. 

  \paragraph{Construcción de la función generatriz:}
  Para obtener una transformación canónica completa, necesitamos establecer una función generatriz que produzca la transformación de coordenadas deseada. Utilizaremos una función generatriz de tipo 2, que depende de las coordenadas antiguas $(x)$, las nuevas $(Y)$ y los momentos nuevos $(P_X)$ y antiguos $(p_y)$:
  \[
  F_2(x,Y;P_X,p_y;t) = xP_X + \left(Y + \frac{1}{2}gt^2\right)p_y
  \]

  \paragraph{Verificación de las relaciones canónicas:}
  A partir de la función generatriz, las relaciones canónicas nos dan:
  \begin{align}
  p_x &= \frac{\partial F_2}{\partial x} = P_X \\
  X &= \frac{\partial F_2}{\partial P_X} = x \\
  P_Y &= \frac{\partial F_2}{\partial Y} = p_y \\
  y &= \frac{\partial F_2}{\partial p_y} = Y + \frac{1}{2}gt^2
  \end{align}

  Estas ecuaciones confirman nuestra transformación de coordenadas inicial y establecen que $P_X = p_x$ y $P_Y = p_y$.

  \paragraph{Cálculo del nuevo hamiltoniano:}
  Según la teoría de transformaciones canónicas dependientes del tiempo, el nuevo hamiltoniano se obtiene como:
  \[
  H'(X,Y;P_X,P_Y) = H(x,y;p_x,p_y) + \frac{\partial F_2}{\partial t}
  \]

  Calculamos cada término:
  \begin{align}
  H(x,y;p_x,p_y) &= \frac{p_x^2+p_y^2}{2m} + mgy \\
  &= \frac{P_X^2+P_Y^2}{2m} + mg\left(Y + \frac{1}{2}gt^2\right)
  \end{align}

  Y la derivada parcial de la función generatriz:
  \[
  \frac{\partial F_2}{\partial t} = \frac{\partial}{\partial t}\left(xP_X + \left(Y + \frac{1}{2}gt^2\right)p_y\right) = gtp_y = gtP_Y
  \]

  Por tanto:
  \begin{align}
  H'(X,Y;P_X,P_Y) &= \frac{P_X^2+P_Y^2}{2m} + mg\left(Y + \frac{1}{2}gt^2\right) + gtP_Y \\
  &= \frac{P_X^2+P_Y^2}{2m} + mgY + \frac{1}{2}mg^2t^2 + gtP_Y
  \end{align}

  Reordenando términos:
  \begin{align}
  H'(X,Y;P_X,P_Y) &= \frac{P_X^2+P_Y^2}{2m} + mgY + P_Y\left(gt\right) + \frac{1}{2}mg^2t^2 \\
  &= \frac{P_X^2+P_Y^2}{2m} + mgY + P_Y\left(gt\right) + \frac{m}{2}\left(gt\right)^2
  \end{align}

  Observemos que:
  \[
  mgY + P_Y(gt) + \frac{m}{2}(gt)^2 = mgY + \frac{P_Y^2}{2m} - \frac{P_Y^2}{2m} + P_Y(gt) + \frac{m}{2}(gt)^2
  \]

  Completando el cuadrado:
  \[
  mgY + P_Y(gt) + \frac{m}{2}(gt)^2 = mgY + \frac{(P_Y + mgt)^2 - P_Y^2}{2m} = mgY + \frac{(P_Y + mgt)^2 - P_Y^2}{2m}
  \]

  Simplificando:
  \[
  H'(X,Y;P_X,P_Y) = \frac{P_X^2}{2m} + \frac{(P_Y + mgt)^2}{2m} - \frac{P_Y^2}{2m} + \frac{P_Y^2}{2m} + mgY = \frac{P_X^2 + P_Y^2}{2m}
  \]

  \paragraph{Interpretación física:}
  El resultado muestra que en el sistema del ascensor (no inercial), el hamiltoniano se reduce a la pura energía cinética, sin términos de potencial gravitatorio. Esto significa que un observador en el ascensor no percibe la gravedad (principio de equivalencia de Einstein): las leyes de la física en un sistema uniformemente acelerado son equivalentes a las de un sistema inercial con un campo gravitatorio uniforme.

  Los hamiltonianos en ambos sistemas son:
  \[
  H_{\text{suelo}} = \frac{p_x^2+p_y^2}{2m} + mgy, \quad
  H_{\text{ascensor}} = \frac{P_X^2+P_Y^2}{2m}
  \]

  En el sistema del ascensor, las ecuaciones de movimiento son las de una partícula libre:
  \[
  \dot{X} = \frac{P_X}{m}, \quad \dot{Y} = \frac{P_Y}{m}, \quad \dot{P}_X = 0, \quad \dot{P}_Y = 0
  \]

  Lo que implica trayectorias rectilíneas con velocidad constante, mientras que en el sistema del suelo se obtienen las conocidas parábolas del tiro parabólico.
  \item Considere un sistema de fuerzas centrales de su elección. Resuelva la ecuación de Hamilton–Jacobi y halle la transformación canónica generada.
  
  \textbf{Solución:}
  Consideraré el potencial de Coulomb (o gravitatorio): $U(r) = -\frac{k}{r}$, donde $k$ es una constante positiva. Para una partícula de masa $m$ en este campo central, el hamiltoniano en coordenadas esféricas es:

  \[
  H = \frac{p_r^2}{2m} + \frac{p_\theta^2}{2mr^2} + \frac{p_\varphi^2}{2mr^2\sin^2\theta} - \frac{k}{r}
  \]

  La ecuación de Hamilton-Jacobi para un sistema conservativo es:
  \[
  H\left(r,\theta,\varphi,\frac{\partial S}{\partial r},\frac{\partial S}{\partial \theta},\frac{\partial S}{\partial \varphi}\right) = E
  \]

  Como la energía es constante, buscamos una solución de la forma:
  \[
  S(r,\theta,\varphi,t) = W(r,\theta,\varphi) - Et
  \]

  Sustituyendo en la ecuación de Hamilton-Jacobi:
  \[
  \frac{1}{2m}\left(\frac{\partial W}{\partial r}\right)^2 + \frac{1}{2mr^2}\left(\frac{\partial W}{\partial \theta}\right)^2 + \frac{1}{2mr^2\sin^2\theta}\left(\frac{\partial W}{\partial \varphi}\right)^2 - \frac{k}{r} = E
  \]

  Podemos separar las variables escribiendo:
  \[
  W(r,\theta,\varphi) = W_r(r) + W_\theta(\theta) + W_\varphi(\varphi)
  \]

  Como el potencial tiene simetría esférica, $\varphi$ es una coordenada cíclica. Por tanto:
  \[
  \frac{\partial W}{\partial \varphi} = \frac{dW_\varphi}{d\varphi} = \alpha_3 = \text{constante}
  \]

  donde $\alpha_3$ se identifica con el momento angular en la dirección $z$.

  Multiplicando la ecuación de Hamilton-Jacobi por $2mr^2$:
  \[
  r^2\left(\frac{dW_r}{dr}\right)^2 + \left(\frac{dW_\theta}{d\theta}\right)^2 + \frac{\alpha_3^2}{\sin^2\theta} = 2mr^2E + 2mkr
  \]

  Para que esta ecuación sea separable, debe existir una constante $\alpha_2^2$ tal que:
  \[
  \left(\frac{dW_\theta}{d\theta}\right)^2 + \frac{\alpha_3^2}{\sin^2\theta} = \alpha_2^2
  \]

  La constante $\alpha_2$ se interpreta como el momento angular total.

  Las ecuaciones separadas son:
  \begin{align}
  \frac{dW_\theta}{d\theta} &= \pm\sqrt{\alpha_2^2 - \frac{\alpha_3^2}{\sin^2\theta}} \\
  \frac{dW_r}{dr} &= \pm\sqrt{2mE + \frac{2mk}{r} - \frac{\alpha_2^2}{r^2}}
  \end{align}

  Resolviendo para $W_\theta$ y $W_\varphi$:
  \begin{align}
  W_\varphi &= \alpha_3\varphi \\
  W_\theta &= \int\sqrt{\alpha_2^2 - \frac{\alpha_3^2}{\sin^2\theta}}\,d\theta
  \end{align}

  Para $W_r$, distinguimos dos casos según el signo de $E$:

  \paragraph{Caso $E < 0$ (órbitas ligadas):}
  Haciendo el cambio de variable $u = 1/r$, $dr = -du/u^2$, la integral se transforma en:
  \begin{align}
  W_r &= -\int\frac{1}{u^2}\sqrt{2mE + 2mku - \alpha_2^2u^2}\,du \\
  &= -\sqrt{\frac{mk^2}{-2E}}\arccos\left(\frac{\alpha_2u-mk}{\alpha_2\sqrt{-2mE/\alpha_2^2}}\right) + \text{constante}
  \end{align}

  \paragraph{Transformación canónica:}
  La función principal de Hamilton completa es:
  \[
  S = W_r(r) + W_\theta(\theta) + \alpha_3\varphi - Et
  \]

  Esta función genera una transformación canónica a nuevas variables $(Q_i,P_i)$ mediante:
  \[
  p_i = \frac{\partial S}{\partial q_i}, \quad Q_i = \frac{\partial S}{\partial P_i}
  \]

  Las nuevas variables son constantes del movimiento:
  \begin{align}
  P_1 &= E \quad \text{(energía)} \\
  P_2 &= \alpha_2 \quad \text{(momento angular total)} \\
  P_3 &= \alpha_3 \quad \text{(componente z del momento angular)}
  \end{align}

  Las variables conjugadas $Q_i$ son ángulos que evolucionan linealmente con el tiempo:
  \begin{align}
  Q_1 &= \frac{\partial S}{\partial E} = -t + \text{términos adicionales} \\
  Q_2 &= \frac{\partial S}{\partial \alpha_2} \\
  Q_3 &= \frac{\partial S}{\partial \alpha_3} = \varphi + \text{términos adicionales}
  \end{align}

  Esta transformación mapea el problema del movimiento en un campo central a un conjunto de variables acción-ángulo, donde las acciones $(P_i)$ son constantes y los ángulos $(Q_i)$ evolucionan linealmente con el tiempo.

  \item Realice los detalles del problema 2 de la sección 40.
  \item Realice los detalles del problema de la sección 50.
  \item ¿A qué velocidad debe moverse un observador hacia una estrella para que la mitad de la luz emitida esté concentrada en un cono de ángulo \(0.01\)\,rad? Puede aproximarse \(\cos\theta\approx1-\theta^2/2\).
  
  \textbf{Solución:} 
  Este problema involucra el efecto relativista de aberración de la luz. Cuando un observador se mueve hacia una fuente de luz, los rayos luminosos parecen concentrarse en la dirección del movimiento.

  En el sistema de referencia en reposo de la estrella, la radiación es emitida isotrópicamente. La mitad de esta radiación se emite dentro del hemisferio frontal, es decir, para ángulos $\theta$ tales que $0 \leq \theta < \pi/2$. Esto corresponde a $\cos\theta = 0$ para el límite de este hemisferio.

  En el sistema del observador que se mueve con velocidad $v = \beta c$ hacia la estrella, la fórmula de aberración relativista relaciona el ángulo $\theta'$ observado con el ángulo original $\theta$ mediante:

  \[
  \cos\theta' = \frac{\cos\theta + \beta}{1 + \beta\cos\theta}
  \]

  Para el límite del hemisferio ($\cos\theta = 0$):

  \[
  \cos\theta' = \frac{\beta}{1} = \beta
  \]

  Queremos que $\theta' = 0.01$ rad, por lo tanto:

  \[
  \cos(0.01) = \beta
  \]

  Utilizando la aproximación $\cos\theta \approx 1-\theta^2/2$:

  \[
  \cos(0.01) \approx 1 - \frac{(0.01)^2}{2} = 1 - 5 \times 10^{-5} = 0.99995
  \]

  Por tanto, la velocidad requerida es:

  \[
  v = \beta c \approx 0.99995c
  \]

  Esta velocidad extremadamente alta, muy cercana a la de la luz, es necesaria para comprimir la mitad de la radiación estelar en un cono tan estrecho de tan solo 0.01 radianes (aproximadamente 0.57 grados).

  \item Considere el lagrangiano
    \[
      L = -mc^2\sqrt{1-\frac{v^2}{c^2}} \;-\; \mathbf a\cdot\mathbf v,
    \]
    con \(\mathbf a\) constante. Halle el hamiltoniano y la ecuación de Hamilton–Jacobi.

  \textbf{Solución:}

  El lagrangiano dado combina el término relativista usual con un término adicional que involucra un vector constante $\mathbf{a}$.

  \paragraph{1. Cálculo del momento conjugado:}
  Primero determinamos el momento conjugado:
  \begin{align}
  \mathbf{p} &= \frac{\partial L}{\partial \mathbf{v}} \\
  &= \frac{\partial}{\partial \mathbf{v}}\left(-mc^2\sqrt{1-\frac{v^2}{c^2}} - \mathbf{a}\cdot\mathbf{v}\right) \\
  &= \frac{m\mathbf{v}}{\sqrt{1-\frac{v^2}{c^2}}} - \mathbf{a}
  \end{align}

  Esto puede reescribirse como:
  \[
  \mathbf{p} + \mathbf{a} = \gamma m\mathbf{v}
  \]
  donde $\gamma = \frac{1}{\sqrt{1-\frac{v^2}{c^2}}}$ es el factor de Lorentz.

  \paragraph{2. Cálculo del hamiltoniano:}

  El hamiltoniano se define como:
  \begin{align}
  H &= \mathbf{p}\cdot\mathbf{v} - L \\
  &= \mathbf{p}\cdot\mathbf{v} + mc^2\sqrt{1-\frac{v^2}{c^2}} + \mathbf{a}\cdot\mathbf{v}
  \end{align}

  Necesitamos expresar $\mathbf{v}$ en términos de $\mathbf{p}$. De la ecuación del momento:
  \[
  \mathbf{v} = \frac{\mathbf{p} + \mathbf{a}}{\gamma m}
  \]

  Sustituyendo en el hamiltoniano:
  \begin{align}
  H &= \mathbf{p}\cdot\frac{\mathbf{p} + \mathbf{a}}{\gamma m} + \frac{mc^2}{\gamma} + \mathbf{a}\cdot\frac{\mathbf{p} + \mathbf{a}}{\gamma m} \\
  &= \frac{1}{\gamma m}\left[\mathbf{p}\cdot(\mathbf{p} + \mathbf{a}) + \mathbf{a}\cdot(\mathbf{p} + \mathbf{a})\right] + \frac{mc^2}{\gamma} \\
  &= \frac{|\mathbf{p} + \mathbf{a}|^2}{\gamma m} + \frac{mc^2}{\gamma}
  \end{align}

  De la relación relativista sabemos que $|\mathbf{p} + \mathbf{a}|^2 = \gamma^2 m^2 v^2 = (\gamma^2 - 1)m^2c^2$. Sustituyendo:
  \begin{align}
  H &= \frac{(\gamma^2 - 1)m^2c^2}{\gamma m} + \frac{mc^2}{\gamma} \\
  &= \frac{m^2c^2\gamma^2 - m^2c^2 + mc^2\gamma}{\gamma m} \\
  &= \gamma mc^2
  \end{align}

  Expresando $\gamma$ en términos de $\mathbf{p}$:
  \begin{align}
  |\mathbf{p} + \mathbf{a}|^2 &= (\gamma^2 - 1)m^2c^2 \\
  \gamma^2 &= 1 + \frac{|\mathbf{p} + \mathbf{a}|^2}{m^2c^2}
  \end{align}

  Por tanto, el hamiltoniano es:
  \[
  H(\mathbf{r}, \mathbf{p}) = mc^2\sqrt{1 + \frac{|\mathbf{p} + \mathbf{a}|^2}{m^2c^2}}
  \]

  \paragraph{3. Ecuación de Hamilton-Jacobi:}

  La ecuación de Hamilton-Jacobi se obtiene sustituyendo $\mathbf{p} = \frac{\partial S}{\partial \mathbf{r}}$ en el hamiltoniano:
  \[
  \frac{\partial S}{\partial t} + mc^2\sqrt{1 + \frac{\left|\frac{\partial S}{\partial \mathbf{r}} + \mathbf{a}\right|^2}{m^2c^2}} = 0
  \]

  Esta ecuación gobierna la evolución de la función principal de Hamilton $S$ para una partícula relativista bajo la influencia del campo externo representado por el vector $\mathbf{a}$.

  \item La ecuación de Schrödinger para una partícula libre en una dimensión es
    \[
      -\frac{\hbar^2}{2m}\,\frac{d^2\psi}{dx^2} \;=\; E\,\psi.
    \]
    Se acostumbra escribir \(E=\hbar^2k^2/(2m)\). Como cualquier función compleja, \(\psi\) se puede escribir
    \[
      \psi(x) = \exp\!\bigl(i\,\Sigma(x)/\hbar\bigr),
    \]
    donde \(\Sigma\) es real. Suponiendo que el problema admite considerar que \(\hbar\) es “pequeña”, expanda \(\Sigma\) en potencias de \(\hbar\):
    \[
      \Sigma = \Sigma_0 + \frac{\hbar}{i}\,\Sigma_1 + \Bigl(\frac{\hbar}{i}\Bigr)^2 \Sigma_2 + \cdots.
    \]
    Muestre que el término de orden cero, \(\Sigma_0\), satisface la ecuación de Hamilton–Jacobi. Resuelva dicha ecuación y obtenga la función de onda \(\psi\) en esta aproximación, comprobando que describe una “onda plana”.

  \textbf{Solución:}

  Partimos de la función de onda expresada como:
  \[
    \psi(x) = \exp\!\bigl(i\,\Sigma(x)/\hbar\bigr)
  \]

  Calculemos sus derivadas para sustituir en la ecuación de Schrödinger:
  \[
    \frac{d\psi}{dx} = \frac{i}{\hbar}\,\frac{d\Sigma}{dx}\,\psi
  \]

  \[
    \frac{d^2\psi}{dx^2} = \frac{i}{\hbar}\,\frac{d}{dx}\left(\frac{d\Sigma}{dx}\,\psi\right) 
    = \frac{i}{\hbar}\,\frac{d^2\Sigma}{dx^2}\,\psi + \frac{i}{\hbar}\,\frac{d\Sigma}{dx}\,\frac{d\psi}{dx}
  \]

  Sustituyendo la expresión de la primera derivada:
  \[
    \frac{d^2\psi}{dx^2} = \frac{i}{\hbar}\,\frac{d^2\Sigma}{dx^2}\,\psi - \frac{1}{\hbar^2}\left(\frac{d\Sigma}{dx}\right)^2\psi
  \]

  Ahora sustituimos en la ecuación de Schrödinger:
  \[
    -\frac{\hbar^2}{2m}\left[\frac{i}{\hbar}\,\frac{d^2\Sigma}{dx^2}\,\psi - \frac{1}{\hbar^2}\left(\frac{d\Sigma}{dx}\right)^2\psi\right] = E\,\psi
  \]

  Dividiendo entre $\psi$ y simplificando:
  \[
    \frac{1}{2m}\left(\frac{d\Sigma}{dx}\right)^2 - \frac{i\hbar}{2m}\frac{d^2\Sigma}{dx^2} = E
  \]

  Ahora sustituimos la expansión en potencias de $\hbar$ para $\Sigma(x)$:
  \[
    \Sigma = \Sigma_0 + \frac{\hbar}{i}\,\Sigma_1 + \Bigl(\frac{\hbar}{i}\Bigr)^2 \Sigma_2 + \cdots
  \]

  Sus derivadas son:
  \[
    \frac{d\Sigma}{dx} = \frac{d\Sigma_0}{dx} + \frac{\hbar}{i}\frac{d\Sigma_1}{dx} + \cdots
  \]

  \[
    \frac{d^2\Sigma}{dx^2} = \frac{d^2\Sigma_0}{dx^2} + \frac{\hbar}{i}\frac{d^2\Sigma_1}{dx^2} + \cdots
  \]

  Al sustituir en la ecuación y tomar solo los términos de orden cero en $\hbar$ (es decir, independientes de $\hbar$), obtenemos:
  \[
    \frac{1}{2m}\left(\frac{d\Sigma_0}{dx}\right)^2 = E
  \]

  Esta es precisamente la ecuación de Hamilton-Jacobi para una partícula libre, donde $\Sigma_0$ juega el papel de la acción reducida o función principal de Hamilton.

  \paragraph{Resolución de la ecuación de Hamilton-Jacobi:}

  De la ecuación:
  \[
    \frac{1}{2m}\left(\frac{d\Sigma_0}{dx}\right)^2 = E
  \]

  Despejamos:
  \[
    \frac{d\Sigma_0}{dx} = \pm\sqrt{2mE}
  \]

  Dado que $E = \frac{\hbar^2k^2}{2m}$, tenemos $\sqrt{2mE} = \hbar k$. Integrando y tomando el signo positivo:
  \[
    \Sigma_0(x) = \hbar kx + \text{constante}
  \]

  Tomando la constante igual a cero (lo que corresponde a una elección de fase), obtenemos:
  \[
    \Sigma_0(x) = \hbar kx
  \]

  \paragraph{Función de onda en la aproximación de orden cero:}

  En esta aproximación, la función de onda es:
  \[
    \psi(x) \approx \exp\left(\frac{i\Sigma_0}{\hbar}\right) = \exp\left(\frac{i\hbar kx}{\hbar}\right) = \exp(ikx)
  \]

  Esta es efectivamente una onda plana con número de onda $k$ y longitud de onda $\lambda = \frac{2\pi}{k}$, que corresponde a una partícula libre con momento $p = \hbar k$ y energía $E = \frac{p^2}{2m}$.

  Esta relación ilustra el principio de correspondencia: en el límite clásico donde $\hbar$ es "pequeña", la función de onda cuántica se aproxima a una forma relacionada con la acción clásica, y las trayectorias cuánticas tienden a las predichas por la mecánica clásica.

  \item Considere un sistema formado por \(N\) partículas, de masas \(m_i\), cuyas posiciones y velocidades son \(\mathbf{r}_i\) y \(\mathbf{v}_i\), para \(i=1,2,\dots,N\). Defina un conjunto \(q=\{q_1,q_2,\dots,q_{3N}\}\) de \(3N\) coordenadas generalizadas por
    \[
      q_1 = x_1\sqrt{m_1},\quad
      q_2 = y_1\sqrt{m_1},\quad
      q_3 = z_1\sqrt{m_1},\quad
      q_4 = x_2\sqrt{m_2},\;\dots,\;
      q_{3N} = z_N\sqrt{m_N}.
    \]
    Defina la distancia infinitesimal en el espacio de configuración \(3N\)-dimensional como
    \[
      ds^2 = \sum_{i=1}^{3N} dq_i^2.
    \]
    Demuestre que la acción de Hamilton (ecuación 44.4) está dada por
    \[
      S_0 \;=\; \int \sqrt{2T}\,ds,
    \]
    donde \(T\) es la energía cinética del sistema de partículas.

  \textbf{Solución:}

  Partamos del análisis de la energía cinética en términos de las coordenadas cartesianas habituales:
  \[
  T = \sum_{i=1}^N \frac{1}{2}m_i(\dot{x}_i^2 + \dot{y}_i^2 + \dot{z}_i^2)
  \]

  Según la definición de las coordenadas generalizadas, para cada partícula tenemos:
  \begin{align}
  q_{3(i-1)+1} &= x_i\sqrt{m_i} \Rightarrow \dot{q}_{3(i-1)+1} = \dot{x}_i\sqrt{m_i} \Rightarrow \dot{x}_i = \frac{\dot{q}_{3(i-1)+1}}{\sqrt{m_i}} \\
  q_{3(i-1)+2} &= y_i\sqrt{m_i} \Rightarrow \dot{q}_{3(i-1)+2} = \dot{y}_i\sqrt{m_i} \Rightarrow \dot{y}_i = \frac{\dot{q}_{3(i-1)+2}}{\sqrt{m_i}} \\
  q_{3(i-1)+3} &= z_i\sqrt{m_i} \Rightarrow \dot{q}_{3(i-1)+3} = \dot{z}_i\sqrt{m_i} \Rightarrow \dot{z}_i = \frac{\dot{q}_{3(i-1)+3}}{\sqrt{m_i}}
  \end{align}

  Sustituyendo estas expresiones en la fórmula de la energía cinética:
  \begin{align}
  T &= \sum_{i=1}^N \frac{1}{2}m_i\left[\left(\frac{\dot{q}_{3(i-1)+1}}{\sqrt{m_i}}\right)^2 + \left(\frac{\dot{q}_{3(i-1)+2}}{\sqrt{m_i}}\right)^2 + \left(\frac{\dot{q}_{3(i-1)+3}}{\sqrt{m_i}}\right)^2\right] \\
  &= \sum_{i=1}^N \frac{1}{2}m_i\left[\frac{\dot{q}_{3(i-1)+1}^2}{m_i} + \frac{\dot{q}_{3(i-1)+2}^2}{m_i} + \frac{\dot{q}_{3(i-1)+3}^2}{m_i}\right] \\
  &= \frac{1}{2}\sum_{i=1}^N \left[\dot{q}_{3(i-1)+1}^2 + \dot{q}_{3(i-1)+2}^2 + \dot{q}_{3(i-1)+3}^2\right] \\
  &= \frac{1}{2}\sum_{j=1}^{3N} \dot{q}_j^2
  \end{align}

  Por lo tanto, la energía cinética adopta una forma particularmente simple en estas coordenadas.

  Calculemos ahora los momentos conjugados:
  \[
  p_j = \frac{\partial L}{\partial \dot{q}_j} = \frac{\partial T}{\partial \dot{q}_j} = \dot{q}_j
  \]

  La acción abreviada de Hamilton (ecuación 44.4) es:
  \[
  S_0 = \int\sum_{j=1}^{3N}p_j\,dq_j = \int\sum_{j=1}^{3N}\dot{q}_j\,dq_j
  \]

  Para relacionarla con el elemento de línea $ds$ en el espacio de configuración, observamos que:
  \begin{align}
  ds^2 &= \sum_{j=1}^{3N}dq_j^2 \\
  &= \sum_{j=1}^{3N}(\dot{q}_j\,dt)^2 \\
  &= dt^2\sum_{j=1}^{3N}\dot{q}_j^2
  \end{align}

  Tomando la raíz cuadrada:
  \[
  ds = dt\sqrt{\sum_{j=1}^{3N}\dot{q}_j^2}
  \]

  Por tanto:
  \[
  dt = \frac{ds}{\sqrt{\sum_{j=1}^{3N}\dot{q}_j^2}}
  \]

  Y:
  \[
  dq_j = \dot{q}_j\,dt = \dot{q}_j\frac{ds}{\sqrt{\sum_{k=1}^{3N}\dot{q}_k^2}}
  \]

  Sustituyendo en la expresión de la acción:
  \begin{align}
  S_0 &= \int\sum_{j=1}^{3N}\dot{q}_j\,dq_j \\
  &= \int\sum_{j=1}^{3N}\dot{q}_j \cdot \dot{q}_j\frac{ds}{\sqrt{\sum_{k=1}^{3N}\dot{q}_k^2}} \\
  &= \int\frac{\sum_{j=1}^{3N}\dot{q}_j^2}{\sqrt{\sum_{k=1}^{3N}\dot{q}_k^2}}ds \\
  &= \int\sqrt{\sum_{j=1}^{3N}\dot{q}_j^2}\,ds
  \end{align}

  Dado que $\sum_{j=1}^{3N}\dot{q}_j^2 = 2T$, finalmente obtenemos:
  \[
  S_0 = \int\sqrt{2T}\,ds
  \]

  Lo que completa la demostración. Esta fórmula expresa la acción abreviada de Hamilton como la integral de la raíz cuadrada del doble de la energía cinética multiplicada por el elemento de línea en el espacio de configuración, y es válida para cualquier sistema de partículas.

  \item Escriba la acción del campo electromagnético
    \[
      S_F = -\frac{1}{16\pi c}\int F_{ik}F^{ik}\,d\Omega,
    \]
    y, de \(\delta S_F=0\), deduzca el segundo par de ecuaciones de Maxwell en ausencia de cargas y corrientes.

  \textbf{Solución:}

  La acción del campo electromagnético libre viene dada por:

  \[
  S_F = -\frac{1}{16\pi c}\int F_{ik}F^{ik}\,d\Omega,
  \]

  donde $F_{ik}$ es el tensor electromagnético, $F^{ik} = g^{im}g^{kn}F_{mn}$ su versión contravariante, $d\Omega$ el elemento de volumen espacio-temporal y $c$ la velocidad de la luz.

  El tensor electromagnético se define en términos del cuadripotencial $A_k$ como:
  \[
  F_{ik} = \partial_i A_k - \partial_k A_i
  \]

  Para derivar las ecuaciones de campo a partir del principio variacional $\delta S_F = 0$, debemos calcular la variación de la acción respecto a pequeñas variaciones $\delta A_k$ del potencial:

  \[
  \delta S_F = -\frac{1}{8\pi c}\int F^{ik}\delta F_{ik}\,d\Omega
  \]

  La variación del tensor de campo es:
  \[
  \delta F_{ik} = \partial_i \delta A_k - \partial_k \delta A_i
  \]

  Sustituyendo:
  \[
  \delta S_F = -\frac{1}{8\pi c}\int F^{ik}(\partial_i \delta A_k - \partial_k \delta A_i)\,d\Omega
  \]

  Reorganizando términos:
  \[
  \delta S_F = -\frac{1}{8\pi c}\int [F^{ik}\partial_i \delta A_k - F^{ik}\partial_k \delta A_i]\,d\Omega
  \]

  Debido a la antisimetría del tensor electromagnético ($F^{ik} = -F^{ki}$), podemos reescribir:
  \[
  \delta S_F = -\frac{1}{8\pi c}\int [F^{ik}\partial_i \delta A_k + F^{ki}\partial_k \delta A_i]\,d\Omega = -\frac{1}{8\pi c}\int [F^{ik}\partial_i \delta A_k + F^{ik}\partial_k \delta A_i]\,d\Omega
  \]

  Aplicando integración por partes y asumiendo que las variaciones se anulan en la frontera:
  \[
  \delta S_F = \frac{1}{8\pi c}\int [(\partial_i F^{ik})\delta A_k + (\partial_k F^{ik})\delta A_i]\,d\Omega
  \]

  Por la antisimetría, $\partial_k F^{ik} = -\partial_k F^{ki} = 0$, y el segundo término es idénticamente nulo. Por lo tanto:
  \[
  \delta S_F = \frac{1}{8\pi c}\int (\partial_i F^{ik})\delta A_k\,d\Omega
  \]

  Para que $\delta S_F = 0$ para variaciones arbitrarias $\delta A_k$, debemos tener:
  \[
  \partial_i F^{ik} = 0
  \]

  Estas son las ecuaciones de Maxwell en ausencia de fuentes (cargas y corrientes), específicamente el primer par (ley de Gauss y ley de Ampère-Maxwell sin corrientes).

  Para obtener el segundo par de ecuaciones de Maxwell, debemos usar la identidad de Bianchi, que se deriva directamente de la definición del tensor electromagnético en términos de potenciales:

  \[
  \partial_i F_{jk} + \partial_j F_{ki} + \partial_k F_{ij} = 0
  \]

  Esta identidad se cumple automáticamente debido a la definición $F_{ik} = \partial_i A_k - \partial_k A_i$, ya que:

  \begin{align*}
  \partial_i F_{jk} + \partial_j F_{ki} + \partial_k F_{ij} &= \partial_i(\partial_j A_k - \partial_k A_j) + \partial_j(\partial_k A_i - \partial_i A_k) + \partial_k(\partial_i A_j - \partial_j A_i) \\
  &= \partial_i\partial_j A_k - \partial_i\partial_k A_j + \partial_j\partial_k A_i - \partial_j\partial_i A_k + \partial_k\partial_i A_j - \partial_k\partial_j A_i
  \end{align*}

  Los términos se cancelan entre sí debido a la igualdad de las derivadas parciales mixtas.

  Esta identidad representa el segundo par de ecuaciones de Maxwell:
  - La ley de Faraday: $\nabla \times \mathbf{E} = -\frac{1}{c}\frac{\partial \mathbf{B}}{\partial t}$
  - La ley de Gauss para el magnetismo: $\nabla \cdot \mathbf{B} = 0$

  Estas ecuaciones no surgen del principio variacional, sino que son consecuencia de la estructura matemática del campo electromagnético como un campo derivado de potenciales.

  \item Considere una esfera de acero que rebota elásticamente entre dos paredes verticales paralelas sin fricción. Evalúe la variable de acción. Demustre la invariancia adiabática.
  
  \textbf{Solución:}  
  Sea $L$ la separación entre las paredes. La esfera de masa $m$ se mueve con velocidad $v=\sqrt{2E/m}$ entre rebotes, de modo que el momento $p=m\,v$ es constante. El movimiento unidimensional tiene periodo 
  \[
  T=\frac{2L}{v}\,.
  \]
  La variable de acción se define como
  \[
  I=\frac{1}{2\pi}\oint p\,dq
  =\frac{1}{2\pi}\bigl(2pL\bigr)
  =\frac{p\,L}{\pi}
  =\frac{L}{\pi}\sqrt{2mE}\,.
  \]
  Si ahora $L=L(t)$ cambia lentamente, $I$ se conserva (invariante adiabático):
  \[
  \frac{dI}{dt}=0
  \quad\Longrightarrow\quad
  E\propto\frac{1}{L^2}\,.
  \]
  Así queda demostrado que la acción $I$ permanece constante frente a variaciones adiabáticas de $L$.
  \item Considere una esfera de acero que salta sobre una placa horizontal sin fricción. Evalúe la variable de acción y exprese el resultado en unidades de \(\hbar\). Discuta la adaptación no clásica.
  
  \textbf{Solución:}
  La esfera de acero realiza un movimiento periódico vertical bajo la influencia del campo gravitatorio. Cuando la esfera rebota sobre la placa horizontal, su velocidad se invierte pero conserva su magnitud debido a que el choque es elástico y no hay fricción.

  \paragraph{1. Descripción del movimiento:}
  La esfera alcanza una altura máxima $h$ determinada por su energía total $E = mgh$. La velocidad varía entre $-v$ y $+v$, donde $v = \sqrt{2gh}$. El período del movimiento completo es $T = 2\sqrt{2h/g}$ (tiempo para subir y bajar).

  \paragraph{2. Cálculo de la variable de acción:}
  La variable de acción se define como:
  \[
  I = \frac{1}{2\pi}\oint p\,dq
  \]

  En nuestro caso unidimensional, $q$ representa la altura $y$ y $p$ el momento vertical $p_y = m\dot{y}$. Para un ciclo completo:
  \[
  I = \frac{1}{2\pi}\oint p_y\,dy
  \]

  El momento vertical se expresa en función de la energía y posición:
  \[
  p_y = \pm\sqrt{2m(E-mgy)}
  \]

  donde el signo cambia en el punto de altura máxima.

  Integrando sobre un ciclo completo (subida y bajada):
  \[
  I = \frac{1}{\pi}\int_0^h \sqrt{2m(mgh-mgy)}\,dy = \frac{1}{\pi}\int_0^h \sqrt{2m^2g(h-y)}\,dy
  \]

  Mediante el cambio de variable $u = h-y$:
  \[
  I = \frac{1}{\pi}\int_0^h \sqrt{2m^2gu}\,du = \frac{1}{\pi}\left.\frac{2}{3}(2m^2gu)^{3/2}\right|_0^h
  \]

  \[
  I = \frac{1}{\pi}\cdot\frac{2}{3}\sqrt{2m^2g}·h^{3/2} = \frac{2}{3\pi}m\sqrt{2gh}·h
  \]

  Usando $v = \sqrt{2gh}$ y $E = mgh$:
  \[
  I = \frac{2}{3\pi}mvh = \frac{2}{3\pi}\frac{mv·E}{mg} = \frac{2}{3\pi}\frac{mv^2}{g}·v
  \]

  Con $E = \frac{1}{2}mv^2$:
  \[
  I = \frac{2}{3\pi}\frac{mv^3}{g}
  \]

  \paragraph{3. Expresión en unidades de $\hbar$:}
  \[
  \frac{I}{\hbar} = \frac{2}{3\pi}\frac{mv^3}{g\hbar}
  \]

  \paragraph{4. Discusión de la adaptación no clásica:}
  Para valores típicos de una esfera de acero:
  \begin{itemize}
  \item $m \approx 10$ g $= 10^{-2}$ kg
  \item $v \approx 1$ m/s
  \item $g = 9.8$ m/s$^2$
  \item $\hbar = 1.05 \times 10^{-34}$ J$\cdot$s
  \end{itemize}

  Sustituyendo estos valores:
  \[
  \frac{I}{\hbar} \approx \frac{2}{3\pi}\frac{10^{-2} \times 1^3}{9.8 \times 1.05 \times 10^{-34}} \approx 10^{33}
  \]

  Este valor es extremadamente grande, lo que indica que el sistema está completamente en el régimen clásico.

  En la teoría cuántica, la variable de acción se cuantiza según la regla de Bohr-Sommerfeld:
  \[
  I = \left(n + \frac{1}{2}\right)\hbar
  \]

  donde $n$ es un entero. El número cuántico correspondiente a nuestro sistema sería del orden de $n \sim 10^{33}$, y la diferencia entre niveles de energía adyacentes sería:
  \[
  \Delta E = \frac{3\pi g\hbar}{2mv^2} \approx 10^{-33} E
  \]

  Estos cálculos confirman que los efectos cuánticos son completamente imperceptibles para este sistema macroscópico. Para observar efectos de cuantización, necesitaríamos considerar partículas microscópicas o temperaturas extremadamente bajas donde la longitud de onda de De Broglie se vuelve comparable con las dimensiones del sistema.

  \item Demuestre
    \[
      \{\mathbf{p},\,l^2\} \;=\; 2\,\mathbf{l}\times\mathbf{p},
    \]
    donde
    \[
      l^2 = l_x^2 + l_y^2 + l_z^2
    \]
    es la magnitud al cuadrado del momento angular de una partícula, 
    \(\mathbf{p}\) es el momento y \(\{\cdot,\cdot\}\) denota el corchete de Poisson.

  \textbf{Solución:}

  Para demostrar esta identidad, calcularemos los corchetes de Poisson componente a componente y luego verificaremos que el resultado es equivalente a $2\,\mathbf{l}\times\mathbf{p}$.

  Recordemos las expresiones para las componentes del momento angular:
  \begin{align}
  l_x &= yp_z - zp_y\\
  l_y &= zp_x - xp_z\\
  l_z &= xp_y - yp_x
  \end{align}

  Calcularemos $\{p_i, l^2\}$ para cada $i=x,y,z$. Usando la regla del producto para los corchetes de Poisson:
  \begin{align}
  \{p_i, l^2\} = \{p_i, l_x^2 + l_y^2 + l_z^2\} = 2l_x\{p_i, l_x\} + 2l_y\{p_i, l_y\} + 2l_z\{p_i, l_z\}
  \end{align}

  Calculemos primero los corchetes básicos $\{p_i, l_j\}$ utilizando las propiedades fundamentales:
  \begin{align}
  \{p_i, p_j\} &= 0\\
  \{p_i, q_j\} &= -\delta_{ij}
  \end{align}

  \underline{Cálculo de $\{p_x, l^2\}$}:

  Primero calculamos:
  \begin{align}
  \{p_x, l_x\} &= \{p_x, yp_z - zp_y\}\\
  &= y\{p_x,p_z\} + p_z\{p_x,y\} - z\{p_x,p_y\} - p_y\{p_x,z\}\\
  &= 0 + 0 - 0 - 0 = 0
  \end{align}

  \begin{align}
  \{p_x, l_y\} &= \{p_x, zp_x - xp_z\}\\
  &= z\{p_x,p_x\} + p_x\{p_x,z\} - x\{p_x,p_z\} - p_z\{p_x,x\}\\
  &= 0 + 0 - 0 - p_z(-1)\\
  &= p_z
  \end{align}

  \begin{align}
  \{p_x, l_z\} &= \{p_x, xp_y - yp_x\}\\
  &= x\{p_x,p_y\} + p_y\{p_x,x\} - y\{p_x,p_x\} - p_x\{p_x,y\}\\
  &= 0 + p_y(-1) - 0 - 0\\
  &= -p_y
  \end{align}

  Sustituyendo estos resultados:
  \begin{align}
  \{p_x, l^2\} &= 2l_x\cdot 0 + 2l_y \cdot p_z + 2l_z \cdot (-p_y)\\
  &= 2(l_yp_z - l_zp_y)
  \end{align}

  Esto corresponde exactamente a la componente $x$ del vector $2\mathbf{l}\times\mathbf{p}$.

  \underline{Cálculo de $\{p_y, l^2\}$}:

  De manera similar:
  \begin{align}
  \{p_y, l_x\} &= \{p_y, yp_z - zp_y\}\\
  &= y\{p_y,p_z\} + p_z\{p_y,y\} - z\{p_y,p_y\} - p_y\{p_y,z\}\\
  &= 0 + p_z(-1) - 0 - 0\\
  &= -p_z
  \end{align}

  \begin{align}
  \{p_y, l_y\} &= \{p_y, zp_x - xp_z\}\\
  &= z\{p_y,p_x\} + p_x\{p_y,z\} - x\{p_y,p_z\} - p_z\{p_y,x\}\\
  &= 0 + 0 - 0 - 0 = 0
  \end{align}

  \begin{align}
  \{p_y, l_z\} &= \{p_y, xp_y - yp_x\}\\
  &= x\{p_y,p_y\} + p_y\{p_y,x\} - y\{p_y,p_x\} - p_x\{p_y,y\}\\
  &= 0 + 0 - 0 - p_x(-1)\\
  &= p_x
  \end{align}

  Por lo tanto:
  \begin{align}
  \{p_y, l^2\} &= 2l_x \cdot (-p_z) + 2l_y \cdot 0 + 2l_z \cdot p_x\\
  &= 2(l_zp_x - l_xp_z)
  \end{align}

  Lo cual corresponde a la componente $y$ del vector $2\mathbf{l}\times\mathbf{p}$.

  \underline{Cálculo de $\{p_z, l^2\}$}:

  Finalmente:
  \begin{align}
  \{p_z, l_x\} &= \{p_z, yp_z - zp_y\}\\
  &= y\{p_z,p_z\} + p_z\{p_z,y\} - z\{p_z,p_y\} - p_y\{p_z,z\}\\
  &= 0 + 0 - 0 - p_y(-1)\\
  &= p_y
  \end{align}

  \begin{align}
  \{p_z, l_y\} &= \{p_z, zp_x - xp_z\}\\
  &= z\{p_z,p_x\} + p_x\{p_z,z\} - x\{p_z,p_z\} - p_z\{p_z,x\}\\
  &= 0 + p_x(-1) - 0 - 0\\
  &= -p_x
  \end{align}

  \begin{align}
  \{p_z, l_z\} &= \{p_z, xp_y - yp_x\}\\
  &= x\{p_z,p_y\} + p_y\{p_z,x\} - y\{p_z,p_x\} - p_x\{p_z,y\}\\
  &= 0 + 0 - 0 - 0 = 0
  \end{align}

  Por lo tanto:
  \begin{align}
  \{p_z, l^2\} &= 2l_x \cdot p_y + 2l_y \cdot (-p_x) + 2l_z \cdot 0\\
  &= 2(l_xp_y - l_yp_x)
  \end{align}

  Lo cual corresponde a la componente $z$ del vector $2\mathbf{l}\times\mathbf{p}$.

  Combinando estos resultados, obtenemos la relación vectorial:
  \[
  \{\mathbf{p}, l^2\} = 2\mathbf{l}\times\mathbf{p}
  \]

  quedando así demostrada la identidad.

  \item Considere el conjunto de todas las transformaciones canónicas. Suponga que están 
    definidas mediante funciones generatrices del tipo \(F\). Es decir,
    \[
      F_1(q,Q,t)\;\text{define la TC}\;(q,p)\to(Q,P), 
      \quad
      F_2(Q,\bar Q,t)\;\text{define la TC}\;(Q,P)\to(\bar Q,\bar P), 
    \]
    \[
      F_3(\bar Q,\bar{\bar Q},t)\;\text{define la TC}\;(\bar Q,\bar P)\to(\bar{\bar Q},\bar{\bar P}), 
      \;\dots
    \]
    Demuestre que ese conjunto tiene las propiedades de un grupo:
    \begin{enumerate}
      \item[(a)] Ley de composición interna: la realización sucesiva de dos TC equivale a una TC.
      \item[(b)] Existe la TC identidad (encuéntrela).
      \item[(c)] Cada TC tiene una TC inversa (encuéntrela).
      \item[(d)] Se cumple la propiedad asociativa.
    \end{enumerate}

  \textbf{Solución:}

  \paragraph{(a) Ley de composición interna:}

  Consideremos dos transformaciones canónicas:
  \begin{itemize}
      \item TC$_1$: $(q,p) \to (Q,P)$ generada por $F_1(q,Q,t)$
      \item TC$_2$: $(Q,P) \to (\bar{Q},\bar{P})$ generada por $F_2(Q,\bar{Q},t)$
  \end{itemize}

  Para TC$_1$, las relaciones canónicas son:
  \[
  p = \frac{\partial F_1}{\partial q}, \quad P = -\frac{\partial F_1}{\partial Q}
  \]

  Para TC$_2$, las relaciones canónicas son:
  \[
  P = \frac{\partial F_2}{\partial Q}, \quad \bar{P} = -\frac{\partial F_2}{\partial \bar{Q}}
  \]

  La composición TC$_2 \circ$ TC$_1$ lleva directamente de $(q,p)$ a $(\bar{Q},\bar{P})$. Definamos una nueva función generatriz:
  \[
  F_3(q,\bar{Q},t) = F_1(q,Q,t) + F_2(Q,\bar{Q},t) - QP
  \]

  donde $Q$ y $P$ se determinan implícitamente a través de las relaciones canónicas. Al derivar $F_3$ respecto a $q$ y $\bar{Q}$:
  \[
  \frac{\partial F_3}{\partial q} = \frac{\partial F_1}{\partial q} = p
  \]
  \[
  \frac{\partial F_3}{\partial \bar{Q}} = \frac{\partial F_2}{\partial \bar{Q}} = -\bar{P}
  \]

  Por lo tanto, $F_3$ genera correctamente la transformación canónica compuesta, mostrando que la composición de dos TC es otra TC.

  \paragraph{(b) Existencia del elemento identidad:}

  La TC identidad debe satisfacer $(q,p) \to (q,p)$. Esta transformación se genera mediante:
  \[
  F_{id}(q,Q,t) = qQ
  \]

  Con esta función generatriz:
  \[
  p = \frac{\partial F_{id}}{\partial q} = Q
  \]
  \[
  P = -\frac{\partial F_{id}}{\partial Q} = -q
  \]

  Si hacemos $Q = p$ y $P = -q$, obtenemos una transformación canónica que es equivalente a la identidad. Aunque las variables $Q$ y $P$ tienen nombres distintos a $q$ y $p$, las ecuaciones canónicas producidas son idénticas.

  Alternativamente, podemos considerar la transformación identidad como el caso donde $Q = q$ y $P = p$, generada por:
  \[
  F_{id}(q,Q,t) = \sum_i q_iP_i
  \]

  \paragraph{(c) Existencia del elemento inverso:}

  Para cada TC definida por $F(q,Q,t)$ que transforma $(q,p) \to (Q,P)$, su inversa debe llevar $(Q,P) \to (q,p)$.

  La función generatriz para la transformación inversa es:
  \[
  F^{-1}(Q,q,t) = -F(q,Q,t)
  \]

  Verificando:
  \[
  P = \frac{\partial F^{-1}}{\partial Q} = -\frac{\partial F}{\partial Q} = P
  \]
  \[
  p = -\frac{\partial F^{-1}}{\partial q} = \frac{\partial F}{\partial q} = p
  \]

  Esta transformación efectivamente revierte la original, demostrando que cada TC tiene una inversa que también es una TC.

  \paragraph{(d) Propiedad asociativa:}

  Consideremos tres TC consecutivas:
  \[
  \text{TC}_1: (q,p) \to (Q,P)
  \]
  \[
  \text{TC}_2: (Q,P) \to (\bar{Q},\bar{P})
  \]
  \[
  \text{TC}_3: (\bar{Q},\bar{P}) \to (\bar{\bar{Q}},\bar{\bar{P}})
  \]

  La propiedad asociativa requiere que:
  \[
  (\text{TC}_3 \circ \text{TC}_2) \circ \text{TC}_1 = \text{TC}_3 \circ (\text{TC}_2 \circ \text{TC}_1)
  \]

  Esta propiedad se cumple naturalmente porque la composición de funciones es asociativa. Las transformaciones canónicas son operadores que actúan sobre el espacio de fases, y la aplicación secuencial de estos operadores sigue la regla asociativa de la composición de funciones. Es decir, no importa si primero aplicamos TC$_1$ y luego la composición de TC$_3 \circ$ TC$_2$, o si primero aplicamos la composición de TC$_2 \circ$ TC$_1$ y luego TC$_3$ - el resultado final es el mismo.

  Por tanto, queda demostrado que el conjunto de todas las transformaciones canónicas forma un grupo matemático.

  \item Calcule la acción para una partícula en caída libre como función de las coordenadas. 
      Compruebe que coincide con el generador de la transformación canónica de evolución temporal 
      e implemente dicha transformación.

  \textbf{Solución:}
  Consideremos una partícula de masa $m$ que se mueve en un campo gravitacional uniforme con aceleración $g$ en dirección vertical. Tomamos el eje $y$ positivo hacia arriba.

  \paragraph{1. Cálculo de la acción:}

  El lagrangiano del sistema es:
  \[
  L = \frac{1}{2}m\dot{x}^2 + \frac{1}{2}m\dot{y}^2 - mgy
  \]

  La ecuación de movimiento para la coordenada $x$ da $\ddot{x}=0$, lo que implica $x(t) = x_0 + v_x(t-t_0)$, mientras que para $y$ obtenemos $\ddot{y}=-g$, por lo que $y(t) = y_0 + v_y(t-t_0) - \frac{1}{2}g(t-t_0)^2$.

  Calculemos la acción entre los instantes $t_0$ y $t$:
  \begin{align}
  S(x,y,t;x_0,y_0,t_0) &= \int_{t_0}^{t} L\,d\tau \\
  &= \int_{t_0}^{t} \left[\frac{1}{2}m\dot{x}^2 + \frac{1}{2}m\dot{y}^2 - mgy\right]\,d\tau
  \end{align}

  Sustituyendo las velocidades:
  \begin{align}
  \dot{x} &= v_x = \frac{x-x_0}{t-t_0} \\
  \dot{y} &= v_y - g(t-\tau) = \frac{y-y_0}{t-t_0} + \frac{1}{2}g(t-t_0) - g(t-\tau)
  \end{align}

  Después de integrar y simplificar, obtenemos:
  \begin{align}
  S(x,y,t;x_0,y_0,t_0) = \frac{m}{2(t-t_0)}[(x-x_0)^2 + (y-y_0)^2] + \frac{mg}{2}(y+y_0)(t-t_0) - \frac{mg^2}{24}(t-t_0)^3
  \end{align}

  \paragraph{2. Verificación como generador de transformación canónica:}

  Para comprobar que esta acción coincide con el generador de una transformación canónica de evolución temporal, calculamos:
  \begin{align}
  p_x &= \frac{\partial S}{\partial x} = \frac{m(x-x_0)}{t-t_0} \\
  p_y &= \frac{\partial S}{\partial y} = \frac{m(y-y_0)}{t-t_0} + \frac{mg(t-t_0)}{2}
  \end{align}

  Estas expresiones coinciden con los momentos físicos: $p_x = m\dot{x}$ y $p_y = m\dot{y}$.

  Además, verificamos:
  \begin{align}
  -\frac{\partial S}{\partial t} &= \frac{m[(x-x_0)^2+(y-y_0)^2]}{2(t-t_0)^2} - \frac{mg(y+y_0)}{2} + \frac{mg^2(t-t_0)^2}{8} \\
  &= \frac{p_x^2+p_y^2}{2m} + mgy = H
  \end{align}

  Esto confirma que $-\partial S/\partial t$ es el hamiltoniano.

  \paragraph{3. Implementación de la transformación canónica:}

  La acción $S$ genera una transformación canónica de $(x_0,y_0,p_{x0},p_{y0})$ a $(x,y,p_x,p_y)$ mediante:
  \begin{align}
  p_{x0} &= -\frac{\partial S}{\partial x_0} = \frac{m(x-x_0)}{t-t_0} = p_x \\
  p_{y0} &= -\frac{\partial S}{\partial y_0} = \frac{m(y-y_0)}{t-t_0} - \frac{mg(t-t_0)}{2} = p_y - mg(t-t_0)
  \end{align}

  La transformación canónica queda:
  \begin{align}
  x &= x_0 + \frac{p_{x0}}{m}(t-t_0) \\
  y &= y_0 + \frac{p_{y0}}{m}(t-t_0) - \frac{1}{2}g(t-t_0)^2 \\
  p_x &= p_{x0} \\
  p_y &= p_{y0} + mg(t-t_0)
  \end{align}

  Esta transformación describe la evolución del sistema desde el estado inicial $(x_0,y_0,p_{x0},p_{y0})$ al instante $t_0$ hasta el estado $(x,y,p_x,p_y)$ al instante $t$. La conservación de $p_x$ y la variación lineal de $p_y$ con el tiempo concuerdan con las propiedades físicas del movimiento bajo gravedad constante.
\end{enumerate}


\section{Teoría Clásica de Campos}
\subsection{Ejercicio 1: Campo Escalar}
Estudie el campo escalar \(\phi(x)\) a partir del lagrangiano:
\[
\mathcal{L} = \frac{1}{2}\partial_\mu\phi\,\partial^\mu\phi - V(\phi).
\]
Derive las ecuaciones de movimiento correspondientes usando el formalismo de Euler-Lagrange.

\subsection{Ejercicio 2: Campo Electromagnético}
Exponga las ecuaciones de Maxwell en el contexto del campo electromagnético y su representación en términos del tensor electromagnético.

\end{document}