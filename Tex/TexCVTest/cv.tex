\documentclass[10pt, letterpaper]{article}

% Packages:
\usepackage[
    ignoreheadfoot,
    top=1.3 cm,
    bottom=1.3 cm,
    left=1.6 cm,
    right=1.6 cm,
    footskip=1.0 cm,
]{geometry}
\usepackage{titlesec}
\usepackage{tabularx}
\usepackage{array}
\usepackage[dvipsnames]{xcolor}
\definecolor{primaryColor}{RGB}{0, 0, 0}
\usepackage{enumitem}
\usepackage{fontawesome5}
\usepackage{amsmath}
\usepackage[
    pdftitle={Juan's CV},
    pdfauthor={Juan},
    pdfcreator={LaTeX with RenderCV},
    colorlinks=true,
    urlcolor=primaryColor
]{hyperref}
\usepackage[pscoord]{eso-pic}
\usepackage{calc}
\usepackage{bookmark}
\usepackage{lastpage}
\usepackage{changepage}
\usepackage{paracol}
\usepackage{ifthen}
\usepackage{needspace}
\usepackage{iftex}

% Ensure that generated pdf is machine readable/ATS parsable:
\ifPDFTeX
    \input{glyphtounicode}
    \pdfgentounicode=1
    \usepackage[T1]{fontenc}
    \usepackage[utf8]{inputenc}
    \usepackage{lmodern}
\fi

\usepackage{charter}

% Some settings:
\raggedright
\AtBeginEnvironment{adjustwidth}{\partopsep0pt}
\pagestyle{empty}
\setcounter{secnumdepth}{0}
\setlength{\parindent}{0pt}
\setlength{\topskip}{0pt}
\setlength{\columnsep}{0.15cm}
\pagenumbering{gobble}

\titleformat{\section}{\needspace{4\baselineskip}\bfseries\large}{}{0pt}{}[\vspace{1pt}\titlerule]
\titlespacing{\section}{
    -1pt
}{ 
    0.3 cm
}{  
    0.2 cm
} 

\renewcommand\labelitemi{$\vcenter{\hbox{\small$\bullet$}}$}

\newenvironment{highlights}{
    \begin{itemize}[
        topsep=0.10 cm,
        parsep=0.10 cm,
        partopsep=0pt,
        itemsep=0pt,
        leftmargin=0 cm + 10pt
    ]
}{
    \end{itemize}
}

\newenvironment{highlightsforbulletentries}{
    \begin{itemize}[
        topsep=0.10 cm,
        parsep=0.10 cm,
        partopsep=0pt,
        itemsep=0pt,
        leftmargin=10pt
    ]
}{
    \end{itemize}
}

\newenvironment{onecolentry}{
    \begin{adjustwidth}{
        0 cm + 0.00001 cm
    }{
        0 cm + 0.00001 cm
    }
}{
    \end{adjustwidth}
}

\newenvironment{twocolentry}[2][]{
    \onecolentry
    \def\secondColumn{#2}
    \setcolumnwidth{\fill, 4.5 cm}
    \begin{paracol}{2}
}{
    \switchcolumn \raggedleft \secondColumn
    \end{paracol}
    \endonecolentry
}

\newenvironment{threecolentry}[3][]{
    \onecolentry
    \def\thirdColumn{#3}
    \setcolumnwidth{, \fill, 4.5 cm}
    \begin{paracol}{3}
    {\raggedright #2} \switchcolumn
}{
    \switchcolumn \raggedleft \thirdColumn
    \end{paracol}
    \endonecolentry
}

\newenvironment{header}{
    \setlength{\topsep}{0pt}\par\kern\topsep\centering\linespread{1.5}
}{
    \par\kern\topsep
}

\newcommand{\placelastupdatedtext}{%
  \AddToShipoutPictureFG*{%
    \put(
        \LenToUnit{\paperwidth-2 cm-0 cm+0.05cm},
        \LenToUnit{\paperheight-1.0 cm}
    ){\vtop{{\null}\makebox[0pt][c]{%
        \small\color{gray}\textit{Last updated in September 2024}\hspace{\widthof{Last updated in September 2024}}
    }}}%
  }%
}

% save the original href command in a new command:
\let\hrefWithoutArrow\href

\begin{document}
    \newcommand{\AND}{\unskip
        \cleaders\copy\ANDbox\hskip\wd\ANDbox
        \ignorespaces
    }
    \newsavebox\ANDbox
    \sbox\ANDbox{$|$}

    \begin{header}
        \fontsize{16 pt}{16 pt}\selectfont Juan Montoya Sanchez

        \vspace{1 pt}

        \normalsize
        \mbox{Medellín, Colombia}%
        \kern 5.0 pt%
        \AND%
        \kern 5.0 pt%
        \mbox{\href{mailto:juan.montoya110@udea.edu.co}{juan.montoya110@udea.edu.co}}%
        \kern 5.0 pt%
        \AND%
        \kern 5.0 pt%
        \mbox{\hrefWithoutArrow{tel:+57 300 366 8854}{+57 300 366 8854}}%
        \kern 5.0 pt%
        \AND%
        \kern 5.0 pt%        
        \mbox{\hrefWithoutArrow{https://orcid.org/0009-0006-6739-8449}{ORCID: 0009-0006-6739-8449}}%
        \kern 5.0 pt%
        \AND%
        \kern 5.0 pt%
        \mbox{\hrefWithoutArrow{https://www.linkedin.com/in/juan-montoya-68262071/}{linkedin.com/in/juan-montoya}}%
        \kern 5.0 pt%
        \AND%
        \kern 5.0 pt%
        \mbox{\hrefWithoutArrow{https://github.com/JuanJ27}{github.com/JuanJ27}}%
    \end{header}

    \vspace{3 pt - 0.3 cm}

    \section{Profile}
    \begin{onecolentry}
        Enthusiastic physics student pursuing a bachelor’s degree in physics, interested in t   he CERN Summer Student Programme. Hands-on experience in experimental and computational techniques, including data analysis, software development (C++, Python), and collaborative research in fundamental science projects.
    \end{onecolentry}

    \section{Education}
    \begin{twocolentry}{
        2019 – \textit{Expected} 2026.
    }
        \textbf{Universidad de Antioquia}
    \end{twocolentry}

    \vspace{0.10 cm}
    \begin{onecolentry}
        \begin{highlights}
            \item Bachelor’s degree in Physics \hfill GPA: 3.8/5.0
            \item \textbf{Relevant coursework}: Big Data in the Cern and Other Contexts, Introduction to fundamental particle physics, Theoretical mechanics, Experimental physics and Computational physics.
        \end{highlights}
    \end{onecolentry}

    \section{Research Experience}
    \begin{twocolentry}{
        2024 – Present.
    }
        \textbf{Undergraduate Research Assistant}, Phenomenology and Fundamental Interactions Group (GFIF) --- Universidad de Antioquia
    \end{twocolentry}

    \vspace{0.10 cm}
    \begin{onecolentry}
        \begin{highlights}
            \item Developed a C++ ROOT script to read branches from \texttt{.root} files, perform necessary calculations, and save the results as plots. This script aimed to characterize the geometric and energetic properties of \(b\)-jets and \(\bar{b}\)-jets at low \(p_T\) (< 30 GeV). This project can be found on \href{https://github.com/JuanJ27/Btagginghep}{\textbf{GitHub}} were my main focus is found in the \textit{V1} folder.
        \end{highlights}
    \end{onecolentry}

    \vspace{0.2 cm}
    \begin{twocolentry}{
        2023 – 2024.
    }
        \textbf{Research Intern}, Condensed Matter Group --- Universidad de Antioquia
    \end{twocolentry}

    \vspace{0.10 cm}
    \begin{onecolentry}
        \begin{highlights}
            \item Contributed to research on quantum dots, focusing on their electronic and optical properties under external fields:
            \begin{itemize}
                \item \href{https://doi.org/10.1140/epjp/s13360-024-05089-z}{\textbf{Electronic and optical properties of tetrapod quantum dots under applied electric and magnetic fields} \\ \textit{European Physical Journal Plus, 2024}}
                \begin{itemize}
                    \item \textbf{My contribution}: Ran half of the COMSOL simulations and exported both numerical and graphical data. Processed simulation outputs in OriginLab, improved figure clarity and references in Overleaf with \LaTeX, and created final figures in Inkscape.
                \end{itemize}

                \item \href{https://doi.org/10.1016/j.physe.2024.116032}{\textbf{Hopf-link GaAs-AlGaAs quantum ring under geometric and external field settings} \\ \textit{Physica E: Low-Dimensional Systems and Nanostructures, 2024}}
                \begin{itemize}
                    \item \textbf{My contribution}: Verified the correct implementation of the potential model in COMSOL and Python. Adjusted the manuscript format in Overleaf to meet the journal’s guidelines.
                \end{itemize}
            \end{itemize}
        \end{highlights}
    \end{onecolentry}

    \section{Conferences \& Presentations}
    \begin{twocolentry}{
        Pasto, December 2024.
    }
        \textbf{9\textsuperscript{th} Colombian Meeting on High Energy Physics (COMHEP)}
    \end{twocolentry}

    \vspace{0.10 cm}
    \begin{onecolentry}
        \begin{highlights}
            \item Oral Presentation: \textit{Systematic Study of the Structure of \(b\)-Jets and \(\bar{b}\)-Jets at Low \(p_T\)} (< 30 GeV). Presented the results of the C++ ROOT script developed during my undergraduate research assistantship.
            \item One of the leaders at the CMS Masterclass activity in Pasto on December 3, 2024. I was responsible for explaining to the attendees how to classify events using graphical tools.
        \end{highlights}
    \end{onecolentry}

    \vspace{0.2 cm}
    \begin{twocolentry}{
        Ibagué, December 2023.
    }
        \textbf{ICTP Physics Without Frontiers: Colombian Network for High Energy Physics School}
    \end{twocolentry}

    \vspace{0.10 cm}
    \begin{onecolentry}
        \begin{highlights}
            \item Attended theoretical and experimental HEP lectures, covering tools such as MadGraph5, applications of neural networks for Higgs signal and background discrimination, and Compton scattering.
            \item Collaboratively developed a neural network for Higgs signal and background discrimination, where I was responsible for cross-validation. After the school, I attended the \textbf{8\textsuperscript{th} COMHEP in 2023}.
        \end{highlights}
    \end{onecolentry}

    \section{Personal Projects}
    \begin{twocolentry}{
        Medellín, November 2024.
    }
        \textbf{United Nations Datathon 2024 -- Sustainable Tourism Analysis} 
        \textit{\href{https://github.com/JuanJ27/UN-Datathon-sisifos}{GitHub link}}
    \end{twocolentry}

    \vspace{0.10 cm}
    \begin{onecolentry}
        \begin{highlights}
            \item Contributed to collecting data, cleaning and preprocessing it, and developing a Python script to analyze the impact of tourism on Medellín. Utilized \texttt{GeoPandas} and \texttt{Plotly} to visualize results in an interactive map.
        \end{highlights}
    \end{onecolentry}

    \begin{twocolentry}{
        Medellín, October 2024.
    }
        \textbf{NASA Space Apps Challenge 2024 -- Community Mapping} 
        \textit{\href{https://github.com/tonnysoyyo/NASA-Space-Apps}{GitHub link}}
    \end{twocolentry}

    \vspace{0.10 cm}
    \begin{onecolentry}
        \begin{highlights}
            \item This project aimed to create a clear representation of socioeconomic conditions in Medellín. My role involved gathering data, cleaning it, and exporting it as GeoJSON files so my teammates could integrate it into a web app.
        \end{highlights}
    \end{onecolentry}

\end{document}
