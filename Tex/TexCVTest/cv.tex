\documentclass[10pt, letterpaper]{article}

% Packages:
\usepackage[
    ignoreheadfoot, % set margins without considering header and footer
    top=2 cm, % seperation between body and page edge from the top
    bottom=2 cm, % seperation between body and page edge from the bottom
    left=2 cm, % seperation between body and page edge from the left
    right=2 cm, % seperation between body and page edge from the right
    footskip=1.0 cm, % seperation between body and footer
    % showframe % for debugging 
]{geometry} % for adjusting page geometry
\usepackage{titlesec} % for customizing section titles
\usepackage{tabularx} % for making tables with fixed width columns
\usepackage{array} % tabularx requires this
\usepackage[dvipsnames]{xcolor} % for coloring text
\definecolor{primaryColor}{RGB}{0, 0, 0} % define primary color
\usepackage{enumitem} % for customizing lists
\usepackage{fontawesome5} % for using icons
\usepackage{amsmath} % for math
\usepackage[
    pdftitle={Juan's CV},
    pdfauthor={Juan},
    pdfcreator={LaTeX with RenderCV},
    colorlinks=true,
    urlcolor=primaryColor
]{hyperref} % for links, metadata and bookmarks
\usepackage[pscoord]{eso-pic} % for floating text on the page
\usepackage{calc} % for calculating lengths
\usepackage{bookmark} % for bookmarks
\usepackage{lastpage} % for getting the total number of pages
\usepackage{changepage} % for one column entries (adjustwidth environment)
\usepackage{paracol} % for two and three column entries
\usepackage{ifthen} % for conditional statements
\usepackage{needspace} % for avoiding page brake right after the section title
\usepackage{iftex} % check if engine is pdflatex, xetex or luatex

% Ensure that generate pdf is machine readable/ATS parsable:
\ifPDFTeX
    \input{glyphtounicode}
    \pdfgentounicode=1
    \usepackage[T1]{fontenc}
    \usepackage[utf8]{inputenc}
    \usepackage{lmodern}
\fi

\usepackage{charter}

% Some settings:
\raggedright
\AtBeginEnvironment{adjustwidth}{\partopsep0pt} % remove space before adjustwidth environment
\pagestyle{empty} % no header or footer
\setcounter{secnumdepth}{0} % no section numbering
\setlength{\parindent}{0pt} % no indentation
\setlength{\topskip}{0pt} % no top skip
\setlength{\columnsep}{0.15cm} % set column seperation
\pagenumbering{gobble} % no page numbering

\titleformat{\section}{\needspace{4\baselineskip}\bfseries\large}{}{0pt}{}[\vspace{1pt}\titlerule]

\titlespacing{\section}{
    % left space:
    -1pt
}{
    % top space:
    0.3 cm
}{
    % bottom space:
    0.2 cm
} % section title spacing

\renewcommand\labelitemi{$\vcenter{\hbox{\small$\bullet$}}$} % custom bullet points
\newenvironment{highlights}{
    \begin{itemize}[
        topsep=0.10 cm,
        parsep=0.10 cm,
        partopsep=0pt,
        itemsep=0pt,
        leftmargin=0 cm + 10pt
    ]
}{
    \end{itemize}
} % new environment for highlights


\newenvironment{highlightsforbulletentries}{
    \begin{itemize}[
        topsep=0.10 cm,
        parsep=0.10 cm,
        partopsep=0pt,
        itemsep=0pt,
        leftmargin=10pt
    ]
}{
    \end{itemize}
} % new environment for highlights for bullet entries

\newenvironment{onecolentry}{
    \begin{adjustwidth}{
        0 cm + 0.00001 cm
    }{
        0 cm + 0.00001 cm
    }
}{
    \end{adjustwidth}
} % new environment for one column entries

\newenvironment{twocolentry}[2][]{
    \onecolentry
    \def\secondColumn{#2}
    \setcolumnwidth{\fill, 4.5 cm}
    \begin{paracol}{2}
}{
    \switchcolumn \raggedleft \secondColumn
    \end{paracol}
    \endonecolentry
} % new environment for two column entries

\newenvironment{threecolentry}[3][]{
    \onecolentry
    \def\thirdColumn{#3}
    \setcolumnwidth{, \fill, 4.5 cm}
    \begin{paracol}{3}
    {\raggedright #2} \switchcolumn
}{
    \switchcolumn \raggedleft \thirdColumn
    \end{paracol}
    \endonecolentry
} % new environment for three column entries

\newenvironment{header}{
    \setlength{\topsep}{0pt}\par\kern\topsep\centering\linespread{1.5}
}{
    \par\kern\topsep
} % new environment for the header

\newcommand{\placelastupdatedtext}{% \placetextbox{<horizontal pos>}{<vertical pos>}{<stuff>}
  \AddToShipoutPictureFG*{% Add <stuff> to current page foreground
    \put(
        \LenToUnit{\paperwidth-2 cm-0 cm+0.05cm},
        \LenToUnit{\paperheight-1.0 cm}
    ){\vtop{{\null}\makebox[0pt][c]{
        \small\color{gray}\textit{Last updated in September 2024}\hspace{\widthof{Last updated in September 2024}}
    }}}%
  }%
}%

% save the original href command in a new command:
\let\hrefWithoutArrow\href

% new command for external links:


\begin{document}
    \newcommand{\AND}{\unskip
        \cleaders\copy\ANDbox\hskip\wd\ANDbox
        \ignorespaces
    }
    \newsavebox\ANDbox
    \sbox\ANDbox{$|$}

    \begin{header}
        \fontsize{25 pt}{25 pt}\selectfont Juan Montoya Sanchez

        \vspace{4 pt}

        \normalsize
        \mbox{Medellín, Colombia}%
        \kern 5.0 pt%
        \AND%
        \kern 5.0 pt%
        \mbox{\href{mailto:juan.montoya110@udea.edu.co}{juan.montoya110@udea.edu.co}}%
        \kern 5.0 pt%
        \AND%
        \kern 5.0 pt%
        \mbox{\hrefWithoutArrow{tel:+57 300 366 8854}{+57 300 366 8854}}%
        \kern 5.0 pt%
        \AND%
        \kern 5.0 pt%        

        \mbox{\hrefWithoutArrow{https://orcid.org/0009-0006-6739-8449}{ORCID: 0009-0006-6739-8449}}%
        \kern 5.0 pt%
        \AND%
        \kern 5.0 pt%
        \mbox{\hrefWithoutArrow{https://www.linkedin.com/in/juan-montoya-68262071/}{linkedin.com/in/juan-montoya}}%
        \kern 5.0 pt%
        \AND%
        \kern 5.0 pt%
        \mbox{\hrefWithoutArrow{https://github.com/JuanJ27}{github.com/JuanJ27}}%
    \end{header}

    \vspace{5 pt - 0.3 cm}


    \section{Profile}
        
        \begin{onecolentry}
            Enthusiastic Physics student interested in High-Energy Physics (HEP). Hands-on experience in data analysis, software development (C++, Python), and collaborative research in an academic group affiliated with the CMS experiment at CERN. Keen to apply theoretical knowledge to real-world research projects and contribute to cutting-edge scientific discoveries.
        \end{onecolentry}
    
    \section{Education}

        
        \begin{twocolentry}{
            2019 – \textit{Expected} 2026.
        }
            \textbf{Universidad de Antioquia} \end{twocolentry}

        \vspace{0.10 cm}
        \begin{onecolentry}
            \begin{highlights}
                \item Bachelor's degree in Physics \hfill GPA: 3.8/5.0
            \end{highlights}
        \end{onecolentry}



    
    \section{Research experience}



        
        \begin{twocolentry}{
            2024 – Present.
        }
            \textbf{Undergraduate Research Assistant}, Phenomenology and Fundamental Interactions Group (GFIF)\end{twocolentry}

        \vspace{0.10 cm}
        \begin{onecolentry}
            \begin{highlights}
                \item Collaborate on low-$p_T$ ($<30$ GeV) $b$-jet analyses using C++ in conjunction with the CMS experiment at CERN.
            \end{highlights}
        \end{onecolentry}


        \vspace{0.2 cm}

        \begin{twocolentry}{
            2023 – Present.
        }
            \textbf{Research Intern}, Condensed Matter Group\end{twocolentry}

        \vspace{0.10 cm}
        \begin{onecolentry}
            \begin{highlights}
                    \item Conduct research on quantum dots and quantum rings, focusing on their electronic and optical properties under external fields. Co-authored two peer-reviewed publications:
                        \begin{itemize}
        \item \textbf{Electronic and optical properties of tetrapod quantum dots under applied electric and magnetic fields} \\
        European Physical Journal Plus, 2024 \\
        \href{https://doi.org/10.1140/epjp/s13360-024-05089-z}{DOI: 10.1140/epjp/s13360-024-05089-z}
        \item \textbf{Hopf-link GaAs-AlGaAs quantum ring under geometric and external field settings} \\
        Physica E: Low-Dimensional Systems and Nanostructures, 2024 \\
        \href{https://doi.org/10.1016/j.physe.2024.116032}{DOI: 10.1016/j.physe.2024.116032}
    \end{itemize}
            \end{highlights}
        \end{onecolentry}


    
    \section{Conferences \& Presentations}



        
        \begin{twocolentry}{
            Pasto, December 2024.
        }
            \textbf{9\textsuperscript{th} Colombian Meeting on High Energy Physics (COMHEP)}\end{twocolentry}

        \vspace{0.10 cm}
        \begin{onecolentry}
            \begin{highlights}
                \item Oral Presentation: \textit{Estudio sistemático de la estructura de jets de $b$ y $\bar{b}$ a bajo $p_T$}.
            \end{highlights}
        \end{onecolentry}


        \vspace{0.2 cm}

        \begin{twocolentry}{
            Ibagué, December 2023.
        }
            \textbf{ICTP Physics Without Frontiers: Colombian Network for High Energy Physics School}\end{twocolentry}

        \vspace{0.10 cm}
        \begin{onecolentry}
            \begin{highlights}
                \item Attended a specialized school focused on theoretical and experimental high-energy physics. Participated in workshops and lectures led by renowned international researchers.
            \end{highlights}
        \end{onecolentry}


        \vspace{0.2 cm}
    
    \section{Projects}
        
        \begin{twocolentry}{
            Medellín, November 2024.
        }
            \textbf{United Nations Datathon 2024 -- Sustainable Tourism Analysis} \textit{\href{https://github.com/JuanJ27/UN-Datathon-sisifos}{GitHub}}\end{twocolentry}

        \vspace{0.10 cm}
        \begin{onecolentry}
            \begin{highlights}
                \item Analyzed large tourism datasets to highlight sustainability metrics. Created interactive data visualizations using Python libraries (\texttt{geopandas}, \texttt{plotly}) for global insights.
            \end{highlights}
        \end{onecolentry}

        \begin{twocolentry}{
            Medellín, October 2024.
        }
            \textbf{NASA Space Apps Challenge 2024 -- Galactic Problem Solver} \textit{\href{https://github.com/tonnysoyyo/NASA-Space-Apps}{GitHub}}\end{twocolentry}

        \vspace{0.10 cm}
        \begin{onecolentry}
            \begin{highlights}
                \item Awarded the “Galactic Problem Solver” certificate for outstanding participation. Developed innovative data visualization techniques representing climate patterns using Python. Collaborated with a multidisciplinary team to address challenges in space and Earth-related contexts.
            \end{highlights}
        \end{onecolentry}

        \vspace{0.2 cm}

    \section{Tutoring Experience}
        
        \begin{twocolentry}{
            November 2024 -- Present.
        }
            \textbf{Tutor at Tutor.com} \end{twocolentry}

        \vspace{0.10 cm}
        \begin{onecolentry}
            \begin{highlights}
                \item Provide online math and physics tutoring to students with diverse academic backgrounds. Tailor explanations to different learning styles, enhancing conceptual understanding.
            \end{highlights}
        \end{onecolentry}


\end{document}