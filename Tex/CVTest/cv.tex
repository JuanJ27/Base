\documentclass[10pt, letterpaper]{article}

% Packages:
\usepackage[
    ignoreheadfoot,
    top=1.3 cm,
    bottom=1.3 cm,
    left=1.6 cm,
    right=1.6 cm,
    footskip=1.0 cm,
]{geometry}
\usepackage{titlesec}
\usepackage{tabularx}
\usepackage{array}  
\usepackage[dvipsnames]{xcolor}
\definecolor{primaryColor}{RGB}{0, 0, 0}
\usepackage{enumitem}
\usepackage{fontawesome5}
\usepackage{amsmath}
\usepackage[
    pdftitle={Juan's CV},
    pdfauthor={Juan},
    pdfcreator={LaTeX with RenderCV},
    colorlinks=true,
    urlcolor=primaryColor
]{hyperref}
\usepackage[pscoord]{eso-pic}
\usepackage{calc}
\usepackage{bookmark}
\usepackage{lastpage}
\usepackage{changepage}
\usepackage{paracol}
\usepackage{ifthen}
\usepackage{needspace}
\usepackage{iftex}

% Ensure that generated pdf is machine readable/ATS parsable:
\ifPDFTeX
    \input{glyphtounicode}
    \pdfgentounicode=1
    \usepackage[T1]{fontenc}
    \usepackage[utf8]{inputenc}
    \usepackage{lmodern}
\fi

\usepackage{charter}

% Some settings:
\raggedright
\AtBeginEnvironment{adjustwidth}{\partopsep0pt}
\pagestyle{empty}
\setcounter{secnumdepth}{0}
\setlength{\parindent}{0pt}
\setlength{\topskip}{0pt}
\setlength{\columnsep}{0.15cm}
\pagenumbering{gobble}

\titleformat{\section}{\needspace{4\baselineskip}\bfseries\large}{}{0pt}{}[\vspace{1pt}\titlerule]
\titlespacing{\section}{
    -1pt
}{ 
    0.3 cm
}{  
    0.2 cm
} 

\renewcommand\labelitemi{$\vcenter{\hbox{\small$\bullet$}}$}

\newenvironment{highlights}{
    \begin{itemize}[
        topsep=0.10 cm,
        parsep=0.10 cm,
        partopsep=0pt,
        itemsep=0pt,
        leftmargin=0 cm + 10pt
    ]
}{
    \end{itemize}
}

\newenvironment{highlightsforbulletentries}{
    \begin{itemize}[
        topsep=0.10 cm,
        parsep=0.10 cm,
        partopsep=0pt,
        itemsep=0pt,
        leftmargin=10pt
    ]
}{
    \end{itemize}
}

\newenvironment{onecolentry}{
    \begin{adjustwidth}{
        0 cm + 0.00001 cm
    }{
        0 cm + 0.00001 cm
    }
}{
    \end{adjustwidth}
}

\newenvironment{twocolentry}[2][]{
    \onecolentry
    \def\secondColumn{#2}
    \setcolumnwidth{\fill, 4.5 cm}
    \begin{paracol}{2}
}{
    \switchcolumn \raggedleft \secondColumn
    \end{paracol}
    \endonecolentry
}

\newenvironment{threecolentry}[3][]{
    \onecolentry
    \def\thirdColumn{#3}
    \setcolumnwidth{, \fill, 4.5 cm}
    \begin{paracol}{3}
    {\raggedright #2} \switchcolumn
}{
    \switchcolumn \raggedleft \thirdColumn
    \end{paracol}
    \endonecolentry
}

\newenvironment{header}{
    \setlength{\topsep}{0pt}\par\kern\topsep\centering\linespread{1.5}
}{
    \par\kern\topsep
}

\newcommand{\placelastupdatedtext}{%
  \AddToShipoutPictureFG*{%
    \put(
        \LenToUnit{\paperwidth-2 cm-0 cm+0.05cm},
        \LenToUnit{\paperheight-1.0 cm}
    ){\vtop{{\null}\makebox[0pt][c]{%
        \small\color{gray}\textit{Last updated in September 2025}\hspace{\widthof{Last updated in September 2025}}
    }}}%
  }%
}

% save the original href command in a new command:
\let\hrefWithoutArrow\href

\begin{document}
    \newcommand{\AND}{\unskip
        \cleaders\copy\ANDbox\hskip\wd\ANDbox
        \ignorespaces
    }
    \newsavebox\ANDbox
    \sbox\ANDbox{$|$}

    \begin{header}
        \fontsize{16 pt}{16 pt}\selectfont Juan Montoya Sanchez

        \normalsize
        \today

        \vspace{1 pt} 
        \mbox{Medellín, Colombia}%
        \kern 5.0 pt%
        \AND%
        \kern 5.0 pt%
        \mbox{\href{mailto:juan.montoya110@udea.edu.co}{juan.montoya110@udea.edu.co}}%
        \kern 5.0 pt%
        \AND%
        \kern 5.0 pt%
        \mbox{\hrefWithoutArrow{tel:+57 300 366 8854}{+57 300 366 8854}}%
        \kern 5.0 pt%
        \AND%
        \kern 5.0 pt%        
        \mbox{\hrefWithoutArrow{https://orcid.org/0009-0006-6739-8449}{ORCID: 0009-0006-6739-8449}}%
        \kern 5.0 pt%
        \AND%
        \kern 5.0 pt%
        \mbox{\hrefWithoutArrow{https://www.linkedin.com/in/juan-montoya-68262071/}{linkedin.com/in/juan-montoya}}%
        \kern 5.0 pt%
        \AND%
        \kern 5.0 pt%
        \mbox{\hrefWithoutArrow{https://github.com/JuanJ27}{github.com/JuanJ27}}%
        \AND%
        \kern 5.0 pt%
        \mbox{\hrefWithoutArrow{https://www.scopus.com/authid/detail.uri?authorId=58973544700}{ScopusID: 58973544700}}%
    \end{header}

    \vspace{3 pt - 0.3 cm}

    \section{Areas of expertise}
    \begin{onecolentry}
        Physics undergraduate specializing in high-energy physics, quantum computing, machine learning, statistical data analysis, cernROOT, Geant4, MadGraph, Python, C++, PennyLane.
    \end{onecolentry}

    \section{Technical Skills}
    \begin{onecolentry}
        \begin{tabularx}{\textwidth}{@{}lX@{}}
            \textbf{Programming:} & C++, Python, BASH, MicroPython, \LaTeX, SQL \\
            \textbf{HEP Software:} & ROOT, Geant4, MadGraph5, uproot \\
            \textbf{ML \& Quantum:} & PyTorch, TensorFlow, scikit-learn, PennyLane \\
            \textbf{Scientific:} & NumPy, SciPy, Pandas, Matplotlib, Jupyter, COMSOL, OriginLab \\
            \textbf{Tools \& Systems:} & Git, Linux, Overleaf \\
        \end{tabularx}
    \end{onecolentry}

    \section{Education}
    \begin{twocolentry}{
        2019 – \textit{Expected} 2026
    }
        \textbf{Universidad de Antioquia} - Medellín, Colombia.
    \end{twocolentry}

    \vspace{0.10 cm}
    \begin{onecolentry}
        \begin{highlights}
            \item Bachelor’s degree in Physics \hfill GPA: 3.85/5.0
            \item \textbf{Relevant coursework}:  \begin{tabularx}{\textwidth}{@{}lX@{}}
        \textbf{Physics:} & Quantum Mechanics, Particle Physics, Statistical Mechanics, Electrodynamics \\
        \textbf{Computing:} & Computational Physics, Introduction to Quantum Computing, Big Data Analysis \\
        \textbf{Mathematics:} & Linear Algebra, Differential Equations, Numerical Analysis \\
        \textbf{Experimental:} & Electronics, Data Acquisition Systems, Signal Processing \\
    \end{tabularx}
        \end{highlights}
    \end{onecolentry}

    \section{Research Experience}
    \begin{twocolentry}{
        2024 – Present
    }
        \textbf{Undergraduate Research Assistant}, Phenomenology and Fundamental Interactions Group (GFIF) --- Universidad de Antioquia
    \end{twocolentry}
    \begin{highlights}
    \item Developed C++ ROOT scripts to characterize geometric and energetic properties of b-jets and \(\bar{b}\)-jets at low \(p_T\) (< 30 GeV), reading branches from \texttt{.root} files, performing necessary calculations, and generating visualization plots
    \item Project code available at \href{https://github.com/JuanJ27/Btagginghep}{\textbf{GitHub}}
    \end{highlights}

    \vspace{0.2 cm}
    \begin{twocolentry}{
        2023 – 2024
    }
        \textbf{Research Intern}, Condensed Matter Group --- Universidad de Antioquia
    \end{twocolentry}

    \vspace{0.10 cm}
    \begin{onecolentry}
        \begin{highlights}
            \item Contributed to research on quantum dots, focusing on their electronic and optical properties under external fields:
            \begin{itemize}

                \item \href{https://doi.org/10.1016/j.physleta.2025.130897}{\textbf{Structural modifications in GaAs/AlGaAs tetrapod nanocrystals under applied pressure and temperature: Electron-impurity properties} \\ \textit{Physics Letters, Section A: General, Atomic and Solid State Physics, 2025}}
                \begin{itemize}
                    \item \textbf{My contribution}: Performed temperature and pressure-dependent calculations for electron-impurity interactions. Analyzed structural modifications using COMSOL and developed visualization scripts for data representation. Collaborated on manuscript preparation and revision. Designed the final figures using Inkscape.
                \end{itemize}

                \item \href{https://doi.org/10.1140/epjp/s13360-024-05089-z}{\textbf{Electronic and optical properties of tetrapod quantum dots under applied electric and magnetic fields} \\ \textit{European Physical Journal Plus, 2024}}
                \begin{itemize}
                    \item \textbf{My contribution}: Ran half of the COMSOL simulations and exported both numerical and graphical data. Processed simulation outputs in OriginLab, improved figure clarity and references in Overleaf with \LaTeX, and created final figures in Inkscape.
                \end{itemize}

                \item \href{https://doi.org/10.1016/j.physe.2024.116032}{\textbf{Hopf-link GaAs-AlGaAs quantum ring under geometric and external field settings} \\ \textit{Physica E: Low-Dimensional Systems and Nanostructures, 2024}}
                \begin{itemize}
                    \item \textbf{My contribution}: Verified the correct implementation of the potential model in COMSOL and Python. Adjusted the manuscript format in Overleaf to meet the journal’s guidelines.
                \end{itemize}
            \end{itemize}
        \end{highlights}
    \end{onecolentry}

    \vspace{0.5 cm}
    \section{Conferences \& Presentations}
    \begin{twocolentry}{
        Pasto, December 2024
    }
        \textbf{9\textsuperscript{th} Colombian Meeting on High Energy Physics (COMHEP)}
    \end{twocolentry}

    \vspace{0.10 cm}
    \begin{onecolentry}
        \begin{highlights}
            \item Oral Presentation: \textit{Systematic Study of the Structure of \(b\)-Jets and \(\bar{b}\)-Jets at Low \(p_T\)} (< 30 GeV). Presented the results of the C++ ROOT script developed during my undergraduate research assistantship.
            \item One of the leaders at the CMS Masterclass activity in Pasto on December 3, 2024. I was responsible for explaining to the attendees how to classify events using graphical tools.
        \end{highlights}
    \end{onecolentry}

    \vspace{0.2 cm}
    \begin{twocolentry}{
        Ibagué, December 2023
    }
        \textbf{ICTP Physics Without Frontiers: Colombian Network for High Energy Physics School}
    \end{twocolentry}

    \vspace{0.10 cm}
    \begin{onecolentry}
        \begin{highlights}
            \item Attended theoretical and experimental HEP lectures, covering tools such as MadGraph5, applications of neural networks for Higgs signal and background discrimination, and Compton scattering.
            \item Collaboratively developed a neural network for Higgs signal and background discrimination, where I was responsible for cross-validation. After the school, I attended the \textbf{8\textsuperscript{th} COMHEP in 2023}.
        \end{highlights}
    \end{onecolentry}

    \section{Personal Projects}

    \begin{twocolentry}{
        2025.
    }
        \textbf{LowPt-Jet-Qml} --- Quantum Machine Learning for Particle Physics \textbf{\textit{\href{https://github.com/JuanJ27/LowPt-Jet-Qml}{GitHub link}}}
    \end{twocolentry}

    \vspace{0.10 cm}
    \begin{onecolentry}
        \begin{highlights}
            \item Implemented and compared Quantum Machine Learning techniques with classical ML for low-pT b-jet tagging
            \item Developed quantum neural networks using PennyLane with Angle Embedding for 16-qubit circuits
            \item Analyzed datasets from different PT ranges using uproot, evaluating models with AUC and tagging power metrics
        \end{highlights}
    \end{onecolentry}

    \begin{twocolentry}{
        2025.
    }
        \textbf{SensorArray} --- Experimental Physics Course Final Project \textbf{\textit{\href{https://github.com/JuanJ27/SensorArray-for-laser-Lissajous-curves}{GitHub link}}}
    \end{twocolentry}

    \vspace{0.10 cm}
    \begin{onecolentry}
        \begin{highlights}
            \item Designed and built a data acquisition system with multiple phototransistors for laser pattern detection
            \item Implemented MicroPython code for ESP32 microcontroller to collect and process sensor data
            \item Developed signal processing algorithms for real-time analysis of light patterns
        \end{highlights}
    \end{onecolentry}

    \begin{twocolentry}{
        2024.
    }
        \textbf{Biospeckle-ML} --- \textit{First Place Winner, Physics Experimental Fair} \textbf{\textit{\href{https://github.com/JuanJ27/Biospeckle-ML}{GitHub link}}}
    \end{twocolentry}

    \vspace{0.10 cm}
    \begin{onecolentry}
        \begin{highlights}
            \item Designed and built an optical system to capture biospeckle phenomenon in blueberries.
            \item Developed C++ algorithms with OpenCV for image processing and statistical pattern analysis
            \item Using scikit-learn and PyTorch, created a neural network classifier achieving 89\% accuracy in blueberry quality assessment
            \item Presented statistical validation of results with rigorous hypothesis testing and confidence intervals
        \end{highlights}
    \end{onecolentry}


    \begin{twocolentry}{
        Medellín, November 2024
    }
        \textbf{United Nations Datathon 2024}-- Sustainable Tourism Analysis, Second Place Winner in student category 
        \textbf{\textit{\href{https://github.com/JuanJ27/UN-Datathon-sisifos}{GitHub link}}}
    \end{twocolentry}

    \vspace{0.10 cm}
    \begin{onecolentry}
        \begin{highlights}
            \item Developed a data analysis pipeline for assessing tourism impact on Medellín's sustainable development
            \item Collected, cleaned and preprocessed multidimensional urban datasets using Python
            \item Created interactive visualizations with \texttt{GeoPandas} and \texttt{Plotly} to represent spatial patterns
            \item Collaborated in a multidisciplinary team to present findings to UN Development Program representatives
        \end{highlights}
    \end{onecolentry}

    \begin{twocolentry}{
        Medellín, October 2024
    }
        \textbf{NASA Space Apps Challenge 2024} -- Community Mapping
        \textbf{\textit{\href{https://github.com/tonnysoyyo/NASA-Space-Apps}{GitHub link}}}
    \end{twocolentry}

    \vspace{0.10 cm}
    \begin{onecolentry}
        \begin{highlights}
            \item Led data acquisition and processing for a socioeconomic mapping project of Medellín
            \item Gathered demographic, economic, and infrastructure data from multiple government sources
            \item Collaborated with frontend developers to ensure seamless data visualization in the final application
        \end{highlights}
    \end{onecolentry}

\end{document}
