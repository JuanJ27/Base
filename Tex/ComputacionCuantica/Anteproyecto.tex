% Anteproyecto.tex
\documentclass[12pt,a4paper]{article}
\usepackage[utf8]{inputenc}
\usepackage[T1]{fontenc}
\usepackage[spanish]{babel}
\usepackage{geometry}
\usepackage{graphicx}
\usepackage{hyperref}
\usepackage{booktabs}
\usepackage{array}
\usepackage{lmodern}

\geometry{
    left=2cm,
    right=2cm,
    top=2cm,
    bottom=2cm
}

\hypersetup{
    colorlinks=true,
    linkcolor=blue,
    citecolor=blue,
    urlcolor=blue
}

\begin{document}

%=== Portada ===%
\begin{center}
    {\Large\bfseries QML para jets en física de altas energías\par}
    \vspace{0.3cm}
    {\normalsize Juan Montoya - Computación Cuántica - Universidad de Antioquia\par}
    \vspace{0.2cm}
\end{center}

%=== Resumen ===%
\section{Resumen}
Este trabajo tiene como objetivo revisar la literatura sobre aprendizaje automático cuántico (QML) aplicado al jet tagging en física de altas energías, con el fin de comprender sus fundamentos teóricos y operativos, comparar su rendimiento frente a técnicas clásicas de clasificación de jets y reproducir los resultados reportados en estudios recientes.

%=== Objetivos ===%
\section{Objetivos}
\begin{itemize}
    \item Entender el rol del QML en jet tagging.
    \item Explorar la aplicación de QML para clasificación de jets (ej. jets de top vs QCD) usando datos simulados.
    % \item Implementar un clasificador cuántico variacional con PennyLane y comparar su rendimiento con métodos clásicos.
    \item Reproducir resultados de papers recientes en HEP.
    % \item Evaluar la viabilidad práctica del QML en el contexto de análisis de datos reales de colisiones.
\end{itemize}

%=== Metodología ===%
\section{Metodología}
\noindent\textbf{Generación de datos:} Usar MadGraph para simular eventos de colisión y Delphes para detectores. Procesar los datos con ROOT para extraer características relevantes de los jets.
\\
\noindent\textbf{Implementación QML:} Desarrollar un clasificador cuántico usando PennyLane, explorando diferentes codificaciones de datos y arquitecturas de circuitos cuánticos.
\\
\noindent\textbf{Benchmark:} Comparar el rendimiento del clasificador cuántico contra métodos clásicos.
\\
\noindent\textbf{Análisis:} Evaluar ventajas/limitaciones del enfoque cuántico, considerando número de parámetros, tiempo de entrenamiento y precisión.

\section{Referencias}
\begin{itemize}
    \item [1] A. Gianelle et al., "Quantum Machine Learning for b-jet charge identification", 
    
    \href{https://arxiv.org/pdf/2202.13943}{arXiv:2202.13943}.
    
    \item [2] H. Elhag et al., "Quantum Convolutional Neural Networks for Jet Images Classification", \href{https://arxiv.org/pdf/2408.08701}{arXiv:2408.08701}.
    \item [3] A. Bal et al., "1 Particle - 1 Qubit: Particle Physics Data Encoding for Quantum Machine Learning", \href{https://arxiv.org/pdf/2502.17301}{arXiv:2502.17301}.
    \item [4] S. Yen-Chi et al., "Hybrid Quantum-Classical Graph Convolutional Network", 
    
    \href{https://arxiv.org/pdf/2101.06189}{arXiv:2101.06189}.
\end{itemize}

% %=== Cronograma ===%
% \section{Cronograma}
% \begin{tabular}{@{} l c @{}} \toprule
% Actividad                  & Tiempo \\ \midrule
% Revisión literatura QML-HEP & Semana 1 \\
% Generación datos con MadGraph/Delphes & Semana 1-2 \\
% Implementación clasificador cuántico & Semana 2 \\
% Comparación con métodos clásicos & Semana 2 \\
% Análisis resultados y reporte & Semana 2 \\ \bottomrule
% \end{tabular}

\end{document}