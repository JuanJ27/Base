\documentclass[a4paper,12pt]{article}
\usepackage[utf8]{inputenc}
\usepackage[T1]{fontenc}
\usepackage[spanish]{babel}
\usepackage{amsmath,amssymb}
\usepackage{braket}
\usepackage{geometry}
\geometry{margin=1in}

\title{Computación Cuántica: Internet Cuántico y Comunicación Superdensa}
\author{Notas de Clase}
\date{}

\begin{document}

\maketitle

\section{Internet Cuántico}

Comunicar información por medio del entrelazamiento cuántico.

\subsection{Estados de Bell}
Los estados de Bell son:
\[
\{\phi^{+},\phi^{-},\psi^{+},\psi^{-}\}
\]
Si cada subsistema se mide en su propia base, el resultado parece aleatorio y con probabilidad $\frac{1}{2}$. Solo si existe correlación clásica, Alice y Bob pueden mostrar que hay correlación entre sus componentes.

Definiciones:
\[
\ket{\phi^+} = \frac{\ket{0}_A\ket{0}_B + \ket{1}_A\ket{1}_B}{\sqrt{2}}
\]
\[
\ket{\psi^\pm} = \frac{\ket{0,1} \pm \ket{1,0}}{\sqrt{2}}
\]
Los estados de Bell pueden clasificarse mediante etiquetas binarias.

\subsection{Canales Cuánticos}
\begin{itemize}
    \item Codificación superdensa (envía dos bits clásicos)
\end{itemize}

\subsection{Compuerta de Hall (Hadamard)}
La compuerta Hadamard se define mediante:
\[
\hat{H} = \frac{\ket{0}+\ket{1}}{\sqrt{2}}\bra{0} + \frac{\ket{0}-\ket{1}}{\sqrt{2}}\bra{1}
\]
Su forma matricial es:
\[
H = \frac{1}{\sqrt{2}}
\begin{pmatrix}
1 & 1 \\
1 & -1 \\
\end{pmatrix}
\]

\subsection{CNOT}
La acción de la puerta CNOT se ilustra por:
\[
\hat{U}_{NOT}\frac{\ket{0,0}+\ket{0,1}}{\sqrt{2}} = \frac{\hat{U}_{NOT}\ket{0,0}+\hat{U}_{NOT}\ket{0,1}}{\sqrt{2}} = \frac{\ket{0,0}+\ket{1,1}}{\sqrt{2}}
\]
Se usan circuitos para preparar estados de Bell.

\subsection{Registro de Estados de Bell}
\[
\ket{\phi^{\pm}} = \frac{\ket{0,0} \pm \ket{1,1}}{\sqrt{2}}
\]
\[
\ket{\psi^{\pm}} = \frac{\ket{0,1} \pm \ket{1,0}}{\sqrt{2}}
\]
Asimismo, podemos formar estados de Bell utilizando Hadamard y CNOT:
\[
\ket{B_{yx}} = \frac{\ket{0,y} + (-1)^x \ket{1,\bar{y}}}{\sqrt{2}}
\]

\subsection{Tabla de Estados de Bell}
\[
\begin{array}{c|c|c}
\text{Bell} & y & x \\ \hline
\phi^{+} & 0 & 0 \\
\phi^{-} & 0 & 1 \\
\psi^{+} & 1 & 0 \\
\psi^{-} & 1 & 1 \\
\end{array}
\]

\section{Comunicación Superdensa}

Protocolo: \\
1. Alice y Bob comparten un par de qubits entrelazados. \\
2. Alice mide su qubit y envía el resultado a Bob. \\
3. Bob decodifica el mensaje utilizando la información recibida de Alice.

\[
\ket{\psi} = \frac{1}{\sqrt{2}}(\ket{00} + \ket{11}) \rightarrow \text{Alice mide } 0 \text{ o } 1
\]

\[
\text{Si Alice mide } 0 \text{, Bob tiene } \ket{00} \text{ y si mide } 1 \text{, Bob tiene } \ket{11}
\]

\section{Ejemplo de Comunicación Superdensa}
\subsection{Ejemplo de Comunicación Superdensa}
Consideremos el siguiente ejemplo práctico:

\begin{enumerate}
    \item Alice y Bob comparten inicialmente el estado de Bell
    \[
    \ket{\phi^+} = \frac{\ket{00} + \ket{11}}{\sqrt{2}}.
    \]
    \item Para enviar el mensaje "10", Alice aplica las siguientes operaciones en su qubit:
    \begin{itemize}
        \item Aplica la compuerta X si el primer bit es 1.
        \item Aplica la compuerta Z si el segundo bit es 1.
    \end{itemize}
    En este caso, como el mensaje es "10", aplica la compuerta X, transformando el estado a:
    \[
    (X \otimes I)\ket{\phi^+} = \ket{\psi^+} = \frac{\ket{01} + \ket{10}}{\sqrt{2}}.
    \]
    \item Bob, al recibir su qubit, realiza una medición en la base de Bell. La medición le permite identificar el estado \(\ket{\psi^+}\) y, por lo tanto, recuperar el mensaje "10".
\end{enumerate}

Así se demuestra cómo, mediante la manipulación de un solo qubit, se pueden transmitir dos bits clásicos utilizando la comunicación superdensa.

\section{Teleportación}
Alice quiere enviar a Bob un bit unico, usando un canal provisto por charlie. Para ello, Alice y Bob comparten un par de qubits entrelazados. Alice mide su qubit con una CNOT y una Hadamard, y envía el resultado a Bob. Bob aplica una compuerta CNOT y una Hadamard en su qubit, obteniendo el estado original de Alice.
\[
\ket{\psi} = \alpha \ket{0} + \beta \ket{1}
\]
\[
\ket{\psi} = \alpha \ket{0} + \beta \ket{1} \rightarrow \text{Alice mide } 0 \text{ o } 1
\]
\[
\text{Si Alice mide } 0 \text{, Bob tiene } \ket{00} \text{ y si mide } 1 \text{, Bob tiene } \ket{11}
\]


\end{document}