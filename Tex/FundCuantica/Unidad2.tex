\documentclass[a4paper,11pt]{article}
\usepackage[utf8]{inputenc}
\usepackage[spanish]{babel}
\usepackage{amsmath,amssymb}
\usepackage{braket}
\usepackage{graphicx}
\usepackage{hyperref}
\usepackage{bookmark}
\usepackage[left=2cm,right=2cm,top=2cm,bottom=2cm]{geometry}

\title{Notas de Clase}
\author{Juan Montoya}
\date{\today}

\begin{document}
\maketitle

\section*{Resumen}
El objetivo de estas notas es ilustrar los postulados fundamentales de la mecánica cuántica mediante el caso de un sistema de espín $1/2$ (por ejemplo, átomos de plata) y el uso del aparato de Stern–Gerlach. Se aborda la preparación de estados, la naturaleza probabilística de las mediciones y la evolución temporal bajo un Hamiltoniano simple.

\section{Operador \texorpdfstring{$S_z$} y espacio de espín}
Al observable $\mathcal{S}_z$ corresponde el operador $S_z$, cuyos autovalores son $\pm\hbar/2$. Denotamos por $\ket{+}$ y $\ket{-}$ los autovectores ortonormales:
\begin{equation}
    \begin{cases}
    S_z \ket{+} = +\dfrac{\hbar}{2}\ket{+},\\
    S_z \ket{-} = -\dfrac{\hbar}{2}\ket{-},
    \end{cases}
\tag{A-10}
\end{equation}
con
\begin{equation}
    \begin{cases}
    \braket{+|+} = \braket{-|-} = 1,\\
    \braket{+|-} = 0,
    \end{cases}
\tag{A-11}
\end{equation}
y la relación de cierre
\begin{equation}
\ket{+}\bra{+} + \ket{-}\bra{-} = \mathbb{I}.
\tag{A-12}
\end{equation}

\subsection*{A-2-b. Los operadores $S_x$, $S_y$ y $S_u$}
Los operadores $S_x$, $S_y$ y $S_u$ tienen los mismos autovalores, $+\hbar/2$ y $-\hbar/2$, que $S_z$. Este resultado era de esperar, ya que siempre es posible girar todo el conjunto del aparato de Stern–Gerlach de modo que el eje definido por el campo magnético quede paralelo a $Ox$, $Oy$ o $\vec u$. Dado que todas las direcciones del espacio son físicamente equivalentes, los fenómenos observados en la placa no cambian bajo tales rotaciones; así, la medición de $S_x$, $S_y$ o $S_u$ sólo puede dar como resultado $+\hbar/2$ o $-\hbar/2$.

En cuanto a los autovectores de $S_x$, $S_y$ y $S_u$, los denotaremos respectivamente por $\ket{\pm}_x$, $\ket{\pm}_y$ y $\ket{\pm}_u$ (el signo en el ket coincide con el del autovalor correspondiente). Sus desarrollos en la base $\{\ket{+},\ket{-}\}$ de $S_z$ se escriben:

\begin{equation}
\ket{\pm}_x = \frac{1}{\sqrt{2}}\bigl(\ket{+}\pm\ket{-}\bigr)
\tag{A-20}
\end{equation}

\begin{equation}
\ket{\pm}_y = \frac{1}{\sqrt{2}}\bigl(\ket{+}\pm i\,\ket{-}\bigr)
\tag{A-21}
\end{equation}

\begin{equation}
\begin{cases}
\ket{+}_u = \cos\dfrac{\theta}{2}\,e^{-i\varphi/2}\,\ket{+}
           + \sin\dfrac{\theta}{2}\,e^{i\varphi/2}\,\ket{-},\\[8pt]
\ket{-}_u = -\,\sin\dfrac{\theta}{2}\,e^{-i\varphi/2}\,\ket{+}
           + \cos\dfrac{\theta}{2}\,e^{i\varphi/2}\,\ket{-}.
\end{cases}
\tag{A-22a,b}
\end{equation}

\section{Estado general y parámetros esféricos}
El estado más general en el espacio de espín es
\begin{equation}
\ket{\psi} = \alpha \ket{+} + \beta \ket{-},
\tag{A-13}
\end{equation}
sujeto a
\begin{equation}
|\alpha|^2 + |\beta|^2 = 1.
\tag{A-14}
\end{equation}
Con la parametrización
\[
\alpha = \cos\frac{\theta}{2}e^{-i\varphi/2},\quad
\beta  = \sin\frac{\theta}{2}e^{i\varphi/2},
\]
podemos asociar cada par $(\alpha,\beta)$ a un vector unitario en la esfera de Bloch.

\section{Mediciones de espín}
Para ilustrar la naturaleza probabilística de las mediciones:
\begin{itemize}
    \item \textbf{Experimento 1:} Con ambos aparatos alineados, si se prepara $\ket{+}$ siempre se obtiene $+\hbar/2$.
    \item \textbf{Experimento 2:} Si se prepara $\ket{+}_u$ (dirección $\vec{u}$) y se mide $S_z$, las probabilidades son
        \[
            P(+\tfrac{\hbar}{2}) = \cos^2\frac{\theta}{2},\quad
            P(-\tfrac{\hbar}{2}) = \sin^2\frac{\theta}{2}.
        \]
    \item \textbf{Experimento 3:} Al rotar el analizador, las probabilidades cambian con el ángulo relativo.
\end{itemize}
El valor medio se corresponde con el resultado clásico:
\[
\langle S_z\rangle = \frac{\hbar}{2}\cos\theta.
\]

\section{Evolución temporal}
En un campo magnético uniforme $\vec{B}_0$, el Hamiltoniano es
\[
\hat H = \omega_0 \hat S_z,\quad \omega_0 = -\gamma B_0.
\]
Sus autoestados $\ket{\pm}$ tienen energías separadas por $\hbar\omega_0$. Si
\[
\ket{\psi(0)} = \cos\frac{\theta}{2}e^{-i\varphi/2}\ket{+}
                             + \sin\frac{\theta}{2}e^{i\varphi/2}\ket{-},
\]
entonces la evolución de Schrödinger produce una precesión de Larmor:
\[
\ket{\psi(t)}
= \cos\frac{\theta}{2}e^{-i(\varphi+\omega_0 t)/2}\ket{+}
+ \sin\frac{\theta}{2}e^{i(\varphi+\omega_0 t)/2}\ket{-}.
\]

\subsection*{Demostración de la existencia de un vector unitario $\mathbf{u}$}
% Demostración de la existencia de un vector unitario u
Vamos a mostrar que existe, para todo $\ket{\psi}$, un vector unitario $\mathbf{u}$ tal que $\ket{\psi}$ es colineal con el ket $\ket{+}_u$. Elegimos por tanto dos números complejos $\alpha$ y $\beta$ que satisfacen la relación
\begin{equation}
|\alpha|^2 + |\beta|^2 = 1
\tag{B-2}
\end{equation}
pero que son arbitrarios en lo demás. Teniendo en cuenta (B-2), existe necesariamente un ángulo $\theta$ tal que
\begin{equation}
\begin{cases}
\cos\dfrac{\theta}{2} = |\alpha|,\\
\sin\dfrac{\theta}{2} = |\beta|.
\end{cases}
\tag{B-3}
\end{equation}
Si, además, imponemos
\begin{equation}
0 \le \theta \le \pi,
\tag{B-4}
\end{equation}
la ecuación
\[
\tan\dfrac{\theta}{2} = \left|\dfrac{\beta}{\alpha}\right|
\]
determina $\theta$ de forma única. Sabemos que sólo la diferencia de fases de $\alpha$ y $\beta$ influye en las predicciones físicas. Definimos entonces
\begin{align}
\varphi &= Arg(\beta) - Arg(\alpha), \tag{B-5}\\
\chi    &= Arg(\beta) + Arg(\alpha). \tag{B-6}
\end{align}
De aquí se sigue
\begin{equation}
    Arg(\beta) = \tfrac12\,\chi + \tfrac12\,\varphi,
    \quad
    Arg(\alpha) = \tfrac12\,\chi - \tfrac12\,\varphi.
\tag{B-7}
\end{equation}
Con esta notación, el ket $\ket{\psi}$ puede escribirse:
\begin{equation}
\ket{\psi}
= e^{i\chi/2}
\Bigl[
\cos\dfrac{\theta}{2}\,e^{-i\varphi/2}\,\ket{+}
+
\sin\dfrac{\theta}{2}\,e^{i\varphi/2}\,\ket{-}
\Bigr].
\tag{B-8}
\end{equation}

\subsection*{Resumen de la Figura 9: Trayectorias clásicas vs.\ superposición cuántica}

Cuando un átomo de plata entra con espín en el estado $\ket{+}$ (Fig.~9-a) o $\ket{-}$ (Fig.~9-b), su función de onda externa está concentrada en un único paquete estrecho cuyo centro recorre una trayectoria que puede describirse clásicamente.  

Sin embargo, si el estado de espín es la superposición  
\[
\ket{\psi} = \cos\frac{\theta}{2}\,\ket{+} + \sin\frac{\theta}{2}\,\ket{-}
\tag{B-9}
\]
el paquete de onda inicial se divide en dos subpaquetes (Fig.~9-c), cada uno localizado cerca de los puntos 1 y 2 al llegar a la pantalla. Aunque cada subpaquete sigue siendo muy estrecho, el átomo ya no tiene una sola trayectoria clásica: la probabilidad de detección se reparte entre ambos lugares.  

Estos dos subpaquetes corresponden a la misma partícula con distinta fase relativa; de hecho, si no se realiza la medición (quitan- do la pantalla) y se aplica un gradiente de campo magnético inverso, podrían recombinarse en un único paquete.

\textbf{Comentario:}\\
(i) Si el campo $\mathbf{B}_{0}$ es paralelo al vector unitario $\mathbf{u}$ de ángulos polares $\theta,\varphi$, la ecuación (B-17) debe reemplazarse por
\begin{equation}
H=\omega_{0}\,S_{u}
\tag{B-20}
\end{equation}
donde $S_{u}=\mathbf{S}\cdot\mathbf{u}$.\\
(ii) Para el átomo de plata $\gamma<0$, luego $\omega_{0}>0$ según (B-16), lo cual explica la disposición de los niveles en la figura.

\subsection*{B-3. Evolución de un espín 1/2 en un campo magnético uniforme}

\subsubsection*{B-3-a. Hamiltoniano de interacción y ecuación de Schrödinger}

Consideremos un átomo de plata en un campo magnético uniforme $\mathbf{B}_{0}$, y tomemos el eje $O z$ paralelo a $\mathbf{B}_{0}$. La energía potencial clásica del momento magnético $\boldsymbol{\mathcal{M}}=\gamma\,\mathbf{S}$ es:
\begin{equation}
W=-\boldsymbol{\mathcal{M}}\cdot\mathbf{B}_{0}
=-\mathcal{M}_{z}\,B_{0}
=-\gamma\,B_{0}\,S_{z}
\tag{B-15}
\end{equation}
donde $B_{0}=\lvert\mathbf{B}_{0}\rvert$. Definimos:
\begin{equation}
\omega_{0}=-\gamma\,B_{0}
\tag{B-16}
\end{equation}
que tiene dimensión de velocidad angular.

Al cuantizar sólo los grados internos, $S_{z}$ se reemplaza por el operador $\hat S_{z}$ y la energía (B-15) pasa a ser el Hamiltoniano
\begin{equation}
H=\omega_{0}\,\hat S_{z}
\tag{B-17}
\end{equation}
Este operador es independiente del tiempo, por lo que resolver la ecuación de Schrödinger equivale a encontrar los autovalores de $H$. Sus autovectores son los mismos de $\hat S_{z}$:
\begin{equation}
\begin{cases}
H\ket{+}=+\dfrac{\hbar\,\omega_{0}}{2}\,\ket{+},\\[6pt]
H\ket{-}=-\dfrac{\hbar\,\omega_{0}}{2}\,\ket{-}.
\end{cases}
\tag{B-18}
\end{equation}
Por tanto, existen dos niveles de energía,
\[
E_{+}=+\frac{\hbar\omega_{0}}{2},
\quad
E_{-}=-\frac{\hbar\omega_{0}}{2},
\]
y su separación $\hbar\omega_{0}$ define la frecuencia de Bohr
\begin{equation}
\nu_{+-}
=\frac{1}{h}\,(E_{+}-E_{-})
=\frac{\omega_{0}}{2\pi}\,.
\tag{B-19}
\end{equation}

\subsection*{C-2. Aspecto estático: efecto del acoplamiento en los estados estacionarios del sistema}

\subsubsection*{C-2-a. Expresiones para los autoestados y autoenergías de \(H\)}

En la base \(\{\ket{\varphi_1},\ket{\varphi_2}\}\), la matriz que representa \(H\) es:
\begin{equation}
H = \begin{pmatrix}
E_1 + W_{11} & W_{12} \\
W_{21}       & E_2 + W_{22}
\end{pmatrix}
\tag{C-7}
\end{equation}

Su diagonalización conduce a los autoenergías:
\begin{equation}
\begin{aligned}
E_+ &= \tfrac12\bigl(E_1 + W_{11} + E_2 + W_{22}\bigr)
       + \tfrac12\sqrt{\bigl(E_1 + W_{11} - E_2 - W_{22}\bigr)^2 + 4|W_{12}|^2},\\
E_- &= \tfrac12\bigl(E_1 + W_{11} + E_2 + W_{22}\bigr)
       - \tfrac12\sqrt{\bigl(E_1 + W_{11} - E_2 - W_{22}\bigr)^2 + 4|W_{12}|^2}.
\end{aligned}
\tag{C-8}
\end{equation}
(si \(W_{ij}=0\), entonces \(E_+=E_1\) y \(E_-=E_2\)).  

Los autoestados asociados son:
\begin{align}
\ket{\psi_+}
&= \cos\frac{\theta}{2}\,e^{-i\varphi/2}\,\ket{\varphi_1}
+ \sin\frac{\theta}{2}\,e^{i\varphi/2}\,\ket{\varphi_2},
\tag{C-9a}\\
\ket{\psi_-}
&= -\sin\frac{\theta}{2}\,e^{-i\varphi/2}\,\ket{\varphi_1}
+ \cos\frac{\theta}{2}\,e^{i\varphi/2}\,\ket{\varphi_2}.
\tag{C-9b}
\end{align}

donde los ángulos \(\theta\) y \(\varphi\) se definen por:
\begin{equation}
\tan\theta
= \frac{2\,|W_{12}|}{E_1 + W_{11} - E_2 - W_{22}},
\quad
0 \le \theta < \pi,
\tag{C-10}
\end{equation}
\begin{equation}
W_{21} = |W_{21}|\,e^{i\varphi}.
\tag{C-11}
\end{equation}

\subsection*{C-3. Aspecto dinámico: oscilación del sistema entre los dos estados no perturbados}

\subsubsection*{C-3-a. Evolución del vector de estado}

Sea el vector de estado en el instante \(t\):
\[
\ket{\psi(t)} = a_1(t)\ket{\varphi_1} + a_2(t)\ket{\varphi_2}
\tag{C-22}
\]
La ecuación de Schrödinger es:
\[
i\hbar\,\frac{d}{dt}\ket{\psi(t)} = \bigl(H_0 + W\bigr)\ket{\psi(t)}\,.
\tag{C-23}
\]
Proyectando en \(\ket{\varphi_1}\) y \(\ket{\varphi_2}\), con \(W_{11}=W_{22}=0\), obtenemos el sistema acoplado:
\[
\begin{aligned}
i\hbar\,\frac{da_1}{dt} &= E_1\,a_1 + W_{12}\,a_2\,,\\
i\hbar\,\frac{da_2}{dt} &= W_{21}\,a_1 + E_2\,a_2\,.
\end{aligned}
\tag{C-24}
\]
La solución se construye descomponiendo \(\ket{\psi(0)}\) en los autoestados \(\ket{\psi_+}\), \(\ket{\psi_-}\) de \(H=H_0+W\):
\[
\ket{\psi(0)}=\lambda\,\ket{\psi_+}+\mu\,\ket{\psi_-}\,,
\tag{C-25}
\]
y asumiendo la condición inicial
\[
\ket{\psi(0)}=\ket{\varphi_1}\,.
\tag{C-27}
\]

\subsubsection*{C-3-b. Cálculo de \(\mathcal{P}_{12}(t)\): fórmula de Rabi}

Expandiendo \(\ket{\psi(0)}\) en la base \(\{\ket{\psi_+},\ket{\psi_-}\}\) (invertir (C-9)), se tiene:
\[
\ket{\psi(0)} = \ket{\varphi_1}
= e^{i\varphi/2}\Bigl[\cos\frac{\theta}{2}\,\ket{\psi_+}
                 -\sin\frac{\theta}{2}\,\ket{\psi_-}\Bigr].
\tag{C-28}
\]
Entonces la evolución temporal es
\[
\ket{\psi(t)}
= e^{i\varphi/2}\Bigl[\cos\frac{\theta}{2}\,e^{-iE_+t/\hbar}\ket{\psi_+}
                      -\sin\frac{\theta}{2}\,e^{-iE_-t/\hbar}\ket{\psi_-}\Bigr].
\tag{C-29}
\]
La amplitud de probabilidad de hallar el sistema en \(\ket{\varphi_2}\) es
\[
\braket{\varphi_2|\psi(t)}
= e^{i\varphi/2}\Bigl[\cos\frac{\theta}{2}\,e^{-iE_+t/\hbar}\braket{\varphi_2|\psi_+}
                     -\sin\frac{\theta}{2}\,e^{-iE_-t/\hbar}\braket{\varphi_2|\psi_-}\Bigr],
\tag{C-30}
\]
y la probabilidad
\[
\mathcal{P}_{12}(t) = \bigl|\braket{\varphi_2|\psi(t)}\bigr|^2
= \tfrac12\sin^2\theta\bigl[1-\cos\bigl((E_+ - E_-)\,t/\hbar\bigr)\bigr]
= \sin^2\theta\;\sin^2\!\bigl((E_+ - E_-)\,t/2\hbar\bigr).
\tag{C-31}
\]
Usando además las expresiones de los ángulos \(\theta\) y \(\varphi\) [(C-12),(C-13)], se reescribe como
\[
\mathcal{P}_{12}(t)
= \frac{4\,|W_{12}|^2}{4\,|W_{12}|^2 + (E_1 - E_2)^2}
  \;\sin^2\!\Bigl[t/(2\hbar)\,\sqrt{4\,|W_{12}|^2 + (E_1 - E_2)^2}\Bigr].
\tag{C-32}
\]
Esta última es la conocida como \emph{fórmula de Rabi}.

\section*{Complemento JIV: Ejercicios}

\subsection*{Ejercicio 1}
Considera una partícula de espín $1/2$ de momento magnético $\mathbf{M}=\gamma\,\mathbf{S}$. El espacio de estados de espín está generado por los vectores $\ket{+}$ y $\ket{-}$, autovectores de $S_z$ con autovalores $+\hbar/2$ y $-\hbar/2$. En $t=0$, el estado del sistema es
\[
\ket{\psi(0)}=\ket{+}.
\]
\begin{enumerate}
  \item Si en $t=0$ medimos el observable $S_z$, ¿qué resultados se pueden obtener y con qué probabilidades?
  \item En lugar de realizar esa medición, dejamos que el sistema evolucione bajo la acción de un campo magnético uniforme paralelo a $Oy$, de módulo $B_0$. Calcula, en la base $\{\ket{+},\ket{-}\}$, el estado del sistema en tiempo $t$.
  \item En ese instante medimos los observables $S_x$, $S_y$ y $S_z$. ¿Qué valores pueden obtenerse y con qué probabilidades? ¿Qué relación debe existir entre $B_0$ y $t$ para que alguno de los resultados sea absolutamente cierto? Da su interpretación física.
\end{enumerate}

\subsection*{Ejercicio 2}
Considera de nuevo una partícula de espín $1/2$ (misma notación).
\begin{enumerate}
  \item En $t=0$ medimos $S_y$ y hallamos $+\hbar/2$. ¿Cuál es el vector de estado $\ket{\psi(0)}$ inmediatamente tras la medición?
  \item Inmediatamente después aplicamos un campo magnético uniforme dependiente del tiempo, paralelo a $Oz$, de modo que
  \[
    H(t)=\omega_0(t)\,S_z\,.
  \]
  Supón que $\omega_0(t)=0$ para $t<0$ y $t>T$, y que crece linealmente de $0$ a $\omega_0$ en el intervalo $0<t<T$ (siendo $T$ un parámetro de tiempo dado). Demuestra que, en cualquier instante $t$, el estado puede escribirse
  \[
    \ket{\psi(t)}
    = \frac{1}{\sqrt{2}}\Bigl[e^{\,i\theta(t)}\ket{+}
      \;+\; i\,e^{-\,i\theta(t)}\ket{-}\Bigr],
  \]
  donde $\theta(t)$ es una función real de $t$ (a calcular).
  \item En $t=T$ medimos $S_y$. ¿Qué resultados pueden hallarse y con qué probabilidades? Determina la relación entre $\omega_0$ y $T$ para garantizar un resultado único. Da su interpretación física.
\end{enumerate}

\textbf{Solución:}
\begin{enumerate}
  \item Tras medir $S_y=+\hbar/2$ el sistema queda en el autovector correspondiente,
    \[
      \ket{\psi(0)} = \ket{+}_y
      = \frac{1}{\sqrt{2}}\bigl(\ket{+} + i\ket{-}\bigr).
    \]
  \item Como $[H(t),H(t')]=0$, la evolución es
    \[
      U(t)
      = \exp\!\Bigl(-\tfrac{i}{\hbar}\!\int_0^tH(t')\,dt'\Bigr).
    \]
    Dado que 
    $H(t)\ket{\pm}=\pm\frac{\hbar}{2}\,\omega_0(t)\ket{\pm}$, definimos
    \[
      \theta(t)=\frac{1}{2}\int_0^t\omega_0(t')\,dt',
    \]
    y se obtiene
    \[
      U(t)\ket{+}=e^{-\,i\theta(t)}\ket{+},\qquad
      U(t)\ket{-}=e^{\,i\theta(t)}\ket{-}.
    \]
    Aplicando a $\ket{\psi(0)}$:
    \[
      \ket{\psi(t)}
      =\frac{1}{\sqrt{2}}\Bigl(e^{-\,i\theta(t)}\ket{+}
        +i\,e^{\,i\theta(t)}\ket{-}\Bigr).
    \]
    Redefiniendo $\theta\to-\theta$ recuperamos la forma deseada.
  \item Escribimos las proyecciones sobre $\ket{\pm}_y$:
    \[
      \ket{+}_y=\tfrac1{\sqrt2}(\ket{+}+i\ket{-}),\quad
      \ket{-}_y=\tfrac1{\sqrt2}(\ket{+}-i\ket{-}).
    \]
    Entonces
    \[
      P\bigl(+\tfrac{\hbar}{2}\bigr)
      =\bigl|\braket{+_y|\psi(T)}\bigr|^2
      =\cos^2\theta(T),
      \quad
      P\bigl(-\tfrac{\hbar}{2}\bigr)=\sin^2\theta(T).
    \]
    Para que el resultado sea siempre $+\hbar/2$ se pide
    \[
      \cos^2\theta(T)=1
      \;\Longrightarrow\;
      \theta(T)=n\pi
      \;\Longrightarrow\;
      \int_0^T\omega_0(t)\,dt=2n\pi.
    \]
    Si $\omega_0(t)=\tfrac{\omega_0}{T}\,t$ en $[0,T]$, entonces
    $\int_0^T\omega_0(t)\,dt=\tfrac12\,\omega_0T$, y la condición es
    $\tfrac12\,\omega_0T=2n\pi$, esto es $\omega_0T=4n\pi$.  
    Físicamente, equivale a girar el espín alrededor de $Oz$ un número entero
    de vueltas completas, de modo que el autovector de $S_y$ regresa a sí mismo.
\end{enumerate}

\subsection*{Ejercicio 3}
Considera una partícula de espín $1/2$ en un campo magnético $\mathbf{B}_0$ con componentes
\[
  B_x=\frac{1}{\sqrt{2}}\,B_0,\quad B_y=0,\quad B_z=\frac{1}{\sqrt{2}}\,B_0.
\]
\begin{enumerate}
  \item Calcula la matriz que representa el Hamiltoniano $H=\gamma\,\mathbf{S}\cdot\mathbf{B}_0$ en la base $\{\ket{+},\ket{-}\}$ de $S_z$.
  \item Halla los autovalores y autovectores de $H$.
  \item Si en $t=0$ el sistema está en el estado $\ket{-}$, ¿qué valores de energía se obtienen y con qué probabilidades?
  \item Determina el vector de estado $\ket{\psi(t)}$ en tiempo $t$. En ese instante medimos $S_x$: calcula el valor medio de la medición y explica su interpretación geométrica.
\end{enumerate}

\subsection*{Ejercicio 4\,d)}
Supón ahora que la velocidad de un átomo es una variable aleatoria, de modo que el tiempo de vuelo $T$ sólo se conoce con una incertidumbre $\Delta T$. Además, el campo $B_0$ es tan intenso que $\omega_0\,\Delta T \gg 1$. Entonces el producto $\omega_0 T$ (módulo $2\pi$) es equiprobable en $[0,2\pi]$.
\begin{enumerate}
  \item ¿Cuál es el operador densidad $\rho_2$ de un átomo justo al entrar en el analizador? ¿Corresponde a un estado puro?
  \item Calcula $Tr(\rho_2\,S_x)$, $Tr(\rho_2\,S_y)$ y $Tr(\rho_2\,S_z)$. ¿Cómo interpretas esos resultados? ¿En qué caso la densidad describe un espín totalmente polarizado? ¿Cuándo uno completamente no polarizado?
  \item Describe cualitativamente los fenómenos observados a la salida del analizador al variar $\omega_0$ desde $0$ hasta el régimen $\omega_0\,\Delta T\gg1$.
\end{enumerate}

\subsection*{A-3. Propiedades generales del Hamiltoniano cuántico}

En mecánica cuántica, las coordenadas clásicas \(x\) y \(p\) se reemplazan por los operadores \(\hat X\) y \(\hat P\), que cumplen la relación de conmutación:
\[
[\hat X,\hat P]=i\hbar
\tag{A-14}
\]
Partiendo de la forma clásica
\[
H=\frac{p^2}{2m} \;+\;\frac12\,m\omega^2\,x^2,
\]
se obtiene el operador Hamiltoniano
\[
\hat H \;=\; \frac{\hat P^2}{2m} \;+\;\frac12\,m\omega^2\,\hat X^2.
\tag{A-15}
\]
Al ser \(\hat H\) independiente del tiempo, resolvemos la ecuación de autovalores
\[
\hat H\ket{\varphi}=E\ket{\varphi}
\tag{A-16}
\]
que en representación \(x\) toma la forma
\[
\Bigl[-\tfrac{\hbar^2}{2m}\tfrac{d^2}{dx^2} + \tfrac12\,m\omega^2\,x^2\Bigr]\varphi(x)
= E\,\varphi(x).
\tag{A-17}
\]
De aquí se deducen las siguientes propiedades:
\begin{enumerate}
  \item Los valores propios de \(\hat H\) son positivos, pues si \(V(x)\ge V_m\) entonces
  \[
    E>V_m.
    \tag{A-18}
  \]
  \item Las autofunciones tienen paridad definida, dado que el potencial es par:
  \[
    V(-x)=V(x).
    \tag{A-19}
  \]
  \item El espectro es discreto, ya que el movimiento queda confinado en una región limitada del eje \(x\).
\end{enumerate}

\subsection*{B-1. Notación y operadores adimensionales}

Para simplificar la resolución, definimos los operadores adimensionales
\[
  \hat X = \sqrt{\frac{m\omega}{\hbar}}\;\hat x,
  \quad
  \hat P = \frac{1}{\sqrt{m\hbar\omega}}\;\hat p,
  \tag{B-1}
\]
que satisfacen el conmutador canónico
\[
  [\hat X,\hat P]=i.
  \tag{B-2}
\]
El Hamiltoniano se factoriza como
\[
  H = \hbar\omega\,\hat H,
  \tag{B-3}
\]
donde el operador adimensional es
\[
  \hat H = \frac12\bigl(\hat X^2 + \hat P^2\bigr).
  \tag{B-4}
\]
Buscamos las soluciones de la ecuación de autovalores
\[
  \hat H\ket{\varphi_n} = \varepsilon_n\,\ket{\varphi_n}.
  \tag{B-5}
\]

\subsubsection*{B-1-a. Operador \(a^\dagger a\) y Hamiltoniano adimensional}

Partiendo de las definiciones
\[
a = \frac{1}{\sqrt2}(\hat X + i\hat P), 
\quad
a^\dagger = \frac{1}{\sqrt2}(\hat X - i\hat P),
\]
calculamos primero
\begin{equation}
\begin{split}
a^\dagger a
&= \tfrac12\,(\hat X - i\hat P)(\hat X + i\hat P)\\
&= \tfrac12\bigl(\hat X^2 + \hat P^2 + i[\hat X,\hat P]\bigr)\\
&= \tfrac12\bigl(\hat X^2 + \hat P^2 -1\bigr)
\end{split}
\tag{B-10}
\end{equation}
Comparando con la forma adimensional del Hamiltoniano
\(\hat H = \tfrac12(\hat X^2 + \hat P^2)\) (Ecuación (B-4)), obtenemos
\begin{equation}
\hat H 
= a^\dagger a + \tfrac12
= \tfrac12\,(\hat X - i\hat P)(\hat X + i\hat P) \;+\;\tfrac12
\tag{B-11}
\end{equation}
y, de manera análoga,
\begin{equation}
\hat H = a\,a^\dagger - \tfrac12.
\tag{B-12}
\end{equation}
Introducimos entonces el operador número
\begin{equation}
N = a^\dagger a,
\tag{B-13}
\end{equation}
que es hermítico, pues
\begin{equation}
N^\dagger = (a^\dagger a)^\dagger = a^\dagger a = N.
\tag{B-14}
\end{equation}
Por tanto, de (B-11) se sigue
\begin{equation}
\hat H = N + \tfrac12.
\tag{B-15}
\end{equation}

\subsubsection*{B-1-b. Operadores \(a\), \(a^\dagger\) y \(N\)}

Si intentáramos factorizar \(\hat X^2 + \hat P^2\) como si \(\hat X,\hat P\) fuesen números, fallaríamos por la no conmutatividad. En su lugar, definimos
\begin{align}
a        &= \tfrac{1}{\sqrt2}(\hat X + i\hat P),
\tag{B-6a}\\
a^\dagger&= \tfrac{1}{\sqrt2}(\hat X - i\hat P).
\tag{B-6b}
\end{align}
Invertir estas relaciones da
\begin{align}
\hat X &= \tfrac{1}{\sqrt2}(a^\dagger + a),
\tag{B-7a}\\
\hat P &= \tfrac{i}{\sqrt2}(a^\dagger - a).
\tag{B-7b}
\end{align}
Su conmutador se calcula con (B-6) y la relación canónica \([\hat X,\hat P]=i\):
\begin{equation}
\begin{split}
[a,a^\dagger]
&= \tfrac12\,[\,\hat X + i\hat P\,,\,\hat X - i\hat P\,]\\
&= \tfrac{i}{2}\,[\hat P,\hat X] \;-\;\tfrac{i}{2}\,[\hat X,\hat P]
=1.
\end{split}
\tag{B-8}
\end{equation}
Es decir,
\begin{equation}
[a,a^\dagger]=1,
\tag{B-9}
\end{equation}
que es equivalente a la conmutación canónica \([\hat X,\hat P]=i\hbar\).```\subsubsection*{B-1-a. Operador \(a^\dagger a\) y Hamiltoniano adimensional}

Partiendo de las definiciones
\[
a = \frac{1}{\sqrt2}(\hat X + i\hat P), 
\quad
a^\dagger = \frac{1}{\sqrt2}(\hat X - i\hat P),
\]
calculamos primero
\begin{equation}
\begin{split}
a^\dagger a
&= \tfrac12\,(\hat X - i\hat P)(\hat X + i\hat P)\\
&= \tfrac12\bigl(\hat X^2 + \hat P^2 + i[\hat X,\hat P]\bigr)\\
&= \tfrac12\bigl(\hat X^2 + \hat P^2 -1\bigr)
\end{split}
\tag{B-10}
\end{equation}
Comparando con la forma adimensional del Hamiltoniano
\(\hat H = \tfrac12(\hat X^2 + \hat P^2)\) (Ecuación (B-4)), obtenemos
\begin{equation}
\hat H 
= a^\dagger a + \tfrac12
= \tfrac12\,(\hat X - i\hat P)(\hat X + i\hat P) \;+\;\tfrac12
\tag{B-11}
\end{equation}
y, de manera análoga,
\begin{equation}
\hat H = a\,a^\dagger - \tfrac12.
\tag{B-12}
\end{equation}
Introducimos entonces el operador número
\begin{equation}
N = a^\dagger a,
\tag{B-13}
\end{equation}
que es hermítico, pues
\begin{equation}
N^\dagger = (a^\dagger a)^\dagger = a^\dagger a = N.
\tag{B-14}
\end{equation}
Por tanto, de (B-11) se sigue
\begin{equation}
\hat H = N + \tfrac12.
\tag{B-15}
\end{equation}

\subsubsection*{B-1-b. Operadores \(a\), \(a^\dagger\) y \(N\)}

Si intentáramos factorizar \(\hat X^2 + \hat P^2\) como si \(\hat X,\hat P\) fuesen números, fallaríamos por la conmutatividad. En su lugar, definimos
\begin{align}
a        &= \tfrac{1}{\sqrt2}(\hat X + i\hat P),
\tag{B-6a}\\
a^\dagger&= \tfrac{1}{\sqrt2}(\hat X - i\hat P).
\tag{B-6b}
\end{align}
Invertir estas relaciones da
\begin{align}
\hat X &= \tfrac{1}{\sqrt2}(a^\dagger + a),
\tag{B-7a}\\
\hat P &= \tfrac{i}{\sqrt2}(a^\dagger - a).
\tag{B-7b}
\end{align}
Su conmutador se calcula con (B-6) y la relación canónica \([\hat X,\hat P]=i\):
\begin{equation}
\begin{split}
[a,a^\dagger]
&= \tfrac12\,[\,\hat X + i\hat P\,,\,\hat X - i\hat P\,]\\
&= \tfrac{i}{2}\,[\hat P,\hat X] \;-\;\tfrac{i}{2}\,[\hat X,\hat P]
=1.
\end{split}
\tag{B-8}
\end{equation}
Es decir,
\begin{equation}
[a,a^\dagger]=1,
\tag{B-9}
\end{equation}
que es equivalente a la conmutación canónica \([\hat X,\hat P]=i\hbar\).

\subsection*{B-2. Determinación del espectro}

Cuando se resuelve la ecuación de autovalores de \(N\),
\[
N\ket{\varphi_\nu}=\nu\ket{\varphi_\nu},
\tag{B-18}
\]
se demuestra que \(\ket{\varphi_\nu}\) es también autovector de \(\hat H\) con autovalor
\[
E_\nu=(\nu+\tfrac12)\hbar,
\tag{B-19}
\]
pues \(\hat H=N+\tfrac12\) [Ecuaciones (B-3) y (B-15)].

\subsubsection*{B-2-a. Lemas}

\paragraph{Lema I (propiedad de los autovalores de \(N\)).}  
Los autovalores \(\nu\) de \(N=a^\dagger a\) son no negativos.  
De hecho, para cualquier autovector \(\ket{\varphi_\nu}\) de \(N\):
\[
\|a\ket{\varphi_\nu}\|^2
=\bra{\varphi_\nu}a^\dagger a\ket{\varphi_\nu}
=\bra{\varphi_\nu}N\ket{\varphi_\nu}
=\nu\braket{\varphi_\nu|\varphi_\nu}\ge0.
\tag{B-20,B-21}
\]
Como \(\braket{\varphi_\nu|\varphi_\nu}>0\), se obtiene \(\nu\ge0\).
  
\paragraph{Lema II (propiedades de \(a\ket{\varphi_\nu}\)).}  
Sea \(\ket{\varphi_\nu}\ne0\) autovector de \(N\) con autovalor \(\nu\); definamos  
\(\ket{\chi}=a\ket{\varphi_\nu}\). Entonces:  
(i) Si \(\nu=0\), \(\|\ket{\chi}\|^2=\nu\|\varphi_\nu\|^2=0\) y \(\ket{\chi}=0\).  
(ii) Si \(\nu>0\), \(\|\ket{\chi}\|^2>0\) y, usando \([N,a]=-a\) [(B-17a)],
\[
N\ket{\chi}=(aN+[N,a])\ket{\varphi_\nu}
=(\nu-1)\ket{\chi}\,,
\tag{B-27}
\]
de modo que \(\ket{\chi}\) es un autovector de \(N\) con autovalor \(\nu-1\).

\paragraph{Lema III (propiedades de \(a^\dagger\ket{\varphi_\nu}\)).}  
Sea \(\ket{\varphi_\nu}\ne0\) autovector de \(N\) con \(\nu\ge0\); definamos  
\(\ket{\chi'}=a^\dagger\ket{\varphi_\nu}\). Entonces:  
(i) \(\|\ket{\chi'}\|^2=\bra{\varphi_\nu}aa^\dagger\ket{\varphi_\nu}
=(\nu+1)\|\varphi_\nu\|^2>0\), luego \(\ket{\chi'}\ne0\).  
(ii) Con \([N,a^\dagger]=a^\dagger\) [(B-17b)] se halla
\[
N\ket{\chi'}
=(a^\dagger N+[N,a^\dagger])\ket{\varphi_\nu}
=(\nu+1)\ket{\chi'}\,.
\tag{B-29}
\]

\subsubsection*{B-2-b. El espectro de \(N\) son enteros no negativos}

Sea \(\nu\) un autovalor de \(N\) y \(\ket{\varphi_\nu}\) su autovector.  

– Si \(\nu\notin\mathbb{Z}\), existe \(k\in\mathbb{Z}\) tal que  
\(\nu-k<0\)  \((\!)\).  
Aplicando sucesivamente \(a\) a \(\ket{\varphi_\nu}\), por el lema II se obtienen autovectores de autovalores \(\nu-1,\nu-2,\dots,\nu-k<0\), lo cual contradice el lema I.  

– Si \(\nu=n\in\{0,1,2,\dots\}\), la cadena  
\[
\ket{\varphi_n},\;a\ket{\varphi_n},\;\dots,\;a^n\ket{\varphi_n},\;
a^{n+1}\ket{\varphi_n}=0
\tag{B-31,B-33}
\]
termina en cero y no produce autovalores negativos.  

Por tanto, los únicos autovalores posibles son los enteros \(\nu=0,1,2,\dots\).

\subsubsection*{B-2-c. Interpretación de los operadores \(a\) y \(a^\dagger\)}

Partiendo de un autovector \(\ket{\varphi_n}\) de \(N\) (y por tanto de \(\hat H\)) con  
\[
\hat H\ket{\varphi_n}=(n+\tfrac12)\hbar\ket{\varphi_n},
\]
la acción de \(a\) y \(a^\dagger\) satisface:
\[
a\ket{\varphi_n}\propto\ket{\varphi_{n-1}},\qquad
a^\dagger\ket{\varphi_n}\propto\ket{\varphi_{n+1}}.
\]
Así, \(a\) “aniquila” un quantum \(\hbar\) de energía y \(a^\dagger\) lo “crea”, de ahí su nombre de operadores de destrucción y creación.

\subsubsection*{B-2-d. Niveles de energía}

De lo anterior y la relación (B-19) se concluye que los niveles de energía del oscilador armónico 1-D son
\[
E_n=\hbar\omega\Bigl(n+\tfrac12\Bigr),
\quad n=0,1,2,\dots
\tag{B-34}
\]
con separación \(\Delta E=\hbar\omega\) y nivel de cero punto \(E_0=\tfrac12\hbar\omega\).````\subsection*{B-2. Determinación del espectro}

Cuando se resuelve la ecuación de autovalores de \(N\),
\[
N\ket{\varphi_\nu}=\nu\ket{\varphi_\nu},
\tag{B-18}
\]
se demuestra que \(\ket{\varphi_\nu}\) es también autovector de \(\hat H\) con autovalor
\[
E_\nu=(\nu+\tfrac12)\hbar,
\tag{B-19}
\]
pues \(\hat H=N+\tfrac12\) [Ecuaciones (B-3) y (B-15)].

\subsubsection*{B-2-a. Lemas}

\paragraph{Lema I (propiedad de los autovalores de \(N\)).}  
Los autovalores \(\nu\) de \(N=a^\dagger a\) son no negativos.  
De hecho, para cualquier autovector \(\ket{\varphi_\nu}\) de \(N\):
\[
\|a\ket{\varphi_\nu}\|^2
=\bra{\varphi_\nu}a^\dagger a\ket{\varphi_\nu}
=\bra{\varphi_\nu}N\ket{\varphi_\nu}
=\nu\braket{\varphi_\nu|\varphi_\nu}\ge0.
\tag{B-20,B-21}
\]
Como \(\braket{\varphi_\nu|\varphi_\nu}>0\), se obtiene \(\nu\ge0\).
  
\paragraph{Lema II (propiedades de \(a\ket{\varphi_\nu}\)).}  
Sea \(\ket{\varphi_\nu}\ne0\) autovector de \(N\) con autovalor \(\nu\); definamos  
\(\ket{\chi}=a\ket{\varphi_\nu}\). Entonces:  
(i) Si \(\nu=0\), \(\|\ket{\chi}\|^2=\nu\|\varphi_\nu\|^2=0\) y \(\ket{\chi}=0\).  
(ii) Si \(\nu>0\), \(\|\ket{\chi}\|^2>0\) y, usando \([N,a]=-a\) [(B-17a)],
\[
N\ket{\chi}=(aN+[N,a])\ket{\varphi_\nu}
=(\nu-1)\ket{\chi}\,,
\tag{B-27}
\]
de modo que \(\ket{\chi}\) es un autovector de \(N\) con autovalor \(\nu-1\).

\paragraph{Lema III (propiedades de \(a^\dagger\ket{\varphi_\nu}\)).}  
Sea \(\ket{\varphi_\nu}\ne0\) autovector de \(N\) con \(\nu\ge0\); definamos  
\(\ket{\chi'}=a^\dagger\ket{\varphi_\nu}\). Entonces:  
(i) \(\|\ket{\chi'}\|^2=\bra{\varphi_\nu}aa^\dagger\ket{\varphi_\nu}
=(\nu+1)\|\varphi_\nu\|^2>0\), luego \(\ket{\chi'}\ne0\).  
(ii) Con \([N,a^\dagger]=a^\dagger\) [(B-17b)] se halla
\[
N\ket{\chi'}
=(a^\dagger N+[N,a^\dagger])\ket{\varphi_\nu}
=(\nu+1)\ket{\chi'}\,.
\tag{B-29}
\]

\subsubsection*{B-2-b. El espectro de \(N\) son enteros no negativos}

Sea \(\nu\) un autovalor de \(N\) y \(\ket{\varphi_\nu}\) su autovector.  

– Si \(\nu\notin\mathbb{Z}\), existe \(k\in\mathbb{Z}\) tal que  
\(\nu-k<0\)  \((\!)\).  
Aplicando sucesivamente \(a\) a \(\ket{\varphi_\nu}\), por el lema II se obtienen autovectores de autovalores \(\nu-1,\nu-2,\dots,\nu-k<0\), lo cual contradice el lema I.  

– Si \(\nu=n\in\{0,1,2,\dots\}\), la cadena  
\[
\ket{\varphi_n},\;a\ket{\varphi_n},\;\dots,\;a^n\ket{\varphi_n},\;
a^{n+1}\ket{\varphi_n}=0
\tag{B-31,B-33}
\]
termina en cero y no produce autovalores negativos.  

Por tanto, los únicos autovalores posibles son los enteros \(\nu=0,1,2,\dots\).

\subsubsection*{B-2-c. Interpretación de los operadores \(a\) y \(a^\dagger\)}

Partiendo de un autovector \(\ket{\varphi_n}\) de \(N\) (y por tanto de \(\hat H\)) con  
\[
\hat H\ket{\varphi_n}=(n+\tfrac12)\hbar\ket{\varphi_n},
\]
la acción de \(a\) y \(a^\dagger\) satisface:
\[
a\ket{\varphi_n}\propto\ket{\varphi_{n-1}},\qquad
a^\dagger\ket{\varphi_n}\propto\ket{\varphi_{n+1}}.
\]
Así, \(a\) “aniquila” un quantum \(\hbar\) de energía y \(a^\dagger\) lo “crea”, de ahí su nombre de operadores de destrucción y creación.

\subsubsection*{B-2-d. Niveles de energía}

De lo anterior y la relación (B-19) se concluye que los niveles de energía del oscilador armónico 1-D son
\[
E_n=\hbar\omega\Bigl(n+\tfrac12\Bigr),
\quad n=0,1,2,\dots
\tag{B-34}
\]
con separación \(\Delta E=\hbar\omega\) y nivel de cero punto \(E_0=\tfrac12\hbar\omega\).


\section*{Conclusión}
El estudio del espín $1/2$ ejemplifica claramente los postulados de la mecánica cuántica: preparación de estados, probabilidades de medición y evolución temporal gobernada por el Hamiltoniano.
\end{document}