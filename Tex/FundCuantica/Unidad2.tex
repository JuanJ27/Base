\documentclass[a4paper,11pt]{article}
\usepackage[utf8]{inputenc}
\usepackage[spanish]{babel}
\usepackage{amsmath,amssymb}
\usepackage{braket}
\usepackage{graphicx}
\usepackage{hyperref}
\usepackage{bookmark}
\usepackage[left=2cm,right=2cm,top=2cm,bottom=2cm]{geometry}

\title{Notas de Clase}
\author{Juan Montoya}
\date{\today}

\begin{document}
\maketitle

\section*{Resumen}
El objetivo de estas notas es ilustrar los postulados fundamentales de la mecánica cuántica mediante el caso de un sistema de espín $1/2$ (por ejemplo, átomos de plata) y el uso del aparato de Stern–Gerlach. Se aborda la preparación de estados, la naturaleza probabilística de las mediciones y la evolución temporal bajo un Hamiltoniano simple.

\section{Operador \texorpdfstring{$S_z$} y espacio de espín}
Al observable $\mathcal{S}_z$ corresponde el operador $S_z$, cuyos autovalores son $\pm\hbar/2$. Denotamos por $\ket{+}$ y $\ket{-}$ los autovectores ortonormales:
\begin{equation}
    \begin{cases}
    S_z \ket{+} = +\dfrac{\hbar}{2}\ket{+},\\
    S_z \ket{-} = -\dfrac{\hbar}{2}\ket{-},
    \end{cases}
\tag{A-10}
\end{equation}
con
\begin{equation}
    \begin{cases}
    \braket{+|+} = \braket{-|-} = 1,\\
    \braket{+|-} = 0,
    \end{cases}
\tag{A-11}
\end{equation}
y la relación de cierre
\begin{equation}
\ket{+}\bra{+} + \ket{-}\bra{-} = \mathbb{I}.
\tag{A-12}
\end{equation}

\subsection*{A-2-b. Los operadores $S_x$, $S_y$ y $S_u$}
Los operadores $S_x$, $S_y$ y $S_u$ tienen los mismos autovalores, $+\hbar/2$ y $-\hbar/2$, que $S_z$. Este resultado era de esperar, ya que siempre es posible girar todo el conjunto del aparato de Stern–Gerlach de modo que el eje definido por el campo magnético quede paralelo a $Ox$, $Oy$ o $\vec u$. Dado que todas las direcciones del espacio son físicamente equivalentes, los fenómenos observados en la placa no cambian bajo tales rotaciones; así, la medición de $S_x$, $S_y$ o $S_u$ sólo puede dar como resultado $+\hbar/2$ o $-\hbar/2$.

En cuanto a los autovectores de $S_x$, $S_y$ y $S_u$, los denotaremos respectivamente por $\ket{\pm}_x$, $\ket{\pm}_y$ y $\ket{\pm}_u$ (el signo en el ket coincide con el del autovalor correspondiente). Sus desarrollos en la base $\{\ket{+},\ket{-}\}$ de $S_z$ se escriben:

\begin{equation}
\ket{\pm}_x = \frac{1}{\sqrt{2}}\bigl(\ket{+}\pm\ket{-}\bigr)
\tag{A-20}
\end{equation}

\begin{equation}
\ket{\pm}_y = \frac{1}{\sqrt{2}}\bigl(\ket{+}\pm i\,\ket{-}\bigr)
\tag{A-21}
\end{equation}

\begin{equation}
\begin{cases}
\ket{+}_u = \cos\dfrac{\theta}{2}\,e^{-i\varphi/2}\,\ket{+}
           + \sin\dfrac{\theta}{2}\,e^{i\varphi/2}\,\ket{-},\\[8pt]
\ket{-}_u = -\,\sin\dfrac{\theta}{2}\,e^{-i\varphi/2}\,\ket{+}
           + \cos\dfrac{\theta}{2}\,e^{i\varphi/2}\,\ket{-}.
\end{cases}
\tag{A-22a,b}
\end{equation}

\section{Estado general y parámetros esféricos}
El estado más general en el espacio de espín es
\begin{equation}
\ket{\psi} = \alpha \ket{+} + \beta \ket{-},
\tag{A-13}
\end{equation}
sujeto a
\begin{equation}
|\alpha|^2 + |\beta|^2 = 1.
\tag{A-14}
\end{equation}
Con la parametrización
\[
\alpha = \cos\frac{\theta}{2}e^{-i\varphi/2},\quad
\beta  = \sin\frac{\theta}{2}e^{i\varphi/2},
\]
podemos asociar cada par $(\alpha,\beta)$ a un vector unitario en la esfera de Bloch.

\section{Mediciones de espín}
Para ilustrar la naturaleza probabilística de las mediciones:
\begin{itemize}
    \item \textbf{Experimento 1:} Con ambos aparatos alineados, si se prepara $\ket{+}$ siempre se obtiene $+\hbar/2$.
    \item \textbf{Experimento 2:} Si se prepara $\ket{+}_u$ (dirección $\vec{u}$) y se mide $S_z$, las probabilidades son
        \[
            P(+\tfrac{\hbar}{2}) = \cos^2\frac{\theta}{2},\quad
            P(-\tfrac{\hbar}{2}) = \sin^2\frac{\theta}{2}.
        \]
    \item \textbf{Experimento 3:} Al rotar el analizador, las probabilidades cambian con el ángulo relativo.
\end{itemize}
El valor medio se corresponde con el resultado clásico:
\[
\langle S_z\rangle = \frac{\hbar}{2}\cos\theta.
\]

\section{Evolución temporal}
En un campo magnético uniforme $\vec{B}_0$, el Hamiltoniano es
\[
\hat H = \omega_0 \hat S_z,\quad \omega_0 = -\gamma B_0.
\]
Sus autoestados $\ket{\pm}$ tienen energías separadas por $\hbar\omega_0$. Si
\[
\ket{\psi(0)} = \cos\frac{\theta}{2}e^{-i\varphi/2}\ket{+}
                             + \sin\frac{\theta}{2}e^{i\varphi/2}\ket{-},
\]
entonces la evolución de Schrödinger produce una precesión de Larmor:
\[
\ket{\psi(t)}
= \cos\frac{\theta}{2}e^{-i(\varphi+\omega_0 t)/2}\ket{+}
+ \sin\frac{\theta}{2}e^{i(\varphi+\omega_0 t)/2}\ket{-}.
\]

\subsection*{Demostración de la existencia de un vector unitario $\mathbf{u}$}
% Demostración de la existencia de un vector unitario u
Vamos a mostrar que existe, para todo $\ket{\psi}$, un vector unitario $\mathbf{u}$ tal que $\ket{\psi}$ es colineal con el ket $\ket{+}_u$. Elegimos por tanto dos números complejos $\alpha$ y $\beta$ que satisfacen la relación
\begin{equation}
|\alpha|^2 + |\beta|^2 = 1
\tag{B-2}
\end{equation}
pero que son arbitrarios en lo demás. Teniendo en cuenta (B-2), existe necesariamente un ángulo $\theta$ tal que
\begin{equation}
\begin{cases}
\cos\dfrac{\theta}{2} = |\alpha|,\\
\sin\dfrac{\theta}{2} = |\beta|.
\end{cases}
\tag{B-3}
\end{equation}
Si, además, imponemos
\begin{equation}
0 \le \theta \le \pi,
\tag{B-4}
\end{equation}
la ecuación
\[
\tan\dfrac{\theta}{2} = \left|\dfrac{\beta}{\alpha}\right|
\]
determina $\theta$ de forma única. Sabemos que sólo la diferencia de fases de $\alpha$ y $\beta$ influye en las predicciones físicas. Definimos entonces
\begin{align}
\varphi &= Arg(\beta) - Arg(\alpha), \tag{B-5}\\
\chi    &= Arg(\beta) + Arg(\alpha). \tag{B-6}
\end{align}
De aquí se sigue
\begin{equation}
    Arg(\beta) = \tfrac12\,\chi + \tfrac12\,\varphi,
    \quad
    Arg(\alpha) = \tfrac12\,\chi - \tfrac12\,\varphi.
\tag{B-7}
\end{equation}
Con esta notación, el ket $\ket{\psi}$ puede escribirse:
\begin{equation}
\ket{\psi}
= e^{i\chi/2}
\Bigl[
\cos\dfrac{\theta}{2}\,e^{-i\varphi/2}\,\ket{+}
+
\sin\dfrac{\theta}{2}\,e^{i\varphi/2}\,\ket{-}
\Bigr].
\tag{B-8}
\end{equation}

\subsection*{Resumen de la Figura 9: Trayectorias clásicas vs.\ superposición cuántica}

Cuando un átomo de plata entra con espín en el estado $\ket{+}$ (Fig.~9-a) o $\ket{-}$ (Fig.~9-b), su función de onda externa está concentrada en un único paquete estrecho cuyo centro recorre una trayectoria que puede describirse clásicamente.  

Sin embargo, si el estado de espín es la superposición  
\[
\ket{\psi} = \cos\frac{\theta}{2}\,\ket{+} + \sin\frac{\theta}{2}\,\ket{-}
\tag{B-9}
\]
el paquete de onda inicial se divide en dos subpaquetes (Fig.~9-c), cada uno localizado cerca de los puntos 1 y 2 al llegar a la pantalla. Aunque cada subpaquete sigue siendo muy estrecho, el átomo ya no tiene una sola trayectoria clásica: la probabilidad de detección se reparte entre ambos lugares.  

Estos dos subpaquetes corresponden a la misma partícula con distinta fase relativa; de hecho, si no se realiza la medición (quitan- do la pantalla) y se aplica un gradiente de campo magnético inverso, podrían recombinarse en un único paquete.

\textbf{Comentario:}\\
(i) Si el campo $\mathbf{B}_{0}$ es paralelo al vector unitario $\mathbf{u}$ de ángulos polares $\theta,\varphi$, la ecuación (B-17) debe reemplazarse por
\begin{equation}
H=\omega_{0}\,S_{u}
\tag{B-20}
\end{equation}
donde $S_{u}=\mathbf{S}\cdot\mathbf{u}$.\\
(ii) Para el átomo de plata $\gamma<0$, luego $\omega_{0}>0$ según (B-16), lo cual explica la disposición de los niveles en la figura.

\subsection*{B-3. Evolución de un espín 1/2 en un campo magnético uniforme}

\subsubsection*{B-3-a. Hamiltoniano de interacción y ecuación de Schrödinger}

Consideremos un átomo de plata en un campo magnético uniforme $\mathbf{B}_{0}$, y tomemos el eje $O z$ paralelo a $\mathbf{B}_{0}$. La energía potencial clásica del momento magnético $\boldsymbol{\mathcal{M}}=\gamma\,\mathbf{S}$ es:
\begin{equation}
W=-\boldsymbol{\mathcal{M}}\cdot\mathbf{B}_{0}
=-\mathcal{M}_{z}\,B_{0}
=-\gamma\,B_{0}\,S_{z}
\tag{B-15}
\end{equation}
donde $B_{0}=\lvert\mathbf{B}_{0}\rvert$. Definimos:
\begin{equation}
\omega_{0}=-\gamma\,B_{0}
\tag{B-16}
\end{equation}
que tiene dimensión de velocidad angular.

Al cuantizar sólo los grados internos, $S_{z}$ se reemplaza por el operador $\hat S_{z}$ y la energía (B-15) pasa a ser el Hamiltoniano
\begin{equation}
H=\omega_{0}\,\hat S_{z}
\tag{B-17}
\end{equation}
Este operador es independiente del tiempo, por lo que resolver la ecuación de Schrödinger equivale a encontrar los autovalores de $H$. Sus autovectores son los mismos de $\hat S_{z}$:
\begin{equation}
\begin{cases}
H\ket{+}=+\dfrac{\hbar\,\omega_{0}}{2}\,\ket{+},\\[6pt]
H\ket{-}=-\dfrac{\hbar\,\omega_{0}}{2}\,\ket{-}.
\end{cases}
\tag{B-18}
\end{equation}
Por tanto, existen dos niveles de energía,
\[
E_{+}=+\frac{\hbar\omega_{0}}{2},
\quad
E_{-}=-\frac{\hbar\omega_{0}}{2},
\]
y su separación $\hbar\omega_{0}$ define la frecuencia de Bohr
\begin{equation}
\nu_{+-}
=\frac{1}{h}\,(E_{+}-E_{-})
=\frac{\omega_{0}}{2\pi}\,.
\tag{B-19}
\end{equation}

\subsection*{C-2. Aspecto estático: efecto del acoplamiento en los estados estacionarios del sistema}

\subsubsection*{C-2-a. Expresiones para los autoestados y autoenergías de \(H\)}

En la base \(\{\ket{\varphi_1},\ket{\varphi_2}\}\), la matriz que representa \(H\) es:
\begin{equation}
H = \begin{pmatrix}
E_1 + W_{11} & W_{12} \\
W_{21}       & E_2 + W_{22}
\end{pmatrix}
\tag{C-7}
\end{equation}

Su diagonalización conduce a los autoenergías:
\begin{equation}
\begin{aligned}
E_+ &= \tfrac12\bigl(E_1 + W_{11} + E_2 + W_{22}\bigr)
       + \tfrac12\sqrt{\bigl(E_1 + W_{11} - E_2 - W_{22}\bigr)^2 + 4|W_{12}|^2},\\
E_- &= \tfrac12\bigl(E_1 + W_{11} + E_2 + W_{22}\bigr)
       - \tfrac12\sqrt{\bigl(E_1 + W_{11} - E_2 - W_{22}\bigr)^2 + 4|W_{12}|^2}.
\end{aligned}
\tag{C-8}
\end{equation}
(si \(W_{ij}=0\), entonces \(E_+=E_1\) y \(E_-=E_2\)).  

Los autoestados asociados son:
\begin{align}
\ket{\psi_+}
&= \cos\frac{\theta}{2}\,e^{-i\varphi/2}\,\ket{\varphi_1}
+ \sin\frac{\theta}{2}\,e^{i\varphi/2}\,\ket{\varphi_2},
\tag{C-9a}\\
\ket{\psi_-}
&= -\sin\frac{\theta}{2}\,e^{-i\varphi/2}\,\ket{\varphi_1}
+ \cos\frac{\theta}{2}\,e^{i\varphi/2}\,\ket{\varphi_2}.
\tag{C-9b}
\end{align}

donde los ángulos \(\theta\) y \(\varphi\) se definen por:
\begin{equation}
\tan\theta
= \frac{2\,|W_{12}|}{E_1 + W_{11} - E_2 - W_{22}},
\quad
0 \le \theta < \pi,
\tag{C-10}
\end{equation}
\begin{equation}
W_{21} = |W_{21}|\,e^{i\varphi}.
\tag{C-11}
\end{equation}

\subsection*{C-3. Aspecto dinámico: oscilación del sistema entre los dos estados no perturbados}

\subsubsection*{C-3-a. Evolución del vector de estado}

Sea el vector de estado en el instante \(t\):
\[
\ket{\psi(t)} = a_1(t)\ket{\varphi_1} + a_2(t)\ket{\varphi_2}
\tag{C-22}
\]
La ecuación de Schrödinger es:
\[
i\hbar\,\frac{d}{dt}\ket{\psi(t)} = \bigl(H_0 + W\bigr)\ket{\psi(t)}\,.
\tag{C-23}
\]
Proyectando en \(\ket{\varphi_1}\) y \(\ket{\varphi_2}\), con \(W_{11}=W_{22}=0\), obtenemos el sistema acoplado:
\[
\begin{aligned}
i\hbar\,\frac{da_1}{dt} &= E_1\,a_1 + W_{12}\,a_2\,,\\
i\hbar\,\frac{da_2}{dt} &= W_{21}\,a_1 + E_2\,a_2\,.
\end{aligned}
\tag{C-24}
\]
La solución se construye descomponiendo \(\ket{\psi(0)}\) en los autoestados \(\ket{\psi_+}\), \(\ket{\psi_-}\) de \(H=H_0+W\):
\[
\ket{\psi(0)}=\lambda\,\ket{\psi_+}+\mu\,\ket{\psi_-}\,,
\tag{C-25}
\]
y asumiendo la condición inicial
\[
\ket{\psi(0)}=\ket{\varphi_1}\,.
\tag{C-27}
\]

\subsubsection*{C-3-b. Cálculo de \(\mathcal{P}_{12}(t)\): fórmula de Rabi}

Expandiendo \(\ket{\psi(0)}\) en la base \(\{\ket{\psi_+},\ket{\psi_-}\}\) (invertir (C-9)), se tiene:
\[
\ket{\psi(0)} = \ket{\varphi_1}
= e^{i\varphi/2}\Bigl[\cos\frac{\theta}{2}\,\ket{\psi_+}
                 -\sin\frac{\theta}{2}\,\ket{\psi_-}\Bigr].
\tag{C-28}
\]
Entonces la evolución temporal es
\[
\ket{\psi(t)}
= e^{i\varphi/2}\Bigl[\cos\frac{\theta}{2}\,e^{-iE_+t/\hbar}\ket{\psi_+}
                      -\sin\frac{\theta}{2}\,e^{-iE_-t/\hbar}\ket{\psi_-}\Bigr].
\tag{C-29}
\]
La amplitud de probabilidad de hallar el sistema en \(\ket{\varphi_2}\) es
\[
\braket{\varphi_2|\psi(t)}
= e^{i\varphi/2}\Bigl[\cos\frac{\theta}{2}\,e^{-iE_+t/\hbar}\braket{\varphi_2|\psi_+}
                     -\sin\frac{\theta}{2}\,e^{-iE_-t/\hbar}\braket{\varphi_2|\psi_-}\Bigr],
\tag{C-30}
\]
y la probabilidad
\[
\mathcal{P}_{12}(t) = \bigl|\braket{\varphi_2|\psi(t)}\bigr|^2
= \tfrac12\sin^2\theta\bigl[1-\cos\bigl((E_+ - E_-)\,t/\hbar\bigr)\bigr]
= \sin^2\theta\;\sin^2\!\bigl((E_+ - E_-)\,t/2\hbar\bigr).
\tag{C-31}
\]
Usando además las expresiones de los ángulos \(\theta\) y \(\varphi\) [(C-12),(C-13)], se reescribe como
\[
\mathcal{P}_{12}(t)
= \frac{4\,|W_{12}|^2}{4\,|W_{12}|^2 + (E_1 - E_2)^2}
  \;\sin^2\!\Bigl[t/(2\hbar)\,\sqrt{4\,|W_{12}|^2 + (E_1 - E_2)^2}\Bigr].
\tag{C-32}
\]
Esta última es la conocida como \emph{fórmula de Rabi}.

\section*{Complemento JIV: Ejercicios}

\subsection*{Ejercicio 1}
Considera una partícula de espín $1/2$ de momento magnético $\mathbf{M}=\gamma\,\mathbf{S}$. El espacio de estados de espín está generado por los vectores $\ket{+}$ y $\ket{-}$, autovectores de $S_z$ con autovalores $+\hbar/2$ y $-\hbar/2$. En $t=0$, el estado del sistema es
\[
\ket{\psi(0)}=\ket{+}.
\]
\begin{enumerate}
  \item Si en $t=0$ medimos el observable $S_z$, ¿qué resultados se pueden obtener y con qué probabilidades?
  \item En lugar de realizar esa medición, dejamos que el sistema evolucione bajo la acción de un campo magnético uniforme paralelo a $Oy$, de módulo $B_0$. Calcula, en la base $\{\ket{+},\ket{-}\}$, el estado del sistema en tiempo $t$.
  \item En ese instante medimos los observables $S_x$, $S_y$ y $S_z$. ¿Qué valores pueden obtenerse y con qué probabilidades? ¿Qué relación debe existir entre $B_0$ y $t$ para que alguno de los resultados sea absolutamente cierto? Da su interpretación física.
\end{enumerate}

\subsection*{Ejercicio 2}
Considera de nuevo una partícula de espín $1/2$ (misma notación).
\begin{enumerate}
  \item En $t=0$ medimos $S_y$ y hallamos $+\hbar/2$. ¿Cuál es el vector de estado $\ket{\psi(0)}$ inmediatamente tras la medición?
  \item Inmediatamente después aplicamos un campo magnético uniforme dependiente del tiempo, paralelo a $Oz$, de modo que
  \[
    H(t)=\omega_0(t)\,S_z\,.
  \]
  Supón que $\omega_0(t)=0$ para $t<0$ y $t>T$, y que crece linealmente de $0$ a $\omega_0$ en el intervalo $0<t<T$ (siendo $T$ un parámetro de tiempo dado). Demuestra que, en cualquier instante $t$, el estado puede escribirse
  \[
    \ket{\psi(t)}
    = \frac{1}{\sqrt{2}}\Bigl[e^{\,i\theta(t)}\ket{+}
      \;+\; i\,e^{-\,i\theta(t)}\ket{-}\Bigr],
  \]
  donde $\theta(t)$ es una función real de $t$ (a calcular).
  \item En $t=T$ medimos $S_y$. ¿Qué resultados pueden hallarse y con qué probabilidades? Determina la relación entre $\omega_0$ y $T$ para garantizar un resultado único. Da su interpretación física.
\end{enumerate}

\textbf{Solución:}
\begin{enumerate}
  \item Tras medir $S_y=+\hbar/2$ el sistema queda en el autovector correspondiente,
    \[
      \ket{\psi(0)} = \ket{+}_y
      = \frac{1}{\sqrt{2}}\bigl(\ket{+} + i\ket{-}\bigr).
    \]
  \item Como $[H(t),H(t')]=0$, la evolución es
    \[
      U(t)
      = \exp\!\Bigl(-\tfrac{i}{\hbar}\!\int_0^tH(t')\,dt'\Bigr).
    \]
    Dado que 
    $H(t)\ket{\pm}=\pm\frac{\hbar}{2}\,\omega_0(t)\ket{\pm}$, definimos
    \[
      \theta(t)=\frac{1}{2}\int_0^t\omega_0(t')\,dt',
    \]
    y se obtiene
    \[
      U(t)\ket{+}=e^{-\,i\theta(t)}\ket{+},\qquad
      U(t)\ket{-}=e^{\,i\theta(t)}\ket{-}.
    \]
    Aplicando a $\ket{\psi(0)}$:
    \[
      \ket{\psi(t)}
      =\frac{1}{\sqrt{2}}\Bigl(e^{-\,i\theta(t)}\ket{+}
        +i\,e^{\,i\theta(t)}\ket{-}\Bigr).
    \]
    Redefiniendo $\theta\to-\theta$ recuperamos la forma deseada.
  \item Escribimos las proyecciones sobre $\ket{\pm}_y$:
    \[
      \ket{+}_y=\tfrac1{\sqrt2}(\ket{+}+i\ket{-}),\quad
      \ket{-}_y=\tfrac1{\sqrt2}(\ket{+}-i\ket{-}).
    \]
    Entonces
    \[
      P\bigl(+\tfrac{\hbar}{2}\bigr)
      =\bigl|\braket{+_y|\psi(T)}\bigr|^2
      =\cos^2\theta(T),
      \quad
      P\bigl(-\tfrac{\hbar}{2}\bigr)=\sin^2\theta(T).
    \]
    Para que el resultado sea siempre $+\hbar/2$ se pide
    \[
      \cos^2\theta(T)=1
      \;\Longrightarrow\;
      \theta(T)=n\pi
      \;\Longrightarrow\;
      \int_0^T\omega_0(t)\,dt=2n\pi.
    \]
    Si $\omega_0(t)=\tfrac{\omega_0}{T}\,t$ en $[0,T]$, entonces
    $\int_0^T\omega_0(t)\,dt=\tfrac12\,\omega_0T$, y la condición es
    $\tfrac12\,\omega_0T=2n\pi$, esto es $\omega_0T=4n\pi$.  
    Físicamente, equivale a girar el espín alrededor de $Oz$ un número entero
    de vueltas completas, de modo que el autovector de $S_y$ regresa a sí mismo.
\end{enumerate}

\subsection*{Ejercicio 3}
Considera una partícula de espín $1/2$ en un campo magnético $\mathbf{B}_0$ con componentes
\[
  B_x=\frac{1}{\sqrt{2}}\,B_0,\quad B_y=0,\quad B_z=\frac{1}{\sqrt{2}}\,B_0.
\]
\begin{enumerate}
  \item Calcula la matriz que representa el Hamiltoniano $H=\gamma\,\mathbf{S}\cdot\mathbf{B}_0$ en la base $\{\ket{+},\ket{-}\}$ de $S_z$.
  \item Halla los autovalores y autovectores de $H$.
  \item Si en $t=0$ el sistema está en el estado $\ket{-}$, ¿qué valores de energía se obtienen y con qué probabilidades?
  \item Determina el vector de estado $\ket{\psi(t)}$ en tiempo $t$. En ese instante medimos $S_x$: calcula el valor medio de la medición y explica su interpretación geométrica.
\end{enumerate}


\subsection*{Ejercicio 4}
Considera el dispositivo experimental descrito en § B-2-b del Capítulo IV: un haz de átomos de espín \(1/2\) atraviesa un “polarizador” que selecciona la dirección que forma un ángulo \(\theta\) con el eje \(Oz\) en el plano \(xOz\), y luego un “analizador” que mide la componente \(S_z\). Entre polarizador y analizador, sobre una longitud \(L\) del haz, se aplica un campo magnético uniforme \(\mathbf{B}_0\parallel Oz\). Sea \(v\) la velocidad de los átomos y \(T=L/v\) el tiempo de interacción. Definimos
\[
\omega_0 = -\gamma\,B_0.
\]

\begin{enumerate}
  \item ¿Cuál es el vector de estado \(\ket{\psi_i}\) de un espín al entrar en el analizador?
  \item Demuestra que, al medir en el analizador, la probabilidad de encontrar \(+\hbar/2\) es
  \[
    \tfrac12\bigl(1+\cos\theta\cos\omega_0 T\bigr),
  \]
  y la de encontrar \(-\hbar/2\) es
  \[
    \tfrac12\bigl(1-\cos\theta\cos\omega_0 T\bigr).
  \]
  Da una interpretación física.
  \item (Esta parte y la siguiente implican el concepto de operador densidad, definido en el Complemento EIV.) Demuestra que la matriz de densidad \(\rho_1\) de una partícula al entrar en el analizador, en la base \(\{\ket{+},\ket{-}\}\), es
  \[
    \rho_1 = \frac12
    \begin{pmatrix}
      1 + \cos\theta\cos(\omega_0 T)                  & \sin\theta + i\,\cos\theta\,\sin(\omega_0 T) \\[6pt]
      \sin\theta - i\,\cos\theta\,\sin(\omega_0 T)     & 1 - \cos\theta\cos(\omega_0 T)
    \end{pmatrix}.
  \]
  Calcula \( Tr(\rho_1 S_x)\), \( Tr(\rho_1 S_y)\) y \( Tr(\rho_1 S_z)\). ¿Describe \(\rho_1\) un estado puro?

\item Ahora supongamos que la velocidad de un átomo es una variable aleatoria, y por tanto el tiempo de vuelo \(T\) sólo se conoce con una incertidumbre \(\Delta T\). Además, el campo \(\mathbf{B}_0\) es tan intenso que 
\[
\omega_0\,\Delta T \gg 1.
\]
Los posibles valores del producto \(\omega_0 T\) (módulo \(2\pi\)) son entonces todos los comprendidos entre \(0\) y \(2\pi\), equiprobables.

  \begin{enumerate}
    \item ¿Cuál es el operador densidad \(\rho_2\) de un átomo en el instante en que entra en el analizador? ¿Corresponde \(\rho_2\) a un estado puro?
    \item Calcula \( Tr(\rho_2\,S_x)\), \( Tr(\rho_2\,S_y)\) y \( Tr(\rho_2\,S_z)\). ¿Cuál es tu interpretación? ¿En qué caso el operador densidad describe un espín completamente polarizado? ¿Uno completamente no polarizado?
    \item Describe cualitativamente los fenómenos observados a la salida del analizador al variar \(\omega_0\) desde \(0\) hasta el régimen \(\omega_0\,\Delta T\gg1\).
  \end{enumerate}

\end{enumerate}

\subsection*{Ejercicio 5. Operador de evolución de un espín \(1/2\) (Complemento FIII)}

Considera un espín \(1/2\) de momento magnético \(\mathbf{M}=\gamma\,\mathbf{S}\) en un campo magnético
\(\mathbf{B}_0\) de componentes
\[
B_x=-\frac{\omega_x}{\gamma},\quad
B_y=-\frac{\omega_y}{\gamma},\quad
B_z=-\frac{\omega_z}{\gamma}.
\]
Definimos 
\[
\omega_0=\sqrt{\omega_x^2+\omega_y^2+\omega_z^2}\,.
\]

\begin{enumerate}
\item Demuestra que el operador de evolución es
\[
U(t,0)=\exp(i M t),
\]
donde
\[
M=\frac{1}{\hbar}\bigl(\omega_x S_x+\omega_y S_y+\omega_z S_z\bigr)
=\frac12\bigl(\omega_x\sigma_x+\omega_y\sigma_y+\omega_z\sigma_z\bigr).
\]
Calcula la matriz de \(M\) en la base \(\{\ket{+},\ket{-}\}\) de autovectores de \(S_z\), y muestra que
\[
M^2=\frac14\bigl(\omega_x^2+\omega_y^2+\omega_z^2\bigr)
=\Bigl(\frac{\omega_0}{2}\Bigr)^2.
\]

\item Escribe \(U(t,0)\) en la forma
\[
U(t,0)=\cos\!\frac{\omega_0t}{2}
\;-\;\frac{2i}{\omega_0}\,M\;\sin\!\frac{\omega_0t}{2}.
\]

\item Supón que en \(t=0\) el espín está en \(\ket{\psi(0)}=\ket{+}\).  
La probabilidad de hallarlo en \(\ket{+}\) a tiempo \(t\) es
\[
\mathcal{P}_{++}(t)=\bigl|\bra{+}U(t,0)\ket{+}\bigr|^2.
\]
Usando la forma de \(U(t,0)\), demuestra que
\[
\mathcal{P}_{++}(t)
=1-\frac{\omega_x^2+\omega_y^2}{\omega_0^2}\,
\sin^2\!\frac{\omega_0 t}{2}.
\]
Proporciona una interpretación geométrica de este resultado.
\end{enumerate}

\subsection*{Ejercicio 6}
Considera el sistema de dos espines \(1/2\), \(\mathbf{S}_1\) y \(\mathbf{S}_2\), en la base de cuatro vectores \(\{\ket{+,+},\ket{+,-},\ket{-,+},\ket{-,-}\}\) definida en el Complemento DIV. En \(t=0\), el estado del sistema es
\[
\ket{\psi(0)} = \frac12\ket{+,+} \;+\;\frac12\ket{+,-}\;+\;\frac1{\sqrt2}\ket{-,-}.
\]
\begin{enumerate}
  \item En \(t=0\) se mide \(S_{1z}\): ¿cuál es la probabilidad de obtener \(-\hbar/2\)? ¿Cuál es el vector de estado tras la medición? Si a continuación medimos \(S_{1x}\), ¿qué resultados pueden obtenerse y con qué probabilidades? Responde de igual forma si la medición de \(S_{1z}\) hubiera dado \(+\hbar/2\).
  \item Cuando el sistema está en el estado \(\ket{\psi(0)}\), se miden simultáneamente \(S_{1x}\) y \(S_{2z}\). ¿Cuál es la probabilidad de encontrar resultados opuestos? ¿Y resultados idénticos?
  \item En lugar de las mediciones anteriores, dejamos que el sistema evolucione bajo el Hamiltoniano
    \[
      H = \omega_1\,S_{1z} \;+\;\omega_2\,S_{2z}.
    \]
    Determina el estado \(\ket{\psi(t)}\) a tiempo \(t\). Calcula en ese instante los valores medios \(\langle S_1\rangle\) y \(\langle S_2\rangle\). Da una interpretación física.
  \item Demuestra que las longitudes de los vectores \(\langle S_1\rangle\) y \(\langle S_2\rangle\) son menores que \(\hbar/2\). ¿Qué forma debe tener \(\ket{\psi(0)}\) para que cada una de esas longitudes sea exactamente \(+\hbar/2\)?
\end{enumerate}

\subsection*{Ejercicio 7}
Considera el mismo sistema de dos espines \(1/2\) del ejercicio anterior; el espacio de estados está generado por la base \(\{\ket{\pm,\pm}\}\).
\begin{enumerate}
  \item Escribe la matriz \(4\times4\) que representa, en esta base, el operador \(S_{1y}\). ¿Cuáles son sus autovalores y autovectores?
  \item El estado normalizado del sistema es
    \[
      \ket{\psi}=\alpha\ket{+,+} + \beta\ket{+,-} + \gamma\ket{-,+} + \delta\ket{-,-},
    \]
    con \(\alpha,\beta,\gamma,\delta\in\mathbb{C}\). Se miden simultáneamente \(S_{1x}\) y \(S_{2y}\). ¿Qué resultados pueden hallarse y con qué probabilidades? ¿Cómo cambian estas probabilidades si \(\ket{\psi}\) es un producto tensorial de un estado del primer espín y un estado del segundo espín?
  \item Repite las preguntas del apartado anterior para una medición de \(S_{1y}\) y \(S_{2y}\).
  \item En lugar de realizar las mediciones anteriores, medimos sólo \(S_{2y}\). Calcula, primero a partir de los resultados del apartado b) y luego de los del c), la probabilidad de encontrar \(-\hbar/2\).
\end{enumerate}

\subsection*{Ejercicio 8}
Considera un electrón en una molécula lineal triatómica formada por tres átomos equidistantes:
\[
A\;-\;B\;-\;C.
\]
Usamos \(\ket{\varphi_A},\ket{\varphi_B},\ket{\varphi_C}\) para denotar tres estados ortonormales localizados en los núcleos de los átomos \(A\), \(B\) y \(C\). Nos restringimos al subespacio generado por \(\{\ket{\varphi_A},\ket{\varphi_B},\ket{\varphi_C}\}\). Si despreciamos el salto del electrón entre núcleos, su energía viene dada por
\[
H_0\ket{\varphi_A}=E_0\ket{\varphi_A},\quad
H_0\ket{\varphi_B}=E_0\ket{\varphi_B},\quad
H_0\ket{\varphi_C}=E_0\ket{\varphi_C}.
\]
El acoplamiento entre estos estados lo describe un Hamiltoniano adicional \(W\):
\[
W\ket{\varphi_A}=-a\,\ket{\varphi_B},\quad
W\ket{\varphi_B}=-a\bigl(\ket{\varphi_A}+\ket{\varphi_C}\bigr),\quad
W\ket{\varphi_C}=-a\,\ket{\varphi_B},
\]
donde \(a>0\) es una constante real. El Hamiltoniano total es
\[
H = H_0 + W.
\]

\begin{enumerate}
  \item Calcula las energías y los estados estacionarios (autovectores) de \(H\).
  \item Si en \(t=0\) el electrón está en el estado \(\ket{\varphi_A}\), discute cualitativamente su localización en tiempos posteriores. ¿Hay instantes \(t\) en que esté perfectamente localizado en \(A\), \(B\) o \(C\)?
  \item Sea el observable \(D\) cuyos autovectores son \(\ket{\varphi_A},\ket{\varphi_B},\ket{\varphi_C}\) con autovalores \(-d,0,+d\), respectivamente. Si se mide \(D\) en tiempo \(t\), ¿qué valores pueden obtenerse y con qué probabilidades?
  \item Si el estado inicial es arbitrario, ¿cuáles son las frecuencias de Bohr que aparecen en la evolución de \(\langle D\rangle\)? Da una interpretación física de estas frecuencias como las de las radiaciones electromagnéticas que la molécula puede absorber o emitir.
\end{enumerate}

\subsection*{Ejercicio 9.a Definición y espectro de \(R\)}

Sea \(\{\ket{\varphi_n}\}_{n=1}^6\) una base ortonormal. Definimos el operador \(R\) por:
\[
R\ket{\varphi_1}=\ket{\varphi_2},\quad
R\ket{\varphi_2}=\ket{\varphi_3},\;\dots,\;
R\ket{\varphi_6}=\ket{\varphi_1}.
\]
Buscamos \(\lambda\in\mathbb{C}\) y \(\ket{\psi}\neq0\) tales que
\[
R\ket{\psi}=\lambda\ket{\psi}.
\]
Si escribimos \(\ket{\psi}=\sum_{n=1}^6c_n\ket{\varphi_n}\), la ecuación anterior da el sistema
\[
c_1=\lambda\,c_6,\quad
c_2=\lambda\,c_1,\;\dots,\;
c_6=\lambda\,c_5.
\]
De ello se deduce \(\lambda^6=1\), es decir
\[
\lambda_k = e^{2\pi i k/6},\quad k=0,1,\dots,5.
\]
Los autovectores normalizados son
\[
\ket{\psi_k}
= \frac{1}{\sqrt6}\sum_{n=1}^6 e^{-2\pi i k(n-1)/6}\,\ket{\varphi_n},
\]
que forman un sistema ortonormal y completo en el espacio de estados.

\subsection*{Ejercicio 9.b Hamiltoneano de «benceno» y conmutación con \(R\)}

Despreciando el salto del electrón entre átomos, el Hamiltoniano no perturbado es
\[
H_0\ket{\varphi_n}=E_0\ket{\varphi_n},\quad n=1,\dots,6.
\]
La perturbación que permite el salto sólo entre vecinos se escribe
\[
W\ket{\varphi_1} = -a\bigl(\ket{\varphi_6}+\ket{\varphi_2}\bigr),\quad
W\ket{\varphi_2} = -a\bigl(\ket{\varphi_1}+\ket{\varphi_3}\bigr),\;\dots,\;
W\ket{\varphi_6} = -a\bigl(\ket{\varphi_5}+\ket{\varphi_1}\bigr).
\]
El Hamiltoniano total es \(H=H_0+W\). Como \(R\) permuta cíclicamente los \(\ket{\varphi_n}\) y \(H_0\) es trivial en esa base, se comprueba fácilmente que
\[
[\,R,H\,]=0.
\]
Por tanto, los autovectores comunes de \(R\) y \(H\) son precisamente los \(\ket{\psi_k}\) hallados en el apartado anterior. Sus energías son
\[
E_k = E_0 + \bra{\psi_k}W\ket{\psi_k}
= E_0 \;-\;2a\cos\!\frac{2\pi k}{6},
\quad k=0,1,\dots,5.
\]
En ninguno de estos estados el electrón queda localizado en un solo átomo; su función de onda se extiende a todo el anillo. Estos resultados son esenciales para explicar el comportamiento electrónico del benceno.



\subsection*{A-3. Propiedades generales del Hamiltoniano cuántico}

En mecánica cuántica, las coordenadas clásicas \(x\) y \(p\) se reemplazan por los operadores \(\hat X\) y \(\hat P\), que cumplen la relación de conmutación:
\[
[\hat X,\hat P]=i\hbar
\tag{A-14}
\]
Partiendo de la forma clásica
\[
H=\frac{p^2}{2m} \;+\;\frac12\,m\omega^2\,x^2,
\]
se obtiene el operador Hamiltoniano
\[
\hat H \;=\; \frac{\hat P^2}{2m} \;+\;\frac12\,m\omega^2\,\hat X^2.
\tag{A-15}
\]
Al ser \(\hat H\) independiente del tiempo, resolvemos la ecuación de autovalores
\[
\hat H\ket{\varphi}=E\ket{\varphi}
\tag{A-16}
\]
que en representación \(x\) toma la forma
\[
\Bigl[-\tfrac{\hbar^2}{2m}\tfrac{d^2}{dx^2} + \tfrac12\,m\omega^2\,x^2\Bigr]\varphi(x)
= E\,\varphi(x).
\tag{A-17}
\]
De aquí se deducen las siguientes propiedades:
\begin{enumerate}
  \item Los valores propios de \(\hat H\) son positivos, pues si \(V(x)\ge V_m\) entonces
  \[
    E>V_m.
    \tag{A-18}
  \]
  \item Las autofunciones tienen paridad definida, dado que el potencial es par:
  \[
    V(-x)=V(x).
    \tag{A-19}
  \]
  \item El espectro es discreto, ya que el movimiento queda confinado en una región limitada del eje \(x\).
\end{enumerate}

\subsection*{B-1. Notación y operadores adimensionales}

Para simplificar la resolución, definimos los operadores adimensionales
\[
  \hat X = \sqrt{\frac{m\omega}{\hbar}}\;\hat x,
  \quad
  \hat P = \frac{1}{\sqrt{m\hbar\omega}}\;\hat p,
  \tag{B-1}
\]
que satisfacen el conmutador canónico
\[
  [\hat X,\hat P]=i.
  \tag{B-2}
\]
El Hamiltoniano se factoriza como
\[
  H = \hbar\omega\,\hat H,
  \tag{B-3}
\]
donde el operador adimensional es
\[
  \hat H = \frac12\bigl(\hat X^2 + \hat P^2\bigr).
  \tag{B-4}
\]
Buscamos las soluciones de la ecuación de autovalores
\[
  \hat H\ket{\varphi_n} = \varepsilon_n\,\ket{\varphi_n}.
  \tag{B-5}
\]

\subsubsection*{B-1-a. Operador \(a^\dagger a\) y Hamiltoniano adimensional}

Partiendo de las definiciones
\[
a = \frac{1}{\sqrt2}(\hat X + i\hat P), 
\quad
a^\dagger = \frac{1}{\sqrt2}(\hat X - i\hat P),
\]
calculamos primero
\begin{equation}
\begin{split}
a^\dagger a
&= \tfrac12\,(\hat X - i\hat P)(\hat X + i\hat P)\\
&= \tfrac12\bigl(\hat X^2 + \hat P^2 + i[\hat X,\hat P]\bigr)\\
&= \tfrac12\bigl(\hat X^2 + \hat P^2 -1\bigr)
\end{split}
\tag{B-10}
\end{equation}
Comparando con la forma adimensional del Hamiltoniano
\(\hat H = \tfrac12(\hat X^2 + \hat P^2)\) (Ecuación (B-4)), obtenemos
\begin{equation}
\hat H 
= a^\dagger a + \tfrac12
= \tfrac12\,(\hat X - i\hat P)(\hat X + i\hat P) \;+\;\tfrac12
\tag{B-11}
\end{equation}
y, de manera análoga,
\begin{equation}
\hat H = a\,a^\dagger - \tfrac12.
\tag{B-12}
\end{equation}
Introducimos entonces el operador número
\begin{equation}
N = a^\dagger a,
\tag{B-13}
\end{equation}
que es hermítico, pues
\begin{equation}
N^\dagger = (a^\dagger a)^\dagger = a^\dagger a = N.
\tag{B-14}
\end{equation}
Por tanto, de (B-11) se sigue
\begin{equation}
\hat H = N + \tfrac12.
\tag{B-15}
\end{equation}

\subsubsection*{B-1-b. Operadores \(a\), \(a^\dagger\) y \(N\)}

Si intentáramos factorizar \(\hat X^2 + \hat P^2\) como si \(\hat X,\hat P\) fuesen números, fallaríamos por la no conmutatividad. En su lugar, definimos
\begin{align}
a        &= \tfrac{1}{\sqrt2}(\hat X + i\hat P),
\tag{B-6a}\\
a^\dagger&= \tfrac{1}{\sqrt2}(\hat X - i\hat P).
\tag{B-6b}
\end{align}
Invertir estas relaciones da
\begin{align}
\hat X &= \tfrac{1}{\sqrt2}(a^\dagger + a),
\tag{B-7a}\\
\hat P &= \tfrac{i}{\sqrt2}(a^\dagger - a).
\tag{B-7b}
\end{align}
Su conmutador se calcula con (B-6) y la relación canónica \([\hat X,\hat P]=i\):
\begin{equation}
\begin{split}
[a,a^\dagger]
&= \tfrac12\,[\,\hat X + i\hat P\,,\,\hat X - i\hat P\,]\\
&= \tfrac{i}{2}\,[\hat P,\hat X] \;-\;\tfrac{i}{2}\,[\hat X,\hat P]
=1.
\end{split}
\tag{B-8}
\end{equation}
Es decir,
\begin{equation}
[a,a^\dagger]=1,
\tag{B-9}
\end{equation}
que es equivalente a la conmutación canónica \([\hat X,\hat P]=i\hbar\).```\subsubsection*{B-1-a. Operador \(a^\dagger a\) y Hamiltoniano adimensional}

Partiendo de las definiciones
\[
a = \frac{1}{\sqrt2}(\hat X + i\hat P), 
\quad
a^\dagger = \frac{1}{\sqrt2}(\hat X - i\hat P),
\]
calculamos primero
\begin{equation}
\begin{split}
a^\dagger a
&= \tfrac12\,(\hat X - i\hat P)(\hat X + i\hat P)\\
&= \tfrac12\bigl(\hat X^2 + \hat P^2 + i[\hat X,\hat P]\bigr)\\
&= \tfrac12\bigl(\hat X^2 + \hat P^2 -1\bigr)
\end{split}
\tag{B-10}
\end{equation}
Comparando con la forma adimensional del Hamiltoniano
\(\hat H = \tfrac12(\hat X^2 + \hat P^2)\) (Ecuación (B-4)), obtenemos
\begin{equation}
\hat H 
= a^\dagger a + \tfrac12
= \tfrac12\,(\hat X - i\hat P)(\hat X + i\hat P) \;+\;\tfrac12
\tag{B-11}
\end{equation}
y, de manera análoga,
\begin{equation}
\hat H = a\,a^\dagger - \tfrac12.
\tag{B-12}
\end{equation}
Introducimos entonces el operador número
\begin{equation}
N = a^\dagger a,
\tag{B-13}
\end{equation}
que es hermítico, pues
\begin{equation}
N^\dagger = (a^\dagger a)^\dagger = a^\dagger a = N.
\tag{B-14}
\end{equation}
Por tanto, de (B-11) se sigue
\begin{equation}
\hat H = N + \tfrac12.
\tag{B-15}
\end{equation}

\subsubsection*{B-1-b. Operadores \(a\), \(a^\dagger\) y \(N\)}

Si intentáramos factorizar \(\hat X^2 + \hat P^2\) como si \(\hat X,\hat P\) fuesen números, fallaríamos por la conmutatividad. En su lugar, definimos
\begin{align}
a        &= \tfrac{1}{\sqrt2}(\hat X + i\hat P),
\tag{B-6a}\\
a^\dagger&= \tfrac{1}{\sqrt2}(\hat X - i\hat P).
\tag{B-6b}
\end{align}
Invertir estas relaciones da
\begin{align}
\hat X &= \tfrac{1}{\sqrt2}(a^\dagger + a),
\tag{B-7a}\\
\hat P &= \tfrac{i}{\sqrt2}(a^\dagger - a).
\tag{B-7b}
\end{align}
Su conmutador se calcula con (B-6) y la relación canónica \([\hat X,\hat P]=i\):
\begin{equation}
\begin{split}
[a,a^\dagger]
&= \tfrac12\,[\,\hat X + i\hat P\,,\,\hat X - i\hat P\,]\\
&= \tfrac{i}{2}\,[\hat P,\hat X] \;-\;\tfrac{i}{2}\,[\hat X,\hat P]
=1.
\end{split}
\tag{B-8}
\end{equation}
Es decir,
\begin{equation}
[a,a^\dagger]=1,
\tag{B-9}
\end{equation}
que es equivalente a la conmutación canónica \([\hat X,\hat P]=i\hbar\).

\subsection*{B-2. Determinación del espectro}

Cuando se resuelve la ecuación de autovalores de \(N\),
\[
N\ket{\varphi_\nu}=\nu\ket{\varphi_\nu},
\tag{B-18}
\]
se demuestra que \(\ket{\varphi_\nu}\) es también autovector de \(\hat H\) con autovalor
\[
E_\nu=(\nu+\tfrac12)\hbar,
\tag{B-19}
\]
pues \(\hat H=N+\tfrac12\) [Ecuaciones (B-3) y (B-15)].

\subsubsection*{B-2-a. Lemas}

\paragraph{Lema I (propiedad de los autovalores de \(N\)).}  
Los autovalores \(\nu\) de \(N=a^\dagger a\) son no negativos.  
De hecho, para cualquier autovector \(\ket{\varphi_\nu}\) de \(N\):
\[
\|a\ket{\varphi_\nu}\|^2
=\bra{\varphi_\nu}a^\dagger a\ket{\varphi_\nu}
=\bra{\varphi_\nu}N\ket{\varphi_\nu}
=\nu\braket{\varphi_\nu|\varphi_\nu}\ge0.
\tag{B-20,B-21}
\]
Como \(\braket{\varphi_\nu|\varphi_\nu}>0\), se obtiene \(\nu\ge0\).
  
\paragraph{Lema II (propiedades de \(a\ket{\varphi_\nu}\)).}  
Sea \(\ket{\varphi_\nu}\ne0\) autovector de \(N\) con autovalor \(\nu\); definamos  
\(\ket{\chi}=a\ket{\varphi_\nu}\). Entonces:  
(i) Si \(\nu=0\), \(\|\ket{\chi}\|^2=\nu\|\varphi_\nu\|^2=0\) y \(\ket{\chi}=0\).  
(ii) Si \(\nu>0\), \(\|\ket{\chi}\|^2>0\) y, usando \([N,a]=-a\) [(B-17a)],
\[
N\ket{\chi}=(aN+[N,a])\ket{\varphi_\nu}
=(\nu-1)\ket{\chi}\,,
\tag{B-27}
\]
de modo que \(\ket{\chi}\) es un autovector de \(N\) con autovalor \(\nu-1\).

\paragraph{Lema III (propiedades de \(a^\dagger\ket{\varphi_\nu}\)).}  
Sea \(\ket{\varphi_\nu}\ne0\) autovector de \(N\) con \(\nu\ge0\); definamos  
\(\ket{\chi'}=a^\dagger\ket{\varphi_\nu}\). Entonces:  
(i) \(\|\ket{\chi'}\|^2=\bra{\varphi_\nu}aa^\dagger\ket{\varphi_\nu}
=(\nu+1)\|\varphi_\nu\|^2>0\), luego \(\ket{\chi'}\ne0\).  
(ii) Con \([N,a^\dagger]=a^\dagger\) [(B-17b)] se halla
\[
N\ket{\chi'}
=(a^\dagger N+[N,a^\dagger])\ket{\varphi_\nu}
=(\nu+1)\ket{\chi'}\,.
\tag{B-29}
\]

\subsubsection*{B-2-b. El espectro de \(N\) son enteros no negativos}

Sea \(\nu\) un autovalor de \(N\) y \(\ket{\varphi_\nu}\) su autovector.  

– Si \(\nu\notin\mathbb{Z}\), existe \(k\in\mathbb{Z}\) tal que  
\(\nu-k<0\)  \((\!)\).  
Aplicando sucesivamente \(a\) a \(\ket{\varphi_\nu}\), por el lema II se obtienen autovectores de autovalores \(\nu-1,\nu-2,\dots,\nu-k<0\), lo cual contradice el lema I.  

– Si \(\nu=n\in\{0,1,2,\dots\}\), la cadena  
\[
\ket{\varphi_n},\;a\ket{\varphi_n},\;\dots,\;a^n\ket{\varphi_n},\;
a^{n+1}\ket{\varphi_n}=0
\tag{B-31,B-33}
\]
termina en cero y no produce autovalores negativos.  

Por tanto, los únicos autovalores posibles son los enteros \(\nu=0,1,2,\dots\).

\subsubsection*{B-2-c. Interpretación de los operadores \(a\) y \(a^\dagger\)}

Partiendo de un autovector \(\ket{\varphi_n}\) de \(N\) (y por tanto de \(\hat H\)) con  
\[
\hat H\ket{\varphi_n}=(n+\tfrac12)\hbar\ket{\varphi_n},
\]
la acción de \(a\) y \(a^\dagger\) satisface:
\[
a\ket{\varphi_n}\propto\ket{\varphi_{n-1}},\qquad
a^\dagger\ket{\varphi_n}\propto\ket{\varphi_{n+1}}.
\]
Así, \(a\) “aniquila” un quantum \(\hbar\) de energía y \(a^\dagger\) lo “crea”, de ahí su nombre de operadores de destrucción y creación.

\subsubsection*{B-2-d. Niveles de energía}

De lo anterior y la relación (B-19) se concluye que los niveles de energía del oscilador armónico 1-D son
\[
E_n=\hbar\omega\Bigl(n+\tfrac12\Bigr),
\quad n=0,1,2,\dots
\tag{B-34}
\]
con separación \(\Delta E=\hbar\omega\) y nivel de cero punto \(E_0=\tfrac12\hbar\omega\).````\subsection*{B-2. Determinación del espectro}

Cuando se resuelve la ecuación de autovalores de \(N\),
\[
N\ket{\varphi_\nu}=\nu\ket{\varphi_\nu},
\tag{B-18}
\]
se demuestra que \(\ket{\varphi_\nu}\) es también autovector de \(\hat H\) con autovalor
\[
E_\nu=(\nu+\tfrac12)\hbar,
\tag{B-19}
\]
pues \(\hat H=N+\tfrac12\) [Ecuaciones (B-3) y (B-15)].

\subsubsection*{B-2-a. Lemas}

\paragraph{Lema I (propiedad de los autovalores de \(N\)).}  
Los autovalores \(\nu\) de \(N=a^\dagger a\) son no negativos.  
De hecho, para cualquier autovector \(\ket{\varphi_\nu}\) de \(N\):
\[
\|a\ket{\varphi_\nu}\|^2
=\bra{\varphi_\nu}a^\dagger a\ket{\varphi_\nu}
=\bra{\varphi_\nu}N\ket{\varphi_\nu}
=\nu\braket{\varphi_\nu|\varphi_\nu}\ge0.
\tag{B-20,B-21}
\]
Como \(\braket{\varphi_\nu|\varphi_\nu}>0\), se obtiene \(\nu\ge0\).
  
\paragraph{Lema II (propiedades de \(a\ket{\varphi_\nu}\)).}  
Sea \(\ket{\varphi_\nu}\ne0\) autovector de \(N\) con autovalor \(\nu\); definamos  
\(\ket{\chi}=a\ket{\varphi_\nu}\). Entonces:  
(i) Si \(\nu=0\), \(\|\ket{\chi}\|^2=\nu\|\varphi_\nu\|^2=0\) y \(\ket{\chi}=0\).  
(ii) Si \(\nu>0\), \(\|\ket{\chi}\|^2>0\) y, usando \([N,a]=-a\) [(B-17a)],
\[
N\ket{\chi}=(aN+[N,a])\ket{\varphi_\nu}
=(\nu-1)\ket{\chi}\,,
\tag{B-27}
\]
de modo que \(\ket{\chi}\) es un autovector de \(N\) con autovalor \(\nu-1\).

\paragraph{Lema III (propiedades de \(a^\dagger\ket{\varphi_\nu}\)).}  
Sea \(\ket{\varphi_\nu}\ne0\) autovector de \(N\) con \(\nu\ge0\); definamos  
\(\ket{\chi'}=a^\dagger\ket{\varphi_\nu}\). Entonces:  
(i) \(\|\ket{\chi'}\|^2=\bra{\varphi_\nu}aa^\dagger\ket{\varphi_\nu}
=(\nu+1)\|\varphi_\nu\|^2>0\), luego \(\ket{\chi'}\ne0\).  
(ii) Con \([N,a^\dagger]=a^\dagger\) [(B-17b)] se halla
\[
N\ket{\chi'}
=(a^\dagger N+[N,a^\dagger])\ket{\varphi_\nu}
=(\nu+1)\ket{\chi'}\,.
\tag{B-29}
\]

\subsubsection*{B-2-b. El espectro de \(N\) son enteros no negativos}

Sea \(\nu\) un autovalor de \(N\) y \(\ket{\varphi_\nu}\) su autovector.  

– Si \(\nu\notin\mathbb{Z}\), existe \(k\in\mathbb{Z}\) tal que  
\(\nu-k<0\)  \((\!)\).  
Aplicando sucesivamente \(a\) a \(\ket{\varphi_\nu}\), por el lema II se obtienen autovectores de autovalores \(\nu-1,\nu-2,\dots,\nu-k<0\), lo cual contradice el lema I.  

– Si \(\nu=n\in\{0,1,2,\dots\}\), la cadena  
\[
\ket{\varphi_n},\;a\ket{\varphi_n},\;\dots,\;a^n\ket{\varphi_n},\;
a^{n+1}\ket{\varphi_n}=0
\tag{B-31,B-33}
\]
termina en cero y no produce autovalores negativos.  

Por tanto, los únicos autovalores posibles son los enteros \(\nu=0,1,2,\dots\).

\subsubsection*{B-2-c. Interpretación de los operadores \(a\) y \(a^\dagger\)}

Partiendo de un autovector \(\ket{\varphi_n}\) de \(N\) (y por tanto de \(\hat H\)) con  
\[
\hat H\ket{\varphi_n}=(n+\tfrac12)\hbar\ket{\varphi_n},
\]
la acción de \(a\) y \(a^\dagger\) satisface:
\[
a\ket{\varphi_n}\propto\ket{\varphi_{n-1}},\qquad
a^\dagger\ket{\varphi_n}\propto\ket{\varphi_{n+1}}.
\]
Así, \(a\) “aniquila” un quantum \(\hbar\) de energía y \(a^\dagger\) lo “crea”, de ahí su nombre de operadores de destrucción y creación.

\subsubsection*{B-2-d. Niveles de energía}

De lo anterior y la relación (B-19) se concluye que los niveles de energía del oscilador armónico 1-D son
\[
E_n=\hbar\omega\Bigl(n+\tfrac12\Bigr),
\quad n=0,1,2,\dots
\tag{B-34}
\]
con separación \(\Delta E=\hbar\omega\) y nivel de cero punto \(E_0=\tfrac12\hbar\omega\).

\section*{Complemento MV: Ejercicios}

Considere un oscilador armónico de masa \(m\) y frecuencia angular \(\omega\). En \(t=0\), el estado del oscilador es
\[
\ket{\psi(0)}=\sum_{n=0}^\infty c_n\ket{\varphi_n},
\]
donde \(\{\ket{\varphi_n}\}\) son estados estacionarios con energías \(E_n=(n+\tfrac12)\hbar\omega\).

\begin{enumerate}
  \item ¿Cuál es la probabilidad \(\mathcal{P}\) de que una medición de la energía realizada en un tiempo arbitrario \(t>0\) dé un resultado mayor que \(2\hbar\omega\)? Cuando \(\mathcal{P}=0\), ¿qué coeficientes \(c_n\) son distintos de cero?
  \item A partir de ahora, suponga que sólo \(c_0\) y \(c_1\) son distintos de cero. Escriba la condición de normalización para \(\ket{\psi(0)}\) y exprese el valor medio \(\langle H\rangle\) de la energía en función de \(c_0\) y \(c_1\). Con el requisito adicional \(\langle H\rangle=\hbar\omega\), calcule \(\lvert c_0\rvert^2\) y \(\lvert c_1\rvert^2\).
  \item Dado que el estado normalizado \(\ket{\psi(0)}\) está definido sólo hasta un factor de fase global, fijamos este factor eligiendo \(c_0\) real y positivo. Sea
  \[
    c_1 = \lvert c_1\rvert\,e^{\,i\theta_1}.
  \]
  Suponemos además que 
  \(\langle H\rangle=\hbar\omega\) y que 
  \(\langle X\rangle=\tfrac12\sqrt{\hbar/(m\omega)}\).  
  Calcule el ángulo de fase \(\theta_1\).
  \item Con \(\ket{\psi(0)}\) así determinado, escriba \(\ket{\psi(t)}\) para \(t>0\) y calcule el valor de \(\theta_1\) en el instante \(t\). Deduzca a partir de ello la evolución del valor medio \(\langle X\rangle(t)\) de la posición.
\end{enumerate}

\subsection*{Ejercicio 2. Oscilador armónico anisótropo en tres dimensiones}

En un problema tridimensional, considere una partícula de masa \(m\) y con energía potencial
\[
V(X,Y,Z)
= \frac{m\omega^2}{2}\Bigl[\bigl(1+\tfrac{2\lambda}{3}\bigr)\,(X^2+Y^2)
                      +\bigl(1-\tfrac{4\lambda}{3}\bigr)\,Z^2\Bigr],
\]
donde \(\omega\ge0\) y \(0\le\lambda<\tfrac34\).

\begin{enumerate}
  \item ¿Cuáles son los estados propios del Hamiltoniano y las energías correspondientes?
  \item Calcule y discuta, en función de \(\lambda\), la variación de la energía, la paridad y el grado de degeneración del estado fundamental y de los dos primeros estados excitados.
\end{enumerate}

\subsection*{Ejercicio 3. Oscilador armónico: dos partículas}

Dos partículas de la misma masa \(m\), con posiciones \(X_1,X_2\) y momentos \(P_1,P_2\), se encuentran sometidas al mismo potencial unidimensional
\[
V(X)=\tfrac12\,m\omega^2X^2.
\]
Las partículas no interactúan entre sí.

\begin{enumerate}
  \item Escriba el operador Hamiltoniano \(H\) del sistema de dos partículas. Demuestre que
  \[
    H = H_1 + H_2,
  \]
  donde \(H_1\) actúa sólo en el espacio de estados de la partícula 1 y \(H_2\) sólo en el de la partícula 2. Calcule los niveles de energía del sistema, sus grados de degeneración y las funciones de onda correspondientes.
  
  \item ¿Forma \(H\) un conjunto completo de operadores que se conmutan mutuamente (C.S.C.O.)? ¿Y el conjunto \(\{H_1,H_2\}\)? Denotemos por \(\ket{\Phi_{n_1,n_2}}\) los autovectores comunes de \(H_1\) y \(H_2\). Escriba las relaciones de ortonormalidad y de cierre para los estados \(\{\ket{\Phi_{n_1,n_2}}\}\).
  
  \item Consideremos un sistema que, en \(t=0\), está en el estado
  \[
    \ket{\psi(0)}
    = \tfrac12\bigl(\ket{\Phi_{0,0}} + \ket{\Phi_{1,0}} + \ket{\Phi_{0,1}} + \ket{\Phi_{1,1}}\bigr).
  \]
  ¿Qué resultados pueden obtenerse, y con qué probabilidades, si en ese instante medimos:
  \begin{itemize}
    \item la energía total del sistema \(H\),
    \item la energía de la partícula 1 (\(H_1\)),
    \item la posición o la velocidad de la partícula 1?
  \end{itemize}
\end{enumerate}

\subsection*{Ejercicio 4 (continuación)}

El sistema de dos partículas, en \(t=0\), se encuentra en el mismo estado \(\ket{\psi(0)}\) definido en el ejercicio anterior.

\begin{enumerate}
  \item En \(t=0\) se mide la energía total \(H\) y el resultado hallado es \(2\hbar\omega\).  
  \begin{enumerate}
    \item Calcule los valores medios de la posición, del momento y de la energía de la partícula 1 para un tiempo arbitrario \(t>0\). Repita para la partícula 2.
    \item Para \(t>0\) se mide la energía de la partícula 1. ¿Qué valores pueden encontrarse y con qué probabilidades? Haga la misma pregunta para la medición de la posición de la partícula 1 y dibuje la densidad de probabilidad resultante.
  \end{enumerate}
  
  \item En lugar de medir la energía total \(H\), en \(t=0\) se mide la energía \(H_2\) de la partícula 2 y el resultado es \(\tfrac12\hbar\omega\). ¿Cómo cambian las respuestas a los apartados $\alpha$) y $\beta$) del punto a)?
\end{enumerate}

\subsection*{Ejercicio 5. Operador de intercambio de dos partículas}

Denotemos por \(\{\ket{\Phi_{n_1,n_2}}\}\) los autovectores comunes de \(H_1\) y \(H_2\), con autovalores \((n_1+\tfrac12)\hbar\omega\) y \((n_2+\tfrac12)\hbar\omega\). Definimos el operador de “intercambio” de partículas \(P_e\) mediante
\[
P_e \ket{\Phi_{n_1,n_2}} = \ket{\Phi_{n_2,n_1}}.
\]

\begin{enumerate}
\item Demuestra que \(P_e^{-1}=P_e\) y que \(P_e\) es unitario. ¿Cuáles son los autovalores de \(P_e\)? Sea \(B\) un observable cualquiera y definamos
\[
B' = P_e\,B\,P_e^\dagger.
\]
Muestra que la condición \(B'=B\) (invarianza de \(B\) bajo el intercambio) es equivalente a \([B,P_e]=0\).

\item Prueba que
\[
P_e \,H_1\,P_e^\dagger = H_2,
\quad
P_e \,H_2\,P_e^\dagger = H_1.
\]
¿Conmuta el Hamiltoniano total \(H=H_1+H_2\) con \(P_e\)? Calcula la acción de \(P_e\) sobre los observables \(X_1,P_1,X_2,P_2\).

\item Construye una base de autovectores comunes a \(H\) y \(P_e\). ¿Forman estos operadores un C.S.C.O.? ¿Qué sucede con el espectro de \(H\) y la degeneración de sus autovalores si se conservan únicamente los autovectores \(\ket{\Phi}\) de \(H\) que satisfacen 
\[
P_e\ket{\Phi}=-\ket{\Phi}\;?
\]
\end{enumerate}

\subsection*{Ejercicio 6. Oscilador armónico cargado en un campo eléctrico variable}

Consideremos un oscilador armónico unidimensional de masa \(m\), carga \(q\) y potencial  
\[
V(X)=\tfrac12\,m\omega^2X^2.
\]
Si se aplica un campo eléctrico \(\mathcal{E}(t)\parallel Oz\), el término de interacción es
\[
W(t)=-\,q\,\mathcal{E}(t)\,X.
\]
\begin{enumerate}
  \item Exprese el Hamiltoniano total \(H(t)=H_0+W(t)\) en términos de los operadores de creación y aniquilación \(a\), \(a^\dagger\). Calcule los conmutadores \([a,H(t)]\) y \([a^\dagger,H(t)]\).
  
  \item Sea
  \[
    \alpha(t)=\bra{\psi(t)}\,a\,\ket{\psi(t)},
  \]
  donde \(\ket{\psi(t)}\) es el estado normalizado de la partícula. Muestre, usando a), que  
  \[
    \frac{d}{dt}\,\alpha(t)
    = -\,i\,\omega\,\alpha(t)\;+\;i\,\lambda(t),
    \quad
    \lambda(t)=\frac{q}{\sqrt{2m\hbar\omega}}\;\mathcal{E}(t).
  \]
  Integre esta ecuación diferencial. A partir de \(\alpha(t)\), determine los valores medios de la posición \(\langle X\rangle(t)\) y del momento \(\langle P\rangle(t)\).

  \item Sea \(\ket{\psi(t)}\) el estado normalizado de un oscilador armónico en un campo eléctrico variable, y definamos
\[
\alpha(t)=\bra{\psi(t)}\,a\,\ket{\psi(t)}.
\]
  \item Definimos el «ket desplazado»
  \[
    \ket{\phi(t)} = \bigl[a - \alpha(t)\bigr]\,\ket{\psi(t)}.
  \]
  Donde \(\alpha(t)\) es el valor calculado en el apartado b). Usando los resultados de a) y b) demuestre que
  \[
    i\hbar\frac{d}{dt}\ket{\phi(t)}
    = \bigl[\,H(t)+\hbar\omega\bigr]\,\ket{\phi(t)}.
  \]
  ¿Cómo varía con el tiempo la norma de \(\ket{\phi(t)}\)?

  \item Suponga que \(\ket{\psi(0)}\) es autovector de \(a\) con autovalor \(\alpha(0)\). Demuestre que \(\ket{\psi(t)}\) sigue siendo autovector de \(a\) y calcule su autovalor \(\alpha(t)\).  
  A continuación, en \(t\) calcule el valor medio del Hamiltoniano no perturbado
  \[
    H_0 = H(t) - W(t)
  \]
  en función de \(\alpha(0)\). Dé además las desviaciones cuadráticas RMS \(\Delta X\), \(\Delta P\) y \(\Delta H_0\). ¿Cómo dependen de \(t\)?

  \item Suponga que en \(t=0\) el oscilador está en el estado fundamental \(\ket{\varphi_0}\). El campo eléctrico actúa entre \(0\) y \(T\) y luego se anula.  
  \begin{enumerate}
    \item Para \(t>T\), ¿cómo evolucionan los valores medios \(\langle X\rangle(t)\) y \(\langle P\rangle(t)\)?  
    \item Aplicación: suponga que durante \(0<t<T\) el campo es  
    \[
      \mathcal{E}(t)=\mathcal{E}_0\cos(\omega' t).
    \]
    Discuta el fenómeno de resonancia en función de \(\Delta\omega=\omega'-\omega\).  
    Si para \(t>T\) se mide la energía, ¿qué resultados pueden encontrarse y con qué probabilidades?
  \end{enumerate}
\end{enumerate}

\subsection*{Ejercicio 7}

Sea un oscilador armónico unidimensional con hamiltoniano \(H\) y estados estacionarios \(\{\ket{\varphi_n}\}\), tales que
\[
  H\ket{\varphi_n}=(n+\tfrac12)\hbar\omega\,\ket{\varphi_n}.
\]
Definimos el operador de traslación
\[
  U(k)=e^{\,i k X},
\]
con \(k\in\mathbb{R}\).

\begin{enumerate}
  \item ¿Es \(U(k)\) unitario? Demuestre que, para todo \(n\), sus elementos de matriz satisfacen
  \[
    \sum_{n'}\bigl|\bra{\varphi_n}U(k)\ket{\varphi_{n'}}\bigr|^2 \;=\;1.
  \]
  
  \item Exprese \(U(k)\) en términos de los operadores \(a\) y \(a^\dagger\). Utilice la fórmula de Glauber (fórmula (63) del Complemento B\,II) para escribir \(U(k)\) como producto de exponentes:
  \[
    U(k)=\exp\bigl(A\,a^\dagger\bigr)\;\exp\bigl(B\,a\bigr)\;\exp\bigl(C\bigr),
  \]
  determinando las constantes \(A,B,C\).
  
  \item Pruebe las siguientes identidades, para un parámetro \(\lambda\in\mathbb{C}\):
  \[
    e^{\lambda\,a}\,\ket{\varphi_0} = \ket{\varphi_0},
    \qquad
    \bra{\varphi_n}\,e^{\lambda\,a^\dagger}\,\ket{\varphi_0}
    = \frac{\lambda^n}{\sqrt{n!}}.
  \]

  \item Encuentre la expresión, en términos de 
  \[
    E_k = \frac{\hbar^2k^2}{2m}
    \quad\text{y}\quad
    E_\omega = \hbar\omega,
  \]
  para el elemento de matriz
  \[
    \bra{\varphi_0}\,U(k)\,\ket{\varphi_n}.
  \]
  ¿Qué sucede cuando \(k\to0\)? ¿Se podía haber predicho este resultado de forma directa?
\end{enumerate}

\subsection*{Ejercicio 8. Operador de evolución del oscilador armónico}

Definimos  
\[
U(t,0)=e^{-\tfrac{i}{\hbar}Ht},\qquad
H=\hbar\omega\Bigl(a^\dagger a + \tfrac12\Bigr).
\]

\begin{enumerate}
\item Defina los operadores en Heisenberg
\[
\widetilde a(t)=U^\dagger(t,0)\,a\,U(t,0),\quad
\widetilde a^\dagger(t)=U^\dagger(t,0)\,a^\dagger\,U(t,0).
\]
Calculando su acción sobre los autovectores \(\ket{\varphi_n}\) de \(H\), obtenga las expresiones de \(\widetilde a(t)\) y \(\widetilde a^\dagger(t)\) en función de \(a\) y \(a^\dagger\).

\item Calcule los operadores  
\[
\widetilde X(t)=U^\dagger(t,0)\,X\,U(t,0),\quad
\widetilde P(t)=U^\dagger(t,0)\,P\,U(t,0).
\]
¿Cómo pueden interpretarse físicamente las relaciones obtenidas?

\item Demuestre que  
\[
U^\dagger\!\Bigl(\tfrac{\pi}{2\omega},0\Bigr)\ket{x}
\]
es un autovector del momento \(P\) y especifique su autovalor.  
De igual modo, pruebe que  
\[
U^\dagger\!\Bigl(\tfrac{\pi}{2\omega},0\Bigr)\ket{p}
\]
es un autovector de la posición \(X\).

\item  En \(t=0\), la función de onda del oscilador es \(\psi(x,0)\). ¿Cómo puede obtenerse a partir de \(\psi(x,0)\) la función de onda \(\psi(x,t_q)\) en los tiempos  
\[
t_q = \frac{q\pi}{2\omega},\qquad q\in\mathbb{N}^+?
\]

\item  Elija para \(\psi(x,0)\) la función de onda \(\varphi_n(x)\) correspondiente a un estado estacionario. A partir de la pregunta anterior, derive la relación que debe existir entre \(\varphi_n(x)\) y su transformada de Fourier \(\widetilde\varphi_n(p)\).

\item  Describa cualitativamente la evolución de la función de onda en los siguientes casos:  
(i) \(\psi(x,0)=e^{ikx}\) con \(k\in\mathbb{R}\).  
(ii) \(\psi(x,0)=e^{-\rho x}\) con \(\rho>0\).

(iii) \[
\psi(x,0)=
\begin{cases}
\dfrac{1}{\sqrt a}, & -\tfrac a2 \le x \le \tfrac a2,\\
0,                   & \text{en otro caso}.
\end{cases}
\]
(iv) \(\psi(x,0)=e^{-\rho^2 x^2}\) con \(\rho\in\mathbb{R}\).

\end{enumerate}

\section*{Conclusión}
El estudio del espín $1/2$ ejemplifica claramente los postulados de la mecánica cuántica: preparación de estados, probabilidades de medición y evolución temporal gobernada por el Hamiltoniano.
\end{document}